\begin{refsection}
\chapter*{Extended Abstract}

\subsection*{Problemstellung}

Die zunehmende Integration generativer KI in wissensintensive Entscheidungsprozesse verändert die Ausgestaltung von Führungsarbeit in Organisationen grundlegend. Während bestehende Forschung vorwiegend Effizienz-, Produktivitäts- und Leistungsaspekte dieser Technologien untersucht \parencite{bankinsMultilevelReviewArtificial2024}, ist bislang unzureichend verstanden, wie Führungskräfte den Einsatz generativer KI motivational erleben. Aus Perspektive des Organizational Behavior ist jedoch zentral, dass Arbeitsgestaltung und subjektives Arbeitserleben eng mit Motivation, Leistungsfähigkeit und Wohlbefinden verknüpft sind \parencite{deci_self-determination_2017,van_den_broeck_review_2016}.

Insbesondere im mittleren Management von Banken, wo Entscheidungsarbeit durch hohe Komplexität, Verantwortung und regulatorische Anforderungen geprägt ist, wird generative KI zunehmend in Entscheidungsvorbereitungsprozesse integriert. Diese Systeme wirken dabei nicht nur als Effizienzwerkzeuge, sondern verändern das sozio-technische Arbeitsumfeld, in dem Führungsarbeit stattfindet. Vor diesem Hintergrund stellt sich die Frage, wie diese Technologien über psychologische Wirkmechanismen das Erleben von Führungsarbeit beeinflussen.

Die Self-Determination Theory (SDT) bietet hierfür einen geeigneten theoretischen Rahmen, da sie postuliert, dass Motivation und Wohlbefinden wesentlich von der Befriedigung der Grundbedürfnisse nach Autonomie, Kompetenz und sozialer Eingebundenheit abhängen 
\parencite{deci_what_2000,deci_self-determination_2017}. Der Einsatz generativer KI kann aus dieser Perspektive sowohl bedürfnisunterstützend als auch bedürfnisfrustrierend wirken, abhängig davon, wie Führungskräfte deren Rolle in Entscheidungsprozessen interpretieren \parencite{gagneUnderstandingShapingFuture2022,edwards_managerial_2024}.

Ziel der Arbeit ist es daher zu verstehen, wie Führungskräfte im mittleren Management von Banken den Einsatz generativer KI in Entscheidungsvorbereitungsprozessen erleben und unter welchen Bedingungen diese Technologien als unterstützend oder einschränkend für Autonomie, Kompetenz und soziale Eingebundenheit wahrgenommen werden. Mittleres Management ist für diese Fragestellung besonders aufschlussreich: Führungskräfte auf dieser Ebene treffen täglich eine hohe Anzahl an Entscheidungen unter Zeitdruck und Ressourcenknappheit, tragen dabei jedoch gleichzeitig substanzielle Verantwortung gegenüber übergeordneten Hierarchieebenen und eigenen Teams \parencite{Mintzberg1973Nature, floydManagingStrategicConsensus1997}. 
Gerade diese Verdichtung von Entscheidungsvolumen, Komplexität und Verantwortungszurechnung macht das mittlere Management zu einem besonders sensiblen Ort, an dem die motivationalen Konsequenzen technologischer Eingriffe in Entscheidungsprozesse sichtbar werden.

\subsection*{Theoretische Verortung}

Die Arbeit ist im Forschungsfeld Organizational Behavior verortet und integriert drei Perspektiven: (1) digitale Transformation als organisationsentwicklungsrelevanter Eingriff in Arbeits- und Entscheidungsprozesse, (2) generative KI als sozio-technisches Arbeitssystem sowie (3) die Self-Determination Theory als motivationspsychologischer Kernrahmen.

Forschung zu KI in Organisationen betont zunehmend, dass algorithmische Systeme nicht isoliert technologisch betrachtet werden dürfen, sondern als Teil sozio-technischer Arrangements, deren Wirkung sich über Wahrnehmungen und Sinnzuschreibungen der Nutzenden entfaltet \parencite{bankinsMultilevelReviewArtificial2024,tongJanusFaceArtificial2021}. Studien zeigen, dass solche Systeme entweder als unterstützende Ressource oder als kontrollierender Eingriff interpretiert werden können, was unterschiedliche motivationale Konsequenzen nach sich zieht \parencite{edwards_managerial_2024}.
SDT erklärt, wie Arbeitsbedingungen über die Befriedigung der Bedürfnisse nach Autonomie, Kompetenz und sozialer Eingebundenheit Motivation und Arbeitsoutcomes beeinflussen \parencite{van_den_broeck_review_2016}. Digitale Systeme können demnach als Formen der Arbeitsgestaltung verstanden werden, die diese Bedürfnisse gezielt unterstützen oder unterminieren \parencite{gagneUnderstandingShapingFuture2022,mcanally_self-determination_2024}.

\subsection*{Methodisches Vorgehen}

Die Untersuchung folgt einem qualitativen Forschungsdesign. Datengrundlage sind leitfadengestützte, problemzentrierte Interviews mit Führungskräften des mittleren Managements aus Banken im DACH-Raum. Dieser Kontext eignet sich besonders, da Entscheidungsprozesse hier durch hohe Komplexität, Dokumentationsanforderungen und Verantwortungszurechnung geprägt sind.

Ziel der Datenerhebung ist es, konkrete Entscheidungssituationen zu rekonstruieren, in denen generative KI eingesetzt wird, und zu erfassen, wie Führungskräfte diese Nutzung in Bezug auf Autonomie, Kompetenz und soziale Eingebundenheit deuten.

Die Datenauswertung erfolgt mittels strukturierender qualitativer Inhaltsanalyse mit deduktiv-induktiver Kategorienbildung. Die drei SDT-Bedürfnisse dienen als deduktive Hauptkategorien, während situative Deutungsmuster, Wahrnehmungen von Unterstützung, Kontrolle, Verantwortung und Zusammenarbeit induktiv aus dem Material entwickelt werden.

\subsection*{Erwarteter Beitrag}

Der Beitrag dieser Arbeit gliedert sich in eine theoretische und eine praktische Dimension. Beide sind aus dem skizzierten Forschungsrahmen abgeleitet.

Theoretisch erweitert die Studie die Anwendung der SDT auf einen bislang kaum untersuchten technologischen Kontext: den Einsatz generativer KI in Entscheidungsprozessen von Führungskräften. Bestehende SDT-Forschung hat digitale Arbeitssysteme als Gestaltungsgröße zwar konzeptuell diskutiert \parencite{gagneUnderstandingShapingFuture2022}, empirische Befunde für die spezifische Konstellation von generativer KI und Führungsarbeit in Entscheidungsprozessen fehlen jedoch weitgehend. Die Arbeit liefert hier erste qualitative Einsichten, ohne den Anspruch zu erheben, generalisierbare Gesetzmäßigkeiten zu formulieren.

Empirisch trägt sie zur Beschreibung einer in der Forschung unterbeschriebenen Gruppe bei: Führungskräfte im mittleren Management des Bankensektors, die generative KI aktiv in komplexe Entscheidungsprozesse einbinden. Durch die rekonstruierende Analyse konkreter Nutzungserfahrungen entsteht ein differenziertes Bild davon, unter welchen Bedingungen diese Technologien als bedürfnisunterstützend oder bedürfnisfrustrierend erlebt werden.

Praktisch liefert die Arbeit Orientierungspunkte für die motivationsgerechte Ausgestaltung generativer KI in Entscheidungsprozessen, nicht im Sinne universeller Handlungsempfehlungen, sondern als Beschreibung relevanter Wahrnehmungsmuster, die in Implementierungsüberlegungen bislang zu kurz kommen \parencite{prasadGenerativeAICatalyst2024, quaquebekeNowNewNext2023}.

\section*{Literaturquellen}
% Literaturverzeichnis
\printbibliography[heading=none]
\end{refsection}