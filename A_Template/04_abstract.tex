%\newpage
%\begin{abstract}
Generative KI verspricht Führungskräften Entlastung, doch während Unternehmen milliardenschwer in KI-Systeme investieren, bleibt unbeantwortet, wie die Menschen, die diese Systeme täglich nutzen, ihre eigene Wirksamkeit erleben. Die technologische Transformation von Führungsarbeit wird primär durch die Linse von Effizienzgewinnen betrachtet, während motivationale Konsequenzen weitgehend unsichtbar bleiben. Diese Studie untersucht auf Grundlage der Self-Determination Theory, wie Führungskräfte im mittleren Management von \gls{DACH}-Banken den Einsatz generativer KI in Entscheidungsvorbereitungsprozessen erleben und unter welchen Bedingungen diese Technologien die psychologischen Grundbedürfnisse nach Autonomie, Kompetenz und sozialer Eingebundenheit unterstützen oder unterminieren. Mittels leitfadengestützter Interviews und qualitativer Inhaltsanalyse wird ein prozessuales Verständnis der motivationalen Wirkung generativer KI in realen Führungspraktiken entwickelt. Die Studie leistet damit einen Beitrag zur Verbindung von KI-Forschung, motivationspsychologischer Perspektive im Organizational Behavior.

