%\section{Ausgangslage: Die Transformation der Wissensarbeit durch generative KI}

Die Einführung generativer künstlicher Intelligenz – insbesondere Large Language Models wie ChatGPT, GPT-4, Claude und Gemini – markiert einen Wendepunkt in der Geschichte der Arbeitsautomatisierung. Anders als frühere Technologiewellen, die primär manuelle und routinebasierte Tätigkeiten adressierten, zielt generative KI auf die Augmentation komplexer kognitiver Prozesse ab, die bislang als Kerndomäne hochqualifizierter Wissensarbeiter galten. Diese Entwicklung stellt Organisationen und insbesondere Führungskräfte vor fundamentale Herausforderungen: Wie lassen sich die erheblichen Produktivitätspotenziale realisieren, ohne dabei die motivationalen Grundlagen nachhaltiger Leistungsfähigkeit zu untergraben?

%\subsection{Generative KI als disruptive kognitive Augmentation}

Generative KI transformiert Wissensarbeit durch die Bereitstellung kognitiver Unterstützung auf Abruf, die sowohl die Zusammensetzung von Aufgaben als auch die Bewertung menschlicher Fähigkeiten verändert. Empirische Feldstudien dokumentieren substanzielle Produktivitätseffekte: In einer gestaffelten Einführung eines konversationellen KI-Assistenten bei 5.000 Kundenservice-Mitarbeitern stieg die Anzahl gelöster Anfragen pro Stunde um etwa \textbf{14\%} \parencite{brynjolfssonGenerativeAIWork2025}. Bemerkenswert ist die Verteilung dieser Gewinne: Die größten Produktivitätssteigerungen zeigten sich bei weniger erfahrenen Mitarbeitern, während hochqualifizierte Experten kaum profitierten. Dieses Muster illustriert den augmentativen Charakter generativer KI, die nicht uniform automatisiert, sondern gezielt Kompetenzlücken adressiert und damit die Leistungsverteilung innerhalb von Teams komprimiert.

Die konzeptionelle Disruption liegt in der Verschiebung kognitiver Arbeit: Während traditionelle Informationssysteme primär Zugang zu Daten ermöglichen, synthetisiert generative KI Wissen und generiert Inhalte. Dies verlagert menschliche Arbeit von der Suche und Synthese hin zur Verifikation, Evaluation und Orchestration \parencite{Robertson2024, Benbya2024}. Neue Kompetenzen – insbesondere \textit{Prompt Engineering} und die aktive Ko-Konstruktion von Wissen mit KI-Systemen – werden zunehmend als zentrale Fähigkeiten für Wissensarbeiter identifiziert.

Aus Perspektive des organisationalen Wissensmanagements eröffnen sich ambivalente Dynamiken. Einerseits ermöglicht generative KI die Dissemination impliziten Expertenwissens an weniger erfahrene Mitarbeiter und kann damit Wissensbarrieren abbauen \parencite{Storey2025}. Andererseits entstehen Risiken der Wissensverluste und übermäßigen Abhängigkeit von algorithmischen Systemen, insbesondere wenn kritische Evaluationsfähigkeiten erodieren oder organisationales Lernen durch unreflektierte KI-Nutzung untergraben wird \parencite{Storey2025}.

%\subsection{Spezifische Relevanz für wissensintensive Tätigkeiten}

Generative KI adressiert gezielt die Kernprozesse wissensintensiver Arbeit, die durch Komplexität, Kreativität und Kommunikation charakterisiert sind:

\textbf{Kreativität und Ideengenerierung.} KI-Systeme können Ideationsprozesse beschleunigen und Entwürfe produzieren, erfordern jedoch menschliches Urteilsvermögen zur Sicherstellung von Neuheit, Domänenpassung und strategischer Relevanz \parencite{Benbya2024}. Die Qualität kreativer Outputs hängt kritisch von der Fähigkeit ab, KI-generierte Vorschläge zu evaluieren, zu rekombinieren und in spezifische Kontexte zu übersetzen.

\textbf{Analyse und Sensemaking.} Durch die Bereitstellung synthetisierter Zusammenfassungen und Kandidatenanalysen verschiebt generative KI die Verteilung kognitiver Arbeit hin zu kritischer Bewertung und Integration \parencite{Storey2025}. Dies erfordert nicht nur technische Kompetenz im Umgang mit KI-Tools, sondern auch metakognitive Fähigkeiten zur Beurteilung der Validität, Vollständigkeit und Verzerrungen KI-generierter Analysen.

\textbf{Kommunikation und Textproduktion.} Effektive Nutzung generativer KI für kommunikative Aufgaben verlangt Prompt-Design-Fähigkeiten und iterative Validierungsprozesse. Trainingsstudien zeigen, dass strukturierte Prompt-Protokolle die Qualität der Mensch-KI-Ko-Konstruktion signifikant verbessern können \parencite{Robertson2024}. Gleichzeitig entstehen neue Herausforderungen in Bezug auf Authentizität, Autorschaft und die Wahrung professioneller Standards in KI-unterstützter Kommunikation.

%\subsection{Führungskräfte des mittleren Managements als primäre Nutzergruppe}

Führungskräfte des mittleren Managements nehmen eine besondere Position in der generativen KI-Transformation ein. Sie sind sowohl intensive Nutzer der Technologie für eigene Aufgaben (Berichtswesen, Datenanalyse, Kommunikation) als auch zentrale Vermittler, die strategische KI-Initiativen in operative Praktiken übersetzen müssen. Diese \textbf{Doppelrolle als Anwender und Implementierungsverantwortliche} prägt die Adoptionsdynamiken und schafft spezifische Spannungsfelder \parencite{JeanBaptiste2024, Dahlgren2024}.

Qualitative Studien dokumentieren aktive Identitätsarbeit von mittleren Führungskräften, die sich von reinen Implementierern zu strategischen Akteuren entwickeln \parencite{Duraipandi2024}. Diese Transition erfordert neue Kompetenzmixe, die technische Literalität mit erweiterten Soft Skills (Coaching, ethische Urteilskraft, Boundary Spanning) verbinden. Gleichzeitig berichten Führungskräfte von Unsicherheiten bezüglich der Zukunft ihrer Rollen und der Bewertung ihrer Expertise in KI-augmentierten Arbeitsumgebungen.

Organisationale Trends reflektieren diese Entwicklung: HR- und Managementforschung zeigt eine zunehmende Betonung der Qualifizierung mittlerer Führungskräfte für KI-Facilitation und Governance als integralen Bestandteil digitaler Transformationsagenden \parencite{Suseno2021}. Dies unterstreicht die strategische Bedeutung dieser Führungsebene für den Erfolg organisationaler KI-Implementierungen.

%\subsection{Organisationale Trends und Diffusionsdynamiken}

Die Diffusion generativer KI in Organisationen vollzieht sich mit beispielloser Geschwindigkeit. Während frühere Technologiewellen Jahrzehnte zur Durchdringung von Organisationen benötigten, berichten Unternehmensbefragungen bereits 2023 – ein Jahr nach der öffentlichen Verfügbarkeit von ChatGPT – von substantiellen Adoptionsraten. Die Kombination aus niedrigschwelligen Zugängen (natürlichsprachliche Interfaces), unmittelbaren Nutzenerlebnissen und strategischem Wettbewerbsdruck treibt eine rasche Integration in vielfältige Geschäftsfunktionen.

Diese Geschwindigkeit stellt jedoch Organisationen vor erhebliche Herausforderungen. Change-Management-Prozesse, die üblicherweise längerfristige Planungs- und Vorbereitungsphasen umfassen, müssen mit der Dynamik technologischer Entwicklung und bottom-up Adoption Schritt halten. Für mittlere Führungskräfte bedeutet dies oft, gleichzeitig eigene Lernkurven zu bewältigen, Teams zu unterstützen und organisationale Richtlinien zu entwickeln – eine dreifache Belastung, die motivationale Ressourcen beansprucht.

\section{Problemstellung und Relevanz}

Während die Effizienzgewinne generativer KI empirisch gut dokumentiert sind, bleibt die Perspektive der subjektiven Nutzererfahrung – insbesondere motivationale Erlebnisse, Wohlbefinden und psychologische Kosten – systematisch unterbelichtet. Diese Asymmetrie birgt erhebliche Risiken für die Nachhaltigkeit organisationaler KI-Implementierungen.

%\subsection{Das Effizienzparadigma der KI-Implementierung}

Organisationale KI-Einführungen werden typischerweise durch ein \textbf{Effizienzparadigma} geleitet: Ziele wie Durchsatzsteigerung, Fehlerreduktion und Skalierbarkeit dominieren Entscheidungsprozesse und Erfolgsmessung. Feldstudien bestätigen messbare Produktivitätsgewinne – etwa die bereits erwähnten 14\% Effizienzsteigerung im Kundenservice \parencite{Brynjolfsson2023} – sowie zusätzliche Vorteile wie verbesserte Kundenzufriedenheit und reduzierte Anforderungen an manageriale Interventionen.

Diese Fokussierung auf operationale Outputs ist nachvollziehbar, reflektiert jedoch eine potenzielle Engführung: Vorherrschende Erfolgsmetriken privilegieren quantifizierbare Effizienzgewinne, während psychosoziale Outcomes systematisch vernachlässigt werden \parencite{Benbya2024}. Die implizite Annahme, dass Produktivitätssteigerungen automatisch zu positiven Mitarbeitererfahrungen führen, erweist sich bei genauerer Betrachtung als problematisch.

%\subsection{Die vernachlässigte Perspektive: Subjektives Erleben und Wohlbefinden}

Eine wachsende Forschungslinie dokumentiert signifikante psychologische Kosten der KI-Adoption, die in effizienzorientierten Implementierungen häufig übersehen werden:

\textbf{KI-bedingte Angst und Wohlbefindenseinbußen.} Studien identifizieren \textit{KI-Angst} als substantiellen Faktor, der das Wohlbefinden von Mitarbeitern negativ beeinflusst \parencite{Suseno2021}. Diese Angst speist sich aus verschiedenen Quellen: Unsicherheit über zukünftige Jobsicherheit, wahrgenommene Bedrohung beruflicher Identität, Sorge vor Überwachung und Leistungstransparenz sowie Überforderung durch neue technologische Anforderungen. Entscheidend ist der Befund, dass organisationale Unterstützungsmaßnahmen und High-Performance-Work-Practices diese negativen Effekte moderieren können – ein Hinweis darauf, dass KI-Angst nicht unvermeidlich ist, sondern durch Gestaltungsentscheidungen beeinflusst werden kann.

\textbf{Technostress und differentielle Stressorwirkungen.} Die Technostress-Forschung differenziert zwischen \textit{Challenge-Stressoren} (Herausforderungen, die als bewältigbar und entwicklungsförderlich wahrgenommen werden) und \textit{Hindrance-Stressoren} (Barrieren, die als behindernd und belastend erlebt werden). Eine Drei-Wellen-Studie mit 301 Teilnehmern zeigte, dass Challenge-Stressoren die Adoptionsintention durch positive Affekte erhöhen können, während Hindrance-Stressoren die Adoptionsbereitschaft durch KI-Angst reduzieren \parencite{Chang2024}. Diese Effekte werden durch technische Selbstwirksamkeit moderiert: Mitarbeiter mit hoher Selbstwirksamkeit erleben technologische Anforderungen eher als Herausforderungen, während Mitarbeiter mit geringer Selbstwirksamkeit diese als Bedrohungen interpretieren.

\textbf{Wahrgenommene Automatisierungsrisiken und affektives Wohlbefinden.} Eine Surveystudie mit 349 Teilnehmern dokumentierte, dass das Bewusstsein für Automatisierungsrisiken (STARA – \textit{Susceptibility to Technology-Assisted Replacement Anxiety}) negativ mit affektivem Arbeitswohlbefinden assoziiert ist, mediiert durch erhöhten Jobstress \parencite{Jin2024}. Psychologische Resilienz fungiert als protektiver Faktor, der diese negativen Effekte abschwächt. Dies unterstreicht die Bedeutung individueller Ressourcen und organisationaler Resilienzförderung für die Bewältigung KI-induzierter Stressoren.

%\subsection{Motivationale Dimensionen als kritischer Erfolgsfaktor}

Motivationale Prozesse spielen eine zentrale, jedoch häufig unterschätzte Rolle für den langfristigen Erfolg von KI-Implementierungen:

\textbf{Engagement und Retention.} Die bereits zitierte Feldstudie von \textcite{Brynjolfsson2023} berichtet nicht nur Produktivitätssteigerungen, sondern auch verbesserte Mitarbeiterretention und reduzierte Anforderungen an manageriale Interventionen. Diese Befunde deuten auf positive motivationale Downstream-Effekte hin, wenn KI die Aufgabenbewältigung unterstützt und Erfolgserlebnisse ermöglicht. Allerdings fehlen bislang systematische Längsschnittstudien, die kausale Mechanismen zwischen KI-Nutzung, motivationalen Zuständen und Retention differenziert untersuchen.

\textbf{Intrinsische Motivation und Aufgabenbedeutsamkeit.} Emergente Forschung verweist auf Veränderungen intrinsischer Motivation und wahrgenommener Aufgabenbedeutsamkeit durch KI-Nutzung \parencite{Oesinghaus2024}. Wenn Routineaspekte von Aufgaben automatisiert werden, kann dies einerseits Raum für bedeutungsvollere Tätigkeiten schaffen. Andererseits besteht das Risiko, dass die Delegation kreativer oder analytischer Teilaufgaben an KI das Erleben von Autonomie und Meisterschaft untergraben – Kernelemente intrinsischer Motivation gemäß der Self-Determination Theory.

\textbf{Randbedingungen motivationaler Effekte.} Technische Selbstwirksamkeit und affektive Einstellungen gegenüber KI moderieren, ob technologische Stressoren in positive (Challenge) oder negative (Hindrance) motivationale Outcomes münden \parencite{Chang2024}. Dies impliziert, dass motivationsförderliche KI-Implementierungen nicht nur technologische, sondern auch psychologische Dimensionen adressieren müssen: Selbstwirksamkeitsförderung, positive Technologieframes und Unterstützung bei der Bewältigung von Unsicherheit sind ebenso zentral wie die Bereitstellung leistungsfähiger Tools.

%\subsection{Spezifische Herausforderungen für das mittlere Management}

Führungskräfte des mittleren Managements sind in besonderem Maße von den Spannungsfeldern der KI-Transformation betroffen:

\textbf{Sandwich-Position und Rollenkonflikte.} Mittlere Manager befinden sich in einer \enquote{Sandwich-Situation}: Sie müssen strategische KI-Direktiven von oben implementieren, während sie gleichzeitig die Moral und Leistungsfähigkeit ihrer Teams schützen und operative Störungen managen \parencite{JeanBaptiste2024, Dahlgren2024}. Diese Doppelbelastung intensiviert Rollenkonflikte und kann zu Identitätsspannungen führen, insbesondere wenn organisationale Erwartungen und die Realität der Teamdynamiken divergieren.

\textbf{Bedrohung der professionellen Identität.} Studien zur Identitätsarbeit dokumentieren, dass mittlere Führungskräfte ihre professionelle Identität als bedroht erleben, wenn Routineaufgaben automatisiert oder umverteilt werden \parencite{Duraipandi2024}. Dies löst Prozesse der Identitätsrekonstruktion aus: Führungskräfte repositionieren sich als Coaches, ethische Wächter und Boundary Spanner zwischen menschlichen und algorithmischen Akteuren. Diese Transition kann als Chance zur Professionalisierung oder als Verlust zentraler Kompetenzbereiche interpretiert werden – die subjektive Bewertung hängt von individuellen Ressourcen, organisationaler Unterstützung und der Qualität des Veränderungsprozesses ab.

\textbf{Widerstand und Change Readiness.} HR-Managerstudien zeigen, dass Überzeugungen und Ängste die Change Readiness vorhersagen, wobei systemische Praktiken (z.B. High-Performance-Work-Systems) diese Effekte moderieren \parencite{Suseno2021}. Widerstand mittlerer Führungskräfte ist somit nicht rein psychologisch, sondern auch institutionell bedingt: Fehlende Ressourcen, unklare Strategien und mangelnde Partizipation in Entscheidungsprozessen verstärken Widerstandsdynamiken. Umgekehrt können partizipative Implementierungsansätze und transparente Kommunikation Widerstände reduzieren und Change Readiness fördern.

%\subsection{Die Spannung zwischen Produktivitätsgewinnen und psychologischen Kosten}

Die empirische Evidenz dokumentiert ein ambivalentes Bild: KI-Adoption kann sowohl Produktivität als auch Wohlbefinden fördern, aber auch zu Stress und motivationalen Einbußen führen. Die folgende Tabelle fasst repräsentative Studien zusammen:

\begin{table}[h]
\centering
\caption{Empirische Evidenz zu Produktivitäts- und Wohlbefindenseffekten der KI-Adoption}
\label{tab:productivity-wellbeing}
\small
\begin{tabular}{@{}p{4.5cm}p{5cm}p{5cm}@{}}
\toprule
\textbf{Studie \& Stichprobe} & \textbf{Produktivitätsbefund} & \textbf{Wohlbefindens-/Stressbefund} \\
\midrule
Brynjolfsson \& Raymond (2023), n=5.000 Agents & +14\% gelöste Anfragen/Stunde mit KI-Zugang & Verbesserte Retention, weniger manageriale Interventionen \\
\addlinespace
Dong et al. (2024), n=313 Wissensarbeiter & KI kann innovatives Arbeitsverhalten durch Challenge-Appraisal fördern & KI kann Innovation durch Hindrance-Appraisal (Stress) hemmen \\
\addlinespace
Chang et al. (2024), n=301, drei Wellen & Challenge-Technostress erhöht Adoptionsintention via positive Affekte & Hindrance-Technostress reduziert Adoption via KI-Angst; moderiert durch Selbstwirksamkeit \\
\addlinespace
Jin et al. (2024), n=349 & – & STARA-Bewusstsein → Jobstress → geringeres affektives Wohlbefinden; Resilienz moderiert \\
\bottomrule
\end{tabular}
\end{table}

\textbf{Doppelschneidige Effekte.} Eine Studie mit 313 Wissensarbeitern zeigt, dass KI-Adoption innovatives Arbeitsverhalten fördern kann, wenn sie als Herausforderung interpretiert wird, es jedoch unterdrücken kann, wenn sie als Hindernis und Stressquelle erlebt wird \parencite{Dong2024}. Diese Ambivalenz unterstreicht die Notwendigkeit differenzierter Implementierungsstrategien, die nicht nur technologische, sondern auch psychologische und soziale Dimensionen adressieren.

\textbf{Integrative Gestaltungsanforderungen.} Ergonomie- und Arbeitspsychologie-Reviews betonen, dass neuartige digitale Systeme und KI-Interaktionen neue Stressoren auf individueller, organisationaler und soziokultureller Ebene generieren \parencite{Matthews2024}. Sie argumentieren für integrative Design- und Unterstützungsinterventionen, die technische Usability, organisationale Ressourcen und individuelle Bewältigungsstrategien systematisch verbinden.

\textbf{Implikation für nachhaltige Implementierung.} Es besteht konsistente empirische Evidenz sowohl für Produktivitätsgewinne als auch für psychosoziale Risiken. Die Vernachlässigung subjektiver Erfahrungen und motivationaler Prozesse riskiert, langfristige und nachhaltige KI-Implementierungen zu untergraben \parencite{Dong2024, Chang2024}. Organisationen, die kurzfristige Effizienzgewinne ohne parallele Investitionen in motivationale Ressourcen und Unterstützungsstrukturen priorisieren, gefährden die sehr Retentions- und Engagementziele, die KI-Einführungen oft erreichen sollen.

%\ssection{Relevanz: Wissenschaftliche und praktische Bedeutung der Untersuchung}

Die systematische Untersuchung motivationaler Erlebnisse von Führungskräften im Umgang mit generativer KI besitzt sowohl erhebliche wissenschaftliche als auch unmittelbare praktische Relevanz.

%\subsection{Wissenschaftliche Relevanz und theoretische Beiträge}

\textbf{Erweiterung der Motivationstheorie.} Die empirischen Muster – insbesondere die Differenzierung zwischen Challenge- und Hindrance-Appraisals sowie die moderierende Rolle von Selbstwirksamkeit – erfordern eine Erweiterung etablierter Motivationsmodelle. Die Self-Determination Theory, die Conservation of Resources Theory und das Challenge-Hindrance-Framework müssen für den Kontext der Mensch-KI-Partnerschaft adaptiert werden \parencite{Chang2024, Jin2024}. Spezifisch stellt sich die Frage, wie die drei psychologischen Grundbedürfnisse (Autonomie, Kompetenz, soziale Eingebundenheit) durch verschiedene Formen der KI-Nutzung beeinflusst werden und welche Gestaltungsparameter Bedürfnisbefriedigung vs. -frustration determinieren.

\textbf{Weiterentwicklung der Technologieakzeptanzforschung.} Klassische Modelle wie TAM (Technology Acceptance Model) und UTAUT (Unified Theory of Acceptance and Use of Technology) fokussieren primär auf kognitive Bewertungen (wahrgenommene Nützlichkeit, Benutzerfreundlichkeit). Die vorliegende Evidenz zu affektiven Mediatoren (positive Affekte vs. KI-Angst) und Randbedingungen (technische Selbstwirksamkeit, organisationale Unterstützung) legt nahe, dass reichhaltigere Modelle erforderlich sind, die emotionale und Identitätsdimensionen systematisch integrieren \parencite{Suseno2021, Dong2024}.

\textbf{Beiträge zur HR- und Organisationsforschung.} Die Identitätsarbeit und Rollenrekonfiguration mittlerer Manager im Kontext der KI-Infusion stellt einen fruchtbaren Forschungsbereich für Theorien institutioneller Translation und mikrofundierter digitaler Transformation dar \parencite{Duraipandi2024, JeanBaptiste2024}. Wie übersetzen individuelle Akteure technologische Affordanzen in organisationale Praktiken? Welche Mikroprozesse vermitteln zwischen technologischen Möglichkeiten und organisationalen Outcomes? Diese Fragen sind zentral für ein differenziertes Verständnis digitaler Transformation jenseits technologiedeterministischer Narrative.

%\subsection{Praktische Relevanz und manageriale Implikationen}

\textbf{Gestaltung von HR-Praktiken.} Die Befunde legen spezifische HR-Interventionen nahe: (1) Gezielte Qualifizierungsmaßnahmen in Prompt Engineering und KI-Validierungskompetenzen; (2) Förderung technischer Selbstwirksamkeit durch strukturierte Lernpfade und Peer-Support; (3) Implementierung von High-Performance-Work-Practices, die Ressourcen bereitstellen und KI-Angst reduzieren \parencite{Suseno2021, Chang2024}. Organisationen, die diese Maßnahmen systematisch implementieren, können motivationale Risiken mitigieren und Change Readiness erhöhen.

\textbf{Change Management und partizipative Implementierung.} Die Integration von Produktivitätsmetriken mit Wohlbefindensindikatoren (Stress, affektives Wohlbefinden, Identitätsspannung) in Rollout-Evaluationen ist zentral für nachhaltige Implementierungen \parencite{Benbya2024, Matthews2024}. Ko-Design und partizipative Ansätze reduzieren Widerstand und bewahren Motivation, indem sie Mitarbeitern Kontrolle und Mitgestaltungsmöglichkeiten bieten. Transparente Kommunikation über Ziele, Grenzen und erwartete Veränderungen reduziert Unsicherheit und ermöglicht proaktive Bewältigungsstrategien.

\textbf{Nachhaltige Implementierungsstrategien.} Um organisationale Ziele mit individuellen Erfahrungen in Einklang zu bringen, sollten Organisationen in drei Bereiche investieren: (1) Organisationale Unterstützungsstrukturen, die Ressourcen, Schulung und psychologische Sicherheit bereitstellen; (2) Resilienzförderung durch Coaching, Peer-Netzwerke und Reflexionsräume; (3) Rollenredesign für mittlere Manager, das den Übergang zu Coaching- und Oversight-Funktionen aktiv unterstützt \parencite{Brynjolfsson2023, Duraipandi2024}.

%\subsection{Die Lücke zwischen organisationalen Zielen und individuellen Erfahrungen}

\textbf{Empirische Forschungslücke.} Während hochwertige Feldevidenz zu Produktivitätseffekten vorliegt (z.B. \textcite{Brynjolfsson2023}), ist die Evidenz zu langfristigen Wohlbefindens- und Identitätseffekten begrenzt und oft querschnittlich angelegt. Längsschnitt- und Mixed-Methods-Studien, die mittlere Manager über Implementierungsphasen hinweg begleiten, sind besonders notwendig \parencite{Dong2024, Jin2024}. Solche Studien könnten dynamische Anpassungsprozesse, Wendepunkte und kritische Erfolgsfaktoren identifizieren.

\textbf{Praktische Implementierungslücke.} Organisationale Adoptionsstrategien, die kurzfristige Effizienzgewinne ohne parallele Investitionen in motivationale Ressourcen und Unterstützungsstrukturen priorisieren, riskieren eine Verschärfung von Technostress, Widerstand und Fluktuation – und konterkarieren damit die Retentionsziele, die KI-Einführungen oft verfolgen \parencite{Chang2024, Suseno2021}. Die systematische Integration motivationaler Perspektiven in KI-Strategien ist somit nicht nur ethisch geboten, sondern auch ökonomisch rational.

%\subsection{Forschungsagenda und erwarteter Erkenntnisbeitrag}

Die vorliegende Arbeit adressiert diese Lücken durch eine systematische, theoretisch fundierte Untersuchung der motivationalen Erlebnisse mittlerer Führungskräfte im Umgang mit generativer KI. Durch die Anwendung der Self-Determination Theory als analytischem Rahmen wird untersucht, wie die drei psychologischen Grundbedürfnisse (Autonomie, Kompetenz, soziale Eingebundenheit) durch KI-Nutzung beeinflusst werden und welche Faktoren Bedürfnisbefriedigung fördern oder behindern.

Der erwartete Erkenntnisbeitrag ist dreifach: (1) \textit{Theoretisch} wird die \gls{SDT} auf den emergenten Kontext generativer KI angewendet und erweitert, was zur Theorieentwicklung in Motivationspsychologie und Technologieakzeptanzforschung beiträgt; (2) \textit{Empirisch} werden differenzierte Einblicke in die subjektiven Erlebnisse einer bislang unterrepräsentierten, aber strategisch zentralen Nutzergruppe gewonnen; (3) \textit{Praktisch} werden evidenzbasierte Gestaltungsempfehlungen für motivationsförderliche KI-Implementierungen entwickelt, die Organisationen bei der nachhaltigen digitalen Transformation unterstützen.

Die Integration rigoroser Produktivitätsevaluation mit systematischer Messung motivationaler Zustände, Technostress-Indikatoren und Identitätsoutcomes wird sowohl theoretische Fortschritte (Motivations- und Technologieakzeptanzmodelle) als auch praktische Roadmaps für menschenzentrierte, nachhaltige GenAI-Adoption produzieren \parencite{Benbya2024, Brynjolfsson2023}.
\section{Zielsetzung und Forschungsfrage}

Die rasante Diffusion generativer KI-Systeme in organisationale Arbeitsprozesse hat eine Vielzahl wissenschaftlicher Untersuchungen angestoßen. Während substanzielle Erkenntnisse zu technologischen Fähigkeiten, Produktivitätseffekten und Implementierungsstrategien vorliegen, zeigt eine systematische Betrachtung des Forschungsstands erhebliche Lücken in der Erfassung subjektiver Erlebensdimensionen – insbesondere motivationaler Prozesse bei Führungskräften.

%\subsection{Bisherige Forschung zu generativer KI in Organisationen}

Die aktuelle Forschungslandschaft zu generativer KI in organisationalen Kontexten lässt sich in mehrere Schwerpunktbereiche gliedern:

\textbf{Produktivitäts- und Leistungsforschung.} Ein substanzieller Forschungsstrang dokumentiert Effizienzgewinne und Leistungssteigerungen durch generative KI-Nutzung. Wie bereits dargelegt, zeigen Feldstudien signifikante Produktivitätszuwächse in verschiedenen Tätigkeitsbereichen \parencite{Brynjolfsson2023, Noy2023, DellAcqua2023}. Diese Studien liefern wertvolle Evidenz für die operationale Wirksamkeit generativer KI, fokussieren jedoch primär auf objektive Leistungsindikatoren und vernachlässigen systematisch die subjektive Qualität der Arbeitserfahrung.

\textbf{Technologieakzeptanz und Adoptionsforschung.} Studien zur Akzeptanz algorithmischer Systeme und KI-Tools wenden typischerweise klassische Modelle wie das Technology Acceptance Model (TAM) oder die Unified Theory of Acceptance and Use of Technology (UTAUT) an. Diese Ansätze identifizieren kognitive Bewertungsdimensionen – insbesondere wahrgenommene Nützlichkeit und Benutzerfreundlichkeit – als zentrale Prädiktoren der Nutzungsintention. Allerdings greifen diese Modelle zu kurz, wenn es um die Erfassung tieferliegender motivationaler Dynamiken und die Befriedigung psychologischer Grundbedürfnisse geht \parencite{Bankins2024}.

\textbf{Algorithmisches Management und Kontrolle.} Ein wachsender Forschungsbereich untersucht algorithmische Managementsysteme, die Arbeitsprozesse steuern, überwachen und evaluieren \parencite{Kadolkar2025, Parent-Rocheleau2024}. Diese Forschung dokumentiert ambivalente Effekte: Während algorithmische Systeme Effizienz steigern können, erzeugen sie gleichzeitig Wahrnehmungen von Kontrolle, reduzierter Autonomie und Dehumanisierung \parencite{Lagios2022}. Die Befunde zu algorithmischem Management sind jedoch nicht unmittelbar auf generative KI übertragbar, da letztere typischerweise als kognitives Unterstützungswerkzeug und nicht als Kontrollsystem implementiert wird.

%\subsection{Forschung zu Motivation und Technologie im Arbeitskontext}

Die Schnittstelle von Technologie und Motivation im Arbeitskontext ist zunehmend Gegenstand wissenschaftlicher Untersuchungen:

\textbf{Self-Determination Theory und digitale Systeme.} Die Self-Determination Theory \parencite{Deci2000, Deci2017} wurde in verschiedenen technologischen Kontexten angewendet, um zu verstehen, wie digitale Systeme psychologische Grundbedürfnisse beeinflussen. Studien zeigen, dass Technologien sowohl bedürfnisunterstützend als auch bedürfnishemmend wirken können, abhängig von ihrer Gestaltung und organisationalen Einbettung \parencite{VanDenBroeck2016, Gagne2022}. Allerdings fehlt es an systematischer Anwendung der \gls{SDT} auf den spezifischen Kontext generativer KI, die sich durch ihre generative, kreative und dialogische Natur von früheren digitalen Systemen unterscheidet.

\textbf{Wahrnehmung algorithmischer Systeme.} Forschung im Bereich Organizational Behavior dokumentiert, dass algorithmische Systeme entweder als unterstützende Ressource (Feedback- und Lernhilfe) oder als kontrollierender Eingriff (managerial control) interpretiert werden können, was mit unterschiedlichen motivationalen und affektiven Konsequenzen verbunden ist \parencite{Bankins2024, Edwards2024}. Wenn ein algorithmisches System primär als unterstützendes Feedbackinstrument interpretiert wird, berichten Mitarbeitende höhere intrinsische Motivation, geringere Erschöpfung und ein stärkeres Erleben eigener Wirksamkeit. Demgegenüber führen Kontrollzuschreibungen zu erhöhter extrinsischer Motivation, wahrgenommenem Druck und motivationalen Kosten \parencite{Edwards2024}. Diese Befunde sind konsistent mit der Self-Determination Theory, wonach Unterstützungserfahrungen zur Befriedigung psychologischer Grundbedürfnisse beitragen, während kontrollierende Kontexte Need Thwarting begünstigen.

\textbf{KI und intrinsische Motivation.} Neuere experimentelle Studien zeigen paradoxe Effekte: Während Mensch-KI-Kollaboration die Aufgabenleistung verbessern kann, besteht gleichzeitig das Risiko, dass sie die intrinsische Motivation untergraben kann \parencite{Wu2025}. Dieser Befund deutet auf die Notwendigkeit differenzierter Analysen hin, die nicht nur Leistungsoutcomes, sondern auch motivationale Prozesse systematisch erfassen.

%\subsection{Forschung zu Führungskräften und digitaler Transformation}

Die Rolle von Führungskräften – insbesondere des mittleren Managements – in digitalen Transformationsprozessen ist Gegenstand einer eigenständigen Forschungslinie:

\textbf{Mittleres Management als strategischer Akteur.} Klassische Arbeiten etablieren mittlere Führungskräfte als zentrale Vermittler zwischen strategischen Intentionen und operativer Umsetzung \parencite{Floyd1997}. In digitalen Transformationskontexten nehmen diese Führungskräfte eine Doppelrolle ein: Sie sind sowohl Nutzer neuer Technologien als auch Implementierungsverantwortliche, die ihre Teams durch Veränderungsprozesse führen müssen.

\textbf{Digitale Führung und KI.} Emergente Forschung zu digitaler Führung thematisiert, wie künstliche Intelligenz Führungsrollen und -praktiken verändert \parencite{Quaquebeke2023}. Allerdings fokussiert diese Forschung primär auf strukturelle und funktionale Veränderungen, während die subjektive Erlebensperspektive der Führungskräfte selbst unterbelichtet bleibt.

\textbf{Arbeitsgestaltung und Führungseffektivität.} Forschung zu Arbeitsgestaltung dokumentiert, dass die Art und Weise, wie Arbeit strukturiert ist, signifikante Auswirkungen auf Leistung und Wohlbefinden hat \parencite{Knight2021}. Work-Design-Interventionen, die psychologische Grundbedürfnisse unterstützen, führen zu verbesserten Outcomes. Diese Erkenntnisse sind potenziell relevant für die Gestaltung KI-augmentierter Führungsarbeit, wurden jedoch bislang nicht systematisch auf generative KI-Kontexte übertragen.

%\subsection{Identifizierte Forschungslücken}

Aus der systematischen Betrachtung des Forschungsstands ergeben sich mehrere substanzielle Lücken:

\textbf{Lücke 1: Vernachlässigung motivationaler Erlebensdimensionen.} Während bestehende Forschung vorwiegend Effizienz- und Leistungsaspekte generativer KI hervorhebt, ist bislang unzureichend verstanden, wie sich ihr Einsatz auf das subjektive Erleben von Wirksamkeit in der Führungsarbeit auswirkt. Insbesondere fehlt es an theoretisch fundierten Untersuchungen, die psychologische Grundbedürfnisse – wie das Kompetenzerleben – systematisch in den Blick nehmen.

\textbf{Lücke 2: Fehlende Integration von \gls{SDT} und generativer KI.} Obwohl die Self-Determination Theory ein etabliertes Framework für die Analyse motivationaler Prozesse im Arbeitskontext darstellt \parencite{VanDenBroeck2016, Gagne2022, McAnally2024}, wurde sie bislang nicht systematisch auf den Kontext generativer KI angewendet. Die spezifischen Affordanzen generativer KI – ihre Fähigkeit zur Ko-Kreation, ihr dialogischer Charakter und ihre Rolle als kognitiver Partner – erfordern eine theoretische Auseinandersetzung, die über bisherige Technologiestudien hinausgeht.

\textbf{Lücke 3: Unzureichende Berücksichtigung vermittelnder Mechanismen.} Die Wirkungen generativer KI in organisationalen Entscheidungsprozessen sind maßgeblich durch subjektive Wahrnehmungen vermittelt: KI wirkt nicht direkt, sondern über Sinnzuschreibungen der Nutzenden sowie darüber, welche Intentionen sie hinter dem Einsatz vermuten. Während dies konzeptionell anerkannt ist, fehlt es an empirischen Studien, die diese vermittelnden Mechanismen – insbesondere die wahrgenommene Unterstützungsqualität – explizit modellieren und testen.

\textbf{Lücke 4: Fokus auf mittleres Management.} Führungskräfte des mittleren Managements sind eine strategisch zentrale, jedoch empirisch unterrepräsentierte Gruppe in der KI-Forschung. Ihre spezifische Position – gekennzeichnet durch hohe Entscheidungskomplexität, Verantwortung für Teamentwicklung und Vermittlung zwischen strategischer und operativer Ebene – macht sie zu einer besonders relevanten Zielgruppe für die Untersuchung motivationaler Effekte generativer KI.

\textbf{Lücke 5: Kontextspezifität und ökologische Validität.} Viele bestehende Studien zu generativer KI nutzen experimentelle Designs mit kurzfristigen Aufgaben. Es mangelt an Untersuchungen in realen organisationalen Kontexten, die die Komplexität und Dynamik tatsächlicher Arbeitsprozesse erfassen. Der Bankensektor – charakterisiert durch hohe Entscheidungskomplexität, Dokumentationsanforderungen und Rechenschaftspflichten – stellt einen besonders geeigneten, jedoch bislang unterforschten Kontext dar.

\section{Theoretischer Rahmen: Self-Determination Theory als Analyselinse}

Die vorliegende Arbeit nutzt die Self-Determination Theory (\gls{SDT}) als zentralen theoretischen Rahmen, um die motivationalen Erlebnisse von Führungskräften im Umgang mit generativer KI zu analysieren. Diese Theoriewahl ist durch mehrere Überlegungen begründet.

%\subsection{Begründung der Theoriewahl}

Die Self-Determination Theory \parencite{Deci2000, Deci2017} bietet einen geeigneten theoretischen Rahmen, um die Effekte generativer KI auf subjektive Arbeitserfahrungen zu analysieren, da sie das Erleben von Kompetenz als zentrales psychologisches Grundbedürfnis beschreibt, das maßgeblich Motivation, Leistungsfähigkeit und Wohlbefinden beeinflusst. Die \gls{SDT} zeichnet sich durch mehrere Charakteristika aus, die sie für die vorliegende Untersuchung besonders geeignet machen:

\textbf{Universalität und empirische Bewährung.} Die \gls{SDT} postuliert, dass alle Menschen drei fundamentale psychologische Bedürfnisse haben: Kompetenz, Autonomie und soziale Eingebundenheit. Die Befriedigung dieser Bedürfnisse fördert autonome Motivation und Wohlbefinden \parencite{Deci2000}. Diese Annahmen wurden in zahlreichen Kontexten – einschließlich Arbeits- und Organisationspsychologie – empirisch bestätigt \parencite{VanDenBroeck2016, Gagne2019, Laguerre2025}.

\textbf{Sensitivität für Kontextfaktoren.} Die \gls{SDT} unterscheidet systematisch zwischen bedürfnisunterstützenden und bedürfnishemmenden Kontexten. Digitale Systeme können beide Funktionen erfüllen: Sie können psychologische Grundbedürfnisse unterstützen (z.B. durch Feedback, Kompetenzentwicklung, Autonomieerweiterung) oder behindern (z.B. durch Kontrolle, Überwachung, Dehumanisierung) \parencite{Gagne2022}. Diese differentielle Perspektive ist zentral für das Verständnis, warum identische Technologien unterschiedliche motivationale Konsequenzen haben können.

\textbf{Anschlussfähigkeit an Arbeitsgestaltungsforschung.} Die \gls{SDT} ist eng mit Forschung zu Arbeitsgestaltung und Work Design verbunden. Studien zeigen, dass Arbeitsgestaltungsinterventionen, die psychologische Grundbedürfnisse unterstützen, zu verbesserten Leistungs- und Wohlbefindensoutcomes führen \parencite{Knight2021}. Diese Integration ermöglicht es, generative KI nicht nur als isoliertes Tool, sondern als Element eines sozio-technischen Arbeitssystems zu konzeptualisieren.

\textbf{Fokus auf subjektive Erlebensdimensionen.} Im Gegensatz zu rein verhaltensbezogenen oder leistungsorientierten Ansätzen fokussiert die \gls{SDT} explizit auf subjektive Erlebensdimensionen. Dies ist zentral für die vorliegende Arbeit, die untersucht, wie Führungskräfte die Nutzung generativer KI motivational erleben – unabhängig von objektiven Leistungseffekten.

%\subsection{Kernkonzepte der Self-Determination Theory}

\textbf{Die drei psychologischen Grundbedürfnisse.} Die \gls{SDT} postuliert drei angeborene, universelle psychologische Bedürfnisse \parencite{Deci2000, Deci2017}:

\begin{enumerate}
    \item \textbf{Autonomie}: Das Bedürfnis, Handlungen als selbstbestimmt und mit den eigenen Werten übereinstimmend zu erleben. Autonomie bedeutet nicht Unabhängigkeit, sondern die Wahrnehmung von Wahlfreiheit und Selbstkonkordanz.
    
    \item \textbf{Kompetenz}: Das Bedürfnis, sich als wirksam und fähig zu erleben, Herausforderungen erfolgreich zu bewältigen. Kompetenzerleben entsteht durch optimale Herausforderungen, konstruktives Feedback und die Erfahrung von Meisterschaft.
    
    \item \textbf{Soziale Eingebundenheit (Relatedness)}: Das Bedürfnis nach Zugehörigkeit, Verbundenheit und bedeutungsvollen sozialen Beziehungen.
\end{enumerate}

\textbf{Bedürfnisbefriedigung und -frustration.} Die \gls{SDT} unterscheidet zwischen Bedürfnisbefriedigung (Need Satisfaction) und Bedürfnisfrustration (Need Thwarting). Befriedigung fördert autonome Motivation, Engagement und Wohlbefinden, während Frustration zu kontrollierter Motivation, Amotivation und Beeinträchtigungen des Wohlbefindens führt \parencite{VanDenBroeck2016}.

\textbf{Autonome vs. kontrollierte Motivation.} Die \gls{SDT} unterscheidet verschiedene Formen der Motivation entlang eines Kontinuums der Selbstbestimmung. Autonome Motivation (intrinsische Motivation und identifizierte Regulation) ist mit positiven Outcomes assoziiert, während kontrollierte Motivation (externe und introjizierte Regulation) mit Druck, Anspannung und geringerem Wohlbefinden verbunden ist \parencite{Gagne2015}.

%\subsection{\gls{SDT} im Arbeitskontext: Empirische Evidenz}

Die Anwendung der \gls{SDT} im Arbeitskontext ist gut etabliert und empirisch fundiert:

\textbf{Meta-analytische Befunde.} Ein umfassendes Review dokumentiert, dass die Befriedigung der drei psychologischen Grundbedürfnisse am Arbeitsplatz positiv mit Arbeitszufriedenheit, Engagement, Leistung und Wohlbefinden assoziiert ist, während Bedürfnisfrustration negative Outcomes vorhersagt \parencite{VanDenBroeck2016}. Diese Befunde sind robust über verschiedene Kulturen, Berufsgruppen und Organisationstypen hinweg.

\textbf{Arbeitsgestaltung und Bedürfnisbefriedigung.} Studien zeigen, dass spezifische Arbeitsgestaltungsmerkmale – wie Aufgabenvielfalt, Autonomie, Feedback und soziale Unterstützung – die Befriedigung psychologischer Grundbedürfnisse fördern \parencite{VanDenBroeck2010, Gagne2019}. Diese Erkenntnisse sind relevant für die Gestaltung KI-augmentierter Arbeitsprozesse.

\textbf{HR-Praktiken und Motivation.} Forschung dokumentiert, dass HR-Praktiken, die Autonomie unterstützen und Kompetenzentwicklung fördern, zu höherer autonomer Motivation und besseren Outcomes führen \parencite{Laguerre2025}. Dies unterstreicht die Bedeutung organisationaler Rahmenbedingungen für motivationale Prozesse.

%\subsection{Kompetenzerleben als zentrales Konstrukt}

Für die vorliegende Arbeit steht das \textbf{Kompetenzerleben} im Zentrum der Analyse. Kompetenz wird definiert als die subjektive Wahrnehmung, wirksam und fähig zu sein, Herausforderungen in der eigenen Arbeitsumwelt erfolgreich zu bewältigen \parencite{Deci2000, VanDenBroeck2016}.

\textbf{Relevanz für Führungsarbeit.} Kompetenzerleben ist für Führungskräfte von besonderer Bedeutung, da ihre Arbeit durch hohe Komplexität, Ambiguität und Verantwortung charakterisiert ist. Das Erleben von Wirksamkeit in Entscheidungsprozessen, Problemlösung und Teamführung ist zentral für Führungsengagement und -effektivität \parencite{Zhang2024}.

\textbf{Quellen des Kompetenzerlebens.} Kompetenzerleben entsteht durch mehrere Faktoren: (1) Erfolgreiche Aufgabenbewältigung und Zielerreichung; (2) Konstruktives Feedback, das Fortschritt und Wirksamkeit signalisiert; (3) Optimale Herausforderungen, die weder Unter- noch Überforderung darstellen; (4) Verfügbarkeit von Ressourcen und Unterstützung, die Aufgabenbewältigung ermöglichen.

\textbf{Konsequenzen des Kompetenzerlebens.} Empirische Studien dokumentieren, dass Kompetenzerleben positiv mit Arbeitsengagement, Leistung, innovativem Verhalten, Arbeitszufriedenheit und psychischem Wohlbefinden assoziiert ist \parencite{VanDenBroeck2016, Gagne2022}. Umgekehrt führt Kompetenzfrustration – das Erleben von Ineffektivität und Versagen – zu Stress, Erschöpfung und reduzierter Leistung.

%\subsection{\gls{SDT} und technologische Systeme}

Die Anwendung der \gls{SDT} auf technologische Kontexte ist ein wachsendes Forschungsfeld:

\textbf{Technologie als bedürfnisrelevanter Faktor.} Digitale Systeme können psychologische Grundbedürfnisse sowohl unterstützen als auch behindern \parencite{Gagne2022}. Technologien, die Autonomie erweitern (z.B. durch Flexibilität, Wahlmöglichkeiten), Kompetenz fördern (z.B. durch Feedback, Lernunterstützung) und soziale Verbundenheit ermöglichen (z.B. durch Kommunikationstools), unterstützen Bedürfnisbefriedigung. Umgekehrt können Technologien, die Kontrolle ausüben, Überwachung implementieren oder Dehumanisierung fördern, Bedürfnisfrustration verursachen \parencite{Lagios2022}.

\textbf{Wahrnehmung als vermittelnder Mechanismus.} Entscheidend ist nicht die objektive Technologie, sondern ihre subjektive Wahrnehmung und Interpretation. Dieselbe Technologie kann je nach Implementierungskontext und individuellen Attributionen als unterstützend oder kontrollierend erlebt werden \parencite{Edwards2024, Bankins2024}. Diese Erkenntnis motiviert den Fokus der vorliegenden Arbeit auf wahrgenommene Unterstützungsqualität als vermittelnden Mechanismus.

%\subsection{Abgrenzung zu alternativen theoretischen Ansätzen}

Während mehrere theoretische Frameworks für die Untersuchung von KI im Arbeitskontext relevant sind, bietet die \gls{SDT} spezifische Vorteile:

\textbf{Im Vergleich zu TAM/UTAUT.} Das Technology Acceptance Model und verwandte Ansätze fokussieren primär auf kognitive Bewertungen (Nützlichkeit, Benutzerfreundlichkeit) und Nutzungsintention. Sie erfassen jedoch nicht systematisch motivationale Erlebensdimensionen und psychologische Grundbedürfnisse.

\textbf{Im Vergleich zu Job Characteristics Model.} Das Job Characteristics Model beschreibt, wie Arbeitsmerkmale psychologische Zustände und Outcomes beeinflussen. Es ist jedoch weniger explizit motivationstheoretisch fundiert als die \gls{SDT} und fokussiert stärker auf strukturelle Arbeitsmerkmale als auf subjektive Bedürfnisbefriedigung.

\textbf{Im Vergleich zu Job Demands-Resources Model.} Das JD-R-Modell fokussiert auf Stressoren und Ressourcen, ist jedoch primär auf Wohlbefinden und Burnout ausgerichtet. Die \gls{SDT} bietet eine breitere motivationstheoretische Perspektive, die sowohl positive (Engagement, autonome Motivation) als auch negative (Frustration, kontrollierte Motivation) Prozesse erfasst.

\section{Zielsetzung der Arbeit}

Die vorliegende Arbeit verfolgt mehrere miteinander verbundene Zielsetzungen, die sowohl wissenschaftliche als auch praktische Relevanz besitzen.

%\subsection{Übergeordnetes Forschungsziel}

Das übergeordnete Ziel dieser Arbeit besteht darin, einen Beitrag zur Verbindung von KI-Forschung, motivationspsychologischer Theorie und arbeitsgestaltungsbezogener Perspektive im Organizational Behavior zu leisten. Konkret wird untersucht, wie die Nutzung generativer KI-Tools in Entscheidungsvorbereitungsprozessen mit dem wahrgenommenen Kompetenzerleben von Führungskräften des mittleren Managements zusammenhängt.

%\subsection{Spezifische Forschungsziele}

\textbf{Ziel 1: Theoretische Integration.} Die Arbeit zielt darauf ab, die Self-Determination Theory systematisch auf den Kontext generativer KI anzuwenden und damit einen Beitrag zur theoretischen Weiterentwicklung sowohl der \gls{SDT} als auch der organisationalen KI-Forschung zu leisten. Durch die explizite Modellierung psychologischer Grundbedürfnisse – insbesondere Kompetenz – als zentrale Outcome-Variable wird eine motivationspsychologische Perspektive eingebracht, die in der bisherigen KI-Forschung unterrepräsentiert ist.

\textbf{Ziel 2: Identifikation vermittelnder Mechanismen.} Ein zentrales Ziel besteht darin, die vermittelnden Mechanismen zu untersuchen, durch die generative KI-Nutzung auf Kompetenzerleben wirkt. Als vermittelnder Mechanismus wird die wahrgenommene Unterstützungsqualität generativer KI konzeptualisiert. Aufbauend auf arbeitsgestaltungsbezogener Forschung \parencite{Knight2021} und Befunden zu algorithmischen Systemen \parencite{Edwards2024} wird argumentiert, dass generative KI das Kompetenzerleben insbesondere dann fördert, wenn sie als qualitativ hochwertige kognitive Unterstützung wahrgenommen wird.

\textbf{Ziel 3: Kontextspezifische Evidenz.} Die Arbeit zielt darauf ab, empirische Evidenz in einem spezifischen, bislang unterforschten Kontext zu generieren: Führungskräfte des mittleren Managements in österreichischen Genossenschaftsbanken. Dieser Kontext ist durch hohe Entscheidungskomplexität, Dokumentationsanforderungen und Rechenschaftspflichten charakterisiert und stellt damit ein besonders geeignetes Untersuchungsfeld für die Analyse kognitiver Unterstützungssysteme dar.

\textbf{Ziel 4: Praktische Implikationen.} Ein weiteres Ziel besteht darin, evidenzbasierte Erkenntnisse zu generieren, die für die Gestaltung motivationsförderlicher KI-Implementierungen in Organisationen relevant sind. Die Identifikation der Bedingungen, unter denen generative KI Kompetenzerleben fördert oder beeinträchtigt, kann Organisationen bei der menschenzentrierten Gestaltung digitaler Transformationsprozesse unterstützen.

\section{Forschungsfragen und Hypothesen}

Aus den dargelegten theoretischen Überlegungen und der identifizierten Forschungslücke leitet sich die zentrale Forschungsfrage dieser Arbeit ab.

%\subsection{Hauptforschungsfrage}

\begin{quote}
    \textit{Wie beeinflusst die Nutzung generativer KI-Tools das wahrgenommene Kompetenzerleben von Führungskräften des mittleren Managements in österreichischen Genossenschaftsbanken?}
\end{quote}

Diese Forschungsfrage adressiert die zentrale Forschungslücke hinsichtlich der motivationalen Erlebensdimensionen generativer KI-Nutzung und fokussiert auf eine strategisch relevante, jedoch empirisch unterrepräsentierte Zielgruppe.

%\subsection{Theoretisch abgeleitete Hypothesen}

Auf Grundlage der Self-Determination Theory und der arbeitsgestaltungsbezogenen Forschung werden zwei zentrale Hypothesen formuliert:

\textbf{Hypothese 1 (Haupteffekt):}
\begin{quote}
    \textit{Die Intensität der Nutzung generativer KI-Tools in Entscheidungsvorbereitungsprozessen ist positiv mit dem wahrgenommenen Kompetenzerleben von Führungskräften des mittleren Managements assoziiert.}
\end{quote}

Diese Hypothese postuliert einen positiven Zusammenhang zwischen KI-Nutzung und Kompetenzerleben. Die theoretische Begründung basiert auf der Annahme, dass generative KI als kognitives Werkzeug Führungskräfte bei komplexen Entscheidungsvorbereitungen unterstützt, indem sie Informationssynthese ermöglicht, Optionen generiert und analytische Prozesse beschleunigt. Diese Unterstützung kann das Erleben von Wirksamkeit und erfolgreicher Aufgabenbewältigung fördern.

\textbf{Hypothese 2 (Mediation):}
\begin{quote}
    \textit{Die wahrgenommene Unterstützungsqualität generativer KI-Tools mediiert den Zusammenhang zwischen der Intensität der KI-Nutzung und dem wahrgenommenen Kompetenzerleben.}
\end{quote}

Diese Hypothese postuliert, dass die Nutzung generativer KI nicht unmittelbar auf das Kompetenzerleben wirkt, sondern dass dieser Effekt indirekt über die wahrgenommene Unterstützungsqualität der KI vermittelt wird. Die theoretische Begründung basiert auf der Erkenntnis, dass technologische Systeme ihre Wirkung über subjektive Wahrnehmungen und Interpretationen entfalten \parencite{Edwards2024, Bankins2024}. Nur wenn generative KI als vertrauenswürdig, transparent und nützlich wahrgenommen wird – d.h. als qualitativ hochwertige Unterstützung – trägt sie zur Befriedigung des Kompetenzbedürfnisses bei.

%\subsection{Konzeptionelles Wirkmodell}

Das entwickelte Wirkmodell integriert die beiden Hypothesen in ein kohärentes theoretisches Framework:

\begin{itemize}
    \item \textbf{Unabhängige Variable:} Intensität der Nutzung generativer KI-Tools in Entscheidungsvorbereitungsprozessen
    \item \textbf{Mediatorvariable:} Wahrgenommene Unterstützungsqualität der generativen KI (charakterisiert durch Vertrauen, Transparenz und Nützlichkeit)
    \item \textbf{Abhängige Variable:} Wahrgenommenes Kompetenzerleben
    \item \textbf{Kontrollvariablen:} KI-Erfahrung, Führungserfahrung, Hierarchieebene
\end{itemize}

Dieses Modell postuliert einen indirekten Effekt: Die Nutzung generativer KI beeinflusst das Kompetenzerleben indirekt über die wahrgenommene Unterstützungsqualität. Diese Mediationshypothese ist theoretisch fundiert in der \gls{SDT}-Annahme, dass Kontextfaktoren (hier: Technologiewahrnehmung) die Befriedigung psychologischer Grundbedürfnisse beeinflussen.

\section{Methodisches Vorgehen (Kurzübersicht)}

Die empirische Untersuchung der formulierten Forschungsfragen und Hypothesen erfolgt mittels eines quantitativen, hypothesentestenden Forschungsdesigns.

%\subsection{Forschungsdesign und Stichprobe}

\textbf{Design:} Die Studie nutzt ein querschnittliches Survey-Design mit standardisierten Erhebungsinstrumenten. Dieses Design ermöglicht die systematische Erfassung der Konstrukte und die statistische Testung der postulierten Zusammenhänge.

\textbf{Stichprobe:} Die Zielgruppe umfasst Führungskräfte des mittleren Managements in österreichischen Genossenschaftsbanken. Die angestrebte Stichprobengröße beträgt N > 200 (mit einer realisierten Stichprobe von N = 119 für den österreichischen Kontext). Die Auswahl dieser Zielgruppe ist durch mehrere Überlegungen begründet: (1) Mittlere Führungskräfte nehmen eine zentrale Position in organisationalen Entscheidungsprozessen ein; (2) Der Bankensektor ist durch hohe Entscheidungskomplexität, Dokumentationsanforderungen und Rechenschaftspflichten charakterisiert, was ihn zu einem besonders geeigneten Kontext für die Untersuchung kognitiver Unterstützungssysteme macht; (3) Genossenschaftsbanken befinden sich in einem aktiven Prozess der digitalen Transformation, was die Relevanz und Aktualität der Untersuchung sicherstellt.

%\subsection{Erhebungsinstrumente}

Die Operationalisierung der Konstrukte erfolgt durch validierte und neu entwickelte Skalen:

\textbf{Generative KI-Nutzung:} Die Intensität der Nutzung generativer KI-Tools in Entscheidungsvorbereitungsprozessen wird mittels speziell entwickelter Items erfasst, die verschiedene Anwendungsbereiche (Informationssuche, Analyse, Dokumentation, Ideengenerierung) abdecken.

\textbf{Wahrgenommene Unterstützungsqualität:} Dieses Konstrukt wird operationalisiert durch wahrnehmungsbasierte Skalen, die auf arbeitsgestaltungsbezogener und unterstützungsbezogener Literatur basieren \parencite{Knight2021, Edwards2024}. Die Skala erfasst die Dimensionen Vertrauen, Transparenz und Nützlichkeit der KI-Unterstützung.

\textbf{Kompetenzerleben:} Das Kompetenzerleben wird mittels der validierten Work-related Basic Need Satisfaction Scale erfasst \parencite{VanDenBroeck2010}. Diese Skala misst das Erleben von Wirksamkeit und erfolgreicher Aufgabenbewältigung im Arbeitskontext.

\textbf{Kontrollvariablen:} Um alternative Erklärungen zu kontrollieren, werden KI-Erfahrung, Führungserfahrung und Hierarchieebene als Kontrollvariablen erfasst.

%\subsection{Datenerhebung und -analyse}

\textbf{Datenerhebung:} Die Datenerhebung erfolgt mittels Online-Survey. Die Teilnahme ist freiwillig und anonym. Ethische Standards der Organisationsforschung werden eingehalten.

\textbf{Datenanalyse:} Die Datenanalyse umfasst mehrere Schritte: (1) Deskriptive Statistiken und Reliabilitätsanalysen zur Überprüfung der Messinstrumente; (2) Korrelationsanalysen zur Untersuchung bivariater Zusammenhänge; (3) Regressionsbasierte Pfadanalysen zur Testung der Hypothesen unter Kontrolle relevanter Drittvariablen; (4) Mediationsanalyse zur Testung der indirekten Effekte mittels etablierter Verfahren (z.B. PROCESS Macro für SPSS/R).

\section{Aufbau der Arbeit}

Die vorliegende Arbeit ist in sechs Hauptkapitel gegliedert, die systematisch von der Problemstellung über theoretische Fundierung und empirische Untersuchung zur Diskussion und Implikation führen.

\textbf{Kapitel 1 – Einleitung:} Das Einleitungskapitel führt in die Problemstellung ein, identifiziert die Forschungslücke, stellt den theoretischen Rahmen vor, formuliert Forschungsfragen und Hypothesen und gibt einen Überblick über das methodische Vorgehen sowie den Aufbau der Arbeit.

\textbf{Kapitel 2 – Theoretischer Rahmen:} Das Theoriekapitel entwickelt systematisch die konzeptionellen Grundlagen der Arbeit. Es gliedert sich in mehrere Abschnitte: (1) Organisationsentwicklung im Kontext digitaler Transformation; (2) Generative KI in organisationalen Entscheidungsprozessen; (3) Self-Determination Theory im Arbeitskontext; (4) Kompetenzerleben im Arbeits- und Führungskontext; (5) Wahrgenommene Unterstützungsqualität als vermittelnder Mechanismus; (6) Kontextbedingungen und theoretische Grenzziehung; (7) Konzeptionelles Wirkmodell und Hypothesenableitung.

\textbf{Kapitel 3 – Methodik:} Das Methodenkapitel beschreibt das empirische Vorgehen im Detail. Es umfasst: (1) Stichprobe und Untersuchungsfeld; (2) Erhebungsinstrumente und Operationalisierung der Konstrukte; (3) Datenerhebungsprozess; (4) Datenanalyseverfahren und statistische Methoden.

\textbf{Kapitel 4 – Ergebnisse:} Das Ergebniskapitel präsentiert die empirischen Befunde. Es umfasst: (1) Deskriptive Statistiken und Stichprobenbeschreibung; (2) Reliabilitäts- und Validitätsanalysen; (3) Testung der Hypothesen (Haupteffekt und Mediation); (4) Zusatzanalysen und Robustheitsprüfungen.

\textbf{Kapitel 5 – Diskussion:} Das Diskussionskapitel interpretiert die empirischen Befunde vor dem Hintergrund der theoretischen Grundlagen und des Forschungsstands. Es umfasst: (1) Zusammenfassung und Interpretation der Hauptbefunde; (2) Theoretische Implikationen für \gls{SDT} und organisationale KI-Forschung; (3) Praktische Implikationen für Organisationen und HR-Management; (4) Limitationen der Studie; (5) Ansatzpunkte für zukünftige Forschung.

\textbf{Kapitel 6 – Fazit:} Das Schlusskapitel fasst die zentralen Erkenntnisse zusammen, reflektiert den Beitrag der Arbeit zur wissenschaftlichen Diskussion und gibt einen Ausblick auf zukünftige Entwicklungen.

\section{Abgrenzung und Limitation des Untersuchungsgegenstands}

Um den Untersuchungsgegenstand zu präzisieren und die Reichweite der Befunde angemessen einzuschätzen, sind mehrere Abgrenzungen und Limitationen zu benennen.

%\subsection{Inhaltliche Abgrenzungen}

\textbf{Fokus auf Kompetenzerleben.} Während die Self-Determination Theory drei psychologische Grundbedürfnisse postuliert (Autonomie, Kompetenz, soziale Eingebundenheit), fokussiert die vorliegende Arbeit spezifisch auf Kompetenzerleben. Diese Fokussierung ist theoretisch begründet: Kompetenz ist das Grundbedürfnis, das am unmittelbarsten mit der Unterstützungsfunktion generativer KI in kognitiven Arbeitsprozessen verbunden ist. Autonomie und soziale Eingebundenheit sind zweifellos ebenfalls relevant, werden jedoch in dieser Arbeit nicht systematisch untersucht.

\textbf{Kontext der Unterstützung, nicht Kontrolle.} Die Studie fokussiert auf Kontexte, in denen generative KI primär als unterstützendes Element für Entscheidungsvorbereitung verstanden wird, wobei die Entscheidungshoheit bei der Führungskraft verbleibt. Dies unterscheidet die Untersuchung explizit von Forschung zu \textit{algorithmischem Management}, die sich auf automatisierte Kontrolle von Arbeitsprozessen konzentriert \parencite{Kadolkar2025, Parent-Rocheleau2024}. Diese Abgrenzung ist wichtig, da Unterstützungs- und Kontrollkontexte unterschiedliche motivationale Dynamiken erzeugen.

\textbf{Mittleres Management als Zielgruppe.} Die Studie konzentriert sich auf Führungskräfte des mittleren Managements. Befunde sind nicht unmittelbar auf Top-Management oder operative Mitarbeiter ohne Führungsverantwortung übertragbar, da diese Gruppen unterschiedliche Aufgabenprofile, Verantwortlichkeiten und Technologienutzungsmuster aufweisen.

\textbf{Branchenspezifität.} Die empirische Untersuchung erfolgt im Bankensektor. Während dieser Sektor aufgrund seiner Charakteristika (hohe Entscheidungskomplexität, Regulierung, Dokumentationsanforderungen) besonders geeignet ist, sind Generalisierungen auf andere Branchen mit Vorsicht zu interpretieren.

%\subsection{Methodische Limitationen}

\textbf{Querschnittsdesign.} Die Studie nutzt ein querschnittliches Design, das keine kausalen Aussagen im strengen Sinne erlaubt. Während die Hypothesen theoretisch begründete Kausalrichtungen postulieren, können diese nicht definitiv durch das Design bestätigt werden. Längsschnittstudien wären erforderlich, um dynamische Prozesse und kausale Mechanismen eindeutiger zu identifizieren.

\textbf{Selbstberichtsdaten.} Die Erhebung basiert auf Selbstberichten, was mit bekannten Limitationen verbunden ist: soziale Erwünschtheit, Common Method Bias und subjektive Verzerrungen. Während diese Limitation durch methodische Vorkehrungen (z.B. Anonymität, validierte Skalen, statistische Kontrollen) abgemildert wird, bleibt sie eine prinzipielle Einschränkung.

\textbf{Stichprobenspezifität.} Die Stichprobe umfasst Führungskräfte in österreichischen Genossenschaftsbanken. Während dies einen klar definierten und relevanten Untersuchungskontext darstellt, sind Generalisierungen auf andere Länder, Organisationsformen oder Führungsebenen eingeschränkt.

%\subsection{Nicht berücksichtigte Moderatoren}

Das entwickelte Modell berücksichtigt nicht explizit mehrere potenziell relevante Moderatoren:

\textbf{Individuelle Differenzen.} Faktoren wie technologische Ängstlichkeit, Lernorientierung, Resilienz oder Persönlichkeitsmerkmale könnten die Zusammenhänge moderieren, werden jedoch in dieser Arbeit nicht systematisch untersucht.

\textbf{Führungsstile und organisationale Kultur.} Die Art und Weise, wie Führung praktiziert wird, und die vorherrschende Organisationskultur könnten beeinflussen, wie generative KI wahrgenommen und genutzt wird. Diese Faktoren werden als Kontextbedingungen anerkannt, aber nicht explizit modelliert.

\textbf{Implementierungscharakteristika.} Merkmale der KI-Implementierung (z.B. Partizipation, Training, Change-Management-Qualität) könnten relevante Moderatoren darstellen, werden jedoch in dieser Arbeit nicht erfasst.

%\subsection{Bewusste Fokussierung als Forschungsstrategie}

Diese Abgrenzungen und Limitationen sind nicht als Defizite, sondern als bewusste Fokussierung zu verstehen. Jede empirische Studie muss Komplexität reduzieren, um spezifische Forschungsfragen präzise zu adressieren. Die vorliegende Arbeit leistet einen fokussierten Beitrag zur Verbindung von \gls{SDT} und generativer KI-Forschung. Zukünftige Forschung kann auf diesen Erkenntnissen aufbauen und die hier nicht berücksichtigten Aspekte systematisch untersuchen.