% \usepackage{tikz}
% \usetikzlibrary{positioning, arrows.meta}

\begin{figure}[ht]
\centering
\begin{tikzpicture}[
  node distance=18mm and 18mm,
  box/.style={draw, rounded corners, align=center, inner sep=6pt, minimum width=55mm, minimum height=20mm},
  ctrl/.style={draw, dashed, rounded corners, align=left, inner sep=6pt, minimum width=55mm},
  arrow/.style={-Latex, line width=0.6pt}
]

\node[box] (use) {Nutzung generativer\\KI-Tools\\(Entscheidungsvorbereitung)};

\node[box, right=of use] (comp) {Kompetenzerleben\\(\gls{SDT})};

\node[box, below=of use] (qual) {Wahrgenommene\\Unterstützungsqualität\\der KI};

\node[ctrl, below=of qual] (controls) {\textbf{Kontrollen:}\\
KI-Erfahrung, Führungserfahrung,\\
Hierarchieniveau};

\draw[arrow] (use) -- node[above, pos=0.5] {H1 (+)} (comp);
\draw[arrow] (use) -- node[left, pos=0.5] {H2a (+)} (qual);
\draw[arrow] (qual) -- node[right, pos=0.5] {H2b (+)} (comp);


\end{tikzpicture}
\caption{Wirkmodell: KI-Nutzung, Unterstützungsqualität und Kompetenzerleben (\gls{SDT}).}
\label{fig:wirkmodell-schlank}
\end{figure}