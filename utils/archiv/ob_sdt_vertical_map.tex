% Preamble (falls noch nicht vorhanden)
% \usepackage{tikz}
% \usetikzlibrary{positioning, arrows.meta, fit, shapes.geometric}

% ------------------------------------------------------------
% FIGURE 1: Vertikale Theorie-Landkarte (OB -> \gls{SDT} -> Kompetenz)
% ------------------------------------------------------------
\begin{figure}[ht]
\centering
\begin{tikzpicture}[
  font=\small,
  node distance=8mm and 14mm,
  box/.style={draw, rounded corners, align=center, inner sep=6pt, minimum width=60mm},
  sub/.style={draw, rounded corners, align=left, inner sep=6pt, minimum width=70mm},
  note/.style={draw, dashed, rounded corners, align=left, inner sep=6pt, minimum width=70mm},
  arrow/.style={-Latex, line width=0.6pt}
]

% Main vertical spine
\node[box] (ob) {Organizational Behavior (OB)\\\emph{Forschungsfeld: Verhalten in Organisationen}};
\node[box, below=of ob] (mot) {Motivation im OB\\\emph{Erklärungen: Warum handeln Menschen engagiert?}};
\node[box, below=of mot] (bridges) {Brückenebene\\Work/Job Design \& Empowerment\\\emph{Arbeitsmerkmale $\rightarrow$ Erleben}};
\node[box, below=of bridges] (sdt) {Self-Determination Theory (\gls{SDT})\\\emph{Bedürfnisse $\rightarrow$ Motivation \& Wohlbefinden}};
\node[box, below=of sdt] (comp) {Kompetenzerleben (AV)\\\emph{Wirksamkeit \& Fähigkeitsgefühl bei Arbeit/Entscheidung}};

\draw[arrow] (ob) -- (mot);
\draw[arrow] (mot) -- (bridges);
\draw[arrow] (bridges) -- (sdt);
\draw[arrow] (sdt) -- (comp);

% Side theories near Motivation
\node[sub, right=of mot, xshift=10mm] (altmot) {\textbf{Alternative Motivationstheorien (relevant, aber nicht Kern)}\\
-- Expectancy / Valence-Ansätze\\
-- Goal Setting\\
-- Agency / Control-Logiken\\
-- JD-R (Ressourcen vs. Belastung)};

\draw[arrow] (altmot.west) -- ++(-8mm,0) |- (mot.east);

% Bridges details
\node[sub, right=of bridges, xshift=10mm] (bridge_detail) {\textbf{Brücken in Richtung \gls{SDT}}\\
-- Job Characteristics / Work Design\\
-- Feedback, Autonomie, Lerngelegenheiten\\
-- Psychological Empowerment (Dimension Kompetenz)};

\draw[arrow] (bridge_detail.west) -- ++(-8mm,0) |- (bridges.east);

% \gls{SDT} details
\node[sub, right=of sdt, xshift=10mm] (sdt_detail) {\textbf{\gls{SDT}-Mechanik (Mikro)}\\
-- Autonomie\\
-- Kompetenz\\
-- soziale Eingebundenheit\\
\emph{Bedürfnisbefriedigung vs. Need thwarting}};

\draw[arrow] (sdt_detail.west) -- ++(-8mm,0) |- (sdt.east);

% \gls{AI} as overlay context (connects to bridges/sdt)
\node[note, left=of bridges, xshift=-10mm] (ai_ctx) {\textbf{Technologiepfad (OB neu)}\\
Algorithmische Systeme / KI verändern\\
Arbeitsgestaltung und Entscheidungsarbeit.\\
Wirkungen sind \emph{wahrnehmungsabhängig}:\\
\emph{Support vs. Control}.};

\draw[arrow] (ai_ctx.east) -- ++(8mm,0) |- (bridges.west);
\draw[arrow] (ai_ctx.east) -- ++(8mm,0) |- (sdt.west);

\end{tikzpicture}
\caption{Vertikale Theorie-Landkarte: Vom Feld Organizational Behavior zur \gls{SDT} und dem Kompetenzerleben.}
\label{fig:ob_sdt_vertical_map}
\end{figure}


% ------------------------------------------------------------
% FIGURE 2: "Spielwiesen" von KI im OB + Eingrenzung auf generative, chatbasierte Augmentierung
% ------------------------------------------------------------
\begin{figure}[ht]
\centering
\begin{tikzpicture}[
  font=\small,
  node distance=8mm and 10mm,
  domainbox/.style={draw, rounded corners, align=left, inner sep=6pt, minimum width=70mm},
  hi/.style={draw, rounded corners, very thick, align=left, inner sep=6pt, minimum width=70mm},
  micro/.style={draw, rounded corners, align=left, inner sep=6pt, minimum width=70mm},
  arrow/.style={-Latex, line width=0.6pt}
]

% \gls{AI} playground domains (left column)
\node[domainbox] (am) {\textbf{Algorithmic Management}\\
Monitoring, Scheduling, Evaluation\\
\emph{typisch: Steuerung/ Kontrolle}};
\node[domainbox, below=of am] (hr) {\textbf{Algorithmic HR / People Analytics}\\
Scoring, Screening, Feedback-Systeme\\
\emph{Risiko: Kontrollattribution}};
\node[domainbox, below=of hr] (teams) {\textbf{Human--\gls{AI} Teams}\\
Koordination, Vertrauen, Rollen\\
\emph{Signaleffekte, Reliance}};
\node[domainbox, below=of teams] (decision) {\textbf{\gls{AI} in Entscheidungen}\\
Empfehlungen, Prognosen, Analytik\\
\emph{Accountability, Transparency}};

% Focus area (right column): Generative \gls{AI} as augmented chat-based support
\node[hi, right=of teams, xshift=18mm] (genai) {\textbf{Fokus dieser Arbeit: Generative KI}\\
\textbf{chatbasiert \& augmentierend}\\
-- kognitive Unterstützung\\
-- Strukturierung / Synthese\\
-- Entscheidungsvorbereitung\\
\emph{Mechanismus: wahrgenommene Unterstützungsqualität}};
\node[micro, below=of genai] (boundary) {\textbf{Explizite Abgrenzung}\\
Nicht im Fokus: verpflichtende\\
algorithmische Steuerung (AM),\\
automatisierte Leistungs-/Kontrollregime.};

% Connections among \gls{AI} playground
\draw[arrow] (am) -- (hr);
\draw[arrow] (hr) -- (teams);
\draw[arrow] (teams) -- (decision);

% Arrow to focus (why focus)
\draw[arrow] (teams.east) -- ++(8mm,0) |- (genai.west);
\draw[arrow] (decision.east) -- ++(8mm,0) |- (genai.west);

% Boundary link
\draw[arrow] (genai) -- (boundary);

% Add a small mechanism box
\node[micro, right=of decision, xshift=18mm, yshift=-12mm] (mech) {\textbf{OB-Mechanismen im KI-Kontext}\\
-- Support vs. Control Attribution\\
-- Vertrauen / Disclosure-Reaktionen\\
-- Bedürfnisbefriedigung (\gls{SDT})};

\draw[arrow] (mech.west) -- ++(-8mm,0) |- (decision.east);
\draw[arrow] (mech.west) -- ++(-8mm,0) |- (genai.east);

\end{tikzpicture}
\caption{Spielwiesen von KI im OB-Kontext und Eingrenzung auf generative, chatbasierte Augmentierung.}
\label{fig:ai_ob_playground}
\end{figure}