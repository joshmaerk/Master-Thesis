\chapter{Theoretischer Rahmen}

Das vorliegende Theoriekapitel dient der konzeptionellen Fundierung der Arbeit und verfolgt das Ziel, die Wirkzusammenhänge zwischen der Nutzung generativer KI in organisationalen Entscheidungsprozessen und dem wahrgenommenen Kompetenzerleben von Führungskräften systematisch herzuleiten. Ausgangspunkt ist die Annahme, dass technologische Innovationen – insbesondere generative KI – ihre organisationalen und individuellen Effekte nicht unmittelbar entfalten, sondern über psychologische Mechanismen vermittelt werden, die im Kontext bestehender Arbeits- und Führungsstrukturen wirksam werden \parencite{bankinsMultilevelReviewArtificial2024}. Eine rein technologisch-funktionale Betrachtung greift daher zu kurz, da sie zentrale subjektive Wahrnehmungs- und Deutungsprozesse ausblendet, die für Motivation, Wohlbefinden und Leistungsfähigkeit von Beschäftigten entscheidend sind.

Zur theoretischen Einordnung wird generative KI zunächst als sozio-technisches Arbeitssystem verstanden, das Entscheidungsarbeit nicht ersetzt, sondern potenziell augmentiert und damit in bestehende Rollen- und Verantwortungslogiken von Führungskräften eingreift \parencite{quaquebeke_now_2023}. Darauf aufbauend wird argumentiert, dass die Wirkung solcher Systeme wesentlich davon abhängt, ob sie von den Nutzenden als unterstützende Ressource oder als kontrollierender Eingriff wahrgenommen werden \parencite{edwards_managerial_2024}. Diese wahrnehmungsabhängige Perspektive bildet die Grundlage für die Annahme, dass motivational relevante Effekte generativer KI nicht direkt, sondern über vermittelnde Mechanismen erklärbar sind.

Als zentraler theoretischer Bezugsrahmen dient die Self-Determination Theory (SDT), die erklärt, wie Arbeitsbedingungen und organisationale Gestaltungsformen über die Befriedigung grundlegender psychologischer Bedürfnisse auf Motivation und Wohlbefinden wirken \parencite{deci_what_2000,deci_self-determination_2017}. Im Fokus steht dabei das Kompetenzerleben als Ausdruck wahrgenommener Wirksamkeit und Fähigkeit im Umgang mit anspruchsvollen Aufgaben. Aufbauend auf SDT wird argumentiert, dass generative KI das Kompetenzerleben von Führungskräften insbesondere dann fördern kann, wenn sie als qualitativ hochwertige Unterstützung in der Entscheidungsarbeit erlebt wird. Entsprechend wird die wahrgenommene Unterstützungsqualität der KI als zentraler vermittelnder Mechanismus in das theoretische Modell integriert.

Das Theoriekapitel ist wie folgt aufgebaut: Zunächst wird generative KI im Kontext organisationaler Entscheidungsprozesse konzeptionell eingeordnet. Anschließend wird die Self-Determination Theory im Arbeitskontext dargestellt und das Kompetenzerleben als zentrale abhängige Variable begrifflich geschärft. Darauf aufbauend wird die wahrgenommene Unterstützungsqualität als vermittelnder Mechanismus theoretisch hergeleitet. Abschließend werden die zentralen Annahmen in einem konzeptionellen Wirkmodell zusammengeführt, aus dem die Hypothesen der empirischen Untersuchung abgeleitet werden.

\begin{figure}[ht]
\centering
\begin{tikzpicture}[
  node distance=10mm,
  every node/.style={draw, rounded corners, align=center, inner sep=6pt},
  macro/.style={minimum width=55mm},
  meso/.style={minimum width=55mm},
  micro/.style={minimum width=55mm},
  arrow/.style={-Latex, thick}
]

% --- Levels ---
\node[macro] (oe) {Organisationsentwicklung (OE)\\
\footnotesize Geplanter organisationaler Eingriff};

\node[macro, below=of oe] (st) {Sozio-technische Systemtheorie\\
\footnotesize Integration von Technik, Arbeit und Mensch};

\node[meso, below=of st] (ai) {Generative KI in Organisationen\\
\footnotesize Human--AI Interaction\\
Entscheidungsvorbereitung};

\node[meso, below=of ai] (wd) {Arbeitsgestaltung / Work Design\\
\footnotesize Wahrgenommene Unterstützungsqualität};

\node[micro, below=of wd] (sdt) {Self-Determination Theory (SDT)\\
\footnotesize Psychologische Grundbedürfnisse};

\node[micro, below=of sdt] (comp) {Kompetenzerleben\\
\footnotesize Wahrgenommene Wirksamkeit};

% --- Arrows ---
\draw[arrow] (oe) -- (st);
\draw[arrow] (st) -- (ai);
\draw[arrow] (ai) -- (wd);
\draw[arrow] (wd) -- (sdt);
\draw[arrow] (sdt) -- (comp);

\end{tikzpicture}
\caption{Theorie-Landkarte der Arbeit: Integration organisationsentwicklungs-, technologie- und motivationsbezogener Ansätze}
\label{fig:theorie-landkarte}
\end{figure}

Abbildung \ref{fig:theorie-landkarte} veranschaulicht den theoretischen Bezugsrahmen der Arbeit und zeigt, wie unterschiedliche theoretische Perspektiven zu einem integrierten Erklärungsmodell zusammengeführt werden. Ausgangspunkt bildet die Organisationsentwicklung, in deren Verständnis der Einsatz generativer KI als geplanter organisationsweiter Eingriff in bestehende Arbeits- und Entscheidungsstrukturen eingeordnet wird. Die sozio-technische Systemtheorie verbindet diesen organisationsbezogenen Rahmen mit der Betrachtung technischer Artefakte und menschlichen Erlebens und legitimiert damit die gleichzeitige Berücksichtigung technologischer und psychologischer Aspekte. Auf dieser Grundlage wird generative KI als Bestandteil organisationaler Entscheidungsarbeit verstanden, deren Wirkung nicht allein durch technische Eigenschaften, sondern durch wahrnehmungsabhängige Veränderungen der Arbeitsgestaltung bestimmt ist. Die wahrgenommene Unterstützungsqualität fungiert hierbei als zentraler vermittelnder Mechanismus, über den KI-Nutzung motivational wirksam wird. Die Self-Determination Theory bildet den theoretischen Kern der Arbeit und erklärt, wie unterstützende Arbeitsbedingungen zur Befriedigung psychologischer Grundbedürfnisse beitragen können. Das Kompetenzerleben stellt schließlich das zentrale Outcome dar und operationalisiert den motivationalen Effekt generativer KI im Arbeitskontext des mittleren Managements.

\section{Organisationsentwicklung im Kontext digitaler Transformation}
% Definition von Organisationsentwicklung als geplanter, systemischer Veränderungsprozess.
% Verbindung individueller, struktureller und kultureller Ebenen.
% Ziel: Legitimation des organisationsentwicklungsbezogenen Rahmens der Arbeit.

Organisationsentwicklung bezeichnet einen geplanten, systematischen und langfristig angelegten Prozess zur gezielten Veränderung von Organisationen, der strukturelle, kulturelle und individuelle Ebenen integriert adressiert \parencite{cummingsOrganizationDevelopmentChange2015,burkeOrganizationChangeTheory2018}. Im Zentrum organisationsentwicklungsbezogener Ansätze steht dabei nicht die isolierte Optimierung einzelner Prozesse, sondern die bewusste Gestaltung organisationaler Rahmenbedingungen (z.\,B. Strukturen, Routinen, Führungs- und Lernprozesse), um nachhaltige Anpassungs- und Entwicklungsfähigkeit zu ermöglichen \parencite{cummingsOrganizationDevelopmentChange2015}. Organisationsentwicklung folgt damit einer sozio-technischen Logik, wonach organisationale Strukturen und Arbeitsprozesse stets mit den Wahrnehmungen, Interaktionen und Deutungsmustern der Organisationsmitglieder verflochten sind; Veränderung entfaltet Wirkung folglich über die wechselseitige Kopplung von Systemgestaltung und menschlichem Erleben \parencite{burkeOrganizationChangeTheory2018,scheinOrganizationalCultureLeadership2017}. Vor diesem Hintergrund stellt Organisationsentwicklung einen geeigneten theoretischen Rahmen dar, um technologische, strukturelle und psychologische Dimensionen organisationaler Veränderung integriert zu analysieren und zu gestalten \parencite{cummingsOrganizationDevelopmentChange2015,burkeOrganizationChangeTheory2018}.


% Abgrenzung zwischen Technologieeinsatz und Organisationsentwicklung.
% Digitale Technologien und KI als Interventionen, die Arbeitsprozesse, Rollen und Entscheidungsstrukturen verändern.
% Ziel: Positionierung von KI als OE-relevantes Phänomen.
Digitale Transformation beschreibt einen tiefgreifenden, organisationsweiten Veränderungsprozess, der über die bloße Einführung neuer Technologien hinausgeht und etablierte Arbeitsprozesse, Rollenbilder, Entscheidungsstrukturen sowie kulturelle Deutungsmuster nachhaltig verändert \parencite{vialUnderstandingDigitalTransformation2019}. Aus organisationsentwicklungsbezogener Perspektive stellt der Einsatz digitaler Technologien – insbesondere datenbasierter und KI-gestützter Systeme – daher keinen rein technischen Implementierungsschritt dar, sondern einen gezielten Eingriff in das sozio-technische System der Organisation \parencite{cummingsOrganizationDevelopmentChange2015,burkeOrganizationChangeTheory2018}. Digitale Transformation beeinflusst, wie Arbeit gestaltet wird, wie Entscheidungen vorbereitet und getroffen werden und wie Verantwortung zwischen Mensch und Technologie verteilt ist \parencite{bankinsMultilevelReviewArtificial2024}. Diese Veränderungen entfalten ihre Wirkung nicht allein über technische Leistungsmerkmale, sondern maßgeblich über subjektive Wahrnehmungen, Interpretationen und Akzeptanzprozesse der betroffenen Akteure \parencite{quaquebeke_now_2023}. Vor diesem Hintergrund ist digitale Transformation als organisationsentwicklungsrelevanter Eingriff zu verstehen, dessen Erfolg wesentlich davon abhängt, inwieweit technologische Neuerungen mit psychologischen Bedürfnissen, bestehenden Arbeitslogiken und organisationalen Entwicklungszielen in Einklang gebracht werden.


% Mittleres Management als Übersetzungs-, Implementierungs- und Sinnstiftungsebene.
% Relevanz für Wahrnehmung und Nutzung neuer Technologien.
% Ziel: Begründung der Fokusgruppe der Untersuchung.

 Das mittlere Management nimmt in organisationsentwicklungsbezogenen Veränderungsprozessen eine zentrale Rolle ein, da es als verbindende Ebene zwischen strategischer Entscheidung und operativer Umsetzung fungiert \parencite{floydManagingStrategicConsensus1997}. Führungskräfte dieser Ebene sind maßgeblich daran beteiligt, organisationale Veränderungen zu interpretieren, in bestehende Arbeitskontexte zu übersetzen und gegenüber Mitarbeitenden zu legitimieren. Gerade im Kontext digitaler Transformation und des Einsatzes generativer KI kommt dem mittleren Management eine besondere Bedeutung zu, da technologische Veränderungen häufig unmittelbar in ihre Entscheidungs-, Koordinations- und Analyseprozesse eingreifen \parencite{bankinsMultilevelReviewArtificial2024}. Gleichzeitig prägen Führungskräfte des mittleren Managements durch ihr eigenes Nutzungsverhalten, ihre Haltung gegenüber neuen Technologien und ihre kommunikative Rahmung maßgeblich, wie digitale Systeme organisational wahrgenommen und akzeptiert werden \parencite{quaquebeke_now_2023}. Aus organisationsentwicklungsbezogener Perspektive stellt das mittlere Management somit eine Schlüsselgruppe dar, über die sich entscheidet, ob digitale Technologien – wie generative KI – als unterstützende Ressource oder als kontrollierender Eingriff in bestehende professionelle Handlungsspielräume erlebt werden.


\section{Generative KI in organisationalen Entscheidungsprozessen}
\subsection{Generative KI als sozio-technisches Arbeitssystem}
Generative KI stellt eine neue Klasse digitaler Arbeitssysteme dar, die sich in ihrer Funktionslogik und organisationalen Wirkung deutlich von klassischer Informations\-technologie und deterministischer Automatisierung unterscheidet. Während traditionelle IT-Systeme primär auf die regelbasierte Ausführung klar definierter Prozesse ausgerichtet sind, zeichnen sich generative KI-Systeme durch ihre Fähigkeit aus, auf Basis probabilistischer Modelle eigenständig Inhalte zu erzeugen, Informationen zu synthetisieren und kontextsensitiv zu interagieren \parencite{bankinsMultilevelReviewArtificial2024}. In der organisationswissenschaftlichen Diskussion wird generative KI daher nicht als isolierte technische Innovation verstanden, sondern als sozio-technisches Arbeitssystem, dessen Effekte sich erst im Zusammenspiel mit organisationalen Strukturen, Rollen und Arbeitspraktiken entfalten \parencite{bankinsMultilevelReviewArtificial2024}. Zentrale Übersichtsarbeiten betonen, dass KI in Organisationen insbesondere dann wirksam wird, wenn sie nicht substituierend eingesetzt wird, sondern menschliche kognitive Fähigkeiten ergänzt und erweitert (Augmentierung statt Substitution) \parencite{bankinsMultilevelReviewArtificial2024}. Damit ist generative KI im Spektrum algorithmischer Systeme als kognitiv-unterstützendes System einzuordnen und von Formen des algorithmischen Managements abzugrenzen, die Arbeit über algorithmische Zielvorgaben, Monitoring, Bewertung oder (teil-)automatisierte Steuerung koordinieren \parencite{kadolkar_algorithmic_2025,parent-rocheleau_creation_2024}. Entsprechend fokussiert die vorliegende Arbeit auf eine freiwillige, unterstützende Nutzung generativer KI in der Entscheidungsvorbereitung und ausdrücklich nicht auf eine direkte Steuerung oder Überwachung von Mitarbeitenden.

\subsection{KI-gestützte Entscheidungsvorbereitung im mittleren Management}

Entscheidungsarbeit im mittleren Management ist typischerweise wissensintensiv, von Unsicherheit geprägt und häufig zeitkritisch, da sie unter unvollständiger Informationslage zwischen strategischer Vorgabe und operativer Umsetzung vermittelt. In diesem Kontext kann generative KI als kognitive Ressource eingesetzt werden, indem sie Informationssuche und -verdichtung unterstützt, Entscheidungsalternativen strukturiert und Argumentationslinien bzw. Handlungsoptionen konsistent aufbereitet \parencite{bankinsMultilevelReviewArtificial2024}. Der Einsatz zielt dabei weniger auf den Wegfall von Führungs- und Entscheidungsrollen als auf deren Veränderung: Aufgaben verschieben sich von der reinen Informationsverarbeitung hin zu Einordnung, Plausibilisierung und verantwortlicher Abwägung.

Zugleich ist konzeptionell klarzustellen, dass sich der Fokus dieser Arbeit auf die Entscheidungsvorbereitung (z.\,h. die Generierung und Aufbereitung von Entscheidungsgrundlagen) und nicht auf die Entscheidungshoheit richtet. Auch bei KI-gestützter Vorbereitung verbleiben Verantwortung und letztliche Entscheidung beim Menschen, was die Anschlussfähigkeit an motivationalpsychologische Perspektiven wie die Self-Determination Theory (insb. Autonomie als erlebter Handlungsspielraum) unterstützt.

Im Managementkontext findet generative KI vor allem in der Vorbereitung von Entscheidungen Anwendung, etwa durch die Analyse umfangreicher Datenbestände, die Strukturierung von Entscheidungsalternativen oder die Aufbereitung komplexer Sachverhalte \parencite{bankinsMultilevelReviewArtificial2024}. Für Führungskräfte des mittleren Managements, die zwischen strategischer Vorgabe und operativer Umsetzung vermitteln, sind Entscheidungsprozesse häufig durch hohe Komplexität, Zeitdruck und Unsicherheit gekennzeichnet. Generative KI kann hier als kognitive Unterstützung fungieren, indem sie Informationsverarbeitung erleichtert und Entscheidungsgrundlagen transparenter macht. Zugleich verdeutlicht Forschung zu digitaler und KI-basierter Führung, dass der Einsatz solcher Systeme nicht wertneutral ist, da er Rollenverständnisse, Verantwortungszuschreibungen und das Erleben professioneller Handlungsspielräume beeinflusst \parencite{quaquebekeNowNewNext2023}. In diesem Sinne verändert generative KI nicht nur die Effizienz der Entscheidungsvorbereitung, sondern auch die inhaltliche Ausgestaltung von Führungs- und Managementarbeit, indem neue Formen der Mensch--KI-Kollaboration entstehen \parencite{smithNavigatingAIConvergence2025}.

\subsection{Wahrnehmungsabhängige Wirkungen von KI}

Die Wirkungen generativer KI in organisationalen Entscheidungsprozessen sind maßgeblich durch subjektive Wahrnehmungen vermittelt: KI wirkt nicht direkt, sondern über Sinnzuschreibungen der Nutzenden sowie darüber, welche Intentionen sie hinter dem Einsatz vermuten. Forschung im Bereich Organizational Behavior zeigt, dass algorithmische Systeme entweder als unterstützende Ressource (z.\,B. Feedback- und Lernhilfe) oder als kontrollierender Eingriff (z.\,B. managerial control) interpretiert werden können, was mit unterschiedlichen motivationalen und affektiven Konsequenzen verbunden ist \parencite{bankinsMultilevelReviewArtificial2024,edwards_managerial_2024}. Während unterstützend erlebte KI-Nutzung mit höherer Akzeptanz, Vertrauen und Lernbereitschaft einhergeht, können kontrollorientierte oder intransparente Einsatzformen Stress, Widerstand und wahrgenommenen Autonomieverlust begünstigen \parencite{klonekDoesAIWork2025}.

Ein zentraler Einflussfaktor dieser Wahrnehmung ist Transparenz bzw. Offenlegung: Ob und wie KI in Entscheidungsprozessen eingesetzt wird, beeinflusst Sensemaking-Prozesse und kann Leistungs- bzw. Einstellungswirkungen in entgegengesetzte Richtungen verschieben (z.\,B. Deployment- vs. Disclosure-Effekte) \parencite{tongJanusFaceArtificial2021}. Ergänzend weisen experimentelle Befunde im Kontext generativer KI darauf hin, dass Mensch--KI-Kollaboration zwar Leistungsgewinne ermöglichen kann, zugleich aber motivational ambivalent sein kann, wenn die Interaktion als fremdbestimmt oder entwertend erlebt wird \parencite{wu_human-generative_2025}.

Schließlich spielt Vertrauen eine zentrale Rolle dafür, ob generative KI als hilfreiche Unterstützung oder als riskante, potenziell kontrollierende Technologie interpretiert wird. Insbesondere im HR- und Managementkontext wird Vertrauen als vermittelnder Mechanismus diskutiert, über den sich die Einführung generativer KI auf Akzeptanz und nachgelagerte arbeitsbezogene Outcomes auswirken kann \parencite{prasadGenerativeAICatalyst2024}. Studien zum algorithmischen Management verdeutlichen zudem, dass KI-basierte Steuerungsmechanismen insbesondere dann als problematisch erlebt werden, wenn sie individuelle Expertise entwerten und Handlungsspielräume einschränken \parencite{kadolkar_algorithmic_2025}. Für den organisationalen Kontext generativer KI folgt daraus, dass deren Effekte nicht allein aus technischen Eigenschaften erklärbar sind, sondern wesentlich durch organisationale Gestaltung, Führung und Sinnzuschreibung bestimmt werden. Diese wahrnehmungsabhängige Wirklogik bereitet die Annahme vor, dass die Nutzung generativer KI in Entscheidungsprozessen über psychologische Mechanismen auf motivational relevante Konstrukte wirkt.

Weitere Forschung weist darauf hin, dass die Auswirkungen KI-gestützter Systeme in Organisationen nicht allein durch ihre funktionalen Eigenschaften erklärbar sind, sondern maßgeblich durch subjektive Wahrnehmungen der Nutzung vermittelt werden. Besonders deutlich wird diese Ambivalenz in der Arbeit von \textcite{tongJanusFaceArtificial2021}, die zwischen einem leistungssteigernden \emph{Deployment Effect} und einem potenziell motivationsmindernden \emph{Disclosure Effect} algorithmischer Systeme unterscheidet. Während der Einsatz KI-basierter Feedback- und Entscheidungssysteme objektiv die Qualität von Informationen und Handlungsempfehlungen erhöhen kann, zeigen die Autoren, dass die Offenlegung algorithmischer Einflussnahme bei Mitarbeitenden zugleich Kontrollwahrnehmungen, Misstrauen und Reaktanz auslösen kann \parencite{tongJanusFaceArtificial2021}.

Ergänzend verdeutlichen neuere Arbeiten, dass Vertrauen in KI-Systeme eine zentrale Voraussetzung für positive organisationale Wirkungen technologischer Unterstützung darstellt. Prasad und De (2024) zeigen im Kontext generativer KI im Human Resource Management, dass wahrgenommene Nützlichkeit und Einsatz generativer KI nur dann zu höherem Commitment und positiven arbeitsbezogenen Einstellungen führen, wenn Mitarbeitende den Systemen vertrauen \parencite{prasadGenerativeAICatalyst2024}. Vertrauen fungiert dabei als vermittelnder psychologischer Mechanismus zwischen technologischer Intervention und motivationalen Outcomes.

Aus organisationspsychologischer Perspektive impliziert dies, dass KI-gestützte Systeme ihre unterstützende Funktion nur dann entfalten können, wenn sie als verlässlich, transparent und wohlwollend wahrgenommen werden. Fehlt dieses Vertrauen, besteht die Gefahr, dass identische technologische Funktionen nicht als Ressource, sondern als potenzielle Bedrohung interpretiert werden. Für den Einsatz generativer KI in Entscheidungsprozessen von Führungskräften bedeutet dies, dass wahrgenommene Unterstützungsqualität nicht ausschließlich von der funktionalen Leistungsfähigkeit der Systeme abhängt, sondern maßgeblich durch Vertrauenszuschreibungen geprägt ist. Die Ergebnisse von Prasad und De (2024) stützen damit die Annahme, dass die Wirkung generativer KI auf motivational relevante Konstrukte indirekt und wahrnehmungsabhängig erfolgt.

Diese Befunde unterstreichen, dass KI in organisationalen Entscheidungsprozessen nicht als neutraler technologischer Input wirkt, sondern als sozio-technisches System, dessen Effekte von Zuschreibungen hinsichtlich Unterstützung versus Kontrolle abhängen. Für Führungskräfte bedeutet dies, dass identische KI-Funktionalitäten entweder als kognitive Entlastung und Kompetenzressource oder als Einschränkung eigener Handlungsspielräume interpretiert werden können. Damit liefern die Ergebnisse von \textcite{tongJanusFaceArtificial2021} eine zentrale theoretische Grundlage für die Annahme, dass wahrgenommene Unterstützungsqualität ein vermittelnder Mechanismus zwischen KI-Nutzung und motivationalen Outcomes ist.

\section{Self-Determination Theory im Arbeitskontext}
\subsection{Grundannahmen der Self-Determination Theory} 
Die Self-Determination Theory (SDT) ist eine umfassende Theorie der menschlichen Motivation, die Motivation als qualitativ differenziertes Konstrukt betrachtet \parencite{deciWhatWhyGoal2000}. Zentral ist die Unterscheidung zwischen autonomer Motivation (d.h. intrinsischer und vollständig internalisierter extrinsischer Motivation) und kontrollierter Motivation (durch äußeren Druck oder introjizierte Verpflichtungsgefühle geprägte Motivation), welche unterschiedliche Auswirkungen auf Leistung und Wohlbefinden haben \parencite{deci_self-determination_2017}. Ferner postuliert SDT, dass alle Menschen über drei grundlegende psychologische Bedürfnisse – Kompetenz, Autonomie und soziale Eingebundenheit – verfügen, deren Befriedigung die Entfaltung autonomer Motivation begünstigt und zugleich das Wohlbefinden fördert \parencite{deciWhatWhyGoal2000, van_den_broeck_review_2016}. SDT liefert somit einen theoretischen Bezugsrahmen, um Motivationsprozesse und deren Einfluss auf positive Outcomes zu erklären, und bildet die Grundlage für die vorliegende Arbeit \parencite{deci_self-determination_2017}.

\subsection{Psychologische Grundbedürfnisse bei der Arbeit}
Auch im Arbeitskontext spielen die drei Grundbedürfnisse der SDT eine zentrale Rolle. Autonomie (Erleben von Selbstbestimmung), Kompetenz (Gefühl von Wirksamkeit) und soziale Eingebundenheit (Gefühl von Zugehörigkeit) gelten als essentielle Voraussetzungen für Motivation und Wohlbefinden von Beschäftigten \parencite{vandenbroeckReviewSelfDeterminationTheorys2016}. Organisationen stellen dabei wichtige „Bedürfnisumwelten“ dar, da Arbeitsbedingungen und Führungsverhalten die Befriedigung dieser Bedürfnisse entweder unterstützen oder beeinträchtigen können \parencite{deci_self-determination_2017}. Empirisch zeigt sich, dass eine hohe Bedürfnisbefriedigung am Arbeitsplatz mit positiven Konsequenzen wie intrinsischer Arbeitsmotivation, Leistungsfähigkeit, Engagement und psychischer Gesundheit einhergeht, wohingegen eine Frustration der Bedürfnisse mit negativen Folgen wie Burnout und Unzufriedenheit verbunden ist \parencite{vandenbroeckReviewSelfDeterminationTheorys2016, deci_self-determination_2017}. Entsprechend wird SDT in der Organisations- und Personalpraxis verstärkt herangezogen, um Arbeitsumfelder und HR-Praktiken bedürfnisförderlich zu gestalten \parencite{laguerre_bringing_2025}.

\paragraph{Führung als Kontextbedingung der Bedürfnisbefriedigung}
Aktuelle Forschung im Bereich Organizational Behavior verdeutlicht, dass die Befriedigung psychologischer Grundbedürfnisse im Sinne der Self-Determination Theory nicht unabhängig vom sozialen und organisatorischen Kontext erfolgt, sondern maßgeblich durch Führungshandeln geprägt wird. Zhang et \parencite[]{zhang_how_2024} zeigen in einer mehrstufigen Studie, dass insbesondere ein humble Leadership-Stil – gekennzeichnet durch Lernorientierung, Offenheit für Feedback und die Anerkennung eigener Grenzen – die Befriedigung der Bedürfnisse nach Autonomie und Kompetenz fördert \parencite{zhang_how_2024}. Führungskräfte fungieren damit als zentrale Kontextakteure, die Arbeitsbedingungen so rahmen, dass Mitarbeitende ihre Fähigkeiten wirksam einsetzen und weiterentwickeln können. Aus SDT-Perspektive wird Kompetenz dabei nicht allein durch Aufgabenanforderungen oder individuelle Fähigkeiten bestimmt, sondern durch das erlebte Zusammenspiel von Herausforderung, Unterstützung und Rückmeldung. Die Ergebnisse von Zhang et al. verdeutlichen, dass Führungshandeln eine vermittelnde Rolle zwischen strukturellen Arbeitsbedingungen und motivationalen Outcomes einnimmt, indem es bestimmt, ob Arbeitsanforderungen als entwicklungsfördernd oder überfordernd wahrgenommen werden. Für den vorliegenden Forschungskontext impliziert dies, dass auch technologische Arbeitsmittel wie generative KI nicht isoliert wirken, sondern ihre motivationalen Effekte im Zusammenspiel mit Führung und organisationaler Sinngebung entfalten.

\subsection{SDT und technologische Arbeitsgestaltung} 
SDT ist dabei nicht als statisches Modell zu verstehen, sondern als Analyseinstrument, um neue Arbeitsformen und Technologien hinsichtlich ihrer motivationalen Qualität zu bewerten. Digitale Systeme (einschließlich KI) können so gestaltet und erlebt werden, dass sie psychologische Grundbedürfnisse unterstützen (need-supportive), etwa durch Autonomie- und Kompetenzunterstützung, oder sie können Bedürfnisse unterminieren (need-thwarting), z.B. durch Fremdsteuerung, Intransparenz oder Entwertung von Expertise \parencite{gagne_understanding_2022,mcanally_self-determination_2024}.

\subsection{Kompetenzerleben als zentraler Fokus der vorliegenden Arbeit}
Das Bedürfnis nach Kompetenz beschreibt das grundlegende Bestreben, sich fähig und wirksam in der Auseinandersetzung mit der Umwelt zu fühlen und neue Fertigkeiten zu entwickeln \parencite{vandenbroeckReviewSelfDeterminationTheorys2016}. In der vorliegenden Arbeit steht das Kompetenzerleben daher im Mittelpunkt, da es gerade in Führungsaufgaben, bei komplexen Entscheidungsprozessen und während organisationaler Veränderungsprozesse ein Schlüsselfaktor für erfolgreiche Anpassung und Leistungsbereitschaft ist. Mitarbeiter können neue Herausforderungen nur dann engagiert und konstruktiv bewältigen, wenn ihr Kompetenzbedürfnis erfüllt ist und sie Vertrauen in die eigene Wirksamkeit haben \parencite{deci_self-determination_2017}. Beispielsweise reagieren Beschäftigte in Veränderungsphasen nachweislich aufgeschlossener, wenn ihre Vorgesetzten einen unterstützenden Führungsstil pflegen, der unter anderem das Kompetenzgefühl der Mitarbeiter stärkt \parencite{deci_self-determination_2017}. Die Konzentration auf das Kompetenzerleben als abhängige Variable der Untersuchung ist somit theoretisch gut begründet, da diesem Bedürfnis im Rahmen der SDT eine zentrale Bedeutung für Motivation und Wohlbefinden zukommt \parencite{deciWhatWhyGoal2000}.

\subsection{Self-Determination Theory als Bezugsrahmen für digitale Arbeitskontexte}
Neuere konzeptionelle Arbeiten unterstreichen die besondere Eignung der Self-Determination Theory zur Analyse motivationaler Effekte in zunehmend digitalisierten Arbeitsumgebungen. McAnally und Hagger (2024) argumentieren, dass technologische Entwicklungen wie algorithmische Systeme und KI-basierte Arbeitsmittel nicht primär über Effizienz- oder Akzeptanzmodelle erklärt werden sollten, sondern über ihre Wirkung auf psychologische Grundbedürfnisse \parencite{mcanallySelfDeterminationTheoryWorkplace2024}. Insbesondere betonen die Autoren, dass digitale Technologien sowohl autonomie- und kompetenzförderlich als auch kontrollierend und bedürfnisfrustrierend wirken können, abhängig von ihrer Gestaltung und organisationalen Einbettung. 

Vor diesem Hintergrund ermöglicht SDT eine differenzierte Analyse technologischer Interventionen, da sie nicht von technologischem Determinismus ausgeht, sondern den Fokus auf subjektive Wahrnehmungen, Sinnzuschreibungen und motivational relevante Wirkmechanismen legt. McAnally und Hagger (2024) heben hervor, dass organisationale Praktiken und Technologien dann nachhaltige positive Effekte entfalten, wenn sie als unterstützend erlebt werden und die Selbstregulation der Beschäftigten stärken. Diese Perspektive ist für die vorliegende Arbeit zentral, da sie begründet, warum die Untersuchung generativer KI nicht auf Nutzungsintensität oder Akzeptanz beschränkt bleibt, sondern das Erleben von Kompetenz als psychologischen Wirkmechanismus in den Mittelpunkt stellt.

\section{Kompetenzerleben im Arbeits- und Führungskontext}
\subsection{Begriffsbestimmung und Abgrenzung des Kompetenzerlebens}
Das Kompetenzerleben bezeichnet die subjektive Wahrnehmung, wirksam und fähig zu sein, Herausforderungen in der eigenen Arbeitsumwelt erfolgreich zu bewältigen. In der Self-Determination Theory wird Kompetenz als grundlegendes psychologisches Bedürfnis verstanden, das sich auf das Erleben von Effektivität und Meisterschaft im Umgang mit Aufgaben und Anforderungen bezieht \parencite{van_den_broeck_capturing_2010}. Dieses Verständnis ist klar von verwandten Konstrukten abzugrenzen. Leistung beschreibt primär objektive oder fremdbeurteilte Ergebnisse, während das Kompetenzerleben eine subjektive Erfahrung darstellt. Selbstwert bezieht sich auf eine globale Bewertung der eigenen Person und ist nicht notwendigerweise auf den Arbeitskontext beschränkt. Selbstwirksamkeit fokussiert auf erwartete Fähigkeiten in Bezug auf spezifische zukünftige Handlungen, während Kompetenzerleben stärker das gegenwärtige affektive Gefühl von Wirksamkeit nach der Ausführung einer Tätigkeit erfasst. Auch gegenüber dem Konzept des Psychological Empowerment ist eine Abgrenzung erforderlich: Während Kompetenz dort eine von mehreren Dimensionen darstellt, wird sie im Rahmen der Self-Determination Theory als eigenständiges motivationales Grundbedürfnis konzeptualisiert \parencite{spreitzerPSYCHOLOGICALEMPOWERMENTWORKPLACE1995}.

\subsection{Kompetenzerleben, Führung und Entscheidungsarbeit}
Im Arbeits- und Führungskontext ist das Kompetenzerleben eng mit Entscheidungsarbeit und wahrgenommener Verantwortung verbunden. Führungskräfte erleben Kompetenz insbesondere dann, wenn sie Entscheidungsprozesse aktiv gestalten, Handlungsspielräume nutzen und Rückmeldungen über die Wirksamkeit ihres Handelns erhalten. Forschung im Bereich Empowering Leadership zeigt, dass die Delegation von Verantwortung, der Zugang zu relevanten Informationen und die Unterstützung eigenständiger Problemlösung das Kompetenzerleben von Mitarbeitenden und Führungskräften gleichermaßen fördern \parencite{spreitzerPSYCHOLOGICALEMPOWERMENTWORKPLACE1995}. Für das mittlere Management ist dieser Zusammenhang besonders relevant, da diese Führungsebene häufig zwischen strategischen Vorgaben und operativer Umsetzung vermittelt und Entscheidungen unter hoher Unsicherheit trifft. Ein hohes Kompetenzerleben kann hier dazu beitragen, Entscheidungsanforderungen als bewältigbar zu erleben und Veränderungsprozesse aktiv mitzugestalten. Empirische Studien zeigen zudem, dass wahrgenommene Kompetenz mit höherer intrinsischer Motivation, Lernbereitschaft und konstruktivem Umgang mit komplexen Entscheidungssituationen verbunden ist \parencite{gagne_multidimensional_2015}.

\subsection{Messbarkeit und Relevanz des Kompetenzerlebens}
Das Kompetenzerleben ist empirisch gut operationalisierbar und stellt damit ein anschlussfähiges Konstrukt für organisationsentwicklungsbezogene Forschung dar. Einen zentralen Messansatz bietet die Work-related Basic Need Satisfaction Scale, die Kompetenz als eigenständige Dimension der psychologischen Grundbedürfnisbefriedigung im Arbeitskontext erfasst und umfangreich validiert wurde \parencite{van_den_broeck_capturing_2010}. Ergänzend existieren etablierte Instrumente aus angrenzenden Forschungssträngen, etwa die Kompetenzdimension des Psychological Empowerment oder motivationsbezogene Skalen wie die Multidimensional Work Motivation Scale, die indirekt mit dem Erleben von Kompetenz verknüpft sind \parencite{gagne_multidimensional_2015,spreitzerPSYCHOLOGICALEMPOWERMENTWORKPLACE1995}. Die Relevanz des Kompetenzerlebens ergibt sich nicht nur aus seiner theoretischen Verankerung in der Self-Determination Theory, sondern auch aus seiner empirisch belegten Bedeutung für Leistung, Wohlbefinden und Veränderungsbereitschaft. Damit eignet sich das Kompetenzerleben als zentrale abhängige Variable, um die motivationalen Effekte organisationaler Gestaltungsmaßnahmen und technologischer Interventionen, wie etwa den Einsatz generativer KI in Entscheidungsprozessen, systematisch zu untersuchen.

\section{Wahrgenommene Unterstützungsqualität als vermittelnder Mechanismus}
\subsection{Arbeitsgestaltung}
Arbeitsgestaltung wirkt nicht unmittelbar über objektive Merkmale, sondern entfaltet ihre Effekte maßgeblich über subjektive Wahrnehmungen der Beschäftigten. Zentrale arbeitsbezogene Ressourcen wie Autonomie, Feedback oder Unterstützung beeinflussen Motivation und Leistung insofern, als sie von Individuen als hilfreich für die Bewältigung ihrer Arbeitsanforderungen interpretiert werden; Unterstützung kann dabei als zentrale Ressource der Arbeitsgestaltung verstanden werden \parencite{knight_how_2021}. In der arbeits- und organisationspsychologischen Forschung wird Unterstützung dabei als zentrale Ressource verstanden, die es Individuen ermöglicht, Anforderungen effektiv zu bewältigen, Lernprozesse zu realisieren und Unsicherheiten zu reduzieren. Insbesondere in wissensintensiven Arbeitskontexten ist wahrgenommene Unterstützung eng mit der Nutzung von Expertise, der Qualität von Problemlösungen und dem Erleben von Wirksamkeit verbunden \parencite{hong_explaining_2018}. Damit bildet wahrgenommene Unterstützungsqualität einen zentralen vermittelnden Mechanismus zwischen Arbeitsgestaltung und motivationalen Outcomes.

\subsection{Unterstützungsqualität digitaler Systeme}
Digitale Systeme können im Arbeitskontext als spezifische Form arbeitsbezogener Unterstützung fungieren, insbesondere wenn sie kognitive Entlastung, Strukturierung von Informationen oder Qualitätsverbesserungen ermöglichen. Forschung zur Arbeitsgestaltung zeigt, dass solche unterstützenden Systeme dann wirksam werden, wenn sie als Ressource wahrgenommen werden, die Arbeitsprozesse erleichtert und nicht zusätzliche Komplexität erzeugt \parencite{knight_how_2021}. Generative KI kann in diesem Sinne als kognitive Unterstützung in Wissens- und Entscheidungsarbeit verstanden werden, indem sie Informationen zusammenführt, Entscheidungsalternativen strukturiert und Reflexionsprozesse anregt. Empirische Arbeiten zur motivierenden Arbeitsgestaltung zeigen, dass wahrgenommene Unterstützung durch Arbeitsmittel und -prozesse mit autonomer Motivation und konstruktivem Wissensverhalten verbunden ist \parencite{gagne_different_2019}. Entscheidend ist dabei weniger die technische Leistungsfähigkeit des Systems als vielmehr die subjektive Einschätzung, ob das System zur effektiven Bewältigung der Arbeitsaufgabe beiträgt.
Eine zentrale Differenzierung hinsichtlich der Wirkungen algorithmischer Systeme liefern Edwards et al. (2024), die zeigen, dass identische digitale Systeme von Beschäftigten entweder als unterstützende Feedbackinstrumente oder als kontrollierende Steuerungsmechanismen wahrgenommen werden können. In ihrer Studie zu algorithmisch unterstützten HR-Systemen argumentieren die Autoren, dass diese Wahrnehmungsattribution entscheidend darüber ist, ob technologische Systeme autonome oder kontrollierte Motivation auslösen \parencite{edwardsManagerialControlFeedback2024}. 

Wird ein algorithmisches System primär als unterstützendes Feedbackinstrument interpretiert, berichten Mitarbeitende höhere intrinsische Motivation, geringere Erschöpfung und ein stärkeres Erleben eigener Wirksamkeit. Demgegenüber führen Kontrollzuschreibungen zu erhöhter extrinsischer Motivation, wahrgenommenem Druck und motivationalen Kosten. Diese Befunde sind konsistent mit der Self-Determination Theory, wonach Unterstützungserfahrungen zur Befriedigung psychologischer Grundbedürfnisse beitragen, während kontrollierende Kontexte Need Thwarting begünstigen. Für generative KI in Entscheidungsprozessen impliziert dies, dass ihre motivationalen Effekte weniger von der technischen Ausgestaltung als von der subjektiven Interpretation ihrer Rolle abhängen. Edwards et al. \parencite[]{edwards_managerial_2024} liefern damit eine zentrale theoretische Grundlage für die Annahme, dass wahrgenommene Unterstützungsqualität als vermittelnder Mechanismus zwischen KI-Nutzung und Kompetenzerleben wirkt.

\paragraph{Abgrenzung zu algorithmischem Management}
In der aktuellen Forschung wird der Einsatz algorithmischer Systeme in Organisationen zunehmend unter dem Begriff des algorithmischen Managements diskutiert. Parent-Rocheleau et al. (2024) entwickeln mit dem Algorithmic Management Questionnaire ein differenziertes Instrument zur Erfassung algorithmischer Steuerungsmechanismen, etwa in den Bereichen Leistungsüberwachung, Zielvorgabe, Einsatzplanung und Leistungsbewertung \parencite{parent-rocheleau_creation_2024}. Die Autoren zeigen, dass eine hohe Exponierung gegenüber algorithmischem Management mit geringerer wahrgenommener Autonomie, reduzierter Arbeitskomplexität und niedrigerem Work Engagement einhergeht. 

Diese Perspektive ist für die vorliegende Arbeit insofern relevant, als sie eine konzeptionelle Abgrenzung erlaubt. Während algorithmisches Management auf die automatisierte Steuerung und Kontrolle von Arbeitsprozessen abzielt, fokussiert die vorliegende Untersuchung auf die freiwillige Nutzung generativer KI als kognitive Unterstützung in der Entscheidungsvorbereitung von Führungskräften. Der Einsatz generativer KI wird hier nicht als Fremdsteuerungsinstrument verstanden, sondern als potenzielle Ressource, deren Wirkung maßgeblich von der subjektiven Wahrnehmung ihrer Unterstützungsqualität abhängt. Durch diese Abgrenzung wird verdeutlicht, dass negative motivationalen Effekte algorithmischer Systeme, wie sie im Kontext algorithmischen Managements beschrieben werden, nicht ohne Weiteres auf unterstützende KI-Anwendungen in Entscheidungsprozessen übertragbar sind.

\subsection{Unterstützungsqualität, Vertrauen und Kompetenzerleben im Sinne der \gls{SDT}}

Die Unterstützungsqualität generativer KI ist nicht als objektives Merkmal der Technologie zu verstehen, sondern als subjektive Zuschreibung der Nutzenden. Unterstützung bezeichnet in diesem Zusammenhang die wahrgenommene Ermöglichung wirksamer Aufgabenbearbeitung, etwa durch verbesserte Informationsgrundlagen, kognitive Entlastung, Strukturierung komplexer Sachverhalte oder die Anregung von Reflexionsprozessen. Damit ist das Konstrukt der Unterstützungsqualität konzeptionell anschlussfähig an arbeits- und motivationspsychologische Ansätze, die den Fokus auf wahrnehmungsbasierte Wirkmechanismen legen.

Aus theoretischer Perspektive ist zwischen der technischen Leistungsfähigkeit eines Systems und der subjektiven Bewertung seiner Unterstützung klar zu unterscheiden. Ob generative KI als unterstützend erlebt wird, hängt nicht allein von ihren funktionalen Eigenschaften ab, sondern davon, inwieweit Nutzende ihr einen Beitrag zur eigenen Wirksamkeit zuschreiben. Vertrauen in die Zuverlässigkeit und Angemessenheit der KI stellt dabei eine notwendige Voraussetzung für die Zuschreibung von Unterstützungsqualität dar, wird jedoch im vorliegenden Modell nicht als eigenständige Variable geführt. Stattdessen ist Vertrauen implizit in der Wahrnehmung von Unterstützung enthalten und wird theoretisch integriert, ohne die Modellkomplexität zu erhöhen.

Aus Sicht der Self-Determination Theory entfalten unterstützende Arbeitsbedingungen insbesondere dann positive motivationale Effekte, wenn sie zur Befriedigung grundlegender psychologischer Bedürfnisse beitragen. Wahrgenommene Unterstützung signalisiert Individuen, dass ihnen ausreichende Ressourcen zur Verfügung stehen, um Anforderungen erfolgreich zu bewältigen, und stärkt dadurch das Erleben von Kompetenz als Gefühl von Wirksamkeit und Fähigkeit. Demgegenüber zeigen empirische Befunde, dass mangelnde, inkonsistente oder entwertende Unterstützung mit einer Frustration psychologischer Bedürfnisse (Need Thwarting) verbunden ist und negative motivationale sowie affektive Konsequenzen nach sich ziehen kann \parencite{lagiosExplainingNegativeConsequences2022}. Ergänzend verdeutlicht ressourcenorientierte Forschung, dass Unterstützung als arbeitsbezogene Ressource dazu beiträgt, Anforderungen als bewältigbar zu erleben und adaptive Handlungsfähigkeit auch unter Belastung aufrechtzuerhalten \parencite{karanika-murray_health-performance_2020}.

Vor diesem theoretischen Hintergrund ist anzunehmen, dass die Nutzung generativer KI in der Entscheidungsarbeit nicht unmittelbar auf das Kompetenzerleben von Führungskräften wirkt. Vielmehr entfaltet sich dieser Effekt indirekt über die wahrgenommene Unterstützungsqualität der KI, die als vermittelnder Mechanismus zwischen technologischem Einsatz und motivationalem Outcome fungiert. Diese Argumentation bildet die theoretische Grundlage der Mediationshypothese (H2) der vorliegenden Arbeit.


\section{Kontextbedingungen und theoretische Grenzziehung}
%\section{Kontextbedingungen und theoretische Grenzziehung}

Zur theoretischen Absicherung des entwickelten Wirkmodells ist es erforderlich, zentrale Kontextbedingungen und potenzielle Einflussfaktoren zu reflektieren, die die Wirkung generativer KI auf motivational relevante Konstrukte beeinflussen können, ohne selbst Bestandteil der empirischen Modellierung zu sein. Ziel dieses Abschnitts ist es, die Gültigkeit und Reichweite des Modells einzugrenzen und zugleich aufzuzeigen, unter welchen Bedingungen die postulierten Zusammenhänge erwartungsgemäß variieren könnten.

Ein erster relevanter Kontextfaktor betrifft die Frage der Autonomiegefährdung durch verpflichtende oder stark normierte Formen der KI-Nutzung. Aus Perspektive der Self-Determination Theory ist Autonomie ein zentrales psychologisches Grundbedürfnis, dessen Frustration negative motivationale Konsequenzen nach sich zieht. Wird der Einsatz generativer KI nicht als freiwillige Unterstützung, sondern als verpflichtende Vorgabe oder implizite Erwartung wahrgenommen, kann dies das Erleben von Selbstbestimmung einschränken und potenziell auch positive Effekte auf das Kompetenzerleben relativieren. Das vorliegende Modell fokussiert daher bewusst auf Kontexte, in denen die Nutzung generativer KI primär als unterstützendes Element der Entscheidungsvorbereitung verstanden wird und die Entscheidungshoheit weiterhin bei der Führungskraft verbleibt.

Ein weiterer Einflussfaktor ist die individuelle Wahrnehmung technologischer Bedrohung, häufig operationalisiert über Konzepte wie Technologieangst oder STARA-Awareness (Smart Technology, Artificial Intelligence, Robotics and Algorithms). Forschung zeigt, dass eine hohe Sensibilität gegenüber der potenziellen Ersetzbarkeit durch Technologie mit negativen Einstellungen, reduziertem Wohlbefinden und geringerer organisationaler Bindung einhergehen kann. In solchen Fällen besteht die Möglichkeit, dass generative KI nicht als Ressource, sondern als Bedrohung der eigenen beruflichen Kompetenz interpretiert wird. Diese individuelle Disposition kann somit die Wahrnehmung der Unterstützungsqualität beeinflussen und stellt eine relevante theoretische Wirkbedingung dar, die jedoch im Rahmen der vorliegenden Arbeit nicht explizit modelliert wird.

Schließlich sind auch Führungsstil und organisationaler sowie kultureller Kontext als bedeutsame Rahmenbedingungen zu berücksichtigen. Empirische Studien zeigen, dass unterstützende, lernorientierte und demütige Führungsstile die Befriedigung psychologischer Grundbedürfnisse fördern und adaptive Leistungsformen begünstigen. In einem solchen Kontext kann der Einsatz generativer KI eher als Gelegenheit zur Kompetenzentwicklung und Reflexion wahrgenommen werden. Demgegenüber können stark kontrollorientierte Führungs- und Organisationskulturen dazu führen, dass KI primär als Überwachungs- oder Steuerungsinstrument interpretiert wird, was die motivationalen Effekte des KI-Einsatzes abschwächen oder umkehren kann.

Zusammenfassend werden Autonomiebedingungen, individuelle technologische Dispositionen sowie Führungs- und Kontextfaktoren als theoretisch relevante Wirkbedingungen anerkannt, ohne sie in das empirische Modell aufzunehmen. Diese bewusste Grenzziehung dient der Fokussierung und Modellökonomie und ermöglicht zugleich eine differenzierte Interpretation der Ergebnisse im Lichte potenzieller Kontextabhängigkeiten.

\section{Konzeptionelles Wirkmodell und Hypothesen}
\begin{itemize}
  \item Finales Wirkmodell
  \item Klare Hypothesenlogik:
  \item H1: KI-Nutzung $\rightarrow$ Kompetenzerleben
  \item H2: Mediation über wahrgenommene Unterstützungsqualität
\end{itemize}