\section{Organizational Behavior im Kontext digitaler Transformation}

\subsection{Digitale Transformation als Treiber organisationaler Veränderung}

Digitale Transformation bezeichnet einen umfassenden, technologiegetriebenen Wandel, der Organisationsstrukturen, Geschäftsprozesse und Arbeitspraktiken grundlegend verändert \parencite{vialUnderstandingDigitalTransformation2019}. Im Unterschied zu früheren Automatisierungswellen, die primär manuelle und repetitive Tätigkeiten betrafen, adressiert die aktuelle Transformationswelle zunehmend wissensintensive, kognitiv anspruchsvolle Aufgaben und verändert damit die Natur professioneller Arbeit selbst \parencite{brynjolfssonGenerativeAIWork2025}.

%%\subsubsection{Konzeptionelle Grundlagen.} 
\textcite{vialUnderstandingDigitalTransformation2019} definiert digitale Transformation als Prozess, durch den digitale Technologien tiefgreifende Veränderungen in Organisationen auslösen, die sich auf Wertschöpfungsprozesse, Organisationsstrukturen und strategische Positionierung erstrecken. Diese Definition betont die Mehrdimensionalität des Phänomens: Digitale Transformation ist nicht bloß technologische Adoption, sondern ein organisationaler Wandelprozess, der strukturelle, kulturelle und individuelle Ebenen integriert \parencite{burkeOrganizationChangeTheory2018, cummingsOrganizationDevelopmentChange2015}.

%%\subsubsection{Phasen und Charakteristika.} 
Empirische Untersuchungen identifizieren typische Phasen technologiegetriebenen Wandels: Initiierung und Sensemaking, Implementierung und Anpassung, sowie Routinisierung und kontinuierliche Optimierung \parencite{vialUnderstandingDigitalTransformation2019}. Charakteristisch für digitale Transformationsprozesse ist ihre Nicht-Linearität: Organisationen durchlaufen iterative Zyklen von Experimentation, Lernen und Anpassung, wobei technologische Möglichkeiten und organisationale Praktiken ko-evolutionär aufeinander einwirken.

%%\subsubsection{Auswirkungen auf Organisationsstrukturen und Arbeitsprozesse.} 
Digitale Technologien ermöglichen neue Formen der Arbeitsorganisation, einschließlich dezentraler Entscheidungsstrukturen, flexibler Teamkonfigurationen und hybrider Arbeitsmodelle. Gleichzeitig können sie jedoch auch zu verstärkter Überwachung, algorithmischer Kontrolle und Verdichtung von Arbeitsprozessen führen \parencite{edwardsManagerialControlFeedback2024}. Neuere Forschung zeigt, dass digitale Transformation eine doppelschneidige Wirkung entfaltet: Sie kann Resilienz durch Thriving-Prozesse fördern, gleichzeitig aber auch Arbeitsplatzunsicherheit und Belastung erzeugen \parencite{liu_enabling_2024}.

Eine Studie mit N = 432 Beschäftigten bestätigte diesen dualen Pfad empirisch: Digitale Transformation wirkte sowohl ermöglichend (durch Förderung von Thriving) als auch belastend (durch erhöhte Jobinsecurity) auf die Resilienz von Mitarbeitenden \parencite{liu_enabling_2024}. Diese Ambivalenz unterstreicht die Notwendigkeit, digitale Transformation nicht als deterministische technologische Entwicklung, sondern als gestaltbaren sozio-technischen Prozess zu verstehen.

%%\subsubsection{Rolle von Führung in digitalen Transformationsprozessen.} 
Führungskräfte spielen eine zentrale Rolle bei der Gestaltung und Vermittlung digitaler Transformation. Sie fungieren als Sensemaker, die technologische Veränderungen interpretieren und in organisationale Narrative einbetten, als Change Agents, die Implementierungsprozesse steuern, und als Rollenmodelle, die durch ihr eigenes Verhalten digitale Praktiken legitimieren \parencite{quaquebekeNowNewNext2023}. 

Empirische Befunde zeigen, dass digitale Führungskompetenz – definiert als Fähigkeit, digitale Tools effektiv zu nutzen und Mitarbeitende bei der digitalen Transformation zu unterstützen – signifikant positiv mit dem Arbeitsengagement von Mitarbeitenden assoziiert ist (N = 559; \parencite{LiYangYang2024}). Zudem moderiert emotionale Intelligenz von Führungskräften die Wirkung digitaler Transformation auf Mitarbeitende: Führungskräfte mit hoher emotionaler Intelligenz können negative Effekte digitaler Veränderungen abpuffern und positive Effekte verstärken \parencite{LiYangYang2024}.

\subsection{Die Rolle des mittleren Managements bei der Implementierung neuer Technologien}

Das mittlere Management nimmt in Transformationsprozessen eine strategisch bedeutsame, aber oft ambivalente Position ein. Mittlere Führungskräfte sind gleichzeitig Treiber und Betroffene digitaler Veränderungen, Vermittler zwischen strategischen Vorgaben und operativer Umsetzung sowie Knotenpunkte organisationaler Kommunikation und Sensemaking \parencite{floydManagingStrategicConsensus1997}.

%\subsubsection{Positionierung in der Organisationshierarchie.} 
Mittlere Führungskräfte operieren an der Schnittstelle zwischen strategischer Planung (Top-Management) und operativer Ausführung (Frontline-Mitarbeitende). Diese Position ermöglicht ihnen einzigartigen Zugang zu Informationen aus verschiedenen organisationalen Ebenen und macht sie zu zentralen Akteuren bei der Integration von Top-down-Strategien und Bottom-up-Feedback \parencite{floydManagingStrategicConsensus1997}.

%\subsubsection{Funktionen: Vermittlung, Übersetzung, Implementation.} 
\textcite{floydManagingStrategicConsensus1997} identifizieren vier zentrale strategische Rollen mittlerer Führungskräfte: (1) \textit{Championing strategic alternatives} – das Einbringen innovativer Ideen in strategische Entscheidungsprozesse; (2) \textit{Synthesizing information} – die Aggregation und Interpretation von Informationen aus verschiedenen Quellen; (3) \textit{Facilitating adaptability} – die Förderung organisationaler Flexibilität und Anpassungsfähigkeit; sowie (4) \textit{Implementing deliberate strategy} – die Umsetzung strategischer Vorgaben in operative Praktiken.

Im Kontext digitaler Transformation übernehmen mittlere Führungskräfte zusätzlich die Funktion der \textit{Übersetzung}: Sie müssen abstrakte digitale Strategien in konkrete Arbeitspraktiken übersetzen, technologische Möglichkeiten und organisationale Anforderungen vermitteln sowie Widerstände und Ängste von Mitarbeitenden adressieren \parencite{klonekDoesAIWork2025}.

%\subsubsection{Spannungsfelder: Top-down-Vorgaben vs. Bottom-up-Feedback.} 
Mittlere Führungskräfte navigieren kontinuierlich zwischen widersprüchlichen Erwartungen: Sie sollen strategische Vorgaben des Top-Managements loyal umsetzen, gleichzeitig aber auch die Perspektiven und Widerstände ihrer Mitarbeitenden repräsentieren und nach oben kommunizieren. Diese Dualität kann zu Rollenkonflikten, Ambiguitätsstress und Identitätsspannungen führen \parencite{floydManagingStrategicConsensus1997}.


%\subsubsection{Mittlere Führungskräfte als "Change Agents" und "Betroffene".} 
Während mittlere Führungskräfte einerseits als Change Agents digitale Transformation vorantreiben sollen, sind sie andererseits selbst von technologischen Veränderungen betroffen – ihre Tätigkeitsprofile, Kompetenzen und professionelle Identität werden durch digitale Tools transformiert \parencite{quaquebekeNowNewNext2023}. Diese Doppelrolle erfordert kontinuierliche Selbstreflexion und Identitätsarbeit.

%\subsubsection{Besondere Anforderungen bei KI-Implementierung.} 
Die Implementierung generativer KI-Systeme stellt spezifische Anforderungen an mittlere Führungskräfte: Sie müssen nicht nur technische Funktionalitäten verstehen, sondern auch ethische Implikationen bewerten, Vertrauen in algorithmische Outputs entwickeln, Transparenz und Erklärbarkeit sicherstellen sowie ihre Mitarbeitenden bei der Adoption unterstützen \parencite{bankinsMultilevelReviewArtificial2024}.

\textcite{koponen_work_2025} identifizierten durch eine systematische Literaturanalyse zentrale Arbeitscharakteristika, die mittlere Führungskräfte benötigen, wenn sie KI-integrierte Service-Teams leiten: (1) Autonomie bei der Gestaltung von Mensch-KI-Interaktionen; (2) Feedback-Mechanismen, die sowohl menschliche als auch KI-Performance transparent machen; (3) Aufgabenvielfalt, die routinierte und strategische Tätigkeiten balanciert; sowie (4) soziale Unterstützung durch Peers und Vorgesetzte bei der Bewältigung KI-bezogener Unsicherheiten.

\subsection{Arbeitsgestaltung und ihre Auswirkungen auf Leistung und Wohlbefinden}

Theorien der Arbeitsgestaltung untersuchen, wie strukturelle und soziale Merkmale von Arbeit motivationale Prozesse, Leistung und Wohlbefinden beeinflussen. Zwei zentrale Rahmenmodelle prägen die arbeitspsychologische Forschung: das Job Characteristics Model (JCM) von \textcite{hackman_motivation_1976} und das Job Demands-Resources (JD-R) Modell \parencite{bakker_job_2007}.

%\subsubsection{Job Characteristics Model (Hackman \& Oldham).} 
Das JCM postuliert, dass fünf Kernmerkmale von Arbeit – Aufgabenvielfalt (skill variety), Aufgabenidentität (task identity), Aufgabenbedeutsamkeit (task significance), Autonomie (autonomy) und Rückmeldung (feedback) – kritische psychologische Zustände auslösen (Erlebte Bedeutsamkeit, Erlebte Verantwortung, Kenntnis der Ergebnisse), die wiederum intrinsische Motivation, Arbeitszufriedenheit und Leistung fördern \parencite{hackman_motivation_1976}.


%\subsubsection{Job Demands-Resources Modell.} 
Das JD-R Modell unterscheidet zwischen Arbeitsanforderungen (demands) – Aspekte der Arbeit, die physische oder psychische Anstrengung erfordern – und Arbeitsressourcen (resources) – Aspekte, die Zielerreichung fördern, Anforderungen reduzieren oder persönliches Wachstum stimulieren \parencite{bakker_job_2007}. Hohe Anforderungen bei geringen Ressourcen führen zu Erschöpfung und Gesundheitsbeeinträchtigungen, während hohe Ressourcen Engagement und Leistung fördern.

Im Kontext digitaler Transformation fungieren Technologien ambivalent: Sie können sowohl als Ressourcen (z.B. durch Automatisierung repetitiver Aufgaben, Bereitstellung von Informationen) als auch als Anforderungen (z.B. durch Lernaufwand, technologische Unsicherheit, Überwachungspotenzial) wirken \parencite{LiuWangLin2023}. Eine Zwei-Wellen-Studie (N = 296) untersuchte, wie wahrgenommene Technologieunsicherheit (perceived technology uncertainty) als Anforderung Technostress erhöht und Arbeitszufriedenheit reduziert. Die negativen Effekte wurden durch Koproduktionskontext und Skill-Flexibilität als Ressourcen moderiert \parencite{LiuWangLin2023}.

%\subsubsection{Autonomie, Aufgabenvielfalt und Bedeutsamkeit als Gestaltungsdimensionen.} 
Autonomie – definiert als Grad der Entscheidungsfreiheit über Arbeitsmethoden und -zeitpunkte – gilt als zentrale Gestaltungsdimension mit konsistent positiven Effekten auf Motivation und Wohlbefinden \parencite{HackmanOldham1976}. Aufgabenvielfalt (variety) reduziert Monotonie und fördert Kompetenzentwicklung, während Bedeutsamkeit (significance) das Erleben von Sinnhaftigkeit und Wertschätzung der eigenen Arbeit stärkt.

Empirische Studien zu Robotisierung zeigen jedoch, dass technologische Automatisierung diese Gestaltungsdimensionen negativ beeinflussen kann. Eine Studie mit Instrumentalvariablen-Schätzungen auf Branchenebene fand, dass eine Verdopplung der Robotisierung zu einem Rückgang der wahrgenommenen Arbeitsbedeutsamkeit um 0,9\% und einem Rückgang der Autonomie um 1\% führte. Bei einer 7,5-fachen Steigerung der Robotisierung wurden Rückgänge von 6,8\% bei Bedeutsamkeit und 7,5\% bei Autonomie projiziert \parencite{nikolova_robots_2024}. Diese Befunde verdeutlichen, dass technologische Transformation nicht automatisch zu besserer Arbeitsgestaltung führt, sondern intentionale Gestaltungsinterventionen erfordert.

%\subsubsection{Zusammenhang zwischen Arbeitsgestaltung und Motivation.} 
Arbeitsgestaltung beeinflusst Motivation primär durch die Befriedigung psychologischer Grundbedürfnisse. Autonomie, Kompetenz und soziale Eingebundenheit – die drei Grundbedürfnisse der Self-Determination Theory (\gls{SDT}) – werden durch spezifische Arbeitsmerkmale gefördert oder frustriert \parencite{gagne_understanding_2022}. Autonomieunterstützende Arbeitsgestaltung (z.B. Entscheidungsspielräume, Partizipation) fördert autonome Motivation, während kontrollierende Arbeitsgestaltung (z.B. rigide Vorgaben, Überwachung) kontrollierte oder amotivationale Zustände erzeugt.

%\subsubsection{Technologie als Gestaltungsparameter.} 
Technologie ist nicht neutral, sondern ein aktiver Gestaltungsparameter, der Arbeitsmerkmale konfiguriert. Digitale Tools können Autonomie erweitern (z.B. durch flexible Arbeitsorganisation) oder einschränken (z.B. durch algorithmische Vorgaben); sie können Feedback verbessern (z.B. durch Echtzeit-Dashboards) oder verschlechtern (z.B. durch intransparente Metriken); sie können Aufgabenvielfalt erhöhen (z.B. durch Zugang zu neuen Informationen) oder reduzieren (z.B. durch Standardisierung) \parencite{knight_how_2021}.

\textcite{knight_how_2021} führten eine systematische Review von Arbeitsgestaltungsinterventionen durch und entwickelten ein evidenzbasiertes Modell, das zeigt, wie Arbeitsgestaltung Leistung beeinflusst. Zentrale Mechanismen umfassen: (1) Motivationale Pfade (erhöhte intrinsische Motivation und Engagement); (2) Kognitive Pfade (verbesserte Aufmerksamkeit und Problemlösefähigkeit); sowie (3) Soziale Pfade (verstärkte Kooperation und soziale Unterstützung). Technologie kann alle drei Pfade sowohl fördern als auch behindern, abhängig von ihrer Gestaltung und Implementierung.


\section{Generative KI als sozio-technisches Arbeitssystem}

\subsection{Definition und Charakteristika generativer KI-Tools}

Generative Künstliche Intelligenz (GenAI) bezeichnet eine Klasse von KI-Systemen, die in der Lage sind, neuartige Inhalte – Texte, Bilder, Code, Audio – auf Basis von Trainingsdaten zu erzeugen. Im Unterschied zu traditionellen, regelbasierten oder klassifikatorischen KI-Systemen zeichnen sich generative Modelle durch ihre Fähigkeit zur kreativen Synthese, Kontextsensitivität und kontinuierlichen Lernfähigkeit aus \parencite{brynjolfssonGenerativeAIWork2025}.

%\subsubsection{Technologische Grundlagen: Large Language Models.} 
Die prominentesten generativen KI-Systeme basieren auf Large Language Models (LLMs) – neuronalen Netzwerken mit Milliarden von Parametern, die auf umfangreichen Textkorpora trainiert werden. LLMs wie GPT-4, Claude oder Gemini nutzen Transformer-Architekturen, um probabilistische Vorhersagen über Textsequenzen zu treffen und dadurch kohärente, kontextuell angemessene Antworten zu generieren \parencite{brynjolfssonGenerativeAIWork2025}. Der Durchbruch von ChatGPT im November 2022 und die Veröffentlichung von GPT-4 im März 2023 markierten einen Wendepunkt in der organisationalen Adoption generativer KI, da diese Modelle erstmals breiten Nutzergruppen zugänglich wurden.

%\subsubsection{Abgrenzung zu traditionellen Informationssystemen.} 
Traditionelle Informationssysteme (z.B. ERP-Systeme, Datenbanken, Dashboards) sind primär darauf ausgelegt, vorhandene Informationen zu speichern, zu verarbeiten und bereitzustellen. Generative KI-Systeme hingegen erzeugen \textit{neue} Inhalte, die nicht explizit in ihren Trainingsdaten enthalten sind. Diese generative Kapazität ermöglicht qualitativ neue Anwendungen: automatisierte Texterstellung, kreative Ideengenerierung, Codegenerierung, Szenarioanalysen und dialogische Interaktionen \parencite{bankinsMultilevelReviewArtificial2024}.

%\subsubsection{Spezifische Eigenschaften: Kreativität, Kontextsensitivität, Lernfähigkeit.} 
Generative KI-Systeme unterscheiden sich von früheren Technologien durch ihre Fähigkeit, probabilistisch neue Inhalte zu erzeugen, Kontext über mehrteilige Dialoge hinweg zu halten und Antworten an spezifische Nutzerbedürfnisse anzupassen \parencite{brynjolfssonGenerativeAIWork2025,bankinsMultilevelReviewArtificial2024}. Zudem lassen sie sich über Fine-Tuning und Retrieval-Augmented Generation (RAG) an organisationsspezifische Kontexte anpassen, während Nutzer durch iteratives Prompting ihre Interaktionskompetenz weiterentwickeln.


%\subsubsection{Anwendungsbereiche in wissensintensiven Tätigkeiten.} 
Generative KI findet primär Anwendung in wissensintensiven, kognitiv anspruchsvollen Tätigkeiten: Dokumentenerstellung und -analyse, Entscheidungsvorbereitung, Strategieentwicklung, Kommunikation, Problemlösung und kreative Arbeit \parencite{brynjolfssonGenerativeAIWork2025}. Eine Feldstudie mit Kundenservice-Mitarbeitenden (N > 5.000) zeigte, dass der Einsatz eines generativen KI-Assistenten die Anzahl gelöster Kundenanfragen pro Stunde um 14\% steigerte, wobei die Produktivitätsgewinne primär bei weniger erfahrenen Mitarbeitenden auftraten \parencite{brynjolfssonGenerativeAIWork2025}. Für hochqualifizierte Experten waren die Effizienzgewinne minimal, was darauf hindeutet, dass GenAI primär als Kompetenz-Augmentation für weniger erfahrene Wissensarbeiter fungiert.

\subsection{KI in Organisationen: Effizienzwerkzeuge vs. integrale Bestandteile von Entscheidungsprozessen}

Die organisationale Einbettung von KI kann aus drei unterschiedlichen Perspektiven konzeptualisiert werden: instrumentell, strategisch und transformativ \parencite{bankinsMultilevelReviewArtificial2024}.

%\subsubsection{Instrumentelle Perspektive: KI als Produktivitätstool.} 
Aus instrumenteller Sicht wird KI als Werkzeug zur Effizienzsteigerung betrachtet. KI-Systeme automatisieren repetitive Aufgaben, beschleunigen Informationsverarbeitung und reduzieren kognitive Belastung. Diese Perspektive dominiert in vielen frühen Adoptionsphasen und spiegelt sich in Metriken wie Zeitersparnis, Kostenreduktion und Output-Steigerung wider \parencite{brynjolfssonGenerativeAIWork2025}.

Die instrumentelle Perspektive birgt jedoch Risiken: Sie kann zu techno-deterministischen Implementierungen führen, die menschliche Faktoren – Motivation, Identität, Sinnerleben – vernachlässigen. Empirische Studien zeigen, dass rein effizienzorientierte KI-Implementierungen häufig Widerstände, Technostress und motivationale Kosten erzeugen \parencite{edwardsManagerialControlFeedback2024}.

%\subsubsection{Strategische Perspektive: KI als Entscheidungsunterstützungssystem.} 
Die strategische Perspektive konzeptualisiert KI als integralen Bestandteil von Entscheidungsprozessen. KI-Systeme liefern nicht nur Informationen, sondern generieren Empfehlungen, Prognosen und Handlungsalternativen, die in menschliche Entscheidungsfindung einfließen \parencite{tongJanusFaceArtificial2021}.

\textcite{tongJanusFaceArtificial2021} untersuchten die "Janus-Gesichtigkeit" von KI-Feedback in einer Feldstudie mit Verkaufsmitarbeitenden. Sie unterschieden zwischen \textit{Deployment-Effekten} (KI wird genutzt, aber Mitarbeitende wissen es nicht) und \textit{Disclosure-Effekten} (Mitarbeitende wissen, dass KI genutzt wird). Die Ergebnisse zeigten, dass KI-basiertes Feedback die Leistung verbesserte, aber nur wenn Mitarbeitende \textit{nicht} wussten, dass es von KI stammte. Die Offenlegung der KI-Quelle reduzierte die Akzeptanz und Wirksamkeit des Feedbacks, vermutlich aufgrund von Vertrauensdefiziten und wahrgenommener Entmenschlichung \parencite{tongJanusFaceArtificial2021}. Diese Befunde unterstreichen, dass die strategische Integration von KI in Entscheidungsprozesse transparente Kommunikation und Vertrauensaufbau erfordert.

%\subsubsection{Transformative Perspektive: KI als Neugestaltung von Arbeitsprozessen.} 
Die transformative Perspektive betrachtet KI als Katalysator fundamentaler Veränderungen in Arbeitsrollen, Organisationsstrukturen und professionellen Identitäten. KI ermöglicht nicht nur Effizienzgewinne innerhalb bestehender Prozesse, sondern eröffnet qualitativ neue Formen der Arbeit: Mensch-KI-Kollaboration, hybride Teamstrukturen, algorithmisch vermittelte Koordination \parencite{bankinsMultilevelReviewArtificial2024}.

\textcite{bankinsMultilevelReviewArtificial2024} entwickelten in ihrer Multilevel-Review ein Framework, das fünf thematische Pfade der KI-Wirkung in Organisationen identifiziert: (1) Mensch-KI-Kollaboration und Komplementarität; (2) Wahrnehmung von KI-Fähigkeiten und -Grenzen; (3) KI als Kontrollmechanismus (algorithmisches Management); (4) Arbeitsmarktimplikationen (Job Displacement, Skill Shifts); sowie (5) ethische und soziale Implikationen. Diese Pfade interagieren auf individueller, Team- und organisationaler Ebene und erzeugen komplexe, oft widersprüchliche Effekte.

%\subsubsection{Implikationen unterschiedlicher Implementierungsstrategien.} 
Die Wahl der Implementierungsstrategie beeinflusst maßgeblich, wie KI von Mitarbeitenden wahrgenommen und genutzt wird. Instrumentelle Implementierungen tendieren dazu, KI als eine Art Werkzeug zu framen, was Akzeptanz erleichtern kann, aber transformative Potenziale ungenutzt lässt. Strategische Implementierungen betonen KI als eine Art Partner in Entscheidungsprozessen, was höhere Anforderungen an Transparenz, Erklärbarkeit und Vertrauen stellt. Transformative Implementierungen erfordern fundamentale organisationale Veränderungen, einschließlich neuer Rollen, Kompetenzen und Governance-Strukturen \parencite{bankinsMultilevelReviewArtificial2024}.

\subsection{Wahrnehmung von KI-Systemen: Unterstützung vs. Kontrolle}

Die Wahrnehmung von KI-Systemen durch Mitarbeitende ist nicht technologisch determiniert, sondern sozial konstruiert und kontextabhängig. Eine zentrale Dimension dieser Wahrnehmung ist die Unterscheidung zwischen KI als \textit{Unterstützungssystem} (support) und KI als \textit{Kontrollsystem} (control) \parencite{edwardsManagerialControlFeedback2024}.

%\subsubsection{Dualität der KI-Wahrnehmung in Organisationen.} 
KI-Systeme können gleichzeitig unterstützende und kontrollierende Funktionen erfüllen – oder als solche wahrgenommen werden. Diese Dualität ist nicht bloß theoretisch, sondern empirisch dokumentiert: Dieselben KI-Tools werden von verschiedenen Nutzern oder in verschiedenen Kontexten unterschiedlich interpretiert \parencite{edwardsManagerialControlFeedback2024, monod_worker_2024}.

%\subsubsection{Vertrauen in KI-Systeme und algorithmische Autorität.} 
Vertrauen ist ein zentraler Faktor für die Akzeptanz und effektive Nutzung von KI-Systemen. Vertrauen in KI umfasst mehrere Dimensionen: Vertrauen in die technische Zuverlässigkeit (reliability), Vertrauen in die Fairness und Unvoreingenommenheit (fairness), Vertrauen in die Transparenz und Nachvollziehbarkeit (transparency) sowie Vertrauen in die organisationale Governance (governance) \parencite{prasadGenerativeAICatalyst2024}.

\textcite{prasad_generative_2024} untersuchten in einer Studie mit N = 1.362 Beschäftigten, wie generative KI HR-Praktiken beeinflusst, mediiert durch Vertrauen. Die Ergebnisse zeigten, dass Vertrauen in GenAI signifikant positiv mit der Akzeptanz KI-basierter HR-Praktiken (Rekrutierung, Leistungsmanagement, Lernen \& Entwicklung) assoziiert war. Vertrauen fungierte als vollständiger Mediator zwischen wahrgenommener KI-Nützlichkeit und Akzeptanz \parencite{prasadGenerativeAICatalyst2024}.

\textit{Algorithmische Autorität} bezeichnet das Phänomen, dass KI-Outputs als objektiv, neutral und autoritativ wahrgenommen werden – selbst wenn sie fehlerhaft oder verzerrt sind. Diese Autorität kann zu unkritischer Übernahme von KI-Empfehlungen führen (Automation Bias) oder umgekehrt zu pauschaler Ablehnung (Algorithm Aversion), abhängig von individuellen Dispositionen und Kontextfaktoren \parencite{bankinsMultilevelReviewArtificial2024}.



\subsection{Auswirkungen auf Arbeitsrollen und Tätigkeitsprofile von Führungskräften}

Generative KI transformiert die Arbeitsrollen und Tätigkeitsprofile von Führungskräften auf mehreren Ebenen: Aufgabenverschiebungen, Kompetenzanforderungen, Mensch-KI-Kollaboration und professionelle Identität.

%\subsubsection{Verschiebung von Routineaufgaben zu strategischen Tätigkeiten.} 
Eine häufig artikulierte Erwartung ist, dass KI Führungskräfte von administrativen und repetitiven Aufgaben entlastet und dadurch Raum für strategische, kreative und relationale Tätigkeiten schafft \parencite{quaquebekeNowNewNext2023}. Empirische Evidenz für diese Verschiebung ist jedoch gemischt.

\textcite{brynjolfssonGenerativeAIWork2025} fanden in ihrer Feldstudie, dass KI primär bei standardisierten, weniger komplexen Aufgaben Produktivitätsgewinne erzeugte, während hochkomplexe, kontextspezifische Aufgaben kaum von KI profitierten. Dies deutet darauf hin, dass KI zwar Routineaufgaben übernehmen kann, aber (noch) nicht in der Lage ist, strategische Führungsaufgaben – wie Sensemaking, Beziehungsaufbau, ethische Abwägungen – zu substituieren.

Gleichzeitig zeigen Studien, dass die Delegation von Aufgaben an KI neue Anforderungen erzeugt: Führungskräfte müssen KI-Outputs bewerten, kuratieren und kontextualisieren; sie müssen Prompts formulieren und iterativ verfeinern; sie müssen entscheiden, welche Aufgaben delegiert werden können und welche menschliches Urteil erfordern \parencite{quaquebekeNowNewNext2023}. Diese Meta-Aufgaben erfordern neue Kompetenzen und können zeitintensiv sein, sodass die erwartete Zeitersparnis möglicherweise nicht realisiert wird.

%\subsubsection{Veränderung von Kompetenzanforderungen.} 
Die Integration von GenAI in Führungsarbeit erfordert neue Kompetenzen, die unter dem Begriff "\gls{AI} Literacy" zusammengefasst werden können. \gls{AI} Literacy umfasst: (1) Technisches Verständnis (Funktionsweise, Grenzen und Verzerrungen von KI); (2) Anwendungskompetenz (effektive Nutzung von KI-Tools, Prompt Engineering); (3) Kritische Bewertung (Validierung von KI-Outputs, Erkennung von Fehlern und Bias); sowie (4) Ethische Reflexion (Berücksichtigung von Fairness, Transparenz und Verantwortung) \parencite{YangHsiHsun2024}.

Eine Studie mit N = 565 Designprofessionellen untersuchte die Akzeptanz von KI-Tools unter Verwendung eines integrierten UTAUT-\gls{SDT}-Modells. Das Modell erklärte 52,1\% der Varianz in der Verhaltensintention, KI-Tools zu nutzen. Interessanterweise moderierte die wahrgenommene Bedrohung durch Jobersatz (job replacement threat) die Beziehung zwischen Leistungserwartung und Nutzungsintention negativ: Personen, die KI als Bedrohung ihrer beruflichen Existenz wahrnahmen, zeigten geringere Nutzungsintention, selbst wenn sie die Leistungsfähigkeit der Tools anerkannten \parencite{YangHsiHsun2024}.

%\subsubsection{Mensch-KI-Kollaboration in Führungsfunktionen.} 
Führungsarbeit wird zunehmend als Mensch-KI-Kollaboration konzeptualisiert, in der menschliche und algorithmische Akteure komplementäre Beiträge leisten \parencite{SmithVanWagonerKeplinger2025}. \textcite{SmithVanWagonerKeplinger2025} entwickelten eine Signaling-Theory-Perspektive auf KI-Konvergenz in Mensch-KI-Teams. Sie argumentieren, dass erfolgreiche Kollaboration erfordert, dass beide Akteure – Mensch und KI – Signale über ihre Fähigkeiten, Intentionen und Zuverlässigkeit senden und interpretieren. Mangelnde Signalklarheit führt zu Missverständnissen, Misstrauen und ineffizienter Zusammenarbeit.

Empirische Studien zeigen, dass Mensch-KI-Kollaboration dann erfolgreich ist, wenn: (1) Rollen und Verantwortlichkeiten klar definiert sind; (2) KI transparent und erklärbar agiert; (3) Menschen Kontrollmöglichkeiten behalten (Human-in-the-Loop); sowie (4) kontinuierliches Lernen und Anpassung ermöglicht werden \parencite{bankinsMultilevelReviewArtificial2024}.

%\subsubsection{Identitätsarbeit und professionelles Selbstverständnis.} 
Die Integration von GenAI in Führungsarbeit stellt etablierte professionelle Identitäten infrage. Führungskräfte definieren ihre Rolle traditionell über Expertise, Erfahrung und soziale Kompetenz. Wenn KI-Systeme Expertise partiell substituieren oder sogar übertreffen, kann dies zu Identitätskrisen, Kompetenzbedrohung und Widerstand führen \parencite{quaquebekeNowNewNext2023}.

\textcite{quaquebeke_now_2023} argumentieren, dass KI die Natur von Führung fundamental transformieren wird: von wissensbasierter Autorität zu facilitativer, emotional-intelligenter Führung. Führungskräfte müssen ihre Rolle neu definieren – nicht als allwissende Experten, sondern als Kuratoren, Coaches und Sinnstifter, die menschliche und algorithmische Ressourcen orchestrieren.

\section{Self-Determination Theory (\gls{SDT}) als motivationstheoretischer Rahmen}

\subsection{Grundannahmen und Kernkonzepte der \gls{SDT}}

Die Self-Determination Theory (\gls{SDT}), entwickelt von Edward L. Deci und Richard M. Ryan, ist eine Meta-Theorie menschlicher Motivation, die erklärt, wie soziale Kontexte die Qualität der Motivation und das psychologische Wohlbefinden beeinflussen \parencite{deciWhatWhyGoal2000,deci_self-determination_2017}.

%\subsubsection{Meta-Theorie menschlicher Motivation.} 
\gls{SDT} postuliert, dass Menschen aktive, wachstumsorientierte Organismen sind, die intrinsisch motiviert sind, ihre Umwelt zu explorieren, Kompetenzen zu entwickeln und soziale Beziehungen aufzubauen. Diese intrinsische Tendenz wird jedoch durch soziale Kontexte entweder gefördert oder frustriert \parencite{deciWhatWhyGoal2000}. \gls{SDT} unterscheidet sich von behavioristischen und kognitiv-rationalen Motivationstheorien durch ihren Fokus auf die \textit{Qualität} der Motivation (nicht nur Intensität) und die Rolle psychologischer Grundbedürfnisse als universelle Nährstoffe für Motivation und Wohlbefinden.

%\subsubsection{Unterscheidung intrinsischer und extrinsischer Motivation.} 
\gls{SDT} differenziert zwischen intrinsischer Motivation – Verhalten, das um seiner selbst willen ausgeführt wird, aus Interesse und Freude – und extrinsischer Motivation – Verhalten, das instrumentell zur Erreichung separater Outcomes (Belohnungen, Vermeidung von Bestrafung) ausgeführt wird \parencite{deciWhatWhyGoal2000}.

Entscheidend ist, dass \gls{SDT} extrinsische Motivation nicht als per se problematisch betrachtet, sondern nach dem Grad der Selbstbestimmung (self-determination) differenziert. Extrinsische Motivation kann mehr oder weniger autonom sein, abhängig davon, inwieweit die Person die Regulation internalisiert hat \parencite{deci_self-determination_2017}.

%\subsubsection{Kontinuum der Selbstbestimmung.} 
\gls{SDT} beschreibt ein Kontinuum der Selbstbestimmung, das von Amotivation (keine Intention zu handeln) über verschiedene Formen extrinsischer Motivation bis zu intrinsischer Motivation reicht \parencite{deciWhatWhyGoal2000}:

Das Selbstbestimmungskontinuum reicht von Amotivation (keine Handlungsintention bzw. kein wahrgenommener Wert oder Kompetenz) über Formen extrinsischer Motivation – externale und introjizierte Regulation als kontrollierte, identifizierte und integrierte Regulation als zunehmend autonom internalisierte Regulationen – bis hin zur intrinsischen Motivation.
Je stärker die Regulation internalisiert ist, desto selbstbestimmter wird das Verhalten erlebt (von Kontrolle durch Belohnung/Bestrafung oder Schuld/Scham hin zu persönlich anerkannten und ins Selbst integrierten Zielen sowie Interesse und Freude).

Je autonomer die Motivation, desto positiver sind die Outcomes: höheres Wohlbefinden, bessere Leistung, größere Persistenz und kreativeres Problemlösen \parencite{deci_self-determination_2017}.


\subsection{Die drei psychologischen Grundbedürfnisse}

\gls{SDT} postuliert drei fundamentale psychologische Grundbedürfnisse, deren Befriedigung essenziell für psychologisches Wachstum, Integrität und Wohlbefinden ist: Autonomie, Kompetenz und soziale Eingebundenheit \parencite{deciWhatWhyGoal2000, van_den_broeck_review_2016}.
\subsubsection{Autonomie (Autonomy).}

Autonomie bezeichnet das Bedürfnis, sich als Ursprung (origin) des eigenen Verhaltens zu erleben – selbstbestimmt, volitional und in Übereinstimmung mit den eigenen Werten und Interessen zu handeln \parencite{deciWhatWhyGoal2000}. Autonomie bedeutet \textit{nicht} Unabhängigkeit oder Isolation, sondern \textit{Selbstbestimmung} – die Möglichkeit, Entscheidungen auf Basis eigener Überzeugungen zu treffen, auch wenn diese durch soziale Kontexte informiert sind.

Autonomie wird frustriert, wenn Verhalten durch externe Kräfte (Belohnungen, Bestrafungen, Überwachung) oder internalisierte Druckmechanismen (Schuld, Scham) kontrolliert wird. Selbstbestimmung erfordert, dass Personen ihre Handlungen als frei gewählt erleben, auch wenn objektive Einschränkungen bestehen \parencite{vandenbroeckReviewSelfDeterminationTheorys2016}.

Praktisch wird Autonomie gefördert durch: Wahlmöglichkeiten (choice), Partizipation an Entscheidungen, Bereitstellung von Rationalen (warum eine Aufgabe wichtig ist), Anerkennung von Perspektiven und Gefühlen sowie Minimierung von Kontrolle und Druck \parencite{deci_self-determination_2017}.

Für Führungskräfte ist Autonomie besonders relevant, da ihre Rolle traditionell mit Entscheidungsfreiheit und strategischem Einfluss assoziiert wird. Technologien, die Entscheidungsspielräume einschränken oder Kontrolle ausüben, können das Autonomieerleben von Führungskräften bedrohen und Reaktanz erzeugen \parencite{edwardsManagerialControlFeedback2024}.

\subsubsection{Kompetenz (Competence).}

Kompetenz bezeichnet das Bedürfnis, sich als effektiv und fähig zu erleben – Herausforderungen erfolgreich zu meistern, gewünschte Outcomes zu erzielen und kontinuierlich zu lernen \parencite{deciWhatWhyGoal2000}. Kompetenzerleben ist nicht identisch mit objektiver Kompetenz, sondern bezieht sich auf die \textit{subjektive Wahrnehmung} von Wirksamkeit und Meisterschaft.

Kompetenz wird gefördert durch: klares, konstruktives Feedback, das Informationen über Fortschritt und Erfolg liefert; erreichbare, aber herausfordernde Ziele; Gelegenheiten zur Kompetenzentwicklung; sowie Anerkennung von Leistungen \parencite{vandenbroeckReviewSelfDeterminationTheorys2016}.

Optimales Kompetenzerleben entsteht, wenn Aufgaben weder zu einfach (Langeweile) noch zu schwierig (Überforderung) sind, sondern im Bereich der "optimalen Herausforderung" liegen. Dieser Bereich entspricht dem Konzept des Flow \parencite{csikszentmihalyiFlowPsychologyOptimal2009}, ist aber breiter: Kompetenzerleben erfordert nicht zwingend vollständige Absorption, sondern das Gefühl, Fortschritte zu machen und Kontrolle über Outcomes zu haben.

Neue Technologien können Kompetenzerleben sowohl fördern als auch frustrieren. Sie fördern Kompetenz, wenn sie Nutzer befähigen, Aufgaben effektiver zu bewältigen, neue Fähigkeiten zu entwickeln und Erfolgserlebnisse zu generieren. Sie frustrieren Kompetenz, wenn sie zu komplex, intransparent oder unzuverlässig sind, oder wenn sie menschliche Expertise entwerten \parencite{gagne_understanding_2022}.

\subsubsection{Soziale Eingebundenheit (Relatedness).}

Soziale Eingebundenheit (Relatedness) bezeichnet das Bedürfnis, sich mit anderen verbunden, zugehörig und sozial integriert zu fühlen – Beziehungen zu pflegen, die durch gegenseitige Fürsorge, Respekt und Vertrauen gekennzeichnet sind \parencite{deciWhatWhyGoal2000}.

Im Arbeitskontext umfasst soziale Eingebundenheit: Zugehörigkeit zu Teams und Organisationen, unterstützende Beziehungen zu Kollegen und Vorgesetzten, Anerkennung und Wertschätzung durch andere sowie das Gefühl, einen Beitrag zu einer größeren Gemeinschaft zu leisten \parencite{vandenbroeckReviewSelfDeterminationTheorys2016}.

Soziale Eingebundenheit wird gefördert durch: respektvolle, empathische Kommunikation; Teamarbeit und Kooperation; soziale Unterstützung in Belastungssituationen; sowie organisationale Praktiken, die Gemeinschaft und Zugehörigkeit betonen \parencite{deci_self-determination_2017}.

Digitale Technologien können soziale Eingebundenheit sowohl erleichtern (z.B. durch Kommunikationstools, virtuelle Teams) als auch beeinträchtigen (z.B. durch Reduktion von Face-to-Face-Interaktion, Isolation). Die Wirkung hängt davon ab, wie Technologien gestaltet und genutzt werden \parencite{gagne_understanding_2022}.

\subsection{Bedürfnisbefriedigung und motivationale Konsequenzen}

Die Befriedigung der drei psychologischen Grundbedürfnisse ist der zentrale Mechanismus, durch den soziale Kontexte Motivation und Wohlbefinden beeinflussen \parencite{vandenbroeckReviewSelfDeterminationTheorys2016}.

%\subsubsection{Zusammenhang zwischen Bedürfnisbefriedigung und Wohlbefinden.} 
Meta-Analysen zeigen konsistent, dass Bedürfnisbefriedigung am Arbeitsplatz positiv mit Wohlbefinden (Lebenszufriedenheit, positive Affekte, Vitalität) und negativ mit Ill-Being (Burnout, Depression, Angst) assoziiert ist \parencite{vandenbroeckReviewSelfDeterminationTheorys2016}. Die Effekte sind robust über Kulturen, Berufe und Messmethoden hinweg.

%\subsubsection{Auswirkungen auf Leistung und Engagement.} 
Bedürfnisbefriedigung fördert nicht nur Wohlbefinden, sondern auch Leistung und Engagement. Eine Meta-Analyse von \textcite{van_den_broeck_review_2016} fand signifikante positive Zusammenhänge zwischen Bedürfnisbefriedigung und Job Performance, organisationalem Commitment, Arbeitsengagement sowie proaktivem Verhalten. Die Effekte werden mediiert durch autonome Motivation: Bedürfnisbefriedigung fördert intrinsische und identifizierte Motivation, die wiederum Leistung und Engagement steigern.

%\subsubsection{Bedürfnisfrustration und negative Outcomes.} 
\gls{SDT} unterscheidet konzeptionell und empirisch zwischen \textit{Bedürfnisbefriedigung} (need satisfaction) und \textit{Bedürfnisfrustration} (need frustration). Bedürfnisfrustration tritt auf, wenn Bedürfnisse aktiv blockiert oder untergraben werden – nicht bloß abwesend sind \parencite{VanDenBroeckVansteenkiste2010}. 

Bedürfnisfrustration ist ein stärkerer Prädiktor für negative Outcomes (Burnout, Ill-Being, kontraproduktives Verhalten) als Bedürfnisbefriedigung für positive Outcomes. Dies deutet auf eine Asymmetrie hin: Bedürfnisfrustration ist besonders schädlich, während Bedürfnisbefriedigung protektiv und förderlich wirkt \parencite{vandenbroeckReviewSelfDeterminationTheorys2016}.

%\subsubsection{Kompensatorische Mechanismen.} 
\gls{SDT} postuliert, dass die drei Bedürfnisse interagieren, aber nicht vollständig substituierbar sind. Hohe Befriedigung eines Bedürfnisses kann niedrige Befriedigung eines anderen Bedürfnisses \textit{teilweise}, aber nicht vollständig kompensieren. Alle drei Bedürfnisse müssen mindestens in einem Basismaß befriedigt sein, um optimales Funktionieren zu ermöglichen \parencite{deciWhatWhyGoal2000}.

Empirische Studien zeigen gemischte Befunde zu Kompensationseffekten. Einige Studien finden additive Effekte (alle drei Bedürfnisse tragen unabhängig bei), andere finden synergistische Effekte (Bedürfnisse verstärken sich gegenseitig) \parencite{vandenbroeckReviewSelfDeterminationTheorys2016}.

\subsection{Anwendung der \gls{SDT} im Arbeitskontext}

\gls{SDT} hat sich als fruchtbarer theoretischer Rahmen für die Arbeits- und Organisationspsychologie etabliert, mit umfangreicher empirischer Evidenz und praktischen Implikationen \parencite{deci_self-determination_2017, gagne_understanding_2022}.

%\subsubsection{\gls{SDT} in der Arbeits- und Organisationspsychologie.} 
\textcite{deci_self-determination_2017} bieten eine umfassende Review des Stands der \gls{SDT}-Forschung in Arbeitsorganisationen. Sie zeigen, dass \gls{SDT} erfolgreich auf diverse Outcomes angewendet wurde: Arbeitsmotivation, Leistung, Kreativität, Wohlbefinden, Burnout, organisationales Commitment, Turnover-Intention sowie proaktives und kontraproduktives Verhalten.

%\subsubsection{Empirische Befunde zu Arbeitsmotivation.} 
Die Multidimensional Work Motivation Scale (MWMS), entwickelt von \textcite{gagneMultidimensionalWorkMotivation2015}, operationalisiert das Selbstbestimmungskontinuum für den Arbeitskontext. Validierungsstudien in sieben Sprachen und neun Ländern (N > 3.000) bestätigen die faktorielle Struktur und prädiktive Validität der Skala \parencite{gagneMultidimensionalWorkMotivation2015}. Autonome Motivation (intrinsisch, integriert, identifiziert) war konsistent positiv assoziiert mit Leistung, Wohlbefinden und organisationalem Commitment, während kontrollierte Motivation (external, introjiziert) schwächere oder inkonsistente Effekte zeigte.

%\subsubsection{Autonomieunterstützende vs. kontrollierende Arbeitsumgebungen.} 
Ein zentrales Konzept der \gls{SDT}-Forschung ist \textit{autonomy support} – Führungsverhalten und organisationale Praktiken, die Autonomie, Kompetenz und Relatedness fördern. Autonomieunterstützung umfasst: Perspektiven anerkennen, Wahlmöglichkeiten bieten, Rationale bereitstellen, Kontrolle minimieren sowie Kompetenzentwicklung und soziale Verbundenheit fördern \parencite{deci_self-determination_2017}.

Meta-Analysen zeigen, dass autonomieunterstützende Führung signifikant positiv mit Bedürfnisbefriedigung, autonomer Motivation, Leistung und Wohlbefinden assoziiert ist \parencite{vandenbroeckReviewSelfDeterminationTheorys2016}. Kontrollierende Führung hingegen ist assoziiert mit kontrollierter Motivation, Bedürfnisfrustration und negativen Outcomes.

%\subsubsection{Führungsverhalten und Bedürfnisbefriedigung.} 
Führungsverhalten ist ein zentraler Antezedent von Bedürfnisbefriedigung. Transformationale Führung, dienende Führung (servant leadership) und authentische Führung fördern Bedürfnisbefriedigung durch Empowerment, Unterstützung und Sinnstiftung \parencite{deci_self-determination_2017}.

\subsection{\gls{SDT} und Technologieakzeptanz}

\gls{SDT} wurde zunehmend auf Technologieakzeptanz und -nutzung angewendet, oft in Integration mit dem Technology Acceptance Model (TAM) und der Unified Theory of Acceptance and Use of Technology (UTAUT) \parencite{KoenigPascal2024, YangHsiHsun2024}.

%\subsubsection{Bisherige Anwendungen der \gls{SDT} auf Technologienutzung.} 
Frühe Studien untersuchten, wie Bedürfnisbefriedigung durch Technologienutzung (z.B. Videospiele, soziale Medien, E-Learning-Plattformen) Engagement und Persistenz beeinflusst. Diese Studien zeigten, dass Technologien, die Autonomie (z.B. durch Wahlmöglichkeiten), Kompetenz (z.B. durch Feedback und Erfolgserlebnisse) und Relatedness (z.B. durch soziale Interaktion) fördern, höhere Nutzungsintention und Zufriedenheit erzeugen \parencite{KoenigPascal2024}.

%\subsubsection{Integration mit Technology Acceptance Model (TAM).} 
TAM postuliert, dass wahrgenommene Nützlichkeit (perceived usefulness) und wahrgenommene Benutzerfreundlichkeit (perceived ease of use) die Nutzungsintention bestimmen. \gls{SDT} erweitert TAM, indem es motivationale Mechanismen spezifiziert: Technologien, die Bedürfnisse befriedigen, fördern autonome Motivation, die wiederum Nutzungsintention und tatsächliche Nutzung steigert \parencite{YangHsiHsun2024}.

Eine Studie mit N = 565 Designprofessionellen integrierte UTAUT und \gls{SDT}, um die Akzeptanz von KI-Tools zu erklären. Das Modell zeigte, dass Leistungserwartung (performance expectancy) und soziale Einflüsse (social influence) Nutzungsintention vorhersagten, mediiert durch wahrgenommene Bedürfnisbefriedigung. Autonomie und Kompetenz waren besonders relevante Mediatoren \parencite{YangHsiHsun2024}.

%\subsubsection{Motivationale Faktoren der Technologieadoption.} 
\gls{SDT} identifiziert motivationale Faktoren, die über rein kognitive Bewertungen (Nützlichkeit, Benutzerfreundlichkeit) hinausgehen: Interesse und Freude (intrinsische Motivation), wahrgenommener Wert und Bedeutung (identifizierte Regulation), soziale Normen und Erwartungen (introjizierte oder identifizierte Regulation) sowie externe Anreize oder Zwänge (externale Regulation) \parencite{KoenigPascal2024}.

\textcite{KoenigPascal2024} entwickelten ein theoretisches Framework, das drei Akzeptanzperspektiven integriert: Nutzerakzeptanz (user acceptance), Delegationsakzeptanz (delegation acceptance) und gesellschaftliche Adoption (societal adoption). Für jede Perspektive identifizierten sie spezifische \gls{SDT}-relevante Faktoren: Nutzerakzeptanz hängt primär von Kompetenz und Autonomie ab; Delegationsakzeptanz von Vertrauen und wahrgenommener Zuverlässigkeit; gesellschaftliche Adoption von kollektiven Werten und sozialen Normen.

%\subsubsection{Forschungslücken im Kontext generativer KI.} 
Während \gls{SDT} auf traditionelle Technologien (Software, Plattformen, Roboter) angewendet wurde, ist ihre Anwendung auf generative KI noch begrenzt. Generative KI unterscheidet sich von früheren Technologien durch ihre Ko-Kreationsfähigkeit, Kontextsensitivität und potenzielle Bedrohung professioneller Identität. Diese Charakteristika erfordern theoretische Erweiterungen \parencite{gagne_understanding_2022}.

Spezifische Forschungslücken umfassen: (1) Wie beeinflusst die dialogische, ko-kreative Natur generativer KI Autonomie- und Kompetenzerleben? (2) Unter welchen Bedingungen fördert vs. frustriert GenAI Bedürfnisse? (3) Welche Rolle spielen Transparenz, Erklärbarkeit und Kontrolle für Bedürfnisbefriedigung? (4) Wie interagieren individuelle Unterschiede (\gls{AI} Literacy, Causality Orientations) mit GenAI-Nutzung?


\section{Integration: Generative KI und psychologische Grundbedürfnisse}

\subsection{Potenzielle Auswirkungen generativer KI auf das Autonomieerleben}

%\subsubsection{KI als Erweiterung vs. Einschränkung von Handlungsspielräumen.} 
Generative KI kann Autonomie sowohl erweitern als auch einschränken. Sie erweitert Autonomie, indem sie Nutzern neue Handlungsmöglichkeiten eröffnet: schnellere Informationsverarbeitung, kreative Ideengenerierung, Zugang zu Expertise, Automatisierung ungeliebter Aufgaben \parencite{brynjolfssonGenerativeAIWork2025}. Diese Erweiterung kann das Gefühl von Selbstwirksamkeit und Kontrolle stärken.

Gleichzeitig kann GenAI Autonomie einschränken, wenn sie als Kontrollmechanismus eingesetzt wird: Überwachung von Leistung, algorithmische Vorgaben, Einschränkung von Entscheidungsspielräumen, Entwertung menschlichen Urteils \parencite{edwardsManagerialControlFeedback2024}. Die Wahrnehmung von KI als kontrollierend führt zu Reaktanz, Widerstand und reduzierter autonomer Motivation.

%\subsubsection{Delegation von Entscheidungen an algorithmische Systeme.} 
Die Delegation von Entscheidungen an KI-Systeme ist ambivalent: Sie kann Entlastung bieten (Reduktion kognitiver Belastung, Zeitersparnis), aber auch Autonomieverlust bedeuten (Abgabe von Kontrolle, Abhängigkeit von Algorithmen). Die Wirkung hängt davon ab, ob Delegation als freiwillig und selbstbestimmt oder als erzwungen und fremdbestimmt erlebt wird \parencite{KoenigPascal2024}.

%\subsubsection{Wahrgenommene Kontrolle über KI-Tools.} Zentral für Autonomieerleben ist die wahrgenommene Kontrolle über KI-Tools: Können Nutzer entscheiden, \textit{ob}, \textit{wann} und \textit{wie} sie KI nutzen? Können sie KI-Outputs anpassen, überschreiben oder ablehnen? Haben sie Einfluss auf die Konfiguration und Parametrisierung von KI-Systemen? \parencite{edwardsManagerialControlFeedback2024}

Studien zeigen, dass \"Human-in-the-Loop\"-Designs, die Nutzern finale Entscheidungsgewalt belassen, Autonomieerleben fördern, während vollautomatisierte Systeme, die menschliche Inputs ignorieren, Autonomie frustrieren \parencite{bankinsMultilevelReviewArtificial2024}.

%\subsubsection{Transparenz und Erklärbarkeit von KI-Outputs.} 
Transparenz und Erklärbarkeit (Explainability) sind Voraussetzungen für informierte Autonomie. Wenn Nutzer nicht verstehen, \textit{wie} KI zu ihren Outputs gelangt, können sie nicht beurteilen, ob sie diesen vertrauen und folgen sollten. Intransparente \" Black-Box\"-Systeme untergraben Autonomie, selbst wenn sie technisch nützlich sind \parencite{prasadGenerativeAICatalyst2024}.

%\subsubsection{Spannungsfeld: Effizienzgewinne vs. Autonomieverlust.} 
Ein zentrales Spannungsfeld besteht zwischen Effizienzgewinnen durch KI und potenziellem Autonomieverlust. Organisationen priorisieren oft Effizienz und implementieren KI-Systeme, die Prozesse standardisieren und Entscheidungsspielräume reduzieren. Dies kann kurzfristig Produktivität steigern, aber langfristig autonome Motivation und Engagement untergraben \parencite{monod_worker_2024}.

\subsection{Kompetenzerleben im Umgang mit generativer KI}

%\subsubsection{Neue Kompetenzanforderungen: "\gls{AI} Literacy".} 
Generative KI erfordert neue Kompetenzen, zusammengefasst unter dem Begriff \"\gls{AI} Literacy\". \gls{AI} Literacy umfasst: (1) Technisches Verständnis (Funktionsweise, Grenzen, Bias von KI); (2) Anwendungskompetenz (Prompt Engineering, effektive Nutzung); (3) Kritische Bewertung (Validierung von Outputs, Erkennung von Fehlern); sowie (4) Ethische Reflexion (Fairness, Transparenz, Verantwortung) \parencite{YangHsiHsun2024}.

%\subsubsection{Erfolgserlebnisse durch KI-unterstützte Aufgabenbewältigung.} GenAI kann Kompetenzerleben fördern, indem sie Nutzern ermöglicht, Aufgaben zu bewältigen, die zuvor außerhalb ihrer Reichweite lagen. Beispiele: Erstellung professioneller Texte, Analyse komplexer Daten, Generierung kreativer Ideen. Diese Erfolgserlebnisse stärken Selbstwirksamkeit und Kompetenzwahrnehmung \parencite{brynjolfssonGenerativeAIWork2025}.

Allerdings zeigt Forschung auch, dass KI-unterstützte Erfolgserlebnisse ambivalent sein können: Wenn Nutzer den Erfolg primär der KI (nicht sich selbst) attribuieren, kann dies Kompetenzerleben untergraben statt stärken \parencite{WuLiuRuan2025}.

\textcite{WuLiuRuan2025} untersuchten in einer experimentellen Studie (N = 15.105), wie Mensch-GenAI-Kollaboration Aufgabenleistung und intrinsische Motivation beeinflusst. Die Ergebnisse zeigten, dass GenAI-Unterstützung die Aufgabenleistung signifikant steigerte, aber gleichzeitig die intrinsische Motivation reduzierte. Dieser negative Effekt wurde durch wahrgenommene Kompetenzbedrohung mediiert: Nutzer attribuierten Erfolg der KI und erlebten dadurch reduzierte Selbstwirksamkeit \parencite{WuLiuRuan2025}.

%\subsubsection{Bedrohung professioneller Identität und Expertise.} 
Generative KI kann professionelle Identität und Expertise bedrohen, insbesondere wenn KI Aufgaben übernimmt, die zuvor als Kern professioneller Kompetenz galten (z.B. Schreiben, Analyse, Beratung). Diese Bedrohung kann zu Kompetenzfrustration, Widerstand und defensiven Reaktionen führen \parencite{quaquebekeNowNewNext2023}.


%\subsubsection{Veränderung der Wertschätzung menschlicher Fähigkeiten.} 
GenAI kann die Wertschätzung bestimmter menschlicher Fähigkeiten verändern. Fähigkeiten, die leicht durch KI replizierbar sind (z.B. Informationssuche, Textgenerierung), verlieren an Wert, während Fähigkeiten, die KI (noch) nicht beherrscht (z.B. ethisches Urteil, emotionale Intelligenz, kreative Synthese), an Bedeutung gewinnen \parencite{quaquebekeNowNewNext2023}. Diese Verschiebung erfordert Neubewertung professioneller Kompetenzen und kann Identitätsarbeit auslösen.

\subsection{Soziale Eingebundenheit in KI-vermittelten Arbeitsumgebungen}

%\subsubsection{Auswirkungen auf Teamdynamiken und Zusammenarbeit.} 
Generative KI beeinflusst Teamdynamiken auf mehreren Ebenen: Sie kann Koordination erleichtern (z.B. durch automatisierte Zusammenfassungen, Übersetzungen), aber auch soziale Interaktion reduzieren (z.B. wenn KI menschliche Kommunikation substituiert) \parencite{SmithVanWagonerKeplinger2025}.

\textcite{SmithVanWagonerKeplinger2025} entwickelten eine Signaling-Theory-Perspektive auf KI-Konvergenz in Mensch-KI-Teams. Sie argumentieren, dass erfolgreiche Teamarbeit erfordert, dass Teammitglieder – menschlich und algorithmisch – klare Signale über Fähigkeiten, Intentionen und Zuverlässigkeit senden. Wenn KI intransparent agiert oder menschliche Signale missinterpretiert, entstehen Koordinationsprobleme und reduziertes Vertrauen, was soziale Eingebundenheit beeinträchtigt.

%\subsubsection{KI als Mediator sozialer Interaktion.} 
KI kann als Mediator sozialer Interaktion fungieren: Sie vermittelt Kommunikation (z.B. durch Übersetzung, Zusammenfassung), moderiert Konflikte (z.B. durch neutrale Analyse), oder koordiniert Aktivitäten (z.B. durch Scheduling). Diese Mediation kann soziale Prozesse effizienter machen, aber auch depersonalisieren \parencite{bankinsMultilevelReviewArtificial2024}.

%\subsubsection{Veränderung von Kommunikationsmustern.} 
GenAI verändert Kommunikationsmuster: Asynchrone, textbasierte Kommunikation (z.B. via KI-generierte E-Mails, Berichte) kann synchrone, persönliche Interaktion ersetzen. Dies kann Effizienz steigern, aber soziale Eingebundenheit reduzieren, wenn persönliche Beziehungen vernachlässigt werden \parencite{gagne_understanding_2022}.

%\subsubsection{Isolation vs. neue Formen der Vernetzung.} 
KI kann sowohl Isolation fördern (wenn sie menschliche Interaktion substituiert) als auch neue Formen der Vernetzung ermöglichen (z.B. durch KI-vermittelte Kollaboration über geografische und sprachliche Grenzen hinweg). Die Wirkung hängt ab von organisationaler Kultur und bewusster Gestaltung sozio-technischer Systeme \parencite{bankinsMultilevelReviewArtificial2024}.

%\subsubsection{Führung in hybriden Mensch-KI-Teams.} 
Führung in hybriden Mensch-KI-Teams erfordert neue Kompetenzen: Orchestrierung menschlicher und algorithmischer Ressourcen, Förderung von Vertrauen und Transparenz, Gestaltung von Rollen und Verantwortlichkeiten sowie Aufrechterhaltung sozialer Kohäsion \parencite{quaquebekeNowNewNext2023}.

\textcite{koponen_work_2025} identifizierten zentrale Arbeitscharakteristika für mittlere Führungskräfte in KI-integrierten Service-Teams, darunter: soziale Unterstützung durch Peers und Vorgesetzte, klare Kommunikation über Rollen und Erwartungen sowie Gelegenheiten für Face-to-Face-Interaktion trotz KI-Vermittlung.

\subsection{Synthesemodell: \gls{SDT}-basierte Analyse der KI-Nutzung durch Führungskräfte}

%\subsubsection{Konzeptionelles Rahmenmodell der Arbeit.} 
Das konzeptionelle Modell dieser Arbeit umfasst als unabhängige Variable die Intensität der generativen KI-Nutzung in der Entscheidungsvorbereitung durch mittlere Führungskräfte, als Mediator die wahrgenommene Unterstützungsqualität generativer KI-Tools (Vertrauen, Transparenz, Nützlichkeit) und als abhängige Variable das wahrgenommene Kompetenzerleben als zentrales psychologisches Grundbedürfnis nach \gls{SDT}.
Als Kontrollvariablen werden KI-Erfahrung, Führungserfahrung und Hierarchieebene berücksichtigt.

%\subsubsection{Wechselwirkungen zwischen den drei Grundbedürfnissen.} 
Obwohl diese Arbeit Kompetenzerleben als fokales Konstrukt untersucht, ist anzuerkennen, dass die drei Grundbedürfnisse (Autonomie, Kompetenz, Relatedness) interagieren. Autonomie und Kompetenz sind konzeptionell und empirisch korreliert: Autonomieunterstützung fördert Kompetenzentwicklung, und Kompetenzerleben stärkt das Gefühl von Selbstbestimmung \parencite{vandenbroeckReviewSelfDeterminationTheorys2016}.

Zukünftige Forschung sollte untersuchen, wie GenAI alle drei Bedürfnisse simultan beeinflusst und wie diese Einflüsse interagieren.

%\subsubsection{Moderatoren und Mediatoren.} 
Das Modell postuliert wahrgenommene Unterstützungsqualität als zentralen Mediator. Zusätzliche Moderatoren könnten umfassen: \gls{AI} Literacy (höhere Literacy könnte positive Effekte verstärken), Causality Orientations (autonomie-orientierte Personen könnten stärker von autonomieunterstützender KI profitieren), organisationale Unterstützung (Training, Ressourcen) sowie Führungsstil (autonomieunterstützende vs. kontrollierende Führung durch Vorgesetzte).

%\subsubsection{Ableitung von Hypothesen.} 
Basierend auf der theoretischen Integration werden folgende Hypothesen abgeleitet:

\subsubsection{Hypothese 1 (Haupteffekt):} Die Intensität der Nutzung generativer KI-Tools ist positiv assoziiert mit dem wahrgenommenen Kompetenzerleben mittlerer Führungskräfte.
    
    \textit{Begründung:} GenAI ermöglicht Führungskräften, Aufgaben effektiver zu bewältigen, Erfolgserlebnisse zu generieren und neue Fähigkeiten zu entwickeln, was Kompetenzerleben fördert \parencite{brynjolfssonGenerativeAIWork2025, gagne_understanding_2022}.
    
\subsubsection{Hypothese 2 (Mediation):} Die wahrgenommene Unterstützungsqualität generativer KI-Tools mediiert den positiven Zusammenhang zwischen der Intensität der KI-Nutzung und dem wahrgenommenen Kompetenzerleben.
    
    \textit{Begründung:} KI-Nutzung fördert Kompetenzerleben primär dann, wenn KI als vertrauenswürdig, transparent und nützlich wahrgenommen wird – d.h. als Unterstützungssystem, nicht als Kontrollsystem \parencite{edwardsManagerialControlFeedback2024, prasad_generative_2024}. Wahrgenommene Unterstützungsqualität ist der Mechanismus, durch den KI-Nutzung in Kompetenzerleben übersetzt wird.


Diese Hypothesen werden im empirischen Teil der Arbeit getestet.