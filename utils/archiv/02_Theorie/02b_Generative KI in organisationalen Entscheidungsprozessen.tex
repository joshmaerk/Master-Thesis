\subsection{Generative KI als sozio-technisches Arbeitssystem}
Generative KI stellt eine neue Klasse digitaler Arbeitssysteme dar, die sich in ihrer Funktionslogik und organisationalen Wirkung deutlich von klassischer Informations\-technologie und deterministischer Automatisierung unterscheidet. Während traditionelle IT-Systeme primär auf die regelbasierte Ausführung klar definierter Prozesse ausgerichtet sind, zeichnen sich generative KI-Systeme durch ihre Fähigkeit aus, auf Basis probabilistischer Modelle eigenständig Inhalte zu erzeugen, Informationen zu synthetisieren und kontextsensitiv zu interagieren \parencite{bankinsMultilevelReviewArtificial2024}. In der organisationswissenschaftlichen Diskussion wird generative KI daher nicht als isolierte technische Innovation verstanden, sondern als sozio-technisches Arbeitssystem, dessen Effekte sich erst im Zusammenspiel mit organisationalen Strukturen, Rollen und Arbeitspraktiken entfalten \parencite{bankinsMultilevelReviewArtificial2024}. Zentrale Übersichtsarbeiten betonen, dass KI in Organisationen insbesondere dann wirksam wird, wenn sie nicht substituierend eingesetzt wird, sondern menschliche kognitive Fähigkeiten ergänzt und erweitert (Augmentierung statt Substitution) \parencite{bankinsMultilevelReviewArtificial2024}. Damit ist generative KI im Spektrum algorithmischer Systeme als kognitiv-unterstützendes System einzuordnen und von Formen des algorithmischen Managements abzugrenzen, die Arbeit über algorithmische Zielvorgaben, Monitoring, Bewertung oder (teil-)automatisierte Steuerung koordinieren \parencite{kadolkar_algorithmic_2025,parent-rocheleau_creation_2024}. Entsprechend fokussiert die vorliegende Arbeit auf eine freiwillige, unterstützende Nutzung generativer KI in der Entscheidungsvorbereitung und ausdrücklich nicht auf eine direkte Steuerung oder Überwachung von Mitarbeitenden.

\subsection{KI-gestützte Entscheidungsvorbereitung im mittleren Management}

Entscheidungsarbeit im mittleren Management ist typischerweise wissensintensiv, von Unsicherheit geprägt und häufig zeitkritisch, da sie unter unvollständiger Informationslage zwischen strategischer Vorgabe und operativer Umsetzung vermittelt. In diesem Kontext kann generative KI als kognitive Ressource eingesetzt werden, indem sie Informationssuche und -verdichtung unterstützt, Entscheidungsalternativen strukturiert und Argumentationslinien bzw. Handlungsoptionen konsistent aufbereitet \parencite{bankinsMultilevelReviewArtificial2024}. Der Einsatz zielt dabei weniger auf den Wegfall von Führungs- und Entscheidungsrollen als auf deren Veränderung: Aufgaben verschieben sich von der reinen Informationsverarbeitung hin zu Einordnung, Plausibilisierung und verantwortlicher Abwägung.

Zugleich ist konzeptionell klarzustellen, dass sich der Fokus dieser Arbeit auf die Entscheidungsvorbereitung (z.\,h. die Generierung und Aufbereitung von Entscheidungsgrundlagen) und nicht auf die Entscheidungshoheit richtet. Auch bei KI-gestützter Vorbereitung verbleiben Verantwortung und letztliche Entscheidung beim Menschen, was die Anschlussfähigkeit an motivationalpsychologische Perspektiven wie die Self-Determination Theory (insb. Autonomie als erlebter Handlungsspielraum) unterstützt.

Im Managementkontext findet generative KI vor allem in der Vorbereitung von Entscheidungen Anwendung, etwa durch die Analyse umfangreicher Datenbestände, die Strukturierung von Entscheidungsalternativen oder die Aufbereitung komplexer Sachverhalte \parencite{bankinsMultilevelReviewArtificial2024}. Für Führungskräfte des mittleren Managements, die zwischen strategischer Vorgabe und operativer Umsetzung vermitteln, sind Entscheidungsprozesse häufig durch hohe Komplexität, Zeitdruck und Unsicherheit gekennzeichnet. Generative KI kann hier als kognitive Unterstützung fungieren, indem sie Informationsverarbeitung erleichtert und Entscheidungsgrundlagen transparenter macht. Zugleich verdeutlicht Forschung zu digitaler und KI-basierter Führung, dass der Einsatz solcher Systeme nicht wertneutral ist, da er Rollenverständnisse, Verantwortungszuschreibungen und das Erleben professioneller Handlungsspielräume beeinflusst \parencite{quaquebekeNowNewNext2023}. In diesem Sinne verändert generative KI nicht nur die Effizienz der Entscheidungsvorbereitung, sondern auch die inhaltliche Ausgestaltung von Führungs- und Managementarbeit, indem neue Formen der Mensch--KI-Kollaboration entstehen \parencite{smithNavigatingAIConvergence2025}.

\subsection{Wahrnehmungsabhängige Wirkungen von KI}

Die Wirkungen generativer KI in organisationalen Entscheidungsprozessen sind maßgeblich durch subjektive Wahrnehmungen vermittelt: KI wirkt nicht direkt, sondern über Sinnzuschreibungen der Nutzenden sowie darüber, welche Intentionen sie hinter dem Einsatz vermuten. Forschung im Bereich Organizational Behavior zeigt, dass algorithmische Systeme entweder als unterstützende Ressource (z.\,B. Feedback- und Lernhilfe) oder als kontrollierender Eingriff (z.\,B. managerial control) interpretiert werden können, was mit unterschiedlichen motivationalen und affektiven Konsequenzen verbunden ist \parencite{bankinsMultilevelReviewArtificial2024,edwards_managerial_2024}. Während unterstützend erlebte KI-Nutzung mit höherer Akzeptanz, Vertrauen und Lernbereitschaft einhergeht, können kontrollorientierte oder intransparente Einsatzformen Stress, Widerstand und wahrgenommenen Autonomieverlust begünstigen \parencite{klonekDoesAIWork2025}.

Ein zentraler Einflussfaktor dieser Wahrnehmung ist Transparenz bzw. Offenlegung: Ob und wie KI in Entscheidungsprozessen eingesetzt wird, beeinflusst Sensemaking-Prozesse und kann Leistungs- bzw. Einstellungswirkungen in entgegengesetzte Richtungen verschieben (z.\,B. Deployment- vs. Disclosure-Effekte) \parencite{tongJanusFaceArtificial2021}. Ergänzend weisen experimentelle Befunde im Kontext generativer KI darauf hin, dass Mensch--KI-Kollaboration zwar Leistungsgewinne ermöglichen kann, zugleich aber motivational ambivalent sein kann, wenn die Interaktion als fremdbestimmt oder entwertend erlebt wird \parencite{wu_human-generative_2025}.

Schließlich spielt Vertrauen eine zentrale Rolle dafür, ob generative KI als hilfreiche Unterstützung oder als riskante, potenziell kontrollierende Technologie interpretiert wird. Insbesondere im HR- und Managementkontext wird Vertrauen als vermittelnder Mechanismus diskutiert, über den sich die Einführung generativer KI auf Akzeptanz und nachgelagerte arbeitsbezogene Outcomes auswirken kann \parencite{prasadGenerativeAICatalyst2024}. Studien zum algorithmischen Management verdeutlichen zudem, dass KI-basierte Steuerungsmechanismen insbesondere dann als problematisch erlebt werden, wenn sie individuelle Expertise entwerten und Handlungsspielräume einschränken \parencite{kadolkar_algorithmic_2025}. Für den organisationalen Kontext generativer KI folgt daraus, dass deren Effekte nicht allein aus technischen Eigenschaften erklärbar sind, sondern wesentlich durch organisationale Gestaltung, Führung und Sinnzuschreibung bestimmt werden. Diese wahrnehmungsabhängige Wirklogik bereitet die Annahme vor, dass die Nutzung generativer KI in Entscheidungsprozessen über psychologische Mechanismen auf motivational relevante Konstrukte wirkt.

Weitere Forschung weist darauf hin, dass die Auswirkungen KI-gestützter Systeme in Organisationen nicht allein durch ihre funktionalen Eigenschaften erklärbar sind, sondern maßgeblich durch subjektive Wahrnehmungen der Nutzung vermittelt werden. Besonders deutlich wird diese Ambivalenz in der Arbeit von \textcite{tongJanusFaceArtificial2021}, die zwischen einem leistungssteigernden \emph{Deployment Effect} und einem potenziell motivationsmindernden \emph{Disclosure Effect} algorithmischer Systeme unterscheidet. Während der Einsatz KI-basierter Feedback- und Entscheidungssysteme objektiv die Qualität von Informationen und Handlungsempfehlungen erhöhen kann, zeigen die Autoren, dass die Offenlegung algorithmischer Einflussnahme bei Mitarbeitenden zugleich Kontrollwahrnehmungen, Misstrauen und Reaktanz auslösen kann \parencite{tongJanusFaceArtificial2021}.

Ergänzend verdeutlichen neuere Arbeiten, dass Vertrauen in KI-Systeme eine zentrale Voraussetzung für positive organisationale Wirkungen technologischer Unterstützung darstellt. Prasad und De (2024) zeigen im Kontext generativer KI im Human Resource Management, dass wahrgenommene Nützlichkeit und Einsatz generativer KI nur dann zu höherem Commitment und positiven arbeitsbezogenen Einstellungen führen, wenn Mitarbeitende den Systemen vertrauen \parencite{prasadGenerativeAICatalyst2024}. Vertrauen fungiert dabei als vermittelnder psychologischer Mechanismus zwischen technologischer Intervention und motivationalen Outcomes.

Aus organisationspsychologischer Perspektive impliziert dies, dass KI-gestützte Systeme ihre unterstützende Funktion nur dann entfalten können, wenn sie als verlässlich, transparent und wohlwollend wahrgenommen werden. Fehlt dieses Vertrauen, besteht die Gefahr, dass identische technologische Funktionen nicht als Ressource, sondern als potenzielle Bedrohung interpretiert werden. Für den Einsatz generativer KI in Entscheidungsprozessen von Führungskräften bedeutet dies, dass wahrgenommene Unterstützungsqualität nicht ausschließlich von der funktionalen Leistungsfähigkeit der Systeme abhängt, sondern maßgeblich durch Vertrauenszuschreibungen geprägt ist. Die Ergebnisse von Prasad und De (2024) stützen damit die Annahme, dass die Wirkung generativer KI auf motivational relevante Konstrukte indirekt und wahrnehmungsabhängig erfolgt.

Diese Befunde unterstreichen, dass KI in organisationalen Entscheidungsprozessen nicht als neutraler technologischer Input wirkt, sondern als sozio-technisches System, dessen Effekte von Zuschreibungen hinsichtlich Unterstützung versus Kontrolle abhängen. Für Führungskräfte bedeutet dies, dass identische KI-Funktionalitäten entweder als kognitive Entlastung und Kompetenzressource oder als Einschränkung eigener Handlungsspielräume interpretiert werden können. Damit liefern die Ergebnisse von \textcite{tongJanusFaceArtificial2021} eine zentrale theoretische Grundlage für die Annahme, dass wahrgenommene Unterstützungsqualität ein vermittelnder Mechanismus zwischen KI-Nutzung und motivationalen Outcomes ist.