%\section{Kontextbedingungen und theoretische Grenzziehung}

Zur theoretischen Absicherung des entwickelten Wirkmodells ist es erforderlich, zentrale Kontextbedingungen und potenzielle Einflussfaktoren zu reflektieren, die die Wirkung generativer KI auf motivational relevante Konstrukte beeinflussen können, ohne selbst Bestandteil der empirischen Modellierung zu sein. Ziel dieses Abschnitts ist es, die Gültigkeit und Reichweite des Modells einzugrenzen und zugleich aufzuzeigen, unter welchen Bedingungen die postulierten Zusammenhänge erwartungsgemäß variieren könnten.

Ein erster relevanter Kontextfaktor betrifft die Frage der Autonomiegefährdung durch verpflichtende oder stark normierte Formen der KI-Nutzung. Aus Perspektive der Self-Determination Theory ist Autonomie ein zentrales psychologisches Grundbedürfnis, dessen Frustration negative motivationale Konsequenzen nach sich zieht. Wird der Einsatz generativer KI nicht als freiwillige Unterstützung, sondern als verpflichtende Vorgabe oder implizite Erwartung wahrgenommen, kann dies das Erleben von Selbstbestimmung einschränken und potenziell auch positive Effekte auf das Kompetenzerleben relativieren. Das vorliegende Modell fokussiert daher bewusst auf Kontexte, in denen die Nutzung generativer KI primär als unterstützendes Element der Entscheidungsvorbereitung verstanden wird und die Entscheidungshoheit weiterhin bei der Führungskraft verbleibt.

Ein weiterer Einflussfaktor ist die individuelle Wahrnehmung technologischer Bedrohung, häufig operationalisiert über Konzepte wie Technologieangst oder STARA-Awareness (Smart Technology, Artificial Intelligence, Robotics and Algorithms). Forschung zeigt, dass eine hohe Sensibilität gegenüber der potenziellen Ersetzbarkeit durch Technologie mit negativen Einstellungen, reduziertem Wohlbefinden und geringerer organisationaler Bindung einhergehen kann. In solchen Fällen besteht die Möglichkeit, dass generative KI nicht als Ressource, sondern als Bedrohung der eigenen beruflichen Kompetenz interpretiert wird. Diese individuelle Disposition kann somit die Wahrnehmung der Unterstützungsqualität beeinflussen und stellt eine relevante theoretische Wirkbedingung dar, die jedoch im Rahmen der vorliegenden Arbeit nicht explizit modelliert wird.

Schließlich sind auch Führungsstil und organisationaler sowie kultureller Kontext als bedeutsame Rahmenbedingungen zu berücksichtigen. Empirische Studien zeigen, dass unterstützende, lernorientierte und demütige Führungsstile die Befriedigung psychologischer Grundbedürfnisse fördern und adaptive Leistungsformen begünstigen. In einem solchen Kontext kann der Einsatz generativer KI eher als Gelegenheit zur Kompetenzentwicklung und Reflexion wahrgenommen werden. Demgegenüber können stark kontrollorientierte Führungs- und Organisationskulturen dazu führen, dass KI primär als Überwachungs- oder Steuerungsinstrument interpretiert wird, was die motivationalen Effekte des KI-Einsatzes abschwächen oder umkehren kann.

Zusammenfassend werden Autonomiebedingungen, individuelle technologische Dispositionen sowie Führungs- und Kontextfaktoren als theoretisch relevante Wirkbedingungen anerkannt, ohne sie in das empirische Modell aufzunehmen. Diese bewusste Grenzziehung dient der Fokussierung und Modellökonomie und ermöglicht zugleich eine differenzierte Interpretation der Ergebnisse im Lichte potenzieller Kontextabhängigkeiten.