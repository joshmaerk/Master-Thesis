\subsection{Arbeitsgestaltung}
Arbeitsgestaltung wirkt nicht unmittelbar über objektive Merkmale, sondern entfaltet ihre Effekte maßgeblich über subjektive Wahrnehmungen der Beschäftigten. Zentrale arbeitsbezogene Ressourcen wie Autonomie, Feedback oder Unterstützung beeinflussen Motivation und Leistung insofern, als sie von Individuen als hilfreich für die Bewältigung ihrer Arbeitsanforderungen interpretiert werden; Unterstützung kann dabei als zentrale Ressource der Arbeitsgestaltung verstanden werden \parencite{knight_how_2021}. In der arbeits- und organisationspsychologischen Forschung wird Unterstützung dabei als zentrale Ressource verstanden, die es Individuen ermöglicht, Anforderungen effektiv zu bewältigen, Lernprozesse zu realisieren und Unsicherheiten zu reduzieren. Insbesondere in wissensintensiven Arbeitskontexten ist wahrgenommene Unterstützung eng mit der Nutzung von Expertise, der Qualität von Problemlösungen und dem Erleben von Wirksamkeit verbunden \parencite{hong_explaining_2018}. Damit bildet wahrgenommene Unterstützungsqualität einen zentralen vermittelnden Mechanismus zwischen Arbeitsgestaltung und motivationalen Outcomes.

\subsection{Unterstützungsqualität digitaler Systeme}
Digitale Systeme können im Arbeitskontext als spezifische Form arbeitsbezogener Unterstützung fungieren, insbesondere wenn sie kognitive Entlastung, Strukturierung von Informationen oder Qualitätsverbesserungen ermöglichen. Forschung zur Arbeitsgestaltung zeigt, dass solche unterstützenden Systeme dann wirksam werden, wenn sie als Ressource wahrgenommen werden, die Arbeitsprozesse erleichtert und nicht zusätzliche Komplexität erzeugt \parencite{knight_how_2021}. Generative KI kann in diesem Sinne als kognitive Unterstützung in Wissens- und Entscheidungsarbeit verstanden werden, indem sie Informationen zusammenführt, Entscheidungsalternativen strukturiert und Reflexionsprozesse anregt. Empirische Arbeiten zur motivierenden Arbeitsgestaltung zeigen, dass wahrgenommene Unterstützung durch Arbeitsmittel und -prozesse mit autonomer Motivation und konstruktivem Wissensverhalten verbunden ist \parencite{gagne_different_2019}. Entscheidend ist dabei weniger die technische Leistungsfähigkeit des Systems als vielmehr die subjektive Einschätzung, ob das System zur effektiven Bewältigung der Arbeitsaufgabe beiträgt.
Eine zentrale Differenzierung hinsichtlich der Wirkungen algorithmischer Systeme liefern Edwards et al. (2024), die zeigen, dass identische digitale Systeme von Beschäftigten entweder als unterstützende Feedbackinstrumente oder als kontrollierende Steuerungsmechanismen wahrgenommen werden können. In ihrer Studie zu algorithmisch unterstützten HR-Systemen argumentieren die Autoren, dass diese Wahrnehmungsattribution entscheidend darüber ist, ob technologische Systeme autonome oder kontrollierte Motivation auslösen \parencite{edwardsManagerialControlFeedback2024}. 

Wird ein algorithmisches System primär als unterstützendes Feedbackinstrument interpretiert, berichten Mitarbeitende höhere intrinsische Motivation, geringere Erschöpfung und ein stärkeres Erleben eigener Wirksamkeit. Demgegenüber führen Kontrollzuschreibungen zu erhöhter extrinsischer Motivation, wahrgenommenem Druck und motivationalen Kosten. Diese Befunde sind konsistent mit der Self-Determination Theory, wonach Unterstützungserfahrungen zur Befriedigung psychologischer Grundbedürfnisse beitragen, während kontrollierende Kontexte Need Thwarting begünstigen. Für generative KI in Entscheidungsprozessen impliziert dies, dass ihre motivationalen Effekte weniger von der technischen Ausgestaltung als von der subjektiven Interpretation ihrer Rolle abhängen. Edwards et al. \parencite[]{edwards_managerial_2024} liefern damit eine zentrale theoretische Grundlage für die Annahme, dass wahrgenommene Unterstützungsqualität als vermittelnder Mechanismus zwischen KI-Nutzung und Kompetenzerleben wirkt.

\paragraph{Abgrenzung zu algorithmischem Management}
In der aktuellen Forschung wird der Einsatz algorithmischer Systeme in Organisationen zunehmend unter dem Begriff des algorithmischen Managements diskutiert. Parent-Rocheleau et al. (2024) entwickeln mit dem Algorithmic Management Questionnaire ein differenziertes Instrument zur Erfassung algorithmischer Steuerungsmechanismen, etwa in den Bereichen Leistungsüberwachung, Zielvorgabe, Einsatzplanung und Leistungsbewertung \parencite{parent-rocheleau_creation_2024}. Die Autoren zeigen, dass eine hohe Exponierung gegenüber algorithmischem Management mit geringerer wahrgenommener Autonomie, reduzierter Arbeitskomplexität und niedrigerem Work Engagement einhergeht. 

Diese Perspektive ist für die vorliegende Arbeit insofern relevant, als sie eine konzeptionelle Abgrenzung erlaubt. Während algorithmisches Management auf die automatisierte Steuerung und Kontrolle von Arbeitsprozessen abzielt, fokussiert die vorliegende Untersuchung auf die freiwillige Nutzung generativer KI als kognitive Unterstützung in der Entscheidungsvorbereitung von Führungskräften. Der Einsatz generativer KI wird hier nicht als Fremdsteuerungsinstrument verstanden, sondern als potenzielle Ressource, deren Wirkung maßgeblich von der subjektiven Wahrnehmung ihrer Unterstützungsqualität abhängt. Durch diese Abgrenzung wird verdeutlicht, dass negative motivationalen Effekte algorithmischer Systeme, wie sie im Kontext algorithmischen Managements beschrieben werden, nicht ohne Weiteres auf unterstützende KI-Anwendungen in Entscheidungsprozessen übertragbar sind.

\subsection{Unterstützungsqualität, Vertrauen und Kompetenzerleben im Sinne der \gls{SDT}}

Die Unterstützungsqualität generativer KI ist nicht als objektives Merkmal der Technologie zu verstehen, sondern als subjektive Zuschreibung der Nutzenden. Unterstützung bezeichnet in diesem Zusammenhang die wahrgenommene Ermöglichung wirksamer Aufgabenbearbeitung, etwa durch verbesserte Informationsgrundlagen, kognitive Entlastung, Strukturierung komplexer Sachverhalte oder die Anregung von Reflexionsprozessen. Damit ist das Konstrukt der Unterstützungsqualität konzeptionell anschlussfähig an arbeits- und motivationspsychologische Ansätze, die den Fokus auf wahrnehmungsbasierte Wirkmechanismen legen.

Aus theoretischer Perspektive ist zwischen der technischen Leistungsfähigkeit eines Systems und der subjektiven Bewertung seiner Unterstützung klar zu unterscheiden. Ob generative KI als unterstützend erlebt wird, hängt nicht allein von ihren funktionalen Eigenschaften ab, sondern davon, inwieweit Nutzende ihr einen Beitrag zur eigenen Wirksamkeit zuschreiben. Vertrauen in die Zuverlässigkeit und Angemessenheit der KI stellt dabei eine notwendige Voraussetzung für die Zuschreibung von Unterstützungsqualität dar, wird jedoch im vorliegenden Modell nicht als eigenständige Variable geführt. Stattdessen ist Vertrauen implizit in der Wahrnehmung von Unterstützung enthalten und wird theoretisch integriert, ohne die Modellkomplexität zu erhöhen.

Aus Sicht der Self-Determination Theory entfalten unterstützende Arbeitsbedingungen insbesondere dann positive motivationale Effekte, wenn sie zur Befriedigung grundlegender psychologischer Bedürfnisse beitragen. Wahrgenommene Unterstützung signalisiert Individuen, dass ihnen ausreichende Ressourcen zur Verfügung stehen, um Anforderungen erfolgreich zu bewältigen, und stärkt dadurch das Erleben von Kompetenz als Gefühl von Wirksamkeit und Fähigkeit. Demgegenüber zeigen empirische Befunde, dass mangelnde, inkonsistente oder entwertende Unterstützung mit einer Frustration psychologischer Bedürfnisse (Need Thwarting) verbunden ist und negative motivationale sowie affektive Konsequenzen nach sich ziehen kann \parencite{lagiosExplainingNegativeConsequences2022}. Ergänzend verdeutlicht ressourcenorientierte Forschung, dass Unterstützung als arbeitsbezogene Ressource dazu beiträgt, Anforderungen als bewältigbar zu erleben und adaptive Handlungsfähigkeit auch unter Belastung aufrechtzuerhalten \parencite{karanika-murray_health-performance_2020}.

Vor diesem theoretischen Hintergrund ist anzunehmen, dass die Nutzung generativer KI in der Entscheidungsarbeit nicht unmittelbar auf das Kompetenzerleben von Führungskräften wirkt. Vielmehr entfaltet sich dieser Effekt indirekt über die wahrgenommene Unterstützungsqualität der KI, die als vermittelnder Mechanismus zwischen technologischem Einsatz und motivationalem Outcome fungiert. Diese Argumentation bildet die theoretische Grundlage der Mediationshypothese (H2) der vorliegenden Arbeit.