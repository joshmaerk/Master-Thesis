\subsection{Begriffsbestimmung und Abgrenzung des Kompetenzerlebens}
Das Kompetenzerleben bezeichnet die subjektive Wahrnehmung, wirksam und fähig zu sein, Herausforderungen in der eigenen Arbeitsumwelt erfolgreich zu bewältigen. In der Self-Determination Theory wird Kompetenz als grundlegendes psychologisches Bedürfnis verstanden, das sich auf das Erleben von Effektivität und Meisterschaft im Umgang mit Aufgaben und Anforderungen bezieht \parencite{van_den_broeck_capturing_2010}. Dieses Verständnis ist klar von verwandten Konstrukten abzugrenzen. Leistung beschreibt primär objektive oder fremdbeurteilte Ergebnisse, während das Kompetenzerleben eine subjektive Erfahrung darstellt. Selbstwert bezieht sich auf eine globale Bewertung der eigenen Person und ist nicht notwendigerweise auf den Arbeitskontext beschränkt. Selbstwirksamkeit fokussiert auf erwartete Fähigkeiten in Bezug auf spezifische zukünftige Handlungen, während Kompetenzerleben stärker das gegenwärtige affektive Gefühl von Wirksamkeit nach der Ausführung einer Tätigkeit erfasst. Auch gegenüber dem Konzept des Psychological Empowerment ist eine Abgrenzung erforderlich: Während Kompetenz dort eine von mehreren Dimensionen darstellt, wird sie im Rahmen der Self-Determination Theory als eigenständiges motivationales Grundbedürfnis konzeptualisiert \parencite{spreitzerPSYCHOLOGICALEMPOWERMENTWORKPLACE1995}.

\subsection{Kompetenzerleben, Führung und Entscheidungsarbeit}
Im Arbeits- und Führungskontext ist das Kompetenzerleben eng mit Entscheidungsarbeit und wahrgenommener Verantwortung verbunden. Führungskräfte erleben Kompetenz insbesondere dann, wenn sie Entscheidungsprozesse aktiv gestalten, Handlungsspielräume nutzen und Rückmeldungen über die Wirksamkeit ihres Handelns erhalten. Forschung im Bereich Empowering Leadership zeigt, dass die Delegation von Verantwortung, der Zugang zu relevanten Informationen und die Unterstützung eigenständiger Problemlösung das Kompetenzerleben von Mitarbeitenden und Führungskräften gleichermaßen fördern \parencite{spreitzerPSYCHOLOGICALEMPOWERMENTWORKPLACE1995}. Für das mittlere Management ist dieser Zusammenhang besonders relevant, da diese Führungsebene häufig zwischen strategischen Vorgaben und operativer Umsetzung vermittelt und Entscheidungen unter hoher Unsicherheit trifft. Ein hohes Kompetenzerleben kann hier dazu beitragen, Entscheidungsanforderungen als bewältigbar zu erleben und Veränderungsprozesse aktiv mitzugestalten. Empirische Studien zeigen zudem, dass wahrgenommene Kompetenz mit höherer intrinsischer Motivation, Lernbereitschaft und konstruktivem Umgang mit komplexen Entscheidungssituationen verbunden ist \parencite{gagne_multidimensional_2015}.

\subsection{Messbarkeit und Relevanz des Kompetenzerlebens}
Das Kompetenzerleben ist empirisch gut operationalisierbar und stellt damit ein anschlussfähiges Konstrukt für organisationsentwicklungsbezogene Forschung dar. Einen zentralen Messansatz bietet die Work-related Basic Need Satisfaction Scale, die Kompetenz als eigenständige Dimension der psychologischen Grundbedürfnisbefriedigung im Arbeitskontext erfasst und umfangreich validiert wurde \parencite{van_den_broeck_capturing_2010}. Ergänzend existieren etablierte Instrumente aus angrenzenden Forschungssträngen, etwa die Kompetenzdimension des Psychological Empowerment oder motivationsbezogene Skalen wie die Multidimensional Work Motivation Scale, die indirekt mit dem Erleben von Kompetenz verknüpft sind \parencite{gagne_multidimensional_2015,spreitzerPSYCHOLOGICALEMPOWERMENTWORKPLACE1995}. Die Relevanz des Kompetenzerlebens ergibt sich nicht nur aus seiner theoretischen Verankerung in der Self-Determination Theory, sondern auch aus seiner empirisch belegten Bedeutung für Leistung, Wohlbefinden und Veränderungsbereitschaft. Damit eignet sich das Kompetenzerleben als zentrale abhängige Variable, um die motivationalen Effekte organisationaler Gestaltungsmaßnahmen und technologischer Interventionen, wie etwa den Einsatz generativer KI in Entscheidungsprozessen, systematisch zu untersuchen.