% Definition von Organisationsentwicklung als geplanter, systemischer Veränderungsprozess.
% Verbindung individueller, struktureller und kultureller Ebenen.
% Ziel: Legitimation des organisationsentwicklungsbezogenen Rahmens der Arbeit.

Organisationsentwicklung bezeichnet einen geplanten, systematischen und langfristig angelegten Prozess zur gezielten Veränderung von Organisationen, der strukturelle, kulturelle und individuelle Ebenen integriert adressiert \parencite{cummingsOrganizationDevelopmentChange2015,burkeOrganizationChangeTheory2018}. Im Zentrum organisationsentwicklungsbezogener Ansätze steht dabei nicht die isolierte Optimierung einzelner Prozesse, sondern die bewusste Gestaltung organisationaler Rahmenbedingungen (z.\,B. Strukturen, Routinen, Führungs- und Lernprozesse), um nachhaltige Anpassungs- und Entwicklungsfähigkeit zu ermöglichen \parencite{cummingsOrganizationDevelopmentChange2015}. Organisationsentwicklung folgt damit einer sozio-technischen Logik, wonach organisationale Strukturen und Arbeitsprozesse stets mit den Wahrnehmungen, Interaktionen und Deutungsmustern der Organisationsmitglieder verflochten sind; Veränderung entfaltet Wirkung folglich über die wechselseitige Kopplung von Systemgestaltung und menschlichem Erleben \parencite{burkeOrganizationChangeTheory2018,scheinOrganizationalCultureLeadership2017}. Vor diesem Hintergrund stellt Organisationsentwicklung einen geeigneten theoretischen Rahmen dar, um technologische, strukturelle und psychologische Dimensionen organisationaler Veränderung integriert zu analysieren und zu gestalten \parencite{cummingsOrganizationDevelopmentChange2015,burkeOrganizationChangeTheory2018}.


% Abgrenzung zwischen Technologieeinsatz und Organisationsentwicklung.
% Digitale Technologien und KI als Interventionen, die Arbeitsprozesse, Rollen und Entscheidungsstrukturen verändern.
% Ziel: Positionierung von KI als OE-relevantes Phänomen.
Digitale Transformation beschreibt einen tiefgreifenden, organisationsweiten Veränderungsprozess, der über die bloße Einführung neuer Technologien hinausgeht und etablierte Arbeitsprozesse, Rollenbilder, Entscheidungsstrukturen sowie kulturelle Deutungsmuster nachhaltig verändert \parencite{vialUnderstandingDigitalTransformation2019}. Aus organisationsentwicklungsbezogener Perspektive stellt der Einsatz digitaler Technologien – insbesondere datenbasierter und KI-gestützter Systeme – daher keinen rein technischen Implementierungsschritt dar, sondern einen gezielten Eingriff in das sozio-technische System der Organisation \parencite{cummingsOrganizationDevelopmentChange2015,burkeOrganizationChangeTheory2018}. Digitale Transformation beeinflusst, wie Arbeit gestaltet wird, wie Entscheidungen vorbereitet und getroffen werden und wie Verantwortung zwischen Mensch und Technologie verteilt ist \parencite{bankinsMultilevelReviewArtificial2024}. Diese Veränderungen entfalten ihre Wirkung nicht allein über technische Leistungsmerkmale, sondern maßgeblich über subjektive Wahrnehmungen, Interpretationen und Akzeptanzprozesse der betroffenen Akteure \parencite{quaquebeke_now_2023}. Vor diesem Hintergrund ist digitale Transformation als organisationsentwicklungsrelevanter Eingriff zu verstehen, dessen Erfolg wesentlich davon abhängt, inwieweit technologische Neuerungen mit psychologischen Bedürfnissen, bestehenden Arbeitslogiken und organisationalen Entwicklungszielen in Einklang gebracht werden.


% Mittleres Management als Übersetzungs-, Implementierungs- und Sinnstiftungsebene.
% Relevanz für Wahrnehmung und Nutzung neuer Technologien.
% Ziel: Begründung der Fokusgruppe der Untersuchung.

 Das mittlere Management nimmt in organisationsentwicklungsbezogenen Veränderungsprozessen eine zentrale Rolle ein, da es als verbindende Ebene zwischen strategischer Entscheidung und operativer Umsetzung fungiert \parencite{floydManagingStrategicConsensus1997}. Führungskräfte dieser Ebene sind maßgeblich daran beteiligt, organisationale Veränderungen zu interpretieren, in bestehende Arbeitskontexte zu übersetzen und gegenüber Mitarbeitenden zu legitimieren. Gerade im Kontext digitaler Transformation und des Einsatzes generativer KI kommt dem mittleren Management eine besondere Bedeutung zu, da technologische Veränderungen häufig unmittelbar in ihre Entscheidungs-, Koordinations- und Analyseprozesse eingreifen \parencite{bankinsMultilevelReviewArtificial2024}. Gleichzeitig prägen Führungskräfte des mittleren Managements durch ihr eigenes Nutzungsverhalten, ihre Haltung gegenüber neuen Technologien und ihre kommunikative Rahmung maßgeblich, wie digitale Systeme organisational wahrgenommen und akzeptiert werden \parencite{quaquebeke_now_2023}. Aus organisationsentwicklungsbezogener Perspektive stellt das mittlere Management somit eine Schlüsselgruppe dar, über die sich entscheidet, ob digitale Technologien – wie generative KI – als unterstützende Ressource oder als kontrollierender Eingriff in bestehende professionelle Handlungsspielräume erlebt werden.