\subsection{Grundannahmen der Self-Determination Theory} 
Die Self-Determination Theory (SDT) ist eine umfassende Theorie der menschlichen Motivation, die Motivation als qualitativ differenziertes Konstrukt betrachtet \parencite{deciWhatWhyGoal2000}. Zentral ist die Unterscheidung zwischen autonomer Motivation (d.h. intrinsischer und vollständig internalisierter extrinsischer Motivation) und kontrollierter Motivation (durch äußeren Druck oder introjizierte Verpflichtungsgefühle geprägte Motivation), welche unterschiedliche Auswirkungen auf Leistung und Wohlbefinden haben \parencite{deci_self-determination_2017}. Ferner postuliert SDT, dass alle Menschen über drei grundlegende psychologische Bedürfnisse – Kompetenz, Autonomie und soziale Eingebundenheit – verfügen, deren Befriedigung die Entfaltung autonomer Motivation begünstigt und zugleich das Wohlbefinden fördert \parencite{deciWhatWhyGoal2000, van_den_broeck_review_2016}. SDT liefert somit einen theoretischen Bezugsrahmen, um Motivationsprozesse und deren Einfluss auf positive Outcomes zu erklären, und bildet die Grundlage für die vorliegende Arbeit \parencite{deci_self-determination_2017}.

\subsection{Psychologische Grundbedürfnisse bei der Arbeit}
Auch im Arbeitskontext spielen die drei Grundbedürfnisse der SDT eine zentrale Rolle. Autonomie (Erleben von Selbstbestimmung), Kompetenz (Gefühl von Wirksamkeit) und soziale Eingebundenheit (Gefühl von Zugehörigkeit) gelten als essentielle Voraussetzungen für Motivation und Wohlbefinden von Beschäftigten \parencite{vandenbroeckReviewSelfDeterminationTheorys2016}. Organisationen stellen dabei wichtige „Bedürfnisumwelten“ dar, da Arbeitsbedingungen und Führungsverhalten die Befriedigung dieser Bedürfnisse entweder unterstützen oder beeinträchtigen können \parencite{deci_self-determination_2017}. Empirisch zeigt sich, dass eine hohe Bedürfnisbefriedigung am Arbeitsplatz mit positiven Konsequenzen wie intrinsischer Arbeitsmotivation, Leistungsfähigkeit, Engagement und psychischer Gesundheit einhergeht, wohingegen eine Frustration der Bedürfnisse mit negativen Folgen wie Burnout und Unzufriedenheit verbunden ist \parencite{vandenbroeckReviewSelfDeterminationTheorys2016, deci_self-determination_2017}. Entsprechend wird SDT in der Organisations- und Personalpraxis verstärkt herangezogen, um Arbeitsumfelder und HR-Praktiken bedürfnisförderlich zu gestalten \parencite{laguerre_bringing_2025}.

\paragraph{Führung als Kontextbedingung der Bedürfnisbefriedigung}
Aktuelle Forschung im Bereich Organizational Behavior verdeutlicht, dass die Befriedigung psychologischer Grundbedürfnisse im Sinne der Self-Determination Theory nicht unabhängig vom sozialen und organisatorischen Kontext erfolgt, sondern maßgeblich durch Führungshandeln geprägt wird. Zhang et \parencite[]{zhang_how_2024} zeigen in einer mehrstufigen Studie, dass insbesondere ein humble Leadership-Stil – gekennzeichnet durch Lernorientierung, Offenheit für Feedback und die Anerkennung eigener Grenzen – die Befriedigung der Bedürfnisse nach Autonomie und Kompetenz fördert \parencite{zhang_how_2024}. Führungskräfte fungieren damit als zentrale Kontextakteure, die Arbeitsbedingungen so rahmen, dass Mitarbeitende ihre Fähigkeiten wirksam einsetzen und weiterentwickeln können. Aus SDT-Perspektive wird Kompetenz dabei nicht allein durch Aufgabenanforderungen oder individuelle Fähigkeiten bestimmt, sondern durch das erlebte Zusammenspiel von Herausforderung, Unterstützung und Rückmeldung. Die Ergebnisse von Zhang et al. verdeutlichen, dass Führungshandeln eine vermittelnde Rolle zwischen strukturellen Arbeitsbedingungen und motivationalen Outcomes einnimmt, indem es bestimmt, ob Arbeitsanforderungen als entwicklungsfördernd oder überfordernd wahrgenommen werden. Für den vorliegenden Forschungskontext impliziert dies, dass auch technologische Arbeitsmittel wie generative KI nicht isoliert wirken, sondern ihre motivationalen Effekte im Zusammenspiel mit Führung und organisationaler Sinngebung entfalten.

\subsection{SDT und technologische Arbeitsgestaltung} 
SDT ist dabei nicht als statisches Modell zu verstehen, sondern als Analyseinstrument, um neue Arbeitsformen und Technologien hinsichtlich ihrer motivationalen Qualität zu bewerten. Digitale Systeme (einschließlich KI) können so gestaltet und erlebt werden, dass sie psychologische Grundbedürfnisse unterstützen (need-supportive), etwa durch Autonomie- und Kompetenzunterstützung, oder sie können Bedürfnisse unterminieren (need-thwarting), z.B. durch Fremdsteuerung, Intransparenz oder Entwertung von Expertise \parencite{gagne_understanding_2022,mcanally_self-determination_2024}.

\subsection{Kompetenzerleben als zentraler Fokus der vorliegenden Arbeit}
Das Bedürfnis nach Kompetenz beschreibt das grundlegende Bestreben, sich fähig und wirksam in der Auseinandersetzung mit der Umwelt zu fühlen und neue Fertigkeiten zu entwickeln \parencite{vandenbroeckReviewSelfDeterminationTheorys2016}. In der vorliegenden Arbeit steht das Kompetenzerleben daher im Mittelpunkt, da es gerade in Führungsaufgaben, bei komplexen Entscheidungsprozessen und während organisationaler Veränderungsprozesse ein Schlüsselfaktor für erfolgreiche Anpassung und Leistungsbereitschaft ist. Mitarbeiter können neue Herausforderungen nur dann engagiert und konstruktiv bewältigen, wenn ihr Kompetenzbedürfnis erfüllt ist und sie Vertrauen in die eigene Wirksamkeit haben \parencite{deci_self-determination_2017}. Beispielsweise reagieren Beschäftigte in Veränderungsphasen nachweislich aufgeschlossener, wenn ihre Vorgesetzten einen unterstützenden Führungsstil pflegen, der unter anderem das Kompetenzgefühl der Mitarbeiter stärkt \parencite{deci_self-determination_2017}. Die Konzentration auf das Kompetenzerleben als abhängige Variable der Untersuchung ist somit theoretisch gut begründet, da diesem Bedürfnis im Rahmen der SDT eine zentrale Bedeutung für Motivation und Wohlbefinden zukommt \parencite{deciWhatWhyGoal2000}.

\subsection{Self-Determination Theory als Bezugsrahmen für digitale Arbeitskontexte}
Neuere konzeptionelle Arbeiten unterstreichen die besondere Eignung der Self-Determination Theory zur Analyse motivationaler Effekte in zunehmend digitalisierten Arbeitsumgebungen. McAnally und Hagger (2024) argumentieren, dass technologische Entwicklungen wie algorithmische Systeme und KI-basierte Arbeitsmittel nicht primär über Effizienz- oder Akzeptanzmodelle erklärt werden sollten, sondern über ihre Wirkung auf psychologische Grundbedürfnisse \parencite{mcanallySelfDeterminationTheoryWorkplace2024}. Insbesondere betonen die Autoren, dass digitale Technologien sowohl autonomie- und kompetenzförderlich als auch kontrollierend und bedürfnisfrustrierend wirken können, abhängig von ihrer Gestaltung und organisationalen Einbettung. 

Vor diesem Hintergrund ermöglicht SDT eine differenzierte Analyse technologischer Interventionen, da sie nicht von technologischem Determinismus ausgeht, sondern den Fokus auf subjektive Wahrnehmungen, Sinnzuschreibungen und motivational relevante Wirkmechanismen legt. McAnally und Hagger (2024) heben hervor, dass organisationale Praktiken und Technologien dann nachhaltige positive Effekte entfalten, wenn sie als unterstützend erlebt werden und die Selbstregulation der Beschäftigten stärken. Diese Perspektive ist für die vorliegende Arbeit zentral, da sie begründet, warum die Untersuchung generativer KI nicht auf Nutzungsintensität oder Akzeptanz beschränkt bleibt, sondern das Erleben von Kompetenz als psychologischen Wirkmechanismus in den Mittelpunkt stellt.