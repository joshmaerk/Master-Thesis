\section{Synthese: Motivationale Wirkung von KI in der Führungsarbeit}

Die vorangegangenen Abschnitte haben drei Perspektiven entfaltet, die nun zusammengeführt werden: generative KI als soziotechnisches System mit spezifischer Eingriffstiefe in wissensintensive Arbeit (Abschnitt 2.1), Führungsarbeit im mittleren Management des Bankensektors mit ihrer charakteristischen Verknüpfung von Entscheidungskompetenz, Expertiseidentität und regulatorischer Einbettung (Abschnitt 2.2) sowie die Self-Determination Theory als motivationspsychologischer Erklärungsrahmen (Abschnitt 2.3). Der vorliegende Abschnitt integriert diese Stränge zu einem konzeptionellen Analyserahmen, identifiziert die zentrale Forschungslücke und formuliert die Fragestellung der Arbeit.

\subsection{Generative KI und Autonomieerleben.}

Das Autonomiebedürfnis mittlerer Führungskräfte wird durch generative KI auf mindestens zwei Wegen berührt. Erstens kann KI als Ermöglicherin von Autonomie wirken: Sie entlastet von Routineaufgaben, beschleunigt Informationssynthesen und erweitert den Handlungsspielraum für strategische Gestaltung. Zweitens kann sie als subtiles Kontrollsystem erfahren werden -- dann nämlich, wenn algorithmische Empfehlungen faktisch den Entscheidungskorridor verengen, wenn Transparenzanforderungen die Begründungslast erhöhen oder wenn die Organisation KI-gestützte Prozesse als Steuerungsinstrument implementiert.

Die empirische Forschung stützt diese Ambivalenz. \textcite{edwards_managerial_2024} zeigten in einer Drei-Wellen-Studie (N\,=\,401), dass die motivationalen Konsequenzen algorithmischer HR-Systeme davon abhängen, wie Beschäftigte deren Zweck attribuieren: Wer das System als Kontrollinstrument der Führung wahrnahm, erlebte erhöhte extrinsische Motivation und priorisierte selektiv messbare Aufgaben. Wer es als Feedback-Instrument deutete, berichtete höhere intrinsische Motivation und geringere emotionale Erschöpfung. Die Attributionsrichtung -- Kontrolle oder Unterstützung -- erwies sich als entscheidender als das System selbst. \textcite{klonek_does_2025} bestätigten dieses Muster in einer Text-Mining-Studie mit über 2.700 Beiträgen: Hohe KI-Kontrolle war signifikant mit erhöhtem Stress assoziiert, während Mensch-KI-Teamprozesse -- insbesondere aktions- und interpersonenbezogene -- Stress reduzierten und den negativen Effekt der Kontrolle abpufferten.

Für mittlere Führungskräfte im Bankensektor verschärft sich diese Dynamik durch die regulatorische Struktur. Compliance-Anforderungen erzeugen bereits ohne KI enge Entscheidungskorridore; GenAI-Systeme, die regulatorisch gebotene Analysen automatisieren, können die erlebte Autonomie weiter einschränken, selbst wenn sie objektiv Entlastung bieten. Die Frage ist nicht, ob Autonomie tangiert wird, sondern wie Führungskräfte die Verschiebung erleben und deuten.

\subsection{Generative KI und Kompetenzerleben.}

Das Kompetenzerleben mittlerer Führungskräfte ist im Kontext generativer KI besonders verwundbar. Im Unterschied zu klassischen Automatisierungstechnologien, die primär manuelle oder standardisierte Aufgaben betreffen, greift GenAI in Kernbereiche professioneller Wissensarbeit ein: Analysen erstellen, Strategien entwerfen, Kommunikation gestalten \parencite{brynjolfssonGenerativeAIWork2025}. Das sind genau die Tätigkeiten, über die mittlere Führungskräfte im Bankensektor traditionell ihre professionelle Identität definieren.

Die experimentelle Studie von \textcite{WuLiuRuan2025} (N\,=\,15.105) hat den Mechanismus aufgedeckt: GenAI-Unterstützung steigerte die Aufgabenleistung, reduzierte aber gleichzeitig die intrinsische Motivation -- weil Teilnehmende den Erfolg der KI statt sich selbst attribuierten. Kompetenzerleben wurde nicht durch Scheitern frustriert, sondern durch den Verlust der Erfolgsattribution. Für Führungskräfte, deren Wirksamkeitserleben eng an persönliche Expertise geknüpft ist, dürfte dieser Mechanismus besonders ausgeprägt sein.

Gleichzeitig eröffnet GenAI neue Kompetenzdomänen. Die Fähigkeit, effektive Prompts zu formulieren, KI-Outputs kritisch zu evaluieren und Mensch-KI-Zusammenarbeit zu gestalten, wird zunehmend als eigenständige professionelle Kompetenz verstanden \parencite{dellacquaNavigatingJaggedTechnological2023}. \textcite{smith_navigating_2025} zeigten in vier experimentellen Studien (N\,$\approx$\,1.100), dass die freiwillige Einholung von KI-Empfehlungen -- im Vergleich zur erzwungenen Konfrontation -- die Bereitschaft signifikant erhöhte, KI-Ratschläge zu akzeptieren. Die Wahlfreiheit erzeugte ein Erleben von Kompetenz und Kontrolle über die Interaktion, das die SDT-Logik direkt bestätigte: Optionale KI-Nutzung stützt sowohl Autonomie- als auch Kompetenzerleben, obligatorische untergräbt beides.

Das resultierende Spannungsfeld ist für den Bankensektor besonders ausgeprägt. Die Entscheidungsarbeit mittlerer Führungskräfte verbindet dort analytische Kompetenz mit kontextuellem Erfahrungswissen -- etwa bei Kreditentscheidungen, in denen formale Risikomodelle und langjährige Kundenkenntnis zusammenfließen. Wenn GenAI die analytische Komponente substituiert, stellt sich die Frage, wie das verbleibende Erfahrungswissen erlebt wird: als aufgewertete Kernkompetenz oder als Residualkategorie.

\subsection{Generative KI und soziale Eingebundenheit.}

Die Auswirkungen generativer KI auf soziale Eingebundenheit sind bislang am wenigsten empirisch untersucht, theoretisch aber plausibel. Drei Wirkpfade lassen sich antizipieren.

Erstens verändert GenAI die Interaktionsstruktur in Teams. Wenn Aufgaben, die bisher kooperativ gelöst wurden -- Berichterstellung, Datenanalyse, Entscheidungsvorbereitung -- nun einzeln an KI-Systeme delegiert werden, reduziert sich die aufgabenbezogene Interdependenz. \textcite{klonek_does_2025} argumentierten, dass gerade interpersonale Mensch-KI-Teamprozesse als soziale Ressource wirken, die Stress reduziert. Fehlen diese Prozesse, entfällt auch der Puffer.

Zweitens kann sich das Verhältnis zur eigenen Führungsrolle verschieben. \textcite{quaquebeke_now_2023} argumentierten, dass Führung in KI-Kontexten zunehmend weniger über Informationsvorsprung und Expertise definiert wird und stärker über Beziehungsqualität und Sinngebung. Das birgt Chancen -- eine Intensivierung genuin menschlicher Führungsarbeit --, setzt aber voraus, dass Führungskräfte diese Rollenverschiebung nicht als Kompetenzverlust, sondern als Rollenerweiterung erleben.

Drittens berührt KI die wahrgenommene organisationale Wertschätzung. \textcite{lagios_explaining_2022} zeigten in einer Zwei-Wellen-Studie (N\,=\,603), dass organisationale Dehumanisierung -- das Erleben, als bloßes Werkzeug behandelt zu werden -- alle drei psychologischen Grundbedürfnisse frustriert, mit negativen Konsequenzen für Wohlbefinden, Zufriedenheit und Bindung. Wenn Organisationen GenAI primär als Effizienzinstrument implementieren, ohne die Perspektive der Führungskräfte einzubeziehen, besteht das Risiko, dass sich deren Wertschätzungserleben verschlechtert -- nicht weil die Technologie inhärent dehumanisierend wirkt, sondern weil der Implementierungsprozess es ist.

\subsection{Das Paradox der Entlastung.}

Die Synthese der empirischen Befunde offenbart ein zentrales Paradox. Generative KI verspricht Führungskräften Entlastung von Routineaufgaben und informationeller Überflutung -- eine Ressource im Sinne der JD-R-Theorie, die Demands reduziert und Handlungsspielräume erweitert \parencite{bakker_job_2007}. Gleichzeitig greift sie in genau jene Tätigkeiten ein, über die Führungskräfte Kompetenz, Autonomie und Identität definieren. Die Entlastung kann zum Kontrollverlust werden, die Effizienzsteigerung zur Expertiseentwertung, die Prozessoptimierung zur Beziehungsverarmung.

Ob das eine oder das andere eintritt, ist -- das legen die Befunde nahe -- kein Determinismus der Technologie. \textcite{tongJanusFaceArtificial2021} beschrieben KI treffend als Janusgesicht: bedrohlich und vielversprechend zugleich. Die motivationale Wirkung hängt von einer Konfiguration aus individuellen, organisationalen und technologischen Faktoren ab \parencite{bankinsMultilevelReviewArtificial2024}. SDT bietet einen Erklärungsrahmen, der über die binäre Logik von Akzeptanz oder Ablehnung hinausgeht, indem sie die Mechanismen offenlegt: Nicht die Technologienutzung per se, sondern deren Wirkung auf Autonomie, Kompetenz und soziale Eingebundenheit bestimmt, ob GenAI als motivationale Ressource oder als Belastung erlebt wird.

\subsection{Forschungslücke und Fragestellung.}

Trotz der wachsenden Literatur zu KI im Arbeitskontext bestehen substanzielle Lücken, die den Ausgangspunkt dieser Arbeit bilden.

Die bestehende empirische Forschung ist überwiegend quantitativ und erfasst breit angelegte Zusammenhänge zwischen KI-Nutzung und motivationalen oder leistungsbezogenen Outcomes -- typischerweise über standardisierte Fragebögen, die aggregierte Bewertungen abbilden \parencite{WuLiuRuan2025, prasad_generative_2024, YangHsiHsun2024}. Was diese Studien konzeptionell nicht leisten können, ist die Rekonstruktion der subjektiven Erfahrungswelt: Wie erleben Führungskräfte den konkreten Moment, in dem ein KI-System eine Analyse erstellt, die sie bisher selbst verantwortet haben? Welche Deutungsmuster mobilisieren sie, um ihre Rolle im Verhältnis zur Technologie zu verorten? Wie verschiebt sich ihr Erleben über die Zeit?

Zudem fokussiert die Mehrzahl der Studien auf operative Beschäftigte oder breite Populationen -- nicht auf mittleres Management, dessen spezifische Positionierung zwischen strategischer Planung und operativer Umsetzung eine besondere Betroffenheit durch GenAI begründet \parencite{klonekDoesAIWork2025}. Die wenigen Studien, die Führungskräfte explizit adressieren, untersuchen deren Umgang mit KI in der Teamsteuerung \parencite{monod_worker_2024}, nicht deren eigenes motivationales Erleben als KI-Nutzer.

Schließlich fehlt eine sektorbezogene Perspektive für den DACH-Raum. Der Bankensektor mit seiner Kombination aus hoher Regulierungsdichte, wissensintensiver Entscheidungsarbeit und ambivalenter Digitalisierungsgeschichte (vgl. Abschnitt 2.1.3) stellt spezifische Kontextbedingungen bereit, die in der bisherigen Forschung nicht abgebildet sind.

Die vorliegende Arbeit adressiert diese Lücken mit folgender Forschungsfrage:

\begin{quote}
\textit{Wie erleben mittlere Führungskräfte im Bankensektor (DACH-Region) die Auswirkungen generativer KI-Tools auf ihre psychologischen Grundbedürfnisse nach Autonomie, Kompetenz und sozialer Eingebundenheit gemäß der Self-Determination Theory?}
\end{quote}

Der qualitative Zugang über problemzentrierte Leitfadeninterviews ermöglicht es, die subjektive Erfahrungswelt der Betroffenen zu rekonstruieren, Deutungsmuster zu identifizieren und die Bedingungskonfigurationen herauszuarbeiten, unter denen generative KI bedürfnisunterstützend oder -frustrierend wirkt. Die SDT-Grundbedürfnisse dienen dabei als deduktive Analyselinse, die durch induktive Subkategorien ergänzt wird (vgl. Kapitel 3).