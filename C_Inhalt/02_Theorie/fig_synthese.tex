% --- Pfeil-Styles (orthogonal, ruhig) ---
\tikzset{
  arrow/.style={-{Latex[length=3mm]}, thick, rounded corners=2mm},
  dashedarrow/.style={-{Latex[length=3mm]}, thick, dashed, rounded corners=2mm},
  hlabel/.style={font=\footnotesize, fill=white, inner sep=1.5pt}
}

% --- Layout-Parameter ---
\newlength{\Pad}     % inner sep
\newlength{\GutterX} % horizontaler Abstand zwischen Spalten
\newlength{\GapY}    % vertikaler Abstand zwischen SDT-Boxen

\newlength{\Wcol}     % Node-Breite je Spalte (inkl. Pad)
\newlength{\WcolTxt}  % Textbreite je Spalte (ohne Pad)

\newlength{\Hmid}      % Gesamthöhe der mittleren Zone (inkl. Pad)
\newlength{\HmidTxt}   % Texthöhe für Midbox (ohne Pad)

\newlength{\Hsmall}    % Node-Höhe SDT-Teilbox (inkl. Pad)
\newlength{\HsmallTxt} % Texthöhe SDT-Teilbox (ohne Pad)

\setlength{\Pad}{6pt}
\setlength{\GutterX}{6mm}
\setlength{\GapY}{3mm}

% Höhe der Zeile mit GenAI/Leadership/SDT (SDT-Stack hat gleiche Gesamthöhe)
\setlength{\Hmid}{7cm}

% 3 Spalten + 2*Gutter = \textwidth
\setlength{\Wcol}{\dimexpr(\textwidth - 2\GutterX)/3\relax}
\setlength{\WcolTxt}{\dimexpr\Wcol - 2\Pad\relax}

% SDT: 3 Boxen + 2*GapY = Hmid
\setlength{\Hsmall}{\dimexpr(\Hmid - 2\GapY)/3\relax}

% Text-/Parbox-Höhen (ohne Pad)
\setlength{\HmidTxt}{\dimexpr\Hmid - 2\Pad\relax}
\setlength{\HsmallTxt}{\dimexpr\Hsmall - 2\Pad\relax}

% --- Box-Styles ---
\tikzset{
  base/.style={
    draw, rectangle, rounded corners=1pt,
    align=left, inner sep=\Pad, outer sep=0pt
  },
  fullwidth/.style={
    base,
    minimum width=\textwidth,
    text width=\dimexpr\textwidth - 2\Pad\relax,
    minimum height=2.2cm
  },
  midbox/.style={
    base,
    minimum width=\Wcol,
    text width=\WcolTxt,
    minimum height=\Hmid
  },
  smallbox/.style={
    base,
    minimum width=\Wcol,
    text width=\WcolTxt,
    minimum height=\Hsmall
  },
}

\begin{figure}[t]
\centering
\begin{tikzpicture}[x=1pt,y=1pt]

  % --- TOP (100%) ---
  \node[fullwidth, anchor=north west] (top) at (0,0)
  {\raggedright\sloppy
   \textbf{Kontextfaktoren (moderierend)}:\\
   Sensemaking, Implementierungsmodus, Freiwilligkeit, Teamprozesse};

  % --- 3-SPALTEN-ZEILE (GenAI | Leadership | SDT) ---
  \coordinate (midNW) at ([yshift=-12mm]top.south west);

  % Spalte 1: GenAI
  \node[midbox, anchor=north west] (genai)
    at (midNW)
  {\raggedright\sloppy
   \parbox[t][\HmidTxt][t]{\WcolTxt}{%
     \textbf{GenAI als soziotechnisches System}\\[0.3em]
     -- Aufgabenautomatisierung/\allowbreak-augmentation\\
     -- Algorithmische Kontrolle/\allowbreak Feedback\\
     -- Mensch--KI-Interaktion
   }};

  % Spalte 2: Leadership
  \node[midbox, anchor=north west] (leadership)
    at ([xshift=\GutterX]genai.north east)
  {\raggedright\sloppy
   \parbox[t][\HmidTxt][t]{\WcolTxt}{%
     \textbf{Führungsarbeit im Bankensektor}\\[0.3em]
     -- Entscheidungsvorbereitung\\
     -- Regulatorische Einbettung\\
     -- Rollen- und Expertiseanforderungen
   }};

  % Spalte 3: SDT (Stack)
  \node[smallbox, anchor=north west] (autonomy)
    at ([xshift=\GutterX]leadership.north east)
  {\raggedright\sloppy
   \parbox[t][\HsmallTxt][t]{\WcolTxt}{%
     \textbf{Autonomie}\\
     Volition, Entscheidungsspielraum
   }};

  \node[smallbox, anchor=north west] (competence)
    at ([yshift=-\GapY]autonomy.south west)
  {\raggedright\sloppy
   \parbox[t][\HsmallTxt][t]{\WcolTxt}{%
     \textbf{Kompetenz}\\
     Wirksamkeit, Expertiseerleben
   }};

  \node[smallbox, anchor=north west] (relatedness)
    at ([yshift=-\GapY]competence.south west)
  {\raggedright\sloppy
   \parbox[t][\HsmallTxt][t]{\WcolTxt}{%
     \textbf{Soziale Eingebundenheit}\\
     Verbundenheit, Wertschätzung
   }};

  % --- BOTTOM (Motivationale Wirkung) ---
  \node[fullwidth, anchor=north west] (bottom)
    at ([yshift=-12mm]genai.south west)
  {\raggedright\sloppy
   \textbf{Motivationale Wirkung}\\
   Bedürfnisunterstützend $\rightarrow$ Engagement, Wohlbefinden\\
   Bedürfnisfrustrierend $\rightarrow$ Erschöpfung, Rückzug};

  % ===== Routing-Koordinaten =====
  \path (autonomy.south) -- (competence.north) coordinate[midway] (sdtmid);
  \coordinate (busX12) at ($(genai.east)+(0.5\GutterX,0)$);         % zwischen GenAI und Leadership
  \coordinate (busX23) at ($(leadership.east)+(0.5\GutterX,0)$);    % zwischen Leadership und SDT
  \coordinate (bottommid) at ([yshift=6mm]bottom.north);

  % ===== Pfeile (orthogonal, Top-Journal) =====
  % H1: GenAI -> Leadership
  \draw[arrow]
    ([xshift=2mm]genai.east) -| (busX12 |- genai.east)
    node[hlabel, near end, above] {H1}
    -- ([xshift=-2mm]leadership.west);

  % H2: Leadership -> SDT
  \draw[arrow]
    ([xshift=2mm]leadership.east) -| (busX23 |- sdtmid)
    node[hlabel, near end, above] {H2}
    -- ([xshift=-2mm]sdtmid);

  % H3: GenAI -> SDT (direkter Pfad, getrennt geroutet)
  \draw[arrow]
    ([xshift=2mm]genai.east) -| (busX23 |- sdtmid)
    node[hlabel, pos=0.75, above] {H3}
    -- ([xshift=-2mm]sdtmid);

  % H4: SDT -> Motivationale Wirkung (gebündelt)
  \draw[arrow] (autonomy.south)   |- (bottommid);
  \draw[arrow] (competence.south) |- (bottommid);
  \draw[arrow] (relatedness.south) |- (bottommid);
  \draw[arrow] (bottommid) -- node[hlabel, right] {H4} (bottom.north);

  % H5: Moderation (Top) -> Eingänge (gestrichelt)
  \draw[dashedarrow] (top.south) |- node[hlabel, right] {H5} ([yshift=2mm]genai.north);
  \draw[dashedarrow] (top.south) |- ([yshift=2mm]leadership.north);
  \draw[dashedarrow] (top.south) |- ([yshift=2mm]autonomy.north);

\end{tikzpicture}

\caption{Integratives konzeptionelles Rahmenmodell zur Verknüpfung von GenAI als soziotechnischem System, bankenspezifischer Führungsarbeit und SDT-Grundbedürfnissen. Eigene Darstellung in Anlehnung an SDT \parencite{deciWhatWhyGoal2000,deciSelfDeterminationTheoryWork2017,vandenbroeckReviewSelfDeterminationTheorys2016} sowie KI/Algorithmus- und Führungs-/OB-Perspektiven \parencite{bankinsMultilevelReviewArtificial2024,edwardsManagerialControlFeedback2024,tongJanusFaceArtificial2021,quaquebekeNowNewNext2023,LiuWangLin2023}.}
\label{fig:integratives-rahmenmodell}
\label{fig:top-journal-model-3col}
\end{figure}