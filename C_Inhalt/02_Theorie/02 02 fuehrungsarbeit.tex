\section{Führungsarbeit im mittleren Management}
\label{sec:fuehrung}

Mittlere Führungskräfte stehen organisational an einer Stelle, an der strategische Absicht und operative Wirklichkeit aufeinandertreffen. Ihre Arbeit ist geprägt von Entscheidungen unter Unsicherheit, der Vermittlung zwischen Hierarchieebenen und einem Tätigkeitsprofil, das generative \gls{ki} in besonderer Weise berührt. Der folgende Abschnitt beschreibt diese Arbeitssituation -- zunächst allgemein, dann mit Blick auf die spezifischen Bedingungen des Bankensektors.

\subsection{Entscheidungsarbeit als Kernaufgabe}

Mittlere Führungskräfte operieren an einer organisationalen Nahtstelle: zwischen strategischer Planung des Top-Managements und operativer Ausführung durch Frontline-Mitarbeitende. Diese Position ist weniger komfortabel, als es die Organigramme vermuten lassen. Wer hier arbeitet, muss strategische Vorgaben in operative Handlungen übersetzen, gleichzeitig aber Rückmeldungen und Widerstände der operativen Ebene nach oben kommunizieren -- oft unter Zeitdruck, mit unvollständigen Informationen und widersprüchlichen Erwartungen \parencite{floydManagingStrategicConsensus1997}.

\textcite{floydManagingStrategicConsensus1997} identifizierten vier strategische Rollen, die mittlere Führungskräfte einnehmen: \textit{Championing strategic alternatives} -- das Einbringen innovativer Ideen in strategische Entscheidungsprozesse; \textit{Synthesizing information} -- die Aggregation und Interpretation von Informationen aus verschiedenen organisationalen Quellen; \textit{Facilitating adaptability} -- die Förderung organisationaler Anpassungsfähigkeit; sowie \textit{Implementing deliberate strategy} -- die Umsetzung strategischer Entscheidungen in operative Praxis. Alle vier Rollen haben eines gemeinsam: Sie erfordern Entscheidungen. Nicht die großen, strategischen Richtungsentscheidungen, die dem Top-Management vorbehalten sind, sondern die unzähligen kleineren Urteile darüber, wie Strategie im Alltag konkret wird -- welche Prioritäten gesetzt, welche Informationen weitergegeben, welche Interpretationsspielräume genutzt werden.

Entscheidungsarbeit im mittleren Management ist dabei selten algorithmisch im Sinne klarer Wenn-dann-Regeln. Sie ist vielmehr geprägt von \textit{Sensemaking}: dem Versuch, aus mehrdeutigen Situationen tragfähige Handlungsgrundlagen abzuleiten \parencite{quaquebekeNowNewNext2023}. Mittlere Führungskräfte interpretieren strategische Vorgaben, kontextualisieren sie für ihre Teams, antizipieren Widerstände und justieren ihre Kommunikation entsprechend. Dieses Sensemaking ist nicht bloß ein kognitiver Prozess; es ist auch ein sozialer und emotionaler. Wer Transformation vermitteln soll, muss selbst verstanden haben, was sich verändert und warum -- und muss gleichzeitig mit der eigenen Unsicherheit umgehen können.

Genau hier wird die Einführung generativer KI relevant. Wenn ein Großteil der Entscheidungsarbeit mittlerer Führungskräfte darin besteht, Informationen zu synthetisieren, Optionen abzuwägen und Handlungsempfehlungen zu formulieren, dann adressiert \gls{genai} den Kern ihres Tätigkeitsprofils. \acrshort{ki}-Systeme können Daten schneller aggregieren, Entscheidungsvorlagen erstellen und Szenarien durchspielen. Ob diese Fähigkeit als Unterstützung oder als Bedrohung erlebt wird, hängt davon ab, wie Führungskräfte ihre eigene Rolle definieren -- und ob sie ihre professionelle Identität an den Prozess der Informationsverarbeitung oder an die Qualität des Urteils knüpfen \parencite{quaquebeke_now_2023}.

\textcite{quaquebeke_now_2023} argumentieren, dass \gls{ki} die Natur von Führung fundamental verschieben wird: weg von wissensbasierter Autorität hin zu facilitativer, emotional-intelligenter Führung. Führungskräfte müssen ihre Rolle neu verhandeln -- nicht als allwissende Experten, sondern als Kuratoren und Orchestratoren, die menschliche und algorithmische Ressourcen zusammenführen. Diese Neuverhandlung berührt alle drei psychologischen Grundbedürfnisse der Self-Determination Theory: Autonomie (Wer entscheidet -- Mensch oder Maschine?), Kompetenz (Wessen Expertise zählt noch?) und soziale Eingebundenheit (Wie verändert sich die Beziehung zum Team, wenn \gls{ki} Kommunikation mediiert?).

Hinzu kommt, dass mittlere Führungskräfte in Technologietransformationen eine paradoxe Doppelrolle einnehmen. Sie sollen als Change Agents die Adoption vorantreiben und gleichzeitig sind sie selbst Betroffene, deren Tätigkeitsprofile, Kompetenzen und professionelle Identität durch die neuen Tools transformiert werden \parencite{quaquebeke_now_2023}. Im Kontext generativer \gls{ki} verschärft sich dieses Paradox: Wer die Technologie in seinem Team implementieren soll, muss sie zunächst selbst in die eigene Arbeitsweise integrieren -- mit allen damit verbundenen Unsicherheiten über Verlässlichkeit, Grenzen und langfristige Konsequenzen für die eigene Position.

\textcite{koponen_work_2025} identifizierten durch eine systematische Literaturanalyse zentrale Arbeitscharakteristika, die mittlere Führungskräfte in \acrshort{ki}-integrierten Teams benötigen: Autonomie bei der Gestaltung von Mensch-\acrshort{ki}-Interaktionen, transparente Feedback-Mechanismen für menschliche und algorithmische Leistung, eine Balance zwischen Routine- und strategischen Aufgaben sowie soziale Unterstützung durch Peers und Vorgesetzte bei \acrshort{ki}-bezogenen Unsicherheiten. Fehlen diese Charakteristika, steigt das Risiko für Rollenkonflikte, Ambiguitätsstress und motivationale Erosion.


\subsection{Besonderheiten des Bankensektors}

Der Bankensektor unterscheidet sich von anderen wissensintensiven Branchen in mehreren Dimensionen, die für die motivationale Wirkung generativer \gls{ki} auf mittlere Führungskräfte unmittelbar relevant sind.

Am offensichtlichsten ist die \textit{regulatorische Dichte}. Banken im \acrshort{dach}-Raum unterliegen der Aufsicht durch \gls{bafin}, \gls{fma} und \gls{finma} sowie europäischen Regulierungsrahmen wie der \gls{crr} und der EU-\acrshort{ki}-Verordnung (AI Act). Für Führungskräfte im mittleren Management bedeutet dies, dass Entscheidungen selten in einem Freiraum getroffen werden, sondern innerhalb eng definierter Compliance-Korridore. Kreditentscheidungen folgen standardisierten Ratingprozessen, Beratungsgespräche werden dokumentationspflichtig geführt, Risikoeinschätzungen müssen nachvollziehbar begründet sein. Wenn generative \gls{ki} in diese Prozesse integriert wird, verschärfen sich die Anforderungen an Transparenz und Erklärbarkeit erheblich -- ein algorithmisch generierter Kreditvorschlag, dessen Zustandekommen nicht lückenlos nachvollziehbar ist, widerspricht den aufsichtsrechtlichen Grundprinzipien [Quelle einfügen\footnote{Hier wäre eine regulatorische Quelle sinnvoll, z.\,B. BaFin (2024): \textit{Maschinelles Lernen in der Finanzbranche -- Aufsichtliche Prinzipien}; oder EU AI Act, Art. 6 zu hochriskanten Anwendungen im Finanzsektor.}].

Zweitens prägt eine ausgeprägte \textit{Hierarchie- und Fachexpertisekultur} das mittlere Management im Bankensektor. Entscheidungsbefugnisse sind an formale Kompetenzstufen gebunden; Unterschriftsberechtigungen für Kreditvergaben beispielsweise sind nach Volumen und Risikokategorie gestaffelt. Fachexpertise -- etwa in Bilanzanalyse, Risikomodellierung oder regulatorischer Compliance -- hat traditionell hohen Stellenwert und legitimiert die Autorität mittlerer Führungskräfte gegenüber ihren Teams. Generative \gls{ki}, die in Sekunden Bilanzanalysen erstellt oder Compliance-Prüfungen automatisiert, berührt damit direkt die Grundlage, auf der professionelle Identität und Führungslegitimation aufgebaut sind \parencite{quaquebeke_now_2023}. Anders als in kreativeren Branchen, wo der Wert einer Idee unabhängig von ihrer Quelle beurteilt wird, knüpft der Bankensektor Entscheidungslegitimation eng an formale Expertise und hierarchische Position.

Drittens ist die \textit{Art der Entscheidungsarbeit} im Bankensektor spezifisch. Mittlere Führungskräfte treffen täglich Entscheidungen, die unmittelbare finanzielle Konsequenzen haben -- für die Bank, für Kunden, für regulatorische Kennzahlen. Im Firmenkundengeschäft beurteilen sie Kreditrisiken, im Private Banking beraten sie vermögende Kunden über Anlagestrategien, im Risikomanagement bewerten sie Portfolioexpositionen. Diese Entscheidungen verlangen zweierlei: solide analytische Kompetenz \textit{und} kontextabhängiges Urteilsvermögen, das formale Modelle ergänzt. Die Frage, ob ein langjähriger Firmenkunde mit temporär verschlechterter Bonität weiterhin Kredit erhält, ist eben nicht vollständig formalisierbar -- sie erfordert Kenntnis der Branche, der persönlichen Geschichte und der strategischen Beziehung. Genau in diesem Spannungsfeld zwischen formalisierbarer Analyse und nicht-formalisierbarem Urteil entfaltet generative \gls{ki} ihre ambivalente Wirkung.

Viertens kennzeichnet den \acrshort{dach}-Bankensektor eine \textit{strukturelle Heterogenität}, die sich auf die \acrshort{ki}-Adoption auswirkt. Neben international agierenden Großbanken mit dedizierten Digital-Innovation-Teams existieren zahlreiche Sparkassen, Genossenschaftsbanken und Regionalbanken, deren Digitalisierungsgrad und Ressourcenausstattung erheblich variieren. Für mittlere Führungskräfte in kleineren Instituten bedeutet \acrshort{genai}-Adoption häufig eine individuell getriebene Exploration ohne institutionelle Rahmung -- sie experimentieren eigenständig mit Tools, ohne klare organisationale Leitlinien zu Nutzung, Grenzen und Verantwortlichkeiten. In größeren Häusern wiederum wird \acrshort{ki}-Adoption als Top-down-Projekt implementiert, was Effizienzgewinne verspricht, aber Autonomiespielräume einschränken kann \parencite{edwards_managerial_2024}.

Schließlich ist der Bankensektor durch eine \textit{Vertrauenskultur} geprägt, die über das Kunden-Berater-Verhältnis hinausreicht. Vertrauen ist das Grundkapital des Bankgeschäfts -- Kunden vertrauen ihre finanziellen Ressourcen der Bank an, Mitarbeitende vertrauen auf die Integrität interner Prozesse, Aufsichtsbehörden vertrauen auf die Selbstregulierungsfähigkeit der Institute. Wenn generative \gls{ki} in Entscheidungsprozesse eintritt, stellt sich die Vertrauensfrage auf einer neuen Ebene: Können Führungskräfte \acrshort{ki}-Outputs so weit vertrauen, dass sie Entscheidungen darauf gründen? Können sie dieses Vertrauen gegenüber Kunden und Vorgesetzten begründen? \textcite{prasad_generative_2024} zeigten, dass Vertrauen den zentralen Mediator zwischen wahrgenommener \acrshort{ki}-Nützlichkeit und tatsächlicher Akzeptanz darstellt -- ein Befund, der im vertrauensintensiven Kontext des Bankwesens besondere Relevanz besitzt.

Diese Besonderheiten -- regulatorische Dichte, Fachexpertisekultur, spezifische Entscheidungscharakteristik, strukturelle Heterogenität und Vertrauensabhängigkeit -- erzeugen ein Spannungsfeld, in dem die motivationalen Effekte generativer \gls{ki} auf mittlere Führungskräfte vermutlich anders verlaufen als in weniger regulierten oder stärker technologieaffinen Branchen. Die vorliegende Arbeit nimmt diesen spezifischen Kontext als Untersuchungsfeld, um die Wechselwirkungen zwischen \acrshort{ki}-Nutzung und den psychologischen Grundbedürfnissen von Führungskräften empirisch zu explorieren.
