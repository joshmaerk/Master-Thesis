\section{Generative KI als soziotechnisches Arbeitssystem}
\label{sec:genai}

Generative KI-Systeme sind in den vergangenen Jahren von einem Forschungsgegenstand zu einem Arbeitsmittel geworden, das kognitive Kernprozesse in Organisationen verändert. Der folgende Abschnitt grenzt den Begriff ab, ordnet die technologische Entwicklung ein und beleuchtet Einsatzszenarien in wissensintensiven Organisationen -- mit besonderem Blick auf den Bankensektor der DACH-Region.

\subsection{Begriffliche Abgrenzung und Entwicklung}

Generative Künstliche Intelligenz (GenAI) bezeichnet eine Klasse von KI-Systemen, die auf Basis umfangreicher Trainingsdaten neuartige Inhalte erzeugen können -- Texte, Bilder, Code, Audio oder multimodale Kombinationen. Der Begriff grenzt sich damit bewusst von früheren KI-Generationen ab, die primär klassifikatorisch oder regelbasiert operierten: Während ein Spam-Filter eingehende E-Mails in Kategorien sortiert, verfasst ein generatives Sprachmodell eigenständig Antworten, Analysen oder Entwürfe \parencite{brynjolfsson_generative_2023}.

Technologisch basieren die derzeit leistungsfähigsten generativen Systeme auf Large Language Models (LLMs). Diese neuronalen Netzwerke mit Milliarden von Parametern werden auf umfangreichen Textkorpora trainiert und nutzen Transformer-Architekturen, um probabilistische Vorhersagen über Textsequenzen zu treffen. Modelle wie GPT-4, Claude oder Gemini generieren auf dieser Grundlage kontextuell kohärente Outputs, die sich sprachlich kaum von menschlich verfassten Texten unterscheiden \parencite{brynjolfsson_generative_2023}. Die Veröffentlichung von ChatGPT im November 2022 markierte einen Wendepunkt: Innerhalb von zwei Monaten erreichte das System 100 Millionen aktive Nutzer und machte generative KI erstmals für breite Anwendergruppen in Organisationen zugänglich.

Was generative KI von traditionellen Informationssystemen -- etwa ERP-Systemen, Business-Intelligence-Dashboards oder Datenbanken -- unterscheidet, ist ihre \textit{generative Kapazität}. Herkömmliche Systeme speichern, verarbeiten und visualisieren vorhandene Informationen. GenAI erzeugt hingegen Inhalte, die so nicht explizit in den Trainingsdaten enthalten sind \parencite{bankinsMultilevelReviewArtificial2024}. Diese Eigenschaft eröffnet qualitativ neue Anwendungsszenarien: automatisierte Texterstellung, Szenarioanalysen, kreative Ideengenerierung und dialogische Interaktionen, bei denen das System auf Rückfragen und Kontextualisierungen reagiert.

Drei Merkmale charakterisieren generative KI-Systeme im organisationalen Kontext besonders. Erstens ihre \textit{probabilistische Kreativität}: Die Outputs sind nicht deterministisch, sondern variieren bei identischen Eingaben -- eine Eigenschaft, die sowohl kreatives Potenzial als auch Unsicherheit erzeugt. Zweitens ihre \textit{Kontextsensitivität}: Moderne LLMs halten Kontext über mehrteilige Dialoge hinweg und passen Antworten an spezifische Nutzerbedürfnisse an. Drittens ihre \textit{Anpassbarkeit}: Durch Fine-Tuning und Retrieval-Augmented Generation (RAG) lassen sich generative Systeme an organisationsspezifische Wissensbasen und Prozesse koppeln, während Nutzer über iteratives Prompting ihre Interaktionskompetenz weiterentwickeln \parencite{brynjolfsson_generative_2023, bankinsMultilevelReviewArtificial2024}.

Diese Merkmale machen GenAI zu einem soziotechnischen Phänomen im engeren Sinne: Das System entfaltet seine Wirkung nicht unabhängig von den Personen, die es nutzen, sondern in einer kontinuierlichen Wechselwirkung zwischen technologischen Möglichkeiten und menschlichen Praktiken. Wie Nutzer Prompts formulieren, welches Vertrauen sie in Outputs setzen, ob sie Ergebnisse kritisch prüfen oder unreflektiert übernehmen -- all das beeinflusst, welche Rolle GenAI in einer Organisation tatsächlich spielt \parencite{bankinsMultilevelReviewArtificial2024}.


\subsection{Einsatzszenarien in wissensintensiven Organisationen}

Generative KI findet primär dort Anwendung, wo Arbeit kognitiv anspruchsvoll, textbasiert und wissensintensiv ist: Dokumentenerstellung und -analyse, Entscheidungsvorbereitung, Strategieentwicklung, Kommunikation und Problemlösung \parencite{brynjolfsson_generative_2023}. Im Unterschied zu früheren Automatisierungswellen, die vorrangig manuelle und repetitive Tätigkeiten adressierten, greift die aktuelle Technologiewelle in den Kern professioneller Wissensarbeit ein.

Empirisch gut dokumentiert ist der Produktivitätseffekt. Eine Feldstudie mit über 5.000 Kundenservice-Mitarbeitenden ergab, dass der Einsatz eines generativen KI-Assistenten die Zahl gelöster Anfragen pro Stunde um 14\,\% steigerte. Aufschlussreich war die Verteilung dieses Effekts: Die Produktivitätsgewinne konzentrierten sich auf weniger erfahrene Mitarbeitende, während hochqualifizierte Experten kaum profitierten \parencite{brynjolfsson_generative_2023}. GenAI scheint demnach als eine Art Kompetenz-Augmentation zu fungieren -- sie hebt das Leistungsniveau weniger Erfahrener an, ohne die Leistung von Experten substanziell zu steigern.

\textcite{bankinsMultilevelReviewArtificial2024} identifizierten in einer Multilevel-Review fünf thematische Pfade, über die KI in Organisationen wirkt: (1) Mensch-KI-Kollaboration und Komplementarität, (2) Wahrnehmung von KI-Fähigkeiten und -Grenzen, (3) KI als Kontrollmechanismus im Sinne algorithmischen Managements, (4) Arbeitsmarktimplikationen wie Job Displacement und Skill Shifts sowie (5) ethische und soziale Implikationen. Diese Pfade interagieren über Analyseebenen hinweg und erzeugen häufig widersprüchliche Effekte -- ein Befund, der vereinfachende Narrative von KI als reinem Effizienzwerkzeug infrage stellt.

Die organisationale Einbettung von GenAI lässt sich entlang dreier Perspektiven konzeptualisieren \parencite{bankinsMultilevelReviewArtificial2024}. Aus \textit{instrumenteller} Sicht wird KI als Produktivitätswerkzeug verstanden: Sie automatisiert repetitive Aufgaben, beschleunigt Informationsverarbeitung und reduziert kognitive Belastung. Organisationen messen den Erfolg hier in Zeitersparnis, Kostenreduktion und Output-Steigerung. Diese Perspektive dominiert in frühen Adoptionsphasen, birgt aber das Risiko, motivationale und identitätsbezogene Effekte zu übersehen \parencite{edwards_managerial_2024}.

Die \textit{strategische} Perspektive geht einen Schritt weiter: KI liefert nicht nur Informationen, sondern generiert Empfehlungen, Prognosen und Handlungsalternativen, die in menschliche Entscheidungsprozesse einfließen. In einer Feldstudie mit Verkaufsmitarbeitenden untersuchten \textcite{tongJanusFaceArtificial2021} diese Integration und fanden einen bemerkenswerten Disclosure-Effekt: KI-basiertes Feedback verbesserte die Leistung, allerdings nur, solange Mitarbeitende nicht wussten, dass es von KI stammte. Die Offenlegung der algorithmischen Quelle reduzierte Akzeptanz und Wirksamkeit -- ein Hinweis darauf, dass strategische KI-Integration mehr erfordert als technische Funktionalität, nämlich Vertrauensaufbau und transparente Kommunikation.

Aus \textit{transformativer} Perspektive schließlich fungiert KI als Katalysator fundamentaler Veränderungen in Arbeitsrollen, Organisationsstrukturen und professionellen Identitäten. Mensch-KI-Kollaboration, hybride Teamstrukturen und algorithmisch vermittelte Koordination sind hier keine Ausnahme mehr, sondern alltägliche Praxis \parencite{bankinsMultilevelReviewArtificial2024}. Die Wahl der Perspektive beeinflusst maßgeblich, wie Mitarbeitende KI wahrnehmen und nutzen: Instrumentelle Framings erleichtern unter Umständen die Akzeptanz, lassen aber transformative Potenziale ungenutzt. Strategische Framings erhöhen Anforderungen an Transparenz und Erklärbarkeit. Transformative Implementierungen erfordern neue Rollen, Kompetenzen und Governance-Strukturen.

Quer zu diesen Perspektiven zeigt sich, dass die Wahrnehmung von KI durch Mitarbeitende nicht technologisch determiniert, sondern sozial konstruiert ist. Eine zentrale Dimension ist die Unterscheidung zwischen KI als Unterstützungs- und als Kontrollsystem \parencite{edwards_managerial_2024}. Dieselben Tools werden in verschiedenen Kontexten unterschiedlich interpretiert: als Arbeitserleichterung oder als Überwachungsinstrument, als Kompetenzerweiterung oder als Bedrohung professioneller Expertise \parencite{monod_worker_2024}. Vertrauen erweist sich dabei als zentraler Mediator. \textcite{prasad_generative_2024} zeigten in einer Studie mit 1.362 Beschäftigten, dass Vertrauen in GenAI die Akzeptanz KI-basierter Praktiken vollständig mediierte: Ohne Vertrauen in Zuverlässigkeit, Fairness und Transparenz blieb auch wahrgenommene Nützlichkeit wirkungslos.


\subsection{Generative KI im Bankensektor (DACH)}

Der Bankensektor zählt international zu den Branchen mit der höchsten GenAI-Adoptionsrate. Branchenerhebungen beziffern den Anteil der Finanzinstitute, die generative KI bereits einsetzen oder dies innerhalb von zwei Jahren planen, auf über 95\,\% \parencite{YourJourneyGenAI}. Typische Anwendungsfelder umfassen Marketing und Kundenkommunikation, Risikomanagement und Compliance, Kreditanalyse sowie interne Wissensmanagement-Prozesse. Für die globale Bankenbranche werden jährliche Produktivitätsgewinne von 200 bis 340 Milliarden US-Dollar prognostiziert \parencite{mckinsey__company_capturing_nodate}.

Im DACH-Raum ergibt sich ein spezifisches Bild, das durch drei Faktoren geprägt wird. Erstens ist der Bankensektor in Deutschland, Österreich und der Schweiz stark reguliert. Die Aufsichtsbehörden -- BaFin, FMA und FINMA -- stellen hohe Anforderungen an Transparenz, Erklärbarkeit und Nachvollziehbarkeit algorithmischer Entscheidungen, insbesondere in der Kreditvergabe und im Risikomanagement. Diese regulatorischen Rahmenbedingungen erzeugen ein Spannungsfeld: Einerseits begrenzen sie die Geschwindigkeit der KI-Adoption, andererseits zwingen sie Organisationen zu einer bewussteren Auseinandersetzung mit Governance-Fragen, die in weniger regulierten Branchen häufig nachgelagert behandelt werden \parencite{hundertmarkIFZGenerativeAI2024}.

Zweitens ist das DACH-Bankensystem durch eine hohe Dichte an Universalbanken, Genossenschaftsbanken und Sparkassen gekennzeichnet. Anders als in angelsächsischen Märkten, wo wenige Großbanken den technologischen Takt vorgeben, existieren im DACH-Raum zahlreiche mittelgroße Institute, deren Digitalisierungsgrad erheblich variiert. Für mittlere Führungskräfte in diesen Organisationen bedeutet dies, dass GenAI-Implementierung selten als Top-down-Projekt mit klarer strategischer Rahmung erfolgt, sondern häufig als bottom-up-getriebene Experimentation einzelner Teams oder Abteilungen.

Drittens unterscheidet sich die Arbeitskultur im DACH-Bankensektor in relevanter Weise. Die Tradition konsensualer Entscheidungsfindung, ausgeprägte Mitbestimmungsstrukturen (insbesondere in Deutschland und Österreich) sowie eine vergleichsweise hohe Bedeutung formaler Qualifikationen und Fachexpertise prägen das Umfeld, in das generative KI eingeführt wird. Wenn ein KI-System Kreditanalysen entwirft, die bislang erfahrene Spezialisten formuliert haben, berührt dies nicht nur Effizienzfragen, sondern auch professionelle Identität und die Legitimation von Expertise \parencite{quaquebeke_now_2023}.

Empirische Studien, die explizit die motivationalen Auswirkungen generativer KI auf Führungskräfte im DACH-Bankensektor untersuchen, liegen bislang nicht vor. Die vorhandene Forschung adressiert entweder den Bankensektor ohne spezifischen Fokus auf Führungsmotivation \parencite{bankinsMultilevelReviewArtificial2024} oder untersucht motivationale Effekte von KI ohne branchenspezifische Differenzierung \parencite{brynjolfsson_generative_2023, edwards_managerial_2024}. Diese Forschungslücke ist insofern bemerkenswert, als der DACH-Bankensektor aufgrund seiner regulatorischen Dichte, organisationalen Heterogenität und kulturellen Spezifika einen Kontext darstellt, in dem die motivationalen Spannungen der KI-Adoption besonders ausgeprägt sein dürften. Die vorliegende Arbeit adressiert dieses Desiderat durch eine qualitative Untersuchung, die das Erleben mittlerer Führungskräfte in diesem spezifischen Branchenkontext in den Mittelpunkt stellt.