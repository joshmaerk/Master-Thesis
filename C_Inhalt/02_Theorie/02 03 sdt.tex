\section{Self-Determination Theory (SDT)}
\label{sec:sdt}

Die bisherigen Abschnitte haben gezeigt, dass generative KI in Kernbereiche professioneller Wissensarbeit eingreift und dass mittlere Führungskräfte im Bankensektor davon in spezifischer Weise betroffen sind. Offen geblieben ist die Frage, über welche psychologischen Mechanismen sich diese Veränderungen auf das motivationale Erleben auswirken. Die Self-Determination Theory bietet dafür einen empirisch breit abgestützten Erklärungsrahmen, der Motivation nicht als Quantität, sondern als Qualität begreift -- und damit einen differenzierteren Zugang eröffnet als reine Akzeptanz- oder Leistungsmodelle.

\subsection{Grundannahmen und Entstehungskontext}

Die Self-Determination Theory (SDT), entwickelt von Edward L. Deci und Richard M. Ryan, ist eine Meta-Theorie menschlicher Motivation. Ihr Ausgangspunkt ist eine Annahme, die im Kontrast zu behavioristischen und rein ökonomischen Motivationsmodellen steht: Menschen sind nicht passive Rezipienten externer Anreize, sondern aktive, wachstumsorientierte Organismen mit einer intrinsischen Tendenz, ihre Umwelt zu explorieren, Kompetenzen zu entwickeln und soziale Beziehungen aufzubauen \parencite{deciWhatWhyGoal2000}. Ob diese Tendenz sich entfaltet oder verkümmert, hängt maßgeblich von den sozialen Kontexten ab, in denen Menschen handeln.

Was SDT von anderen Motivationstheorien unterscheidet, ist ihr Fokus auf die \textit{Qualität} der Motivation, nicht bloß deren Intensität. Zwei Personen können gleich viel Energie auf eine Aufgabe verwenden und dennoch fundamental unterschiedlich motiviert sein: die eine aus genuinem Interesse, die andere aus Angst vor negativen Konsequenzen. SDT argumentiert, dass diese Unterscheidung nicht trivial ist -- sie sagt systematisch vorher, wie nachhaltig, kreativ und gesundheitsförderlich das resultierende Verhalten ausfällt \parencite{deciSelfDeterminationTheoryWork2017}.

Konkret differenziert SDT zwischen \textit{intrinsischer Motivation} -- Verhalten, das um seiner selbst willen ausgeführt wird, aus Interesse und Freude -- und \textit{extrinsischer Motivation} -- Verhalten, das instrumentell auf separate Outcomes gerichtet ist. Der theoretische Beitrag liegt darin, extrinsische Motivation nicht pauschal als defizitär zu behandeln, sondern nach dem Grad der Internalisierung zu differenzieren \parencite{deciWhatWhyGoal2000}. Das resultierende \textit{Selbstbestimmungskontinuum} reicht von Amotivation (keine Handlungsintention) über kontrollierte Formen extrinsischer Motivation -- externale Regulation durch Belohnung und Bestrafung, introjizierte Regulation durch Schuld und Selbstwertdruck -- bis hin zu autonomen Formen: identifizierte Regulation (persönliche Anerkennung des Werts einer Handlung), integrierte Regulation (Übereinstimmung mit dem Selbstkonzept) und schließlich intrinsische Motivation.

Je autonomer die Motivationsform, desto günstiger die Konsequenzen: höheres Wohlbefinden, bessere Leistung, größere Persistenz und kreativeres Problemlösen \parencite{deciSelfDeterminationTheoryWork2017}. Das ist keine bloß theoretische Unterscheidung. In einer Validierungsstudie über sieben Sprachen und neun Länder (N\,>\,3.000) bestätigten \textcite{gagneMultidimensionalWorkMotivation2015} die faktorielle Struktur des Kontinuums und zeigten, dass autonome Motivationsformen konsistent positiv mit Leistung, Wohlbefinden und organisationalem Commitment assoziiert waren, während kontrollierte Motivation schwächere oder inkonsistente Effekte aufwies.

Der entscheidende Mechanismus, der erklärt, \textit{warum} bestimmte Kontexte autonome Motivation fördern und andere sie untergraben, liegt in der Befriedigung psychologischer Grundbedürfnisse -- dem Kernstück der Theorie.


\subsection{Die drei psychologischen Grundbedürfnisse}

SDT postuliert drei fundamentale psychologische Grundbedürfnisse, deren Befriedigung für psychologisches Wachstum, Integrität und Wohlbefinden essenziell ist: Autonomie, Kompetenz und soziale Eingebundenheit \parencite{deciWhatWhyGoal2000, vandenbroeckReviewSelfDeterminationTheorys2016}. Die Theorie versteht diese nicht als individuelle Präferenzen, die von Person zu Person variieren, sondern als universelle Nährstoffe -- vergleichbar mit physiologischen Bedürfnissen, deren Frustration unabhängig von kulturellem Kontext oder persönlicher Disposition zu Beeinträchtigungen führt.

\subsubsection{Autonomie}

Autonomie bezeichnet das Bedürfnis, sich als Ursprung des eigenen Verhaltens zu erleben -- selbstbestimmt und in Übereinstimmung mit den eigenen Werten zu handeln \parencite{deciWhatWhyGoal2000}. Der Begriff wird häufig missverstanden: Autonomie meint nicht Unabhängigkeit oder Isolation, sondern \textit{Volition}. Eine Führungskraft, die eine strategische Vorgabe umsetzt, handelt autonom, solange sie die Vorgabe als sinnvoll anerkennt und den Umsetzungsweg selbst gestalten kann. Dieselbe Vorgabe wird zum Autonomiefresser, wenn sie als willkürliche Kontrolle erlebt wird, der man sich fügen muss.

Autonomie wird gefördert durch Wahlmöglichkeiten, Partizipation an Entscheidungen, Bereitstellung von Rationalen und Minimierung von Kontrolle und Druck. Sie wird frustriert, wenn Verhalten durch externe Kräfte oder internalisierte Druckmechanismen kontrolliert wird \parencite{vandenbroeckReviewSelfDeterminationTheorys2016}. Für Führungskräfte ist Autonomie besonders relevant, da ihre Rolle traditionell mit Entscheidungsfreiheit und strategischem Gestaltungsspielraum assoziiert wird. Technologien, die Entscheidungsspielräume einschränken oder algorithmische Kontrolle ausüben, können daher Reaktanz erzeugen, die weit über das Maß hinausgeht, das bei operativen Mitarbeitenden zu beobachten wäre \parencite{edwardsManagerialControlFeedback2024}.

\subsubsection{Kompetenz}

Kompetenz bezeichnet das Bedürfnis, sich als effektiv und fähig zu erleben -- Herausforderungen erfolgreich zu meistern und kontinuierlich zu lernen \parencite{deciWhatWhyGoal2000}. Kompetenzerleben ist nicht identisch mit objektiver Kompetenz; es bezieht sich auf die subjektive Wahrnehmung von Wirksamkeit und Meisterschaft. Eine Führungskraft kann objektiv kompetent sein und sich dennoch inkompetent fühlen, wenn ein KI-System dieselbe Analyse in Sekunden erstellt, für die sie Stunden benötigt.

Optimales Kompetenzerleben entsteht, wenn Aufgaben weder zu einfach noch zu überfordernd sind, sondern im Bereich der optimalen Herausforderung liegen -- ein Konzept, das Parallelen zum Flow-Erleben aufweist \parencite{csikszentmihalyiFlowPsychologyOptimal2009}, aber breiter gefasst ist. Kompetenz wird gefördert durch klares, konstruktives Feedback, erreichbare aber herausfordernde Ziele und Gelegenheiten zur Kompetenzentwicklung \parencite{vandenbroeckReviewSelfDeterminationTheorys2016}. Sie wird frustriert durch Überforderung, intransparente Bewertungskriterien oder die Entwertung erworbener Expertise.

\subsubsection{Soziale Eingebundenheit}

Soziale Eingebundenheit (Relatedness) bezeichnet das Bedürfnis, sich mit anderen verbunden und zugehörig zu fühlen -- Beziehungen zu pflegen, die durch gegenseitige Fürsorge, Respekt und Vertrauen gekennzeichnet sind \parencite{deciWhatWhyGoal2000}. Im Arbeitskontext umfasst dies Zugehörigkeit zu Teams und Organisationen, unterstützende Beziehungen zu Kollegen und Vorgesetzten sowie das Gefühl, einen Beitrag zu einer größeren Gemeinschaft zu leisten \parencite{vandenbroeckReviewSelfDeterminationTheorys2016}.

Im Vergleich zu Autonomie und Kompetenz wird soziale Eingebundenheit in der SDT-Forschung zum Arbeitskontext manchmal als nachrangig behandelt. \textcite{deciSelfDeterminationTheoryWork2017} argumentieren jedoch, dass Relatedness eine notwendige Voraussetzung für nachhaltige Internalisierung extrinsischer Motivation darstellt: Menschen übernehmen Werte und Praktiken ihrer sozialen Umgebung eher, wenn sie sich dieser Umgebung zugehörig fühlen. Für Führungskräfte, deren Wirksamkeit wesentlich auf Beziehungsqualität beruht -- gegenüber Teams, Peers und Vorgesetzten --, ist dieses Bedürfnis keineswegs sekundär.


\subsection{SDT im Arbeitskontext}

SDT hat sich als produktiver theoretischer Rahmen für die Arbeits- und Organisationspsychologie etabliert, mit umfangreicher empirischer Evidenz zu Arbeitsmotivation, Leistung, Kreativität, Wohlbefinden, Burnout und organisationalem Commitment \parencite{deciSelfDeterminationTheoryWork2017, gagneUnderstandingShapingFuture2022}.

Der zentrale Wirkmechanismus ist gut belegt: Arbeitskontexte, die die drei Grundbedürfnisse befriedigen, fördern autonome Motivation, die wiederum positive Outcomes begünstigt. Meta-Analysen zeigen konsistent, dass Bedürfnisbefriedigung am Arbeitsplatz positiv mit Wohlbefinden und negativ mit Burnout und Ill-Being assoziiert ist -- robust über Kulturen, Berufe und Messmethoden hinweg \parencite{vandenbroeckReviewSelfDeterminationTheorys2016}. Bedürfnisbefriedigung fördert dabei nicht nur Wohlbefinden, sondern auch Leistung und Engagement: \textcite{vandenbroeckReviewSelfDeterminationTheorys2016} fanden signifikante positive Zusammenhänge mit Job Performance, organisationalem Commitment und proaktivem Verhalten.

Konzeptionell und empirisch bedeutsam ist die Unterscheidung zwischen Bedürfnisbefriedigung (need satisfaction) und Bedürfnisfrustration (need frustration). Frustration tritt auf, wenn Bedürfnisse aktiv blockiert oder untergraben werden -- nicht bloß abwesend sind \parencite{vandenbroeckCapturingAutonomyCompetence2010}. Dieser Unterschied ist nicht akademisch: Bedürfnisfrustration erweist sich als stärkerer Prädiktor für negative Outcomes (Burnout, kontraproduktives Verhalten) als Bedürfnisbefriedigung für positive \parencite{vandenbroeckReviewSelfDeterminationTheorys2016}. Die Asymmetrie hat praktische Implikationen: Technologien, die Bedürfnisse aktiv frustrieren -- etwa durch Kontrollerleben oder Expertiseentwertung --, dürften gravierendere motivationale Konsequenzen haben als Technologien, die Bedürfnisse lediglich nicht fördern.

Ein zentrales Konzept der angewandten SDT-Forschung ist \textit{Autonomy Support} -- Führungsverhalten und organisationale Praktiken, die Autonomie, Kompetenz und Relatedness fördern. Autonomieunterstützung umfasst: Perspektiven anerkennen, Wahlmöglichkeiten bieten, Rationale bereitstellen und Kontrolle minimieren \parencite{deciSelfDeterminationTheoryWork2017}. Meta-Analysen bestätigen, dass autonomieunterstützende Führung signifikant positiv mit Bedürfnisbefriedigung, autonomer Motivation und Leistung assoziiert ist, während kontrollierende Führung mit Bedürfnisfrustration und negativen Outcomes einhergeht \parencite{vandenbroeckReviewSelfDeterminationTheorys2016}.

Für die vorliegende Arbeit ist dieser Befund in doppelter Hinsicht relevant. Erstens sind mittlere Führungskräfte selbst Empfänger von Autonomieunterstützung (oder -kontrolle) durch ihre Vorgesetzten und durch organisationale Strukturen -- zu denen zunehmend auch algorithmische Systeme gehören. Zweitens sind sie Gestalter von Autonomieunterstützung für ihre Teams, wobei KI-Tools diese Gestaltungsarbeit sowohl erleichtern als auch erschweren können.


\subsection{Digitale Arbeitssysteme als bedürfnisunterstützende oder -frustrierende Faktoren}

SDT wurde zunehmend auf Technologieakzeptanz und -nutzung angewendet, häufig in Integration mit dem Technology Acceptance Model (TAM) und der Unified Theory of Acceptance and Use of Technology (UTAUT) \parencite{KoenigPascal2024, YangHsiHsun2024}. Die Kernfrage lautet: Unter welchen Bedingungen unterstützen oder frustrieren digitale Arbeitssysteme die drei psychologischen Grundbedürfnisse?

SDT erweitert rein kognitive Akzeptanzmodelle um eine motivationale Erklärungsebene. TAM und UTAUT postulieren, dass wahrgenommene Nützlichkeit und Benutzerfreundlichkeit die Nutzungsintention bestimmen. SDT fragt darüber hinaus: \textit{Wie} wird Technologie erlebt? Fördert sie das Gefühl von Selbstbestimmung, oder erzeugt sie Kontrollerleben? Stärkt sie die eigene Wirksamkeit, oder untergräbt sie professionelle Identität? Verbindet sie Menschen, oder isoliert sie? \parencite{KoenigPascal2024}

Empirisch zeigen sich ambivalente Befunde. Auf der einen Seite können digitale Arbeitssysteme Autonomie erweitern (etwa durch flexible Arbeitsorganisation), Kompetenz fördern (durch Zugang zu Informationen und Feedback-Systemen) und Relatedness stärken (durch Kommunikationstools und kollaborative Plattformen) \parencite{gagneUnderstandingShapingFuture2022}. Auf der anderen Seite dokumentiert die Forschung konsistent auch negative Pfade: Automatisierung kann wahrgenommene Arbeitsbedeutsamkeit und Autonomie reduzieren -- eine Studie fand bei einer 7,5-fachen Steigerung der Robotisierung projizierte Rückgänge von 6,8\,\% bei Bedeutsamkeit und 7,5\,\% bei Autonomie \parencite{nikolovaRobotsMeaningSelfdetermination2024}. Wahrgenommene Technologieunsicherheit fungiert als Demand, die Technostress erhöht und Arbeitszufriedenheit reduziert \parencite{LiuWangLin2023}.

\textcite{KoenigPascal2024} entwickelten ein theoretisches Framework, das drei Akzeptanzperspektiven integriert: Nutzerakzeptanz, Delegationsakzeptanz und gesellschaftliche Adoption. Für jede Perspektive identifizierten sie spezifische SDT-relevante Faktoren: Nutzerakzeptanz hängt primär von Kompetenz- und Autonomieerleben ab; Delegationsakzeptanz von Vertrauen und wahrgenommener Zuverlässigkeit; gesellschaftliche Adoption von kollektiven Werten und sozialen Normen.

Speziell für generative KI zeigt eine Studie mit 565 Designprofessionellen die Erklärungskraft eines integrierten UTAUT-SDT-Modells: Es klärte 52,1\,\% der Varianz in der Verhaltensintention auf. Bemerkenswert war, dass die wahrgenommene Bedrohung durch Jobersatz die Beziehung zwischen Leistungserwartung und Nutzungsintention negativ moderierte -- selbst Personen, die die Leistungsfähigkeit der Tools anerkannten, zeigten geringere Nutzungsbereitschaft, wenn sie KI als Bedrohung ihrer beruflichen Existenz wahrnahmen \parencite{YangHsiHsun2024}.

Besonders aufschlussreich für die vorliegende Arbeit ist ein experimenteller Befund von \textcite{WuLiuRuan2025}. In einer Studie mit 15.105 Teilnehmenden steigerte GenAI-Unterstützung die Aufgabenleistung signifikant, reduzierte aber gleichzeitig die intrinsische Motivation. Der Mechanismus: Nutzer attribuierten den Erfolg der KI statt sich selbst und erlebten dadurch reduzierte Selbstwirksamkeit. Kompetenzerleben wurde also nicht durch objektiven Misserfolg frustriert, sondern durch die Verschiebung der Erfolgsattribution -- ein Paradox, das für Führungskräfte besonders relevant sein dürfte, deren professionelle Identität eng an persönliche Expertise geknüpft ist.

Trotz dieser Fortschritte bleibt die Anwendung von SDT auf generative KI im Arbeitskontext begrenzt. Generative KI unterscheidet sich von den bisher untersuchten Technologien durch drei Eigenschaften, die theoretische Erweiterungen erfordern \parencite{gagneUnderstandingShapingFuture2022}: Erstens ihre \textit{Ko-Kreationsfähigkeit} -- GenAI ist nicht bloß Werkzeug, sondern Dialogpartner, der Inhalte mitgestaltet. Zweitens ihre \textit{Kontextsensitivität} -- die Qualität der Interaktion hängt von der Kompetenz der nutzenden Person ab (Prompt Engineering), was eine rekursive Beziehung zwischen Technologienutzung und Kompetenzerleben erzeugt. Drittens ihr \textit{Potenzial zur Identitätsbedrohung} -- GenAI kann Kernaufgaben professioneller Wissensarbeit substituieren, nicht nur Routinetätigkeiten.

Diese drei Eigenschaften machen generative KI zu einem Untersuchungsgegenstand, der über die bisherige SDT-Technologieforschung hinausgeht und spezifische empirische Zugänge erfordert -- eine Lücke, die die vorliegende Arbeit qualitativ zu adressieren versucht.

\subsection{Grenzen und Randbedingungen der SDT im Kontext organisationaler KI-Forschung}
\label{subsec:sdt-grenzen}

Die bisherige Darstellung hat die Self-Determination Theory als tragfähigen Rahmen für die Analyse motivationaler Wirkungen digitaler Arbeitssysteme eingeführt. Bevor dieser Rahmen auf den spezifischen Untersuchungsgegenstand -- generative KI in der Führungsarbeit -- angewendet wird, verdienen einige theoretische Grenzen Beachtung. Sie schmälern nicht die Eignung der SDT für die vorliegende Fragestellung, verweisen aber auf Bereiche, in denen die Theorie unterspezifiziert bleibt und explorative Zugänge erfordert.

\paragraph{Kulturelle Universalität und der Autonomiebegriff.}
SDT postuliert die drei psychologischen Grundbedürfnisse als universell -- eine Annahme, die wiederholt auf kulturelle Einwände gestoßen ist. Chirkov et al. (2003) zeigten zwar, dass Autonomie auch in kollektivistischen Kontexten motivational wirksam bleibt, wiesen aber zugleich nach, dass die \textit{Art und Weise}, wie Autonomie erlebt und ausgedrückt wird, kulturell variiert.\footnote{%
  Chirkov, V., Ryan, R. M., Kim, Y. \& Kaplan, U. (2003). Differentiating autonomy from individualism and independence: A self-determination theory perspective on internalization of cultural orientations and well-being. \textit{Journal of Personality and Social Psychology}, \textit{84}(1), 97--110. \url{https://doi.org/10.1037/0022-3514.84.1.97} -- [Quelle in Zotero ergänzen]}
Für die vorliegende Arbeit ist das keineswegs trivial: Die DACH-Region zeichnet sich durch eine ausgeprägte Konsens- und Mitbestimmungskultur aus, in der Autonomie weniger als individuelle Unabhängigkeit, sondern als Partizipation an Entscheidungsprozessen verstanden werden dürfte. Ob und wie Führungskräfte in diesem Kontext die KI-induzierte Verschiebung von Entscheidungsstrukturen als autonomiefördernd oder -einschränkend erleben, lässt sich aus der SDT allein nicht ableiten -- hier braucht es empirische Rekonstruktion.

\paragraph{Korrelative Evidenzbasis.}
McAnally und Hagger (2024) haben in einem konzeptionellen Review drei zentrale methodische Schwächen der arbeitspsychologischen SDT-Forschung identifiziert: eine Dominanz querschnittlicher Designs, den Mangel an Interventionsstudien mit explizitem Mediationstest und die unzureichende Berücksichtigung moderierender Variablen \parencite[vgl.][]{mcanallySelfdeterminationTheoryWorkplace2024}. Die allermeisten Befunde zur Bedürfnisbefriedigung am Arbeitsplatz beruhen auf korrelativen Daten, die keine kausalen Schlüsse erlauben. Auch die Debatte um den sogenannten Undermining-Effekt -- die These, dass extrinsische Anreize intrinsische Motivation verdrängen -- wird durch widersprüchliche Befunde aus Labor- und Feldstudien verkompliziert. Saini, Uppal und Howard (2025) zeigten in einer aktuellen Metaanalyse, dass der Effekt unter kontrollierten Laborbedingungen robust auftritt, in Feldstudien mit realen Arbeitsbedingungen jedoch deutlich schwächer und inkonsistenter ausfällt.\footnote{%
  Saini, G. K., Uppal, N. \& Howard, J. L. (2025). The undermining effect of extrinsic rewards: Fact or artefact? \textit{Journal of Occupational and Organizational Psychology}, \textit{98}, e70000. -- [Quelle in Zotero ergänzen]}
Für die vorliegende Studie folgt daraus: Die SDT liefert ein konzeptionell überzeugendes, aber empirisch noch unvollständig gesichertes Erklärungsmodell. Gerade deshalb erscheint ein qualitativer Zugang sinnvoll, der Wirkzusammenhänge nicht voraussetzt, sondern rekonstruiert.

\paragraph{Fehlende Identitäts- und Institutionenperspektive.}
Eine auffällige Leerstelle der SDT betrifft die Rolle professioneller Identität. Die vorangegangenen Abschnitte haben gezeigt, dass generative KI Kompetenzerleben nicht nur über Aufgabenbewältigung, sondern auch über die wahrgenommene Entwertung beruflicher Expertise beeinflusst. Dieser Mechanismus -- die Bedrohung des Selbstverständnisses als fachliche Autorität -- wird von der SDT nicht abgebildet. Sie modelliert Kompetenz als funktionales Erleben von Wirksamkeit, nicht als identitätsgebundene Zuschreibung. Ähnlich verhält es sich mit institutionellen Rahmenbedingungen: Der Bankensektor ist durch eine dichte regulatorische Umgebung geprägt, die Handlungsspielräume nicht nur faktisch einschränkt, sondern auch normativ rahmt. SDT verfügt über keinen Mechanismus, der erklärt, wie regulatorischer Druck die Bedürfnisbefriedigung moderiert -- ob etwa Compliance-Anforderungen als externe Kontrolle (und damit autonomiefrustrierend) oder als professionelle Selbstverständlichkeit (und damit motivational neutral) erlebt werden, hängt von Kontextfaktoren ab, die außerhalb der Reichweite der Theorie liegen.

\paragraph{Teamebene als blinder Fleck.}
Grenier et al. (2024) haben darauf hingewiesen, dass die SDT trotz ihrer breiten Rezeption in der Arbeitspsychologie auf Teamebene kaum systematisch angewendet wurde.\footnote{%
  Grenier, S., Bhatt, M. \& Bhimani, R. (2024). A systematic review of self-determination theory in teams. \textit{Applied Psychology}, \textit{73}(3), 1286--1312. \url{https://doi.org/10.1111/apps.12526} -- [Quelle in Zotero ergänzen]}
Das Bedürfnis nach sozialer Eingebundenheit wird vorwiegend als dyadisches oder individuelles Konstrukt operationalisiert; wie sich Teamdynamiken -- etwa geteilte Verantwortung, kollektive Kompetenzzuschreibung oder die soziale Aushandlung von KI-Nutzung -- auf die Bedürfnisbefriedigung auswirken, bleibt unterbestimmt. Für Führungskräfte im mittleren Management, deren Arbeit wesentlich durch die Koordination von Teams und die Vermittlung zwischen Hierarchieebenen geprägt ist, stellt diese Lücke eine relevante Einschränkung dar.

\paragraph{Konsequenzen für das Forschungsdesign.}
Diese Grenzen entwerten die SDT nicht als Analyserahmen -- sie präzisieren den Anspruch, mit dem die Theorie in dieser Arbeit eingesetzt wird. Die drei psychologischen Grundbedürfnisse strukturieren die Analyse als \textit{heuristische Kategorien}, nicht als kausaltheoretisches Erklärungsmodell. Gerade weil die SDT an den beschriebenen Stellen unterspezifiziert ist, braucht es einen explorativen Zugang, der empirisch rekonstruiert, wie Führungskräfte die motivationalen Wirkungen generativer KI tatsächlich erleben und deuten. Die qualitative Methodik dieser Arbeit -- problemzentrierte Interviews mit induktiver Subkategorienbildung (vgl. Kapitel~3) -- ist darauf angelegt, genau diese Leerstellen zu adressieren, indem sie Phänomene sichtbar macht, die deduktiv aus der SDT allein nicht ableitbar wären.