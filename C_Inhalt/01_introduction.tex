\chapter{Einführung}

Generative \gls{ki} verändert, wie in Organisationen gearbeitet, entschieden und geführt wird. Das vorliegende Kapitel umreißt die Problemstellung, die sich daraus für das motivationale Erleben von Führungskräften ergibt, formuliert die Zielsetzung und Forschungsfrage der Arbeit und gibt einen Überblick über den Gang der Argumentation.

\section{Problemstellung und Relevanz}

Generative \gls{ki} hat innerhalb weniger Jahre den Weg aus der Forschungsliteratur in den Arbeitsalltag von Millionen von Wissensarbeiterinnen und Wissensarbeitern gefunden. Was lange als futuristisches Versprechen galt, ist in Unternehmen längst operative Realität: Sprachmodelle wie GPT-4, Claude oder Gemini unterstützen bei der Analyse von Dokumenten, der Vorbereitung von Entscheidungen, der Kommunikation mit Stakeholdern und der Zusammenfassung komplexer Sachverhalte \parencite{brynjolfsson_generative_2023, noyExperimentalEvidenceProductivity2023}. Das Effizienzversprechen, das mit dieser Technologie verbunden wird, ist empirisch nicht ohne Grundlage: Feldstudien dokumentieren substanzielle Produktivitätsgewinne, insbesondere bei wissensintensiven und kommunikativen Tätigkeiten \parencite{dellacqua_navigating_2023}.


Doch hinter der Fassade des Effizienzgewinns tut sich ein blinder Fleck auf. Die Frage, wie Führungskräfte den Einsatz dieser Systeme motivational erleben, bleibt in der bisherigen Forschung weitgehend unbeantwortet. Bestehende Studien konzentrieren sich auf Leistungsoutcomes, Adoptionsraten und organisationale Effizienz \parencite{bankinsMultilevelReviewArtificial2024} -- das subjektive Erleben von Menschen, die täglich mit generativer \gls{ki} arbeiten, gerät dabei in den Hintergrund. Dabei ist aus Sicht des Organizational Behavior längst bekannt, dass Motivation, Wohlbefinden und Leistungsfähigkeit untrennbar miteinander verbunden sind und wesentlich davon abhängen, wie Menschen ihre Arbeitsbedingungen erleben \parencite{deciSelfDeterminationTheoryWork2017}.

Besonders auffällig ist diese Lücke mit Blick auf Führungskräfte des mittleren Managements. Sie nehmen eine strukturell exponierte Stellung ein: Als strategische Übersetzer zwischen Unternehmensführung und operativer Ebene müssen sie \acrshort{ki}-gestützte Arbeitsprozesse nicht nur selbst navigieren, sondern auch in ihren Teams einführen und begleiten \parencite{floydManagingStrategicConsensus1997}. Gleichzeitig ist ihre Arbeit durch hohe Komplexität, Verantwortung und den Anspruch geprägt, Entscheidungen in einem Umfeld zu treffen, das Rechenschaftspflicht und fachliches Urteilsvermögen verlangt. In diesem Kontext verändert generative \gls{ki} die Bedingungen von Führungsarbeit grundlegend -- und damit potenziell auch das Erleben der eigenen Wirksamkeit, Autonomie und Eingebundenheit.

Für den Bankensektor im \acrshort{dach}-Raum gilt das in besonderer Weise. Entscheidungsprozesse sind dort durch regulatorische Anforderungen, Dokumentationspflichten und Verantwortungszurechnung geprägt -- ein Umfeld, in dem kognitive Unterstützungssysteme besonders sichtbare Spuren hinterlassen. Generative \gls{ki} wirkt hier nicht als neutrales Effizienzwerkzeug: Sie verändert, wer welchen Teil eines Entscheidungsprozesses verantwortet, wie Expertise wahrgenommen wird und in welchem Verhältnis menschliches Urteil und algorithmische Ausgabe zueinanderstehen. Ob diese Veränderungen als Entlastung oder als Kontrollverlust erlebt werden, ist nicht technologisch determiniert -- es hängt davon ab, wie Führungskräfte den Einsatz dieser Systeme deuten \parencite{edwardsManagerialControlFeedback2024, tongJanusFaceArtificial2021}.


Genau hier setzt die \gls{sdt} an. Sie beschreibt Motivation und Wohlbefinden als Funktionen dreier psychologischer Grundbedürfnisse: Autonomie -- das Erleben, Entscheidungen selbstbestimmt zu treffen; Kompetenz -- das Erleben, wirksam und fähig zu sein; und soziale Eingebundenheit -- das Erleben von Verbundenheit und Zugehörigkeit \parencite{deciWhatWhyGoal2000, deciSelfDeterminationTheoryWork2017}. Werden diese Bedürfnisse durch Arbeitsbedingungen unterstützt, entstehen Engagement und intrinsische Motivation; werden sie frustriert, folgen Rückzug, Erschöpfung und Amotivation \parencite{vandenbroeckReviewSelfDeterminationTheorys2016}. Generative KI kann, je nach Wahrnehmung und Implementierungskontext, in beide Richtungen wirken \parencite{gagneUnderstandingShapingFuture2022, klonekDoesAIWork2025}. Dieses ambivalente Potenzial macht sie zu einem theoretisch besonders interessanten Gegenstand für die \acrshort{sdt}-Forschung.



Die vorliegende Arbeit nimmt diese Spannung zum Ausgangspunkt. Sie fragt nicht, ob generative KI die Führungsarbeit effizienter macht -- das ist gut dokumentiert. Sie fragt, was dieser Technologieeinsatz mit Menschen macht, die täglich Entscheidungsverantwortung tragen.

\section{Zielsetzung und Forschungsfrage}

Ziel der Arbeit ist es zu verstehen, wie Führungskräfte im mittleren Management von Banken im \acrshort{dach}-Raum den Einsatz generativer \gls{ki} in Entscheidungsvorbereitungsprozessen motivational erleben und unter welchen Bedingungen diese Technologien als unterstützend oder einschränkend für die psychologischen Grundbedürfnisse nach Autonomie, Kompetenz und sozialer Eingebundenheit wahrgenommen werden.

Damit richtet sich die Arbeit gegen eine doppelte Verkürzung im bisherigen Forschungsstand: zum einen gegen die Reduktion von \acrshort{ki}-Wirkungen auf Effizienzmetriken, zum anderen gegen eine technologiedeterministische Betrachtung, die die deutungsabhängige Natur dieser Wirkungen ausblendet. Statt Hypothesen zu testen, verfolgt die Arbeit ein prozessuales Erkenntnisinteresse: Sie will aufzeigen, wie sich motivationale Wirkungen generativer KI in konkreten Führungspraktiken entfalten und welche situativen, organisationalen und individuellen Faktoren dabei eine Rolle spielen.

Die leitende Forschungsfrage lautet:

\begin{quote}
    \textit{Wie erleben Führungskräfte im mittleren Management von Banken den Einsatz generativer \gls{ki} in Entscheidungsvorbereitungsprozessen, und unter welchen Bedingungen wird diese Technologie als unterstützend oder einschränkend für Autonomie, Kompetenz und soziale Eingebundenheit wahrgenommen?}
\end{quote}

Diese Frage trägt zur Schließung mehrerer identifizierbarer Forschungslücken bei. Erstens fehlt es an theoretisch fundierten Untersuchungen, die die \gls{sdt} systematisch auf den Kontext generativer \gls{ki} anwenden -- obwohl das Framework dafür konzeptionell gut geeignet ist \parencite{mcanallySelfDeterminationTheoryWorkplace2024, gagneUnderstandingShapingFuture2022}. Zweitens ist die Gruppe der mittleren Führungskräfte in der \acrshort{ki}-Forschung trotz ihrer strategischen Bedeutung empirisch unterrepräsentiert. Drittens mangelt es an Studien, die das subjektive Erleben in realen organisationalen Kontexten erfassen, statt sich auf experimentelle Settings mit artifiziellen Aufgaben zu beschränken \parencite{bankinsMultilevelReviewArtificial2024}.



Der erwartete Beitrag der Arbeit ist dreifach: Theoretisch erweitert sie die Anwendung der SDT auf einen bislang wenig untersuchten technologischen Kontext. Empirisch gewinnt sie differenzierte Einblicke in das subjektive Erleben einer strategisch zentralen, in der Forschung jedoch unterbeschriebenen Gruppe. Praktisch liefert sie Orientierungspunkte für die motivationsgerechte Gestaltung generativer KI in Führungskontexten -- ein Aspekt, der in den meisten \acrshort{ki}-Implementierungsstrategien bislang zu kurz kommt \parencite{prasadGenerativeAICatalyst2024, quaquebekeNowNewNext2023}.


\section{Aufbau der Arbeit}

Die Arbeit gliedert sich in sechs Kapitel. Das vorliegende Einführungskapitel legt Problemstellung, Relevanz und Forschungsfrage dar.

Kapitel~2 entwickelt den theoretischen Rahmen. Es führt zunächst in generative \gls{ki} als soziotechnisches Arbeitssystem ein, beschreibt ihre Einsatzszenarien in wissensintensiven Organisationen und geht auf spezifische Entwicklungen im Bankensektor der \acrshort{dach}-Region ein. Anschließend wird Führungsarbeit im mittleren Management beleuchtet -- mit Fokus auf Entscheidungsarbeit als Kernaufgabe und den Besonderheiten des Bankenumfelds. Der dritte theoretische Abschnitt entfaltet die \gls{sdt}: Grundannahmen, die drei psychologischen Grundbedürfnisse, ihre Anwendung im Arbeitskontext und die Frage, wie digitale Arbeitssysteme bedürfnisunterstützend oder -frustrierend wirken können. Das Kapitel schließt mit einer theoretischen Synthese, die die motivationale Wirkung generativer KI in der Führungsarbeit konzeptionell rahmt.

Kapitel~3 beschreibt das methodische Vorgehen. Es begründet die Wahl eines qualitativen Forschungsdesigns, stellt den problemzentrierten Interviewansatz vor und erläutert Leitfadenentwicklung, Sampling und Durchführung der Erhebung. Daran schließen sich die Beschreibung des Transkriptionsverfahrens sowie die Darstellung der strukturierenden qualitativen Inhaltsanalyse nach Kuckartz an, die als Auswertungsmethode eingesetzt wird. Das Kapitel endet mit Überlegungen zu Gütekriterien und ethischen Aspekten der Untersuchung.

Kapitel~4 präsentiert die empirischen Ergebnisse. Nach einer Beschreibung der Interviewpartnerinnen und -partner werden die Befunde entlang der drei \acrshort{sdt}-Grundbedürfnisse strukturiert: Autonomieerleben, Kompetenzerleben und soziale Eingebundenheit im Kontext generativer \gls{ki}. Übergreifende Deutungsmuster und situative Bedingungen werden abschließend zusammengeführt.

Kapitel~5 diskutiert die Ergebnisse vor dem Hintergrund der \gls{sdt} und des Forschungsstands, benennt Implikationen für die Gestaltung von \gls{ki} in Führungskontexten und reflektiert die Limitationen der Studie.

Kapitel~6 fasst die zentralen Befunde zusammen, benennt den wissenschaftlichen Beitrag der Arbeit, formuliert praktische Handlungsempfehlungen und skizziert offene Fragen für zukünftige Forschung.