\section{Forschungsdesign und -paradigma}
\label{met:forschungsdesign}
Die vorliegende Arbeit folgt einem qualitativen Forschungsdesign. Diese Entscheidung ergibt sich aus der Natur der Forschungsfrage: Im Mittelpunkt steht das subjektive Erleben von Führungskräften im Umgang mit generativer \gls{ki} -- also Sinnzuschreibungen, Deutungsmuster und psychologische Wahrnehmungsprozesse, die sich einer standardisierten 
Messung grundsätzlich entziehen \parencite{flick2017, mayring2016}. Qualitative Methoden bieten sich dort an, wo soziale Phänomene aus der Innenperspektive der Beteiligten erschlossen werden sollen und wo der theoretische Zugang noch vergleichsweise offen ist \parencite{przyborski2021}.
Wissenschaftstheoretisch bewegt sich die Arbeit im interpretativen Paradigma. Ausgangspunkt ist die Überzeugung, dass soziale Wirklichkeit -- hier konkret: das motivationale Erleben von Führungsarbeit unter dem Einsatz generativer KI -- nicht einfach vorgefunden, sondern durch Wahrnehmungen und Deutungen der handelnden Personen erst hervorgebracht wird \parencite{berger1969}. In diesem konstruktivistischen Verständnis geht es nicht darum, ob generative KI die psychologischen Grundbedürfnisse nach Autonomie, Kompetenz und sozialer Eingebundenheit objektiv verändert. Interessanter ist die Frage, wie Führungskräfte solche Veränderungen wahrnehmen und welche Bedeutung sie ihnen in konkreten Handlungssituationen beimessen.
Die Entscheidung für ein qualitatives Design lässt sich auch mit dem Stand der Forschung begründen. Studien zu \gls{ki} in Organisationen setzen bislang überwiegend auf quantitative Designs und arbeiten mit aggregierten Effektmaßen \parencite{bankinsMultilevelReviewArtificial2024}. 
Was dabei weitgehend fehlt, sind Untersuchungen, die die prozessualen und situativen Bedingungen motivationaler Wirkungen in realen Führungskontexten nachzeichnen. Die vorliegende Arbeit versteht sich als explorativer Beitrag mit dem Ziel, theoretisch gehaltvolle Einsichten zu gewinnen -- ohne dabei den Anspruch statistischer Verallgemeinerung zu erheben \parencite{strauss1996}.