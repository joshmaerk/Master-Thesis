\section{Datenerhebung: Problemzentrierte Leitfadeninterviews}

Zur Erhebung der empirischen Daten wurden problemzentrierte Leitfadeninterviews eingesetzt. Diese Methode verbindet die Offenheit qualitativer Gesprächsführung mit einer thematischen Fokussierung auf ein vorab definiertes Problemfeld -- in diesem Fall 
das motivationale Erleben von Führungskräften im Umgang mit generativer \gls{ki} \parencite{witzelProblemcenteredInterview2000}. Im Vergleich zu vollständig narrativen oder unstrukturierten Formaten ermöglicht das problemzentrierte Interview eine gezieltere Erschließung theoretisch relevanter Inhaltsbereiche, ohne die Gesprächspartner in 
vorgegebene Antwortkategorien zu drängen \parencite{flick2017}.

\subsection{Interviewleitfaden und Entwicklungsprozess}

Der Interviewleitfaden wurde in mehreren Schritten entwickelt und dabei konsequent an der Forschungsfrage sowie den zentralen Konstrukten der Selbstbestimmungstheorie ausgerichtet \parencite{deciSelfDeterminationTheoryWork2017}. In einem ersten Schritt wurden auf Basis der theoretischen Vorüberlegungen thematische Blöcke formuliert, die die drei psychologischen Grundbedürfnisse -- Autonomie, Kompetenz und soziale 
Eingebundenheit -- als Strukturierungsprinzip aufgreifen. Ergänzt wurden diese durch einen einleitenden Block zur beruflichen Situation und zum konkreten \acrshort{ki}-Einsatz im Arbeitsalltag sowie durch einen abschließenden Block, der Raum für persönliche Bewertungen und weiterführende Gedanken ließ.

Die Fragen wurden bewusst offen formuliert, um Erzählimpulse zu setzen, anstatt Antwortrichtungen vorzugeben. Auf suggestive oder wertende Formulierungen wurde verzichtet. Nach einer ersten Fassung des Leitfadens erfolgte ein kognitives Pretest-Verfahren, in dem der Leitfaden hinsichtlich Verständlichkeit, Gesprächsfluss und thematischer Vollständigkeit überprüft wurde. Auf dieser 
Grundlage wurden einzelne Fragen umformuliert und die Reihenfolge der Blöcke angepasst. Der finale Leitfaden umfasst [X] Hauptfragen mit jeweils vorbereiteten Nachfragen und dient im Interview als Orientierungsrahmen, nicht als starres Skript \parencite{helfferich_qualitat_2011}.

\subsection{Sampling und Stichprobenbeschreibung}

Die Auswahl der Interviewpersonen folgt einer bewussten, theoriegeleiteten Samplinglogik. Da das Ziel der Arbeit nicht die statistische Repräsentativität, sondern die theoretische Sättigung relevanter Perspektiven ist, wurde ein purposives Sampling-Verfahren gewählt \parencite{flick2017}. Als Einschlusskriterien galten: 
eine Führungsposition im mittleren Management, eine Tätigkeit im Bankensektor der \acrshort{dach}-Region sowie nachweisliche Berührungspunkte mit generativer KI im beruflichen Kontext -- sei es durch aktive Nutzung oder durch die Begleitung entsprechender Implementierungsprozesse im eigenen Verantwortungsbereich.
Die Stichprobe umfasst [X] Führungskräfte aus [X] Instituten unterschiedlicher Größe und Ausrichtung, darunter Universalbanken, Sparkassen und genossenschaftliche Institute. Die Variation entlang dieser Dimensionen wurde bewusst angestrebt, um ein möglichst breites Spektrum organisationaler Kontexte abzubilden. Hinsichtlich 
Geschlecht, Alter und Führungsspanne weist die Stichprobe [kurze Beschreibung der Zusammensetzung] auf. Der Zugang zu den Interviewpersonen erfolgte über [Beschreibung des Zugangswegs, z.\,B. berufliche Netzwerke, direkte Ansprache, Verbandsstrukturen]. Alle Teilnehmenden wurden vorab schriftlich über Ziel, Ablauf und Datenschutz der Studie informiert und gaben ihr Einverständnis zur 
Aufzeichnung und wissenschaftlichen Auswertung der Gespräche.

\subsection{Durchführung und Aufzeichnung}
Die Interviews wurden im Zeitraum [Monat/Jahr -- Monat/Jahr] durchgeführt. Auf Wunsch der Teilnehmenden fanden [X] Gespräche als Videokonferenz statt, [X] wurden persönlich vor Ort geführt. Die Wahl des Settings oblag den Interviewpersonen, um eine möglichst vertraute und störungsarme Gesprächsatmosphäre zu gewährleisten. Die durchschnittliche Gesprächsdauer betrug [X] Minuten, bei einer Spanne von [X] bis [X] Minuten.

Alle Interviews wurden -- nach ausdrücklicher Zustimmung -- vollständig aufgezeichnet. Die Aufzeichnungen wurden anschließend nach einem einheitlichen Transkriptionsschema verschriftlicht \parencite{dresingTranskriptionenQualitativerDaten2017}. Zur Wahrung der Anonymität wurden Namen und identifizierende Angaben bereits im Transkript durch neutrale Kürzel ersetzt. Die fertigen Transkripte wurden den Interviewpersonen auf Wunsch zur Durchsicht und Freigabe zur Verfügung gestellt (Member Checking). Sämtliche Materialien werden gemäß den datenschutzrechtlichen Vorgaben aufbewahrt und nach Abschluss der Arbeit fristgerecht gelöscht.