\chapter{Theoretischer Hintergrund}

Die Frage, wie generative KI das motivationale Erleben von Führungskräften beeinflusst, lässt sich nicht aus einer einzelnen Disziplin heraus beantworten. Sie erfordert eine Verbindung technologischer, organisationaler und psychologischer Perspektiven. Das folgende Kapitel legt dieses theoretische Fundament in vier Schritten: Zunächst wird generative KI als soziotechnisches Arbeitssystem charakterisiert, das weit über ein reines Effizienzwerkzeug hinausgeht (Abschnitt~\ref{sec:genai}). Anschließend richtet sich der Blick auf die spezifische Arbeitssituation mittlerer Führungskräfte im Bankensektor, deren Entscheidungsarbeit den unmittelbaren Wirkungskontext der Technologie bildet (Abschnitt~\ref{sec:fuehrung}). Der dritte Abschnitt entfaltet die Self-Determination Theory als motivationspsychologischen Erklärungsrahmen (Abschnitt~\ref{sec:sdt}). Eine abschließende Synthese führt die drei Stränge zusammen und leitet die Forschungslücke ab, die diese Arbeit adressiert (Abschnitt~\ref{sec:synthese}).

\section{Generative KI als soziotechnisches Arbeitssystem}
\label{sec:genai}

Generative KI-Systeme sind in den vergangenen Jahren von einem Forschungsgegenstand zu einem Arbeitsmittel geworden, das kognitive Kernprozesse in Organisationen verändert. Der folgende Abschnitt grenzt den Begriff ab, ordnet die technologische Entwicklung ein und beleuchtet Einsatzszenarien in wissensintensiven Organisationen -- mit besonderem Blick auf den Bankensektor der DACH-Region.

\subsection{Begriffliche Abgrenzung und Entwicklung}

Generative Künstliche Intelligenz (GenAI) bezeichnet eine Klasse von KI-Systemen, die auf Basis umfangreicher Trainingsdaten neuartige Inhalte erzeugen können -- Texte, Bilder, Code, Audio oder multimodale Kombinationen. Der Begriff grenzt sich damit bewusst von früheren KI-Generationen ab, die primär klassifikatorisch oder regelbasiert operierten: Während ein Spam-Filter eingehende E-Mails in Kategorien sortiert, verfasst ein generatives Sprachmodell eigenständig Antworten, Analysen oder Entwürfe \parencite{brynjolfsson_generative_2023}.

Technologisch basieren die derzeit leistungsfähigsten generativen Systeme auf Large Language Models (LLMs). Diese neuronalen Netzwerke mit Milliarden von Parametern werden auf umfangreichen Textkorpora trainiert und nutzen Transformer-Architekturen, um probabilistische Vorhersagen über Textsequenzen zu treffen. Modelle wie GPT-4, Claude oder Gemini generieren auf dieser Grundlage kontextuell kohärente Outputs, die sich sprachlich kaum von menschlich verfassten Texten unterscheiden \parencite{brynjolfsson_generative_2023}. Die Veröffentlichung von ChatGPT im November 2022 markierte einen Wendepunkt: Innerhalb von zwei Monaten erreichte das System 100 Millionen aktive Nutzer und machte generative KI erstmals für breite Anwendergruppen in Organisationen zugänglich.

Was generative KI von traditionellen Informationssystemen -- etwa ERP-Systemen, Business-Intelligence-Dashboards oder Datenbanken -- unterscheidet, ist ihre \textit{generative Kapazität}. Herkömmliche Systeme speichern, verarbeiten und visualisieren vorhandene Informationen. GenAI erzeugt hingegen Inhalte, die so nicht explizit in den Trainingsdaten enthalten sind \parencite{bankinsMultilevelReviewArtificial2024}. Diese Eigenschaft eröffnet qualitativ neue Anwendungsszenarien: automatisierte Texterstellung, Szenarioanalysen, kreative Ideengenerierung und dialogische Interaktionen, bei denen das System auf Rückfragen und Kontextualisierungen reagiert.

Drei Merkmale charakterisieren generative KI-Systeme im organisationalen Kontext besonders. Erstens ihre \textit{probabilistische Kreativität}: Die Outputs sind nicht deterministisch, sondern variieren bei identischen Eingaben -- eine Eigenschaft, die sowohl kreatives Potenzial als auch Unsicherheit erzeugt. Zweitens ihre \textit{Kontextsensitivität}: Moderne LLMs halten Kontext über mehrteilige Dialoge hinweg und passen Antworten an spezifische Nutzerbedürfnisse an. Drittens ihre \textit{Anpassbarkeit}: Durch Fine-Tuning und Retrieval-Augmented Generation (RAG) lassen sich generative Systeme an organisationsspezifische Wissensbasen und Prozesse koppeln, während Nutzer über iteratives Prompting ihre Interaktionskompetenz weiterentwickeln \parencite{brynjolfsson_generative_2023, bankinsMultilevelReviewArtificial2024}.

Diese Merkmale machen GenAI zu einem soziotechnischen Phänomen im engeren Sinne: Das System entfaltet seine Wirkung nicht unabhängig von den Personen, die es nutzen, sondern in einer kontinuierlichen Wechselwirkung zwischen technologischen Möglichkeiten und menschlichen Praktiken. Wie Nutzer Prompts formulieren, welches Vertrauen sie in Outputs setzen, ob sie Ergebnisse kritisch prüfen oder unreflektiert übernehmen -- all das beeinflusst, welche Rolle GenAI in einer Organisation tatsächlich spielt \parencite{bankinsMultilevelReviewArtificial2024}.


\subsection{Einsatzszenarien in wissensintensiven Organisationen}

Generative KI findet primär dort Anwendung, wo Arbeit kognitiv anspruchsvoll, textbasiert und wissensintensiv ist: Dokumentenerstellung und -analyse, Entscheidungsvorbereitung, Strategieentwicklung, Kommunikation und Problemlösung \parencite{brynjolfsson_generative_2023}. Im Unterschied zu früheren Automatisierungswellen, die vorrangig manuelle und repetitive Tätigkeiten adressierten, greift die aktuelle Technologiewelle in den Kern professioneller Wissensarbeit ein.

Empirisch gut dokumentiert ist der Produktivitätseffekt. Eine Feldstudie mit über 5.000 Kundenservice-Mitarbeitenden ergab, dass der Einsatz eines generativen KI-Assistenten die Zahl gelöster Anfragen pro Stunde um 14\,\% steigerte. Aufschlussreich war die Verteilung dieses Effekts: Die Produktivitätsgewinne konzentrierten sich auf weniger erfahrene Mitarbeitende, während hochqualifizierte Experten kaum profitierten \parencite{brynjolfsson_generative_2023}. GenAI scheint demnach als eine Art Kompetenz-Augmentation zu fungieren -- sie hebt das Leistungsniveau weniger Erfahrener an, ohne die Leistung von Experten substanziell zu steigern.

\textcite{bankinsMultilevelReviewArtificial2024} identifizierten in einer Multilevel-Review fünf thematische Pfade, über die KI in Organisationen wirkt: (1) Mensch-KI-Kollaboration und Komplementarität, (2) Wahrnehmung von KI-Fähigkeiten und -Grenzen, (3) KI als Kontrollmechanismus im Sinne algorithmischen Managements, (4) Arbeitsmarktimplikationen wie Job Displacement und Skill Shifts sowie (5) ethische und soziale Implikationen. Diese Pfade interagieren über Analyseebenen hinweg und erzeugen häufig widersprüchliche Effekte -- ein Befund, der vereinfachende Narrative von KI als reinem Effizienzwerkzeug infrage stellt.

Die organisationale Einbettung von GenAI lässt sich entlang dreier Perspektiven konzeptualisieren \parencite{bankinsMultilevelReviewArtificial2024}. Aus \textit{instrumenteller} Sicht wird KI als Produktivitätswerkzeug verstanden: Sie automatisiert repetitive Aufgaben, beschleunigt Informationsverarbeitung und reduziert kognitive Belastung. Organisationen messen den Erfolg hier in Zeitersparnis, Kostenreduktion und Output-Steigerung. Diese Perspektive dominiert in frühen Adoptionsphasen, birgt aber das Risiko, motivationale und identitätsbezogene Effekte zu übersehen \parencite{edwards_managerial_2024}.

Die \textit{strategische} Perspektive geht einen Schritt weiter: KI liefert nicht nur Informationen, sondern generiert Empfehlungen, Prognosen und Handlungsalternativen, die in menschliche Entscheidungsprozesse einfließen. In einer Feldstudie mit Verkaufsmitarbeitenden untersuchten \textcite{tongJanusFaceArtificial2021} diese Integration und fanden einen bemerkenswerten Disclosure-Effekt: KI-basiertes Feedback verbesserte die Leistung, allerdings nur, solange Mitarbeitende nicht wussten, dass es von KI stammte. Die Offenlegung der algorithmischen Quelle reduzierte Akzeptanz und Wirksamkeit -- ein Hinweis darauf, dass strategische KI-Integration mehr erfordert als technische Funktionalität, nämlich Vertrauensaufbau und transparente Kommunikation.

Aus \textit{transformativer} Perspektive schließlich fungiert KI als Katalysator fundamentaler Veränderungen in Arbeitsrollen, Organisationsstrukturen und professionellen Identitäten. Mensch-KI-Kollaboration, hybride Teamstrukturen und algorithmisch vermittelte Koordination sind hier keine Ausnahme mehr, sondern alltägliche Praxis \parencite{bankinsMultilevelReviewArtificial2024}. Die Wahl der Perspektive beeinflusst maßgeblich, wie Mitarbeitende KI wahrnehmen und nutzen: Instrumentelle Framings erleichtern unter Umständen die Akzeptanz, lassen aber transformative Potenziale ungenutzt. Strategische Framings erhöhen Anforderungen an Transparenz und Erklärbarkeit. Transformative Implementierungen erfordern neue Rollen, Kompetenzen und Governance-Strukturen.

Quer zu diesen Perspektiven zeigt sich, dass die Wahrnehmung von KI durch Mitarbeitende nicht technologisch determiniert, sondern sozial konstruiert ist. Eine zentrale Dimension ist die Unterscheidung zwischen KI als Unterstützungs- und als Kontrollsystem \parencite{edwards_managerial_2024}. Dieselben Tools werden in verschiedenen Kontexten unterschiedlich interpretiert: als Arbeitserleichterung oder als Überwachungsinstrument, als Kompetenzerweiterung oder als Bedrohung professioneller Expertise \parencite{monod_worker_2024}. Vertrauen erweist sich dabei als zentraler Mediator. \textcite{prasad_generative_2024} zeigten in einer Studie mit 1.362 Beschäftigten, dass Vertrauen in GenAI die Akzeptanz KI-basierter Praktiken vollständig mediierte: Ohne Vertrauen in Zuverlässigkeit, Fairness und Transparenz blieb auch wahrgenommene Nützlichkeit wirkungslos.


\subsection{Generative KI im Bankensektor (DACH)}

Der Bankensektor zählt international zu den Branchen mit der höchsten GenAI-Adoptionsrate. Branchenerhebungen beziffern den Anteil der Finanzinstitute, die generative KI bereits einsetzen oder dies innerhalb von zwei Jahren planen, auf über 95\,\% \parencite{YourJourneyGenAI}. Typische Anwendungsfelder umfassen Marketing und Kundenkommunikation, Risikomanagement und Compliance, Kreditanalyse sowie interne Wissensmanagement-Prozesse. Für die globale Bankenbranche werden jährliche Produktivitätsgewinne von 200 bis 340 Milliarden US-Dollar prognostiziert \parencite{mckinsey__company_capturing_nodate}.

Im DACH-Raum ergibt sich ein spezifisches Bild, das durch drei Faktoren geprägt wird. Erstens ist der Bankensektor in Deutschland, Österreich und der Schweiz stark reguliert. Die Aufsichtsbehörden -- BaFin, FMA und FINMA -- stellen hohe Anforderungen an Transparenz, Erklärbarkeit und Nachvollziehbarkeit algorithmischer Entscheidungen, insbesondere in der Kreditvergabe und im Risikomanagement. Diese regulatorischen Rahmenbedingungen erzeugen ein Spannungsfeld: Einerseits begrenzen sie die Geschwindigkeit der KI-Adoption, andererseits zwingen sie Organisationen zu einer bewussteren Auseinandersetzung mit Governance-Fragen, die in weniger regulierten Branchen häufig nachgelagert behandelt werden \parencite{hundertmarkIFZGenerativeAI2024}.

Zweitens ist das DACH-Bankensystem durch eine hohe Dichte an Universalbanken, Genossenschaftsbanken und Sparkassen gekennzeichnet. Anders als in angelsächsischen Märkten, wo wenige Großbanken den technologischen Takt vorgeben, existieren im DACH-Raum zahlreiche mittelgroße Institute, deren Digitalisierungsgrad erheblich variiert. Für mittlere Führungskräfte in diesen Organisationen bedeutet dies, dass GenAI-Implementierung selten als Top-down-Projekt mit klarer strategischer Rahmung erfolgt, sondern häufig als bottom-up-getriebene Experimentation einzelner Teams oder Abteilungen.

Drittens unterscheidet sich die Arbeitskultur im DACH-Bankensektor in relevanter Weise. Die Tradition konsensualer Entscheidungsfindung, ausgeprägte Mitbestimmungsstrukturen (insbesondere in Deutschland und Österreich) sowie eine vergleichsweise hohe Bedeutung formaler Qualifikationen und Fachexpertise prägen das Umfeld, in das generative KI eingeführt wird. Wenn ein KI-System Kreditanalysen entwirft, die bislang erfahrene Spezialisten formuliert haben, berührt dies nicht nur Effizienzfragen, sondern auch professionelle Identität und die Legitimation von Expertise \parencite{quaquebeke_now_2023}.

Empirische Studien, die explizit die motivationalen Auswirkungen generativer KI auf Führungskräfte im DACH-Bankensektor untersuchen, liegen bislang nicht vor. Die vorhandene Forschung adressiert entweder den Bankensektor ohne spezifischen Fokus auf Führungsmotivation \parencite{bankinsMultilevelReviewArtificial2024} oder untersucht motivationale Effekte von KI ohne branchenspezifische Differenzierung \parencite{brynjolfsson_generative_2023, edwards_managerial_2024}. Diese Forschungslücke ist insofern bemerkenswert, als der DACH-Bankensektor aufgrund seiner regulatorischen Dichte, organisationalen Heterogenität und kulturellen Spezifika einen Kontext darstellt, in dem die motivationalen Spannungen der KI-Adoption besonders ausgeprägt sein dürften. Die vorliegende Arbeit adressiert dieses Desiderat durch eine qualitative Untersuchung, die das Erleben mittlerer Führungskräfte in diesem spezifischen Branchenkontext in den Mittelpunkt stellt.
\section{Führungsarbeit im mittleren Management}
\label{sec:fuehrung}

Mittlere Führungskräfte stehen organisational an einer Stelle, an der strategische Absicht und operative Wirklichkeit aufeinandertreffen. Ihre Arbeit ist geprägt von Entscheidungen unter Unsicherheit, der Vermittlung zwischen Hierarchieebenen und einem Tätigkeitsprofil, das generative \gls{ki} in besonderer Weise berührt. Der folgende Abschnitt beschreibt diese Arbeitssituation -- zunächst allgemein, dann mit Blick auf die spezifischen Bedingungen des Bankensektors.

\subsection{Entscheidungsarbeit als Kernaufgabe}

Mittlere Führungskräfte operieren an einer organisationalen Nahtstelle: zwischen strategischer Planung des Top-Managements und operativer Ausführung durch Frontline-Mitarbeitende. Diese Position ist weniger komfortabel, als es die Organigramme vermuten lassen. Wer hier arbeitet, muss strategische Vorgaben in operative Handlungen übersetzen, gleichzeitig aber Rückmeldungen und Widerstände der operativen Ebene nach oben kommunizieren -- oft unter Zeitdruck, mit unvollständigen Informationen und widersprüchlichen Erwartungen \parencite{floydManagingStrategicConsensus1997}.

\textcite{floydManagingStrategicConsensus1997} identifizierten vier strategische Rollen, die mittlere Führungskräfte einnehmen: \textit{Championing strategic alternatives} -- das Einbringen innovativer Ideen in strategische Entscheidungsprozesse; \textit{Synthesizing information} -- die Aggregation und Interpretation von Informationen aus verschiedenen organisationalen Quellen; \textit{Facilitating adaptability} -- die Förderung organisationaler Anpassungsfähigkeit; sowie \textit{Implementing deliberate strategy} -- die Umsetzung strategischer Entscheidungen in operative Praxis. Alle vier Rollen haben eines gemeinsam: Sie erfordern Entscheidungen. Nicht die großen, strategischen Richtungsentscheidungen, die dem Top-Management vorbehalten sind, sondern die unzähligen kleineren Urteile darüber, wie Strategie im Alltag konkret wird -- welche Prioritäten gesetzt, welche Informationen weitergegeben, welche Interpretationsspielräume genutzt werden.

Entscheidungsarbeit im mittleren Management ist dabei selten algorithmisch im Sinne klarer Wenn-dann-Regeln. Sie ist vielmehr geprägt von \textit{Sensemaking}: dem Versuch, aus mehrdeutigen Situationen tragfähige Handlungsgrundlagen abzuleiten \parencite{quaquebekeNowNewNext2023}. Mittlere Führungskräfte interpretieren strategische Vorgaben, kontextualisieren sie für ihre Teams, antizipieren Widerstände und justieren ihre Kommunikation entsprechend. Dieses Sensemaking ist nicht bloß ein kognitiver Prozess; es ist auch ein sozialer und emotionaler. Wer Transformation vermitteln soll, muss selbst verstanden haben, was sich verändert und warum -- und muss gleichzeitig mit der eigenen Unsicherheit umgehen können.

Genau hier wird die Einführung generativer KI relevant. Wenn ein Großteil der Entscheidungsarbeit mittlerer Führungskräfte darin besteht, Informationen zu synthetisieren, Optionen abzuwägen und Handlungsempfehlungen zu formulieren, dann adressiert \gls{genai} den Kern ihres Tätigkeitsprofils. \acrshort{ki}-Systeme können Daten schneller aggregieren, Entscheidungsvorlagen erstellen und Szenarien durchspielen. Ob diese Fähigkeit als Unterstützung oder als Bedrohung erlebt wird, hängt davon ab, wie Führungskräfte ihre eigene Rolle definieren -- und ob sie ihre professionelle Identität an den Prozess der Informationsverarbeitung oder an die Qualität des Urteils knüpfen \parencite{quaquebeke_now_2023}.

\textcite{quaquebeke_now_2023} argumentieren, dass \gls{ki} die Natur von Führung fundamental verschieben wird: weg von wissensbasierter Autorität hin zu facilitativer, emotional-intelligenter Führung. Führungskräfte müssen ihre Rolle neu verhandeln -- nicht als allwissende Experten, sondern als Kuratoren und Orchestratoren, die menschliche und algorithmische Ressourcen zusammenführen. Diese Neuverhandlung berührt alle drei psychologischen Grundbedürfnisse der Self-Determination Theory: Autonomie (Wer entscheidet -- Mensch oder Maschine?), Kompetenz (Wessen Expertise zählt noch?) und soziale Eingebundenheit (Wie verändert sich die Beziehung zum Team, wenn \gls{ki} Kommunikation mediiert?).

Hinzu kommt, dass mittlere Führungskräfte in Technologietransformationen eine paradoxe Doppelrolle einnehmen. Sie sollen als Change Agents die Adoption vorantreiben und gleichzeitig sind sie selbst Betroffene, deren Tätigkeitsprofile, Kompetenzen und professionelle Identität durch die neuen Tools transformiert werden \parencite{quaquebeke_now_2023}. Im Kontext generativer \gls{ki} verschärft sich dieses Paradox: Wer die Technologie in seinem Team implementieren soll, muss sie zunächst selbst in die eigene Arbeitsweise integrieren -- mit allen damit verbundenen Unsicherheiten über Verlässlichkeit, Grenzen und langfristige Konsequenzen für die eigene Position.

\textcite{koponen_work_2025} identifizierten durch eine systematische Literaturanalyse zentrale Arbeitscharakteristika, die mittlere Führungskräfte in \acrshort{ki}-integrierten Teams benötigen: Autonomie bei der Gestaltung von Mensch-\acrshort{ki}-Interaktionen, transparente Feedback-Mechanismen für menschliche und algorithmische Leistung, eine Balance zwischen Routine- und strategischen Aufgaben sowie soziale Unterstützung durch Peers und Vorgesetzte bei \acrshort{ki}-bezogenen Unsicherheiten. Fehlen diese Charakteristika, steigt das Risiko für Rollenkonflikte, Ambiguitätsstress und motivationale Erosion.


\subsection{Besonderheiten des Bankensektors}

Der Bankensektor unterscheidet sich von anderen wissensintensiven Branchen in mehreren Dimensionen, die für die motivationale Wirkung generativer \gls{ki} auf mittlere Führungskräfte unmittelbar relevant sind.

Am offensichtlichsten ist die \textit{regulatorische Dichte}. Banken im \acrshort{dach}-Raum unterliegen der Aufsicht durch \gls{bafin}, \gls{fma} und \gls{finma} sowie europäischen Regulierungsrahmen wie der \gls{crr} und der EU-\acrshort{ki}-Verordnung (AI Act). Für Führungskräfte im mittleren Management bedeutet dies, dass Entscheidungen selten in einem Freiraum getroffen werden, sondern innerhalb eng definierter Compliance-Korridore. Kreditentscheidungen folgen standardisierten Ratingprozessen, Beratungsgespräche werden dokumentationspflichtig geführt, Risikoeinschätzungen müssen nachvollziehbar begründet sein. Wenn generative \gls{ki} in diese Prozesse integriert wird, verschärfen sich die Anforderungen an Transparenz und Erklärbarkeit erheblich -- ein algorithmisch generierter Kreditvorschlag, dessen Zustandekommen nicht lückenlos nachvollziehbar ist, widerspricht den aufsichtsrechtlichen Grundprinzipien [Quelle einfügen\footnote{Hier wäre eine regulatorische Quelle sinnvoll, z.\,B. BaFin (2024): \textit{Maschinelles Lernen in der Finanzbranche -- Aufsichtliche Prinzipien}; oder EU AI Act, Art. 6 zu hochriskanten Anwendungen im Finanzsektor.}].

Zweitens prägt eine ausgeprägte \textit{Hierarchie- und Fachexpertisekultur} das mittlere Management im Bankensektor. Entscheidungsbefugnisse sind an formale Kompetenzstufen gebunden; Unterschriftsberechtigungen für Kreditvergaben beispielsweise sind nach Volumen und Risikokategorie gestaffelt. Fachexpertise -- etwa in Bilanzanalyse, Risikomodellierung oder regulatorischer Compliance -- hat traditionell hohen Stellenwert und legitimiert die Autorität mittlerer Führungskräfte gegenüber ihren Teams. Generative \gls{ki}, die in Sekunden Bilanzanalysen erstellt oder Compliance-Prüfungen automatisiert, berührt damit direkt die Grundlage, auf der professionelle Identität und Führungslegitimation aufgebaut sind \parencite{quaquebeke_now_2023}. Anders als in kreativeren Branchen, wo der Wert einer Idee unabhängig von ihrer Quelle beurteilt wird, knüpft der Bankensektor Entscheidungslegitimation eng an formale Expertise und hierarchische Position.

Drittens ist die \textit{Art der Entscheidungsarbeit} im Bankensektor spezifisch. Mittlere Führungskräfte treffen täglich Entscheidungen, die unmittelbare finanzielle Konsequenzen haben -- für die Bank, für Kunden, für regulatorische Kennzahlen. Im Firmenkundengeschäft beurteilen sie Kreditrisiken, im Private Banking beraten sie vermögende Kunden über Anlagestrategien, im Risikomanagement bewerten sie Portfolioexpositionen. Diese Entscheidungen verlangen zweierlei: solide analytische Kompetenz \textit{und} kontextabhängiges Urteilsvermögen, das formale Modelle ergänzt. Die Frage, ob ein langjähriger Firmenkunde mit temporär verschlechterter Bonität weiterhin Kredit erhält, ist eben nicht vollständig formalisierbar -- sie erfordert Kenntnis der Branche, der persönlichen Geschichte und der strategischen Beziehung. Genau in diesem Spannungsfeld zwischen formalisierbarer Analyse und nicht-formalisierbarem Urteil entfaltet generative \gls{ki} ihre ambivalente Wirkung.

Viertens kennzeichnet den \acrshort{dach}-Bankensektor eine \textit{strukturelle Heterogenität}, die sich auf die \acrshort{ki}-Adoption auswirkt. Neben international agierenden Großbanken mit dedizierten Digital-Innovation-Teams existieren zahlreiche Sparkassen, Genossenschaftsbanken und Regionalbanken, deren Digitalisierungsgrad und Ressourcenausstattung erheblich variieren. Für mittlere Führungskräfte in kleineren Instituten bedeutet \acrshort{genai}-Adoption häufig eine individuell getriebene Exploration ohne institutionelle Rahmung -- sie experimentieren eigenständig mit Tools, ohne klare organisationale Leitlinien zu Nutzung, Grenzen und Verantwortlichkeiten. In größeren Häusern wiederum wird \acrshort{ki}-Adoption als Top-down-Projekt implementiert, was Effizienzgewinne verspricht, aber Autonomiespielräume einschränken kann \parencite{edwards_managerial_2024}.

Schließlich ist der Bankensektor durch eine \textit{Vertrauenskultur} geprägt, die über das Kunden-Berater-Verhältnis hinausreicht. Vertrauen ist das Grundkapital des Bankgeschäfts -- Kunden vertrauen ihre finanziellen Ressourcen der Bank an, Mitarbeitende vertrauen auf die Integrität interner Prozesse, Aufsichtsbehörden vertrauen auf die Selbstregulierungsfähigkeit der Institute. Wenn generative \gls{ki} in Entscheidungsprozesse eintritt, stellt sich die Vertrauensfrage auf einer neuen Ebene: Können Führungskräfte \acrshort{ki}-Outputs so weit vertrauen, dass sie Entscheidungen darauf gründen? Können sie dieses Vertrauen gegenüber Kunden und Vorgesetzten begründen? \textcite{prasad_generative_2024} zeigten, dass Vertrauen den zentralen Mediator zwischen wahrgenommener \acrshort{ki}-Nützlichkeit und tatsächlicher Akzeptanz darstellt -- ein Befund, der im vertrauensintensiven Kontext des Bankwesens besondere Relevanz besitzt.

Diese Besonderheiten -- regulatorische Dichte, Fachexpertisekultur, spezifische Entscheidungscharakteristik, strukturelle Heterogenität und Vertrauensabhängigkeit -- erzeugen ein Spannungsfeld, in dem die motivationalen Effekte generativer \gls{ki} auf mittlere Führungskräfte vermutlich anders verlaufen als in weniger regulierten oder stärker technologieaffinen Branchen. Die vorliegende Arbeit nimmt diesen spezifischen Kontext als Untersuchungsfeld, um die Wechselwirkungen zwischen \acrshort{ki}-Nutzung und den psychologischen Grundbedürfnissen von Führungskräften empirisch zu explorieren.

\section{Self-Determination Theory (SDT)}
\label{sec:sdt}

Die bisherigen Abschnitte haben gezeigt, dass generative KI in Kernbereiche professioneller Wissensarbeit eingreift und dass mittlere Führungskräfte im Bankensektor davon in spezifischer Weise betroffen sind. Offen geblieben ist die Frage, über welche psychologischen Mechanismen sich diese Veränderungen auf das motivationale Erleben auswirken. Die Self-Determination Theory bietet dafür einen empirisch breit abgestützten Erklärungsrahmen, der Motivation nicht als Quantität, sondern als Qualität begreift -- und damit einen differenzierteren Zugang eröffnet als reine Akzeptanz- oder Leistungsmodelle.

\subsection{Grundannahmen und Entstehungskontext}

Die Self-Determination Theory (SDT), entwickelt von Edward L. Deci und Richard M. Ryan, ist eine Meta-Theorie menschlicher Motivation. Ihr Ausgangspunkt ist eine Annahme, die im Kontrast zu behavioristischen und rein ökonomischen Motivationsmodellen steht: Menschen sind nicht passive Rezipienten externer Anreize, sondern aktive, wachstumsorientierte Organismen mit einer intrinsischen Tendenz, ihre Umwelt zu explorieren, Kompetenzen zu entwickeln und soziale Beziehungen aufzubauen \parencite{deciWhatWhyGoal2000}. Ob diese Tendenz sich entfaltet oder verkümmert, hängt maßgeblich von den sozialen Kontexten ab, in denen Menschen handeln.

Was SDT von anderen Motivationstheorien unterscheidet, ist ihr Fokus auf die \textit{Qualität} der Motivation, nicht bloß deren Intensität. Zwei Personen können gleich viel Energie auf eine Aufgabe verwenden und dennoch fundamental unterschiedlich motiviert sein: die eine aus genuinem Interesse, die andere aus Angst vor negativen Konsequenzen. SDT argumentiert, dass diese Unterscheidung nicht trivial ist -- sie sagt systematisch vorher, wie nachhaltig, kreativ und gesundheitsförderlich das resultierende Verhalten ausfällt \parencite{deciSelfDeterminationTheoryWork2017}.

Konkret differenziert SDT zwischen \textit{intrinsischer Motivation} -- Verhalten, das um seiner selbst willen ausgeführt wird, aus Interesse und Freude -- und \textit{extrinsischer Motivation} -- Verhalten, das instrumentell auf separate Outcomes gerichtet ist. Der theoretische Beitrag liegt darin, extrinsische Motivation nicht pauschal als defizitär zu behandeln, sondern nach dem Grad der Internalisierung zu differenzieren \parencite{deciWhatWhyGoal2000}. Das resultierende \textit{Selbstbestimmungskontinuum} reicht von Amotivation (keine Handlungsintention) über kontrollierte Formen extrinsischer Motivation -- externale Regulation durch Belohnung und Bestrafung, introjizierte Regulation durch Schuld und Selbstwertdruck -- bis hin zu autonomen Formen: identifizierte Regulation (persönliche Anerkennung des Werts einer Handlung), integrierte Regulation (Übereinstimmung mit dem Selbstkonzept) und schließlich intrinsische Motivation.

Je autonomer die Motivationsform, desto günstiger die Konsequenzen: höheres Wohlbefinden, bessere Leistung, größere Persistenz und kreativeres Problemlösen \parencite{deciSelfDeterminationTheoryWork2017}. Das ist keine bloß theoretische Unterscheidung. In einer Validierungsstudie über sieben Sprachen und neun Länder (N\,>\,3.000) bestätigten \textcite{gagneMultidimensionalWorkMotivation2015} die faktorielle Struktur des Kontinuums und zeigten, dass autonome Motivationsformen konsistent positiv mit Leistung, Wohlbefinden und organisationalem Commitment assoziiert waren, während kontrollierte Motivation schwächere oder inkonsistente Effekte aufwies.

Der entscheidende Mechanismus, der erklärt, \textit{warum} bestimmte Kontexte autonome Motivation fördern und andere sie untergraben, liegt in der Befriedigung psychologischer Grundbedürfnisse -- dem Kernstück der Theorie.


\subsection{Die drei psychologischen Grundbedürfnisse}

SDT postuliert drei fundamentale psychologische Grundbedürfnisse, deren Befriedigung für psychologisches Wachstum, Integrität und Wohlbefinden essenziell ist: Autonomie, Kompetenz und soziale Eingebundenheit \parencite{deciWhatWhyGoal2000, vandenbroeckReviewSelfDeterminationTheorys2016}. Die Theorie versteht diese nicht als individuelle Präferenzen, die von Person zu Person variieren, sondern als universelle Nährstoffe -- vergleichbar mit physiologischen Bedürfnissen, deren Frustration unabhängig von kulturellem Kontext oder persönlicher Disposition zu Beeinträchtigungen führt.

\subsubsection{Autonomie}

Autonomie bezeichnet das Bedürfnis, sich als Ursprung des eigenen Verhaltens zu erleben -- selbstbestimmt und in Übereinstimmung mit den eigenen Werten zu handeln \parencite{deciWhatWhyGoal2000}. Der Begriff wird häufig missverstanden: Autonomie meint nicht Unabhängigkeit oder Isolation, sondern \textit{Volition}. Eine Führungskraft, die eine strategische Vorgabe umsetzt, handelt autonom, solange sie die Vorgabe als sinnvoll anerkennt und den Umsetzungsweg selbst gestalten kann. Dieselbe Vorgabe wird zum Autonomiefresser, wenn sie als willkürliche Kontrolle erlebt wird, der man sich fügen muss.

Autonomie wird gefördert durch Wahlmöglichkeiten, Partizipation an Entscheidungen, Bereitstellung von Rationalen und Minimierung von Kontrolle und Druck. Sie wird frustriert, wenn Verhalten durch externe Kräfte oder internalisierte Druckmechanismen kontrolliert wird \parencite{vandenbroeckReviewSelfDeterminationTheorys2016}. Für Führungskräfte ist Autonomie besonders relevant, da ihre Rolle traditionell mit Entscheidungsfreiheit und strategischem Gestaltungsspielraum assoziiert wird. Technologien, die Entscheidungsspielräume einschränken oder algorithmische Kontrolle ausüben, können daher Reaktanz erzeugen, die weit über das Maß hinausgeht, das bei operativen Mitarbeitenden zu beobachten wäre \parencite{edwardsManagerialControlFeedback2024}.

\subsubsection{Kompetenz}

Kompetenz bezeichnet das Bedürfnis, sich als effektiv und fähig zu erleben -- Herausforderungen erfolgreich zu meistern und kontinuierlich zu lernen \parencite{deciWhatWhyGoal2000}. Kompetenzerleben ist nicht identisch mit objektiver Kompetenz; es bezieht sich auf die subjektive Wahrnehmung von Wirksamkeit und Meisterschaft. Eine Führungskraft kann objektiv kompetent sein und sich dennoch inkompetent fühlen, wenn ein KI-System dieselbe Analyse in Sekunden erstellt, für die sie Stunden benötigt.

Optimales Kompetenzerleben entsteht, wenn Aufgaben weder zu einfach noch zu überfordernd sind, sondern im Bereich der optimalen Herausforderung liegen -- ein Konzept, das Parallelen zum Flow-Erleben aufweist \parencite{csikszentmihalyiFlowPsychologyOptimal2009}, aber breiter gefasst ist. Kompetenz wird gefördert durch klares, konstruktives Feedback, erreichbare aber herausfordernde Ziele und Gelegenheiten zur Kompetenzentwicklung \parencite{vandenbroeckReviewSelfDeterminationTheorys2016}. Sie wird frustriert durch Überforderung, intransparente Bewertungskriterien oder die Entwertung erworbener Expertise.

\subsubsection{Soziale Eingebundenheit}

Soziale Eingebundenheit (Relatedness) bezeichnet das Bedürfnis, sich mit anderen verbunden und zugehörig zu fühlen -- Beziehungen zu pflegen, die durch gegenseitige Fürsorge, Respekt und Vertrauen gekennzeichnet sind \parencite{deciWhatWhyGoal2000}. Im Arbeitskontext umfasst dies Zugehörigkeit zu Teams und Organisationen, unterstützende Beziehungen zu Kollegen und Vorgesetzten sowie das Gefühl, einen Beitrag zu einer größeren Gemeinschaft zu leisten \parencite{vandenbroeckReviewSelfDeterminationTheorys2016}.

Im Vergleich zu Autonomie und Kompetenz wird soziale Eingebundenheit in der SDT-Forschung zum Arbeitskontext manchmal als nachrangig behandelt. \textcite{deciSelfDeterminationTheoryWork2017} argumentieren jedoch, dass Relatedness eine notwendige Voraussetzung für nachhaltige Internalisierung extrinsischer Motivation darstellt: Menschen übernehmen Werte und Praktiken ihrer sozialen Umgebung eher, wenn sie sich dieser Umgebung zugehörig fühlen. Für Führungskräfte, deren Wirksamkeit wesentlich auf Beziehungsqualität beruht -- gegenüber Teams, Peers und Vorgesetzten --, ist dieses Bedürfnis keineswegs sekundär.


\subsection{SDT im Arbeitskontext}

SDT hat sich als produktiver theoretischer Rahmen für die Arbeits- und Organisationspsychologie etabliert, mit umfangreicher empirischer Evidenz zu Arbeitsmotivation, Leistung, Kreativität, Wohlbefinden, Burnout und organisationalem Commitment \parencite{deciSelfDeterminationTheoryWork2017, gagneUnderstandingShapingFuture2022}.

Der zentrale Wirkmechanismus ist gut belegt: Arbeitskontexte, die die drei Grundbedürfnisse befriedigen, fördern autonome Motivation, die wiederum positive Outcomes begünstigt. Meta-Analysen zeigen konsistent, dass Bedürfnisbefriedigung am Arbeitsplatz positiv mit Wohlbefinden und negativ mit Burnout und Ill-Being assoziiert ist -- robust über Kulturen, Berufe und Messmethoden hinweg \parencite{vandenbroeckReviewSelfDeterminationTheorys2016}. Bedürfnisbefriedigung fördert dabei nicht nur Wohlbefinden, sondern auch Leistung und Engagement: \textcite{vandenbroeckReviewSelfDeterminationTheorys2016} fanden signifikante positive Zusammenhänge mit Job Performance, organisationalem Commitment und proaktivem Verhalten.

Konzeptionell und empirisch bedeutsam ist die Unterscheidung zwischen Bedürfnisbefriedigung (need satisfaction) und Bedürfnisfrustration (need frustration). Frustration tritt auf, wenn Bedürfnisse aktiv blockiert oder untergraben werden -- nicht bloß abwesend sind \parencite{vandenbroeckCapturingAutonomyCompetence2010}. Dieser Unterschied ist nicht akademisch: Bedürfnisfrustration erweist sich als stärkerer Prädiktor für negative Outcomes (Burnout, kontraproduktives Verhalten) als Bedürfnisbefriedigung für positive \parencite{vandenbroeckReviewSelfDeterminationTheorys2016}. Die Asymmetrie hat praktische Implikationen: Technologien, die Bedürfnisse aktiv frustrieren -- etwa durch Kontrollerleben oder Expertiseentwertung --, dürften gravierendere motivationale Konsequenzen haben als Technologien, die Bedürfnisse lediglich nicht fördern.

Ein zentrales Konzept der angewandten SDT-Forschung ist \textit{Autonomy Support} -- Führungsverhalten und organisationale Praktiken, die Autonomie, Kompetenz und Relatedness fördern. Autonomieunterstützung umfasst: Perspektiven anerkennen, Wahlmöglichkeiten bieten, Rationale bereitstellen und Kontrolle minimieren \parencite{deciSelfDeterminationTheoryWork2017}. Meta-Analysen bestätigen, dass autonomieunterstützende Führung signifikant positiv mit Bedürfnisbefriedigung, autonomer Motivation und Leistung assoziiert ist, während kontrollierende Führung mit Bedürfnisfrustration und negativen Outcomes einhergeht \parencite{vandenbroeckReviewSelfDeterminationTheorys2016}.

Für die vorliegende Arbeit ist dieser Befund in doppelter Hinsicht relevant. Erstens sind mittlere Führungskräfte selbst Empfänger von Autonomieunterstützung (oder -kontrolle) durch ihre Vorgesetzten und durch organisationale Strukturen -- zu denen zunehmend auch algorithmische Systeme gehören. Zweitens sind sie Gestalter von Autonomieunterstützung für ihre Teams, wobei KI-Tools diese Gestaltungsarbeit sowohl erleichtern als auch erschweren können.


\subsection{Digitale Arbeitssysteme als bedürfnisunterstützende oder -frustrierende Faktoren}

SDT wurde zunehmend auf Technologieakzeptanz und -nutzung angewendet, häufig in Integration mit dem Technology Acceptance Model (TAM) und der Unified Theory of Acceptance and Use of Technology (UTAUT) \parencite{KoenigPascal2024, YangHsiHsun2024}. Die Kernfrage lautet: Unter welchen Bedingungen unterstützen oder frustrieren digitale Arbeitssysteme die drei psychologischen Grundbedürfnisse?

SDT erweitert rein kognitive Akzeptanzmodelle um eine motivationale Erklärungsebene. TAM und UTAUT postulieren, dass wahrgenommene Nützlichkeit und Benutzerfreundlichkeit die Nutzungsintention bestimmen. SDT fragt darüber hinaus: \textit{Wie} wird Technologie erlebt? Fördert sie das Gefühl von Selbstbestimmung, oder erzeugt sie Kontrollerleben? Stärkt sie die eigene Wirksamkeit, oder untergräbt sie professionelle Identität? Verbindet sie Menschen, oder isoliert sie? \parencite{KoenigPascal2024}

Empirisch zeigen sich ambivalente Befunde. Auf der einen Seite können digitale Arbeitssysteme Autonomie erweitern (etwa durch flexible Arbeitsorganisation), Kompetenz fördern (durch Zugang zu Informationen und Feedback-Systemen) und Relatedness stärken (durch Kommunikationstools und kollaborative Plattformen) \parencite{gagneUnderstandingShapingFuture2022}. Auf der anderen Seite dokumentiert die Forschung konsistent auch negative Pfade: Automatisierung kann wahrgenommene Arbeitsbedeutsamkeit und Autonomie reduzieren -- eine Studie fand bei einer 7,5-fachen Steigerung der Robotisierung projizierte Rückgänge von 6,8\,\% bei Bedeutsamkeit und 7,5\,\% bei Autonomie \parencite{nikolovaRobotsMeaningSelfdetermination2024}. Wahrgenommene Technologieunsicherheit fungiert als Demand, die Technostress erhöht und Arbeitszufriedenheit reduziert \parencite{LiuWangLin2023}.

\textcite{KoenigPascal2024} entwickelten ein theoretisches Framework, das drei Akzeptanzperspektiven integriert: Nutzerakzeptanz, Delegationsakzeptanz und gesellschaftliche Adoption. Für jede Perspektive identifizierten sie spezifische SDT-relevante Faktoren: Nutzerakzeptanz hängt primär von Kompetenz- und Autonomieerleben ab; Delegationsakzeptanz von Vertrauen und wahrgenommener Zuverlässigkeit; gesellschaftliche Adoption von kollektiven Werten und sozialen Normen.

Speziell für generative KI zeigt eine Studie mit 565 Designprofessionellen die Erklärungskraft eines integrierten UTAUT-SDT-Modells: Es klärte 52,1\,\% der Varianz in der Verhaltensintention auf. Bemerkenswert war, dass die wahrgenommene Bedrohung durch Jobersatz die Beziehung zwischen Leistungserwartung und Nutzungsintention negativ moderierte -- selbst Personen, die die Leistungsfähigkeit der Tools anerkannten, zeigten geringere Nutzungsbereitschaft, wenn sie KI als Bedrohung ihrer beruflichen Existenz wahrnahmen \parencite{YangHsiHsun2024}.

Besonders aufschlussreich für die vorliegende Arbeit ist ein experimenteller Befund von \textcite{WuLiuRuan2025}. In einer Studie mit 15.105 Teilnehmenden steigerte GenAI-Unterstützung die Aufgabenleistung signifikant, reduzierte aber gleichzeitig die intrinsische Motivation. Der Mechanismus: Nutzer attribuierten den Erfolg der KI statt sich selbst und erlebten dadurch reduzierte Selbstwirksamkeit. Kompetenzerleben wurde also nicht durch objektiven Misserfolg frustriert, sondern durch die Verschiebung der Erfolgsattribution -- ein Paradox, das für Führungskräfte besonders relevant sein dürfte, deren professionelle Identität eng an persönliche Expertise geknüpft ist.

Trotz dieser Fortschritte bleibt die Anwendung von SDT auf generative KI im Arbeitskontext begrenzt. Generative KI unterscheidet sich von den bisher untersuchten Technologien durch drei Eigenschaften, die theoretische Erweiterungen erfordern \parencite{gagneUnderstandingShapingFuture2022}: Erstens ihre \textit{Ko-Kreationsfähigkeit} -- GenAI ist nicht bloß Werkzeug, sondern Dialogpartner, der Inhalte mitgestaltet. Zweitens ihre \textit{Kontextsensitivität} -- die Qualität der Interaktion hängt von der Kompetenz der nutzenden Person ab (Prompt Engineering), was eine rekursive Beziehung zwischen Technologienutzung und Kompetenzerleben erzeugt. Drittens ihr \textit{Potenzial zur Identitätsbedrohung} -- GenAI kann Kernaufgaben professioneller Wissensarbeit substituieren, nicht nur Routinetätigkeiten.

Diese drei Eigenschaften machen generative KI zu einem Untersuchungsgegenstand, der über die bisherige SDT-Technologieforschung hinausgeht und spezifische empirische Zugänge erfordert -- eine Lücke, die die vorliegende Arbeit qualitativ zu adressieren versucht.

\subsection{Grenzen und Randbedingungen der SDT im Kontext organisationaler KI-Forschung}
\label{subsec:sdt-grenzen}

Die bisherige Darstellung hat die Self-Determination Theory als tragfähigen Rahmen für die Analyse motivationaler Wirkungen digitaler Arbeitssysteme eingeführt. Bevor dieser Rahmen auf den spezifischen Untersuchungsgegenstand -- generative KI in der Führungsarbeit -- angewendet wird, verdienen einige theoretische Grenzen Beachtung. Sie schmälern nicht die Eignung der SDT für die vorliegende Fragestellung, verweisen aber auf Bereiche, in denen die Theorie unterspezifiziert bleibt und explorative Zugänge erfordert.

\paragraph{Kulturelle Universalität und der Autonomiebegriff.}
SDT postuliert die drei psychologischen Grundbedürfnisse als universell -- eine Annahme, die wiederholt auf kulturelle Einwände gestoßen ist. Chirkov et al. (2003) zeigten zwar, dass Autonomie auch in kollektivistischen Kontexten motivational wirksam bleibt, wiesen aber zugleich nach, dass die \textit{Art und Weise}, wie Autonomie erlebt und ausgedrückt wird, kulturell variiert.\footnote{%
  Chirkov, V., Ryan, R. M., Kim, Y. \& Kaplan, U. (2003). Differentiating autonomy from individualism and independence: A self-determination theory perspective on internalization of cultural orientations and well-being. \textit{Journal of Personality and Social Psychology}, \textit{84}(1), 97--110. \url{https://doi.org/10.1037/0022-3514.84.1.97} -- [Quelle in Zotero ergänzen]}
Für die vorliegende Arbeit ist das keineswegs trivial: Die DACH-Region zeichnet sich durch eine ausgeprägte Konsens- und Mitbestimmungskultur aus, in der Autonomie weniger als individuelle Unabhängigkeit, sondern als Partizipation an Entscheidungsprozessen verstanden werden dürfte. Ob und wie Führungskräfte in diesem Kontext die KI-induzierte Verschiebung von Entscheidungsstrukturen als autonomiefördernd oder -einschränkend erleben, lässt sich aus der SDT allein nicht ableiten -- hier braucht es empirische Rekonstruktion.

\paragraph{Korrelative Evidenzbasis.}
McAnally und Hagger (2024) haben in einem konzeptionellen Review drei zentrale methodische Schwächen der arbeitspsychologischen SDT-Forschung identifiziert: eine Dominanz querschnittlicher Designs, den Mangel an Interventionsstudien mit explizitem Mediationstest und die unzureichende Berücksichtigung moderierender Variablen \parencite[vgl.][]{mcanallySelfdeterminationTheoryWorkplace2024}. Die allermeisten Befunde zur Bedürfnisbefriedigung am Arbeitsplatz beruhen auf korrelativen Daten, die keine kausalen Schlüsse erlauben. Auch die Debatte um den sogenannten Undermining-Effekt -- die These, dass extrinsische Anreize intrinsische Motivation verdrängen -- wird durch widersprüchliche Befunde aus Labor- und Feldstudien verkompliziert. Saini, Uppal und Howard (2025) zeigten in einer aktuellen Metaanalyse, dass der Effekt unter kontrollierten Laborbedingungen robust auftritt, in Feldstudien mit realen Arbeitsbedingungen jedoch deutlich schwächer und inkonsistenter ausfällt.\footnote{%
  Saini, G. K., Uppal, N. \& Howard, J. L. (2025). The undermining effect of extrinsic rewards: Fact or artefact? \textit{Journal of Occupational and Organizational Psychology}, \textit{98}, e70000. -- [Quelle in Zotero ergänzen]}
Für die vorliegende Studie folgt daraus: Die SDT liefert ein konzeptionell überzeugendes, aber empirisch noch unvollständig gesichertes Erklärungsmodell. Gerade deshalb erscheint ein qualitativer Zugang sinnvoll, der Wirkzusammenhänge nicht voraussetzt, sondern rekonstruiert.

\paragraph{Fehlende Identitäts- und Institutionenperspektive.}
Eine auffällige Leerstelle der SDT betrifft die Rolle professioneller Identität. Die vorangegangenen Abschnitte haben gezeigt, dass generative KI Kompetenzerleben nicht nur über Aufgabenbewältigung, sondern auch über die wahrgenommene Entwertung beruflicher Expertise beeinflusst. Dieser Mechanismus -- die Bedrohung des Selbstverständnisses als fachliche Autorität -- wird von der SDT nicht abgebildet. Sie modelliert Kompetenz als funktionales Erleben von Wirksamkeit, nicht als identitätsgebundene Zuschreibung. Ähnlich verhält es sich mit institutionellen Rahmenbedingungen: Der Bankensektor ist durch eine dichte regulatorische Umgebung geprägt, die Handlungsspielräume nicht nur faktisch einschränkt, sondern auch normativ rahmt. SDT verfügt über keinen Mechanismus, der erklärt, wie regulatorischer Druck die Bedürfnisbefriedigung moderiert -- ob etwa Compliance-Anforderungen als externe Kontrolle (und damit autonomiefrustrierend) oder als professionelle Selbstverständlichkeit (und damit motivational neutral) erlebt werden, hängt von Kontextfaktoren ab, die außerhalb der Reichweite der Theorie liegen.

\paragraph{Teamebene als blinder Fleck.}
Grenier et al. (2024) haben darauf hingewiesen, dass die SDT trotz ihrer breiten Rezeption in der Arbeitspsychologie auf Teamebene kaum systematisch angewendet wurde.\footnote{%
  Grenier, S., Bhatt, M. \& Bhimani, R. (2024). A systematic review of self-determination theory in teams. \textit{Applied Psychology}, \textit{73}(3), 1286--1312. \url{https://doi.org/10.1111/apps.12526} -- [Quelle in Zotero ergänzen]}
Das Bedürfnis nach sozialer Eingebundenheit wird vorwiegend als dyadisches oder individuelles Konstrukt operationalisiert; wie sich Teamdynamiken -- etwa geteilte Verantwortung, kollektive Kompetenzzuschreibung oder die soziale Aushandlung von KI-Nutzung -- auf die Bedürfnisbefriedigung auswirken, bleibt unterbestimmt. Für Führungskräfte im mittleren Management, deren Arbeit wesentlich durch die Koordination von Teams und die Vermittlung zwischen Hierarchieebenen geprägt ist, stellt diese Lücke eine relevante Einschränkung dar.

\paragraph{Konsequenzen für das Forschungsdesign.}
Diese Grenzen entwerten die SDT nicht als Analyserahmen -- sie präzisieren den Anspruch, mit dem die Theorie in dieser Arbeit eingesetzt wird. Die drei psychologischen Grundbedürfnisse strukturieren die Analyse als \textit{heuristische Kategorien}, nicht als kausaltheoretisches Erklärungsmodell. Gerade weil die SDT an den beschriebenen Stellen unterspezifiziert ist, braucht es einen explorativen Zugang, der empirisch rekonstruiert, wie Führungskräfte die motivationalen Wirkungen generativer KI tatsächlich erleben und deuten. Die qualitative Methodik dieser Arbeit -- problemzentrierte Interviews mit induktiver Subkategorienbildung (vgl. Kapitel~3) -- ist darauf angelegt, genau diese Leerstellen zu adressieren, indem sie Phänomene sichtbar macht, die deduktiv aus der SDT allein nicht ableitbar wären.
\section{Synthese: Motivationale Wirkung von KI in der Führungsarbeit}

Die vorangegangenen Abschnitte haben drei Perspektiven entfaltet, die nun zusammengeführt werden: generative KI als soziotechnisches System mit spezifischer Eingriffstiefe in wissensintensive Arbeit (Abschnitt 2.1), Führungsarbeit im mittleren Management des Bankensektors mit ihrer charakteristischen Verknüpfung von Entscheidungskompetenz, Expertiseidentität und regulatorischer Einbettung (Abschnitt 2.2) sowie die Self-Determination Theory als motivationspsychologischer Erklärungsrahmen (Abschnitt 2.3). Der vorliegende Abschnitt integriert diese Stränge zu einem konzeptionellen Analyserahmen, identifiziert die zentrale Forschungslücke und formuliert die Fragestellung der Arbeit.

\subsection{Generative KI und Autonomieerleben.}

Das Autonomiebedürfnis mittlerer Führungskräfte wird durch generative KI auf mindestens zwei Wegen berührt. Erstens kann KI als Ermöglicherin von Autonomie wirken: Sie entlastet von Routineaufgaben, beschleunigt Informationssynthesen und erweitert den Handlungsspielraum für strategische Gestaltung. Zweitens kann sie als subtiles Kontrollsystem erfahren werden -- dann nämlich, wenn algorithmische Empfehlungen faktisch den Entscheidungskorridor verengen, wenn Transparenzanforderungen die Begründungslast erhöhen oder wenn die Organisation KI-gestützte Prozesse als Steuerungsinstrument implementiert.

Die empirische Forschung stützt diese Ambivalenz. \textcite{edwards_managerial_2024} zeigten in einer Drei-Wellen-Studie (N\,=\,401), dass die motivationalen Konsequenzen algorithmischer HR-Systeme davon abhängen, wie Beschäftigte deren Zweck attribuieren: Wer das System als Kontrollinstrument der Führung wahrnahm, erlebte erhöhte extrinsische Motivation und priorisierte selektiv messbare Aufgaben. Wer es als Feedback-Instrument deutete, berichtete höhere intrinsische Motivation und geringere emotionale Erschöpfung. Die Attributionsrichtung -- Kontrolle oder Unterstützung -- erwies sich als entscheidender als das System selbst. \textcite{klonek_does_2025} bestätigten dieses Muster in einer Text-Mining-Studie mit über 2.700 Beiträgen: Hohe KI-Kontrolle war signifikant mit erhöhtem Stress assoziiert, während Mensch-KI-Teamprozesse -- insbesondere aktions- und interpersonenbezogene -- Stress reduzierten und den negativen Effekt der Kontrolle abpufferten.

Für mittlere Führungskräfte im Bankensektor verschärft sich diese Dynamik durch die regulatorische Struktur. Compliance-Anforderungen erzeugen bereits ohne KI enge Entscheidungskorridore; GenAI-Systeme, die regulatorisch gebotene Analysen automatisieren, können die erlebte Autonomie weiter einschränken, selbst wenn sie objektiv Entlastung bieten. Die Frage ist nicht, ob Autonomie tangiert wird, sondern wie Führungskräfte die Verschiebung erleben und deuten.

\subsection{Generative KI und Kompetenzerleben.}

Das Kompetenzerleben mittlerer Führungskräfte ist im Kontext generativer KI besonders verwundbar. Im Unterschied zu klassischen Automatisierungstechnologien, die primär manuelle oder standardisierte Aufgaben betreffen, greift GenAI in Kernbereiche professioneller Wissensarbeit ein: Analysen erstellen, Strategien entwerfen, Kommunikation gestalten \parencite{brynjolfssonGenerativeAIWork2025}. Das sind genau die Tätigkeiten, über die mittlere Führungskräfte im Bankensektor traditionell ihre professionelle Identität definieren.

Die experimentelle Studie von \textcite{WuLiuRuan2025} (N\,=\,15.105) hat den Mechanismus aufgedeckt: GenAI-Unterstützung steigerte die Aufgabenleistung, reduzierte aber gleichzeitig die intrinsische Motivation -- weil Teilnehmende den Erfolg der KI statt sich selbst attribuierten. Kompetenzerleben wurde nicht durch Scheitern frustriert, sondern durch den Verlust der Erfolgsattribution. Für Führungskräfte, deren Wirksamkeitserleben eng an persönliche Expertise geknüpft ist, dürfte dieser Mechanismus besonders ausgeprägt sein.

Gleichzeitig eröffnet GenAI neue Kompetenzdomänen. Die Fähigkeit, effektive Prompts zu formulieren, KI-Outputs kritisch zu evaluieren und Mensch-KI-Zusammenarbeit zu gestalten, wird zunehmend als eigenständige professionelle Kompetenz verstanden \parencite{dellacquaNavigatingJaggedTechnological2023}. \textcite{smith_navigating_2025} zeigten in vier experimentellen Studien (N\,$\approx$\,1.100), dass die freiwillige Einholung von KI-Empfehlungen -- im Vergleich zur erzwungenen Konfrontation -- die Bereitschaft signifikant erhöhte, KI-Ratschläge zu akzeptieren. Die Wahlfreiheit erzeugte ein Erleben von Kompetenz und Kontrolle über die Interaktion, das die SDT-Logik direkt bestätigte: Optionale KI-Nutzung stützt sowohl Autonomie- als auch Kompetenzerleben, obligatorische untergräbt beides.

Das resultierende Spannungsfeld ist für den Bankensektor besonders ausgeprägt. Die Entscheidungsarbeit mittlerer Führungskräfte verbindet dort analytische Kompetenz mit kontextuellem Erfahrungswissen -- etwa bei Kreditentscheidungen, in denen formale Risikomodelle und langjährige Kundenkenntnis zusammenfließen. Wenn GenAI die analytische Komponente substituiert, stellt sich die Frage, wie das verbleibende Erfahrungswissen erlebt wird: als aufgewertete Kernkompetenz oder als Residualkategorie.

\subsection{Generative KI und soziale Eingebundenheit.}

Die Auswirkungen generativer KI auf soziale Eingebundenheit sind bislang am wenigsten empirisch untersucht, theoretisch aber plausibel. Drei Wirkpfade lassen sich antizipieren.

Erstens verändert GenAI die Interaktionsstruktur in Teams. Wenn Aufgaben, die bisher kooperativ gelöst wurden -- Berichterstellung, Datenanalyse, Entscheidungsvorbereitung -- nun einzeln an KI-Systeme delegiert werden, reduziert sich die aufgabenbezogene Interdependenz. \textcite{klonek_does_2025} argumentierten, dass gerade interpersonale Mensch-KI-Teamprozesse als soziale Ressource wirken, die Stress reduziert. Fehlen diese Prozesse, entfällt auch der Puffer.

Zweitens kann sich das Verhältnis zur eigenen Führungsrolle verschieben. \textcite{quaquebeke_now_2023} argumentierten, dass Führung in KI-Kontexten zunehmend weniger über Informationsvorsprung und Expertise definiert wird und stärker über Beziehungsqualität und Sinngebung. Das birgt Chancen -- eine Intensivierung genuin menschlicher Führungsarbeit --, setzt aber voraus, dass Führungskräfte diese Rollenverschiebung nicht als Kompetenzverlust, sondern als Rollenerweiterung erleben.

Drittens berührt KI die wahrgenommene organisationale Wertschätzung. \textcite{lagios_explaining_2022} zeigten in einer Zwei-Wellen-Studie (N\,=\,603), dass organisationale Dehumanisierung -- das Erleben, als bloßes Werkzeug behandelt zu werden -- alle drei psychologischen Grundbedürfnisse frustriert, mit negativen Konsequenzen für Wohlbefinden, Zufriedenheit und Bindung. Wenn Organisationen GenAI primär als Effizienzinstrument implementieren, ohne die Perspektive der Führungskräfte einzubeziehen, besteht das Risiko, dass sich deren Wertschätzungserleben verschlechtert -- nicht weil die Technologie inhärent dehumanisierend wirkt, sondern weil der Implementierungsprozess es ist.

\subsection{Das Paradox der Entlastung.}

Die Synthese der empirischen Befunde offenbart ein zentrales Paradox. Generative KI verspricht Führungskräften Entlastung von Routineaufgaben und informationeller Überflutung -- eine Ressource im Sinne der JD-R-Theorie, die Demands reduziert und Handlungsspielräume erweitert \parencite{bakker_job_2007}. Gleichzeitig greift sie in genau jene Tätigkeiten ein, über die Führungskräfte Kompetenz, Autonomie und Identität definieren. Die Entlastung kann zum Kontrollverlust werden, die Effizienzsteigerung zur Expertiseentwertung, die Prozessoptimierung zur Beziehungsverarmung.

Ob das eine oder das andere eintritt, ist -- das legen die Befunde nahe -- kein Determinismus der Technologie. \textcite{tongJanusFaceArtificial2021} beschrieben KI treffend als Janusgesicht: bedrohlich und vielversprechend zugleich. Die motivationale Wirkung hängt von einer Konfiguration aus individuellen, organisationalen und technologischen Faktoren ab \parencite{bankinsMultilevelReviewArtificial2024}. SDT bietet einen Erklärungsrahmen, der über die binäre Logik von Akzeptanz oder Ablehnung hinausgeht, indem sie die Mechanismen offenlegt: Nicht die Technologienutzung per se, sondern deren Wirkung auf Autonomie, Kompetenz und soziale Eingebundenheit bestimmt, ob GenAI als motivationale Ressource oder als Belastung erlebt wird.

\subsection{Forschungslücke und Fragestellung.}

Trotz der wachsenden Literatur zu KI im Arbeitskontext bestehen substanzielle Lücken, die den Ausgangspunkt dieser Arbeit bilden.

Die bestehende empirische Forschung ist überwiegend quantitativ und erfasst breit angelegte Zusammenhänge zwischen KI-Nutzung und motivationalen oder leistungsbezogenen Outcomes -- typischerweise über standardisierte Fragebögen, die aggregierte Bewertungen abbilden \parencite{WuLiuRuan2025, prasad_generative_2024, YangHsiHsun2024}. Was diese Studien konzeptionell nicht leisten können, ist die Rekonstruktion der subjektiven Erfahrungswelt: Wie erleben Führungskräfte den konkreten Moment, in dem ein KI-System eine Analyse erstellt, die sie bisher selbst verantwortet haben? Welche Deutungsmuster mobilisieren sie, um ihre Rolle im Verhältnis zur Technologie zu verorten? Wie verschiebt sich ihr Erleben über die Zeit?

Zudem fokussiert die Mehrzahl der Studien auf operative Beschäftigte oder breite Populationen -- nicht auf mittleres Management, dessen spezifische Positionierung zwischen strategischer Planung und operativer Umsetzung eine besondere Betroffenheit durch GenAI begründet \parencite{klonekDoesAIWork2025}. Die wenigen Studien, die Führungskräfte explizit adressieren, untersuchen deren Umgang mit KI in der Teamsteuerung \parencite{monod_worker_2024}, nicht deren eigenes motivationales Erleben als KI-Nutzer.

Schließlich fehlt eine sektorbezogene Perspektive für den DACH-Raum. Der Bankensektor mit seiner Kombination aus hoher Regulierungsdichte, wissensintensiver Entscheidungsarbeit und ambivalenter Digitalisierungsgeschichte (vgl. Abschnitt 2.1.3) stellt spezifische Kontextbedingungen bereit, die in der bisherigen Forschung nicht abgebildet sind.

Die vorliegende Arbeit adressiert diese Lücken mit folgender Forschungsfrage:

\begin{quote}
\textit{Wie erleben mittlere Führungskräfte im Bankensektor (DACH-Region) die Auswirkungen generativer KI-Tools auf ihre psychologischen Grundbedürfnisse nach Autonomie, Kompetenz und sozialer Eingebundenheit gemäß der Self-Determination Theory?}
\end{quote}

Der qualitative Zugang über problemzentrierte Leitfadeninterviews ermöglicht es, die subjektive Erfahrungswelt der Betroffenen zu rekonstruieren, Deutungsmuster zu identifizieren und die Bedingungskonfigurationen herauszuarbeiten, unter denen generative KI bedürfnisunterstützend oder -frustrierend wirkt. Die SDT-Grundbedürfnisse dienen dabei als deduktive Analyselinse, die durch induktive Subkategorien ergänzt wird (vgl. Kapitel 3).