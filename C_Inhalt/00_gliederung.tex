\chapter{Ergebnisse}

Die folgenden Abschnitte stellen die empirischen Befunde der Interviewanalyse dar. Nach einer Beschreibung der Interviewpartner:innen werden die Ergebnisse entlang der drei SDT-Grundbedürfnisse -- Autonomie, Kompetenz und soziale Eingebundenheit -- strukturiert präsentiert. Abschließend werden übergreifende Deutungsmuster und situative Bedingungen zusammengeführt, die sich quer zu den einzelnen Bedürfnisdimensionen zeigten.

\section{Darstellung der Interviewpartner:innen}
\section{Autonomieerleben im Kontext generativer KI}
\section{Kompetenzerleben im Kontext generativer KI}
\section{Soziale Eingebundenheit im Kontext generativer KI}
\section{Übergreifende Deutungsmuster und situative Bedingungen}

\chapter{Diskussion}

Die empirischen Befunde aus Kapitel~4 werden im Folgenden in den theoretischen und empirischen Kontext eingeordnet. Das Kapitel interpretiert die Ergebnisse vor dem Hintergrund der SDT, diskutiert Anknüpfungspunkte und Abweichungen zum bisherigen Forschungsstand, leitet Implikationen für die Praxis ab und reflektiert die Grenzen der Untersuchung.

\section{Interpretation der Ergebnisse vor dem Hintergrund der SDT}
\section{Einordnung in den Forschungsstand}
\section{Implikationen für die Gestaltung von KI in Führungskontexten}
\section{Limitationen der Studie}

\chapter{Zusammenfassung und Ausblick}

Das abschließende Kapitel bündelt die zentralen Erkenntnisse der Arbeit, ordnet ihren wissenschaftlichen Beitrag ein und formuliert Handlungsempfehlungen für die Praxis. Ein Ausblick auf offene Fragen schließt die Arbeit ab.

\section{Zusammenfassung der zentralen Befunde}
\section{Beitrag zur Forschung}
\section{Praktische Handlungsempfehlungen}
\section{Zukünftiger Forschungsbedarf}
