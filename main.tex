


%BASICS%
\documentclass[12pt,a4paper]{report}
\usepackage[utf8]{inputenc}
\usepackage[left=3.5cm, right=2.5cm, top=2.5cm, bottom=2.5cm]{geometry}
\usepackage{setspace}
\usepackage{titlesec}
\usepackage{fancyhdr}
\usepackage{graphicx}
\usepackage{pdflscape}
\usepackage{caption}
\usepackage{footmisc}
\usepackage{hyperref}
\usepackage{etoolbox}
\usepackage{csquotes}
\usepackage{pdfpages}
\DeclareUnicodeCharacter{202F}{\,} % narrow no-break space (e.g., "z. B.")

% Tabelle
\usepackage{tabularx}
\usepackage{booktabs}
\usepackage{array}

% Flowchart
\usepackage[utf8]{inputenc}
\usepackage[T1]{fontenc}
\usepackage{lmodern}
\usepackage{tikz}
\usetikzlibrary{arrows,arrows.meta,positioning}


% Schriftart und Zeilenabstand
%\renewcommand{\familydefault}{\sfdefault} % Schriftart (Arial-ähnlich)
\setlength{\parindent}{0pt} % Kein Absatzeinzug
\setlength{\parskip}{0.5em} % Abstand zwischen Absätzen
\onehalfspacing % Zeilenabstand 1,5

% Überschriftenformate anpassen: Gleiche Schriftgröße wie der Text
\titleformat{\chapter}%[hang]
  {\normalfont\bfseries\normalsize} % Normale Textgröße (normalsize) für Kapitelüberschrift
  {\thechapter.} % Kapitelnummerierung mit Punkt
  {1em} % Abstand zwischen Nummer und Titel
  {}

\titleformat{\section}
  {\normalfont\bfseries\normalsize} % Normale Textgröße (normalsize) für Abschnittsüberschrift
  {\thesection.} % Abschnittsnummerierung mit Punkt
  {1em} % Abstand zwischen Nummer und Titel
  {}

\titleformat{\subsection}
  {\normalfont\bfseries\normalsize} % Normale Textgröße (normalsize) für Unterabschnittsüberschrift
  {\thesubsection.} % Unterabschnittsnummerierung mit Punkt
  {1em} % Abstand zwischen Nummer und Titel
  {}

% Fußnotenformat
\renewcommand{\footnotelayout}{\setstretch{1}} % Zeilenabstand in Fußnoten

% Kopf-/Fußzeilen
\pagestyle{fancy}
\makeatletter
\let\ps@plain\ps@fancy % Überschreibt den plain-Stil mit dem fancy-Stil
\makeatother
\fancyhf{} % Setzt Kopf- und Fußzeilen zurück
\fancyfoot[R]{\thepage} % Seitennummer rechts unten
\renewcommand{\headrulewidth}{0pt} % Entfernt die Linie in der Kopfzeile
\renewcommand{\footrulewidth}{0pt} % Entfernt die Linie in der Fußzeile

% Beschriftungsformat für Tabellen anpassen
\captionsetup[table]{
    labelformat=simple,
    labelsep=colon, % Trennt "Tabelle Nr" und Titel mit einem Doppelpunkt
    name=Tabelle % Festlegen des Präfixes für die Beschriftung
}

% Nummerierung der Tabellen ohne Abschnittsnummer
\renewcommand{\thetable}{\arabic{table}} % Tabellen nur fortlaufend nummerieren

% Beschriftungsformat für Abbildungen anpassen
\captionsetup[figure]{
    labelformat=simple,
    labelsep=colon, % Trennt "Abbildung Nr" und Titel mit einem Doppelpunkt
    name=Abbildung % Festlegen des Präfixes für die Beschriftung
}

% Nummerierung der Abbildungen ohne Abschnittsnummer
\renewcommand{\thefigure}{\arabic{figure}} % Abbildungen nur fortlaufend nummerieren

\usepackage{tocloft} % Für Anpassungen im Inhalts-, Abbildungs- und Tabellenverzeichnis


% Anpassung der Verzeichnisüberschriften
\renewcommand{\contentsname}{Inhaltsverzeichnis}
\renewcommand{\listfigurename}{Abbildungsverzeichnis}
\renewcommand{\listtablename}{Tabellenverzeichnis}
\renewcommand{\bibname}{Literaturverzeichnis}

\usepackage{tocloft} % Für Anpassungen der Verzeichnisüberschriften

% Schriftgröße für Verzeichnisüberschriften auf 11pt setzen
\renewcommand{\cfttoctitlefont}{\normalfont\normalsize\bfseries} % Inhaltsverzeichnis
\renewcommand{\cftloftitlefont}{\normalfont\normalsize\bfseries} % Abbildungsverzeichnis
\renewcommand{\cftlottitlefont}{\normalfont\normalsize\bfseries} % Tabellenverzeichnis

% Einheitliches Layout für alle Seiten
\fancyhf{} % Setzt Kopf- und Fußzeilen zurück
\fancyfoot[R]{\thepage} % Seitennummer rechts unten
\renewcommand{\headrulewidth}{0pt} % Entfernt die Linie in der Kopfzeile
\renewcommand{\footrulewidth}{0pt} % Entfernt die Linie in der Fußzeile

% Kapitelüberschrift: Gleicher Abstand nach unten wie bei Abschnittsüberschriften
\titlespacing{\chapter}{0pt}{0.5cm}{0.5cm} % Abstand oben und unten 0.5cm

% Abschnittsüberschrift: Gleicher Abstand wie Kapitelüberschrift
\titlespacing{\section}{0pt}{0.5cm}{0.5cm}

% Unterabschnittsüberschrift: Optional anpassen, falls nötig
\titlespacing{\subsection}{0pt}{0.5cm}{0.5cm}

\usepackage[style=apa,backend=biber,language=ngerman]{biblatex} % APA-Format und deutsche Sprache
\usepackage[ngerman]{babel}
\addbibresource{B_Literatur/literatur.bib} % Pfad zur Literaturdatenbank
\DefineBibliographyStrings{ngerman}{%
    bibliography = {Literaturverzeichnis}, % Überschrift ändern
}


\usepackage[en-GB]{datetime2} % or ngerman
\usepackage{draftwatermark}

\SetWatermarkText{DRAFT\\\DTMtoday\\\DTMcurrenttime}
\SetWatermarkScale{0.4}
\SetWatermarkLightness{0.95}
%'''''%


%BASICS%
\documentclass[12pt,a4paper]{report}
\usepackage[utf8]{inputenc}
\usepackage[left=3.5cm, right=2.5cm, top=2.5cm, bottom=2.5cm]{geometry}
\usepackage{setspace}
\usepackage{titlesec}
\usepackage{fancyhdr}
\usepackage{graphicx}
\usepackage{pdflscape}
\usepackage{caption}
\usepackage{footmisc}
\usepackage{hyperref}
\usepackage{etoolbox}
\usepackage{csquotes}
\usepackage{pdfpages}
\DeclareUnicodeCharacter{202F}{\,} % narrow no-break space (e.g., "z. B.")

% Tabelle
\usepackage{tabularx}
\usepackage{booktabs}
\usepackage{array}

% Flowchart
\usepackage[T1]{fontenc}
\usepackage{lmodern}
\usepackage{tikz}
\usetikzlibrary{arrows,arrows.meta,positioning}


% Schriftart und Zeilenabstand
%\renewcommand{\familydefault}{\sfdefault} % Schriftart (Arial-ähnlich)
\setlength{\parindent}{0pt} % Kein Absatzeinzug
\setlength{\parskip}{0.5em} % Abstand zwischen Absätzen
\onehalfspacing % Zeilenabstand 1,5

% Überschriftenformate anpassen: Gleiche Schriftgröße wie der Text
\titleformat{\chapter}%[hang]
  {\normalfont\bfseries\normalsize} % Normale Textgröße (normalsize) für Kapitelüberschrift
  {\thechapter.} % Kapitelnummerierung mit Punkt
  {1em} % Abstand zwischen Nummer und Titel
  {}

\titleformat{\section}
  {\normalfont\bfseries\normalsize} % Normale Textgröße (normalsize) für Abschnittsüberschrift
  {\thesection.} % Abschnittsnummerierung mit Punkt
  {1em} % Abstand zwischen Nummer und Titel
  {}

\titleformat{\subsection}
  {\normalfont\bfseries\normalsize} % Normale Textgröße (normalsize) für Unterabschnittsüberschrift
  {\thesubsection.} % Unterabschnittsnummerierung mit Punkt
  {1em} % Abstand zwischen Nummer und Titel
  {}

% Fußnotenformat
\renewcommand{\footnotelayout}{\setstretch{1}} % Zeilenabstand in Fußnoten

% Kopf-/Fußzeilen
\pagestyle{fancy}
\makeatletter
\let\ps@plain\ps@fancy % Überschreibt den plain-Stil mit dem fancy-Stil
\makeatother
\fancyhf{} % Setzt Kopf- und Fußzeilen zurück
\fancyfoot[R]{\thepage} % Seitennummer rechts unten
\renewcommand{\headrulewidth}{0pt} % Entfernt die Linie in der Kopfzeile
\renewcommand{\footrulewidth}{0pt} % Entfernt die Linie in der Fußzeile

% Beschriftungsformat für Tabellen anpassen
\captionsetup[table]{
    labelformat=simple,
    labelsep=colon, % Trennt "Tabelle Nr" und Titel mit einem Doppelpunkt
    name=Tabelle % Festlegen des Präfixes für die Beschriftung
}

% Nummerierung der Tabellen ohne Abschnittsnummer
\renewcommand{\thetable}{\arabic{table}} % Tabellen nur fortlaufend nummerieren

% Beschriftungsformat für Abbildungen anpassen
\captionsetup[figure]{
    labelformat=simple,
    labelsep=colon, % Trennt "Abbildung Nr" und Titel mit einem Doppelpunkt
    name=Abbildung % Festlegen des Präfixes für die Beschriftung
}

% Nummerierung der Abbildungen ohne Abschnittsnummer
\renewcommand{\thefigure}{\arabic{figure}} % Abbildungen nur fortlaufend nummerieren

\usepackage{tocloft} % Für Anpassungen im Inhalts-, Abbildungs- und Tabellenverzeichnis

% Anpassung der Verzeichnisüberschriften
\renewcommand{\contentsname}{Inhaltsverzeichnis}
\renewcommand{\listfigurename}{Abbildungsverzeichnis}
\renewcommand{\listtablename}{Tabellenverzeichnis}
\renewcommand{\bibname}{Literaturverzeichnis}

% Schriftgröße für Verzeichnisüberschriften auf 11pt setzen
\renewcommand{\cfttoctitlefont}{\normalfont\normalsize\bfseries} % Inhaltsverzeichnis
\renewcommand{\cftloftitlefont}{\normalfont\normalsize\bfseries} % Abbildungsverzeichnis
\renewcommand{\cftlottitlefont}{\normalfont\normalsize\bfseries} % Tabellenverzeichnis

% Kapitelüberschrift: Gleicher Abstand nach unten wie bei Abschnittsüberschriften
\titlespacing{\chapter}{0pt}{0.5cm}{0.5cm} % Abstand oben und unten 0.5cm

% Abschnittsüberschrift: Gleicher Abstand wie Kapitelüberschrift
\titlespacing{\section}{0pt}{0.5cm}{0.5cm}

% Unterabschnittsüberschrift: Optional anpassen, falls nötig
\titlespacing{\subsection}{0pt}{0.5cm}{0.5cm}

\usepackage[style=apa,backend=biber,language=ngerman]{biblatex} % APA-Format und deutsche Sprache
\usepackage[ngerman]{babel}
\addbibresource{B_Literatur/literatur.bib} % Pfad zur Literaturdatenbank
\DefineBibliographyStrings{ngerman}{%
    bibliography = {Literaturverzeichnis}, % Überschrift ändern
}


\usepackage[en-GB]{datetime2} % or ngerman
\usepackage{draftwatermark}

\SetWatermarkText{DRAFT\\\DTMtoday\\\DTMcurrenttime}
\SetWatermarkScale{0.4}
\SetWatermarkLightness{0.95}
%'''''%
% Variablen für die Arbeit
\newcommand{\ThesisTitle}{Arbeitstitel: Zwischen Effizienzversprechen und Motivation: Generative KI in der Führungsarbeit}
\newcommand{\Module}{Master Thesis}
\newcommand{\AuthorName}{Joshua Märker}
\newcommand{\MatriculationNumber}{52416743}
\newcommand{\Supervisor}{Assoz. FH-Prof. Mag. (FH) Martina Kohlberger, PhD}
\newcommand{\SubmissionDate}{28.08.2026} % Lädt die Variablen aus der Datei variables.tex

\begin{document}

% Sperrvermerk
% \chapter*{Sperrvermerk}
Die vorliegende Masterarbeit beinhaltet in den Kapiteln XYZ vertrauliche Informationen der Firma XYZ. 
Diese Kapitel sind daher nur den Gutachtern sowie den Mitgliedern des Prüfungsausschusses zu Prüfungszwecken zugänglich zu machen. Veröffentlichungen und Vervielfältigungen der betroffenen Kapitel sind ohne ausdrückliche Genehmigung des Unternehmens nicht gestattet.
\medskip
Dieser Sperrvermerk gilt XYZ Jahre ab dem Einreichungsdatum der Arbeit beim Prüfungsamt.
\vspace{4cm}
\hfill Rohrbach, den \today \hrulefill
% \thispagestyle{empty}

% Deckblatt
\begin{titlepage}
    \raggedright % Flattersatz nur für das Deckblatt
    % Deckblatt
\begin{titlepage}
    \centering
    {\fontsize{20pt}{26pt}\selectfont \textbf{\ThesisTitle}}\\[2cm]
    %{\large im Rahmen der Lehrveranstaltung}\\[0cm]
    %{\large \textbf{\Module}}\\[1cm]
    {\large\textbf{Executive Education}\\[0cm] MCI | Die Unternehmerische Hochschule®}\\[10cm]
    {\large Betreuer: \textbf{\\[0cm]\Supervisor}}\\[2cm]
     {\large Verfasser:in: \textbf{\\[0cm]\AuthorName \\[0cm]\MatriculationNumber}} \\[2cm]
    {\large Abgabedatum: \\[0cm]\textbf{\SubmissionDate}}\\[2cm]
\end{titlepage} % Einbindung der ausgelagerten Datei
\end{titlepage}
\thispagestyle{empty}

% Einfügen der Eidesstattlichen Erklärung
% \newpage
\chapter*{Eidesstattliche Erklärung}
\chaptermark{Eidesstattliche Erklärung}

Hiermit erkläre ich, Ein Autor, dass ich die vorliegende Masterarbeit selbstständig und ohne unerlaubte Hilfe angefertigt, andere als die angegebenen Quellen und Hilfsmittel nicht benutzt und die den benutzten Quellen wörtlich oder inhaltlich entnommenen Stellen als solche kenntlich gemacht habe.

Die Arbeit wurde bisher in gleicher oder ähnlicher Form keiner anderen Prüfungsbehörde vorgelegt und auch nicht veröffentlicht.
\vspace{1cm}

\hfill Rohrbach, den \today \hrulefill
% \thispagestyle{empty}

% Einfügen der Kurzfassung/Abstract
%\chapter*{Abstract}
%%\newpage
%\begin{abstract}
Generative KI verspricht Führungskräften Entlastung, doch während Unternehmen milliardenschwer in KI-Systeme investieren, bleibt unbeantwortet, wie die Menschen, die diese Systeme täglich nutzen, ihre eigene Wirksamkeit erleben. Die technologische Transformation von Führungsarbeit wird primär durch die Linse von Effizienzgewinnen betrachtet, während motivationale Konsequenzen weitgehend unsichtbar bleiben. Diese Studie untersucht auf Grundlage der Self-Determination Theory, wie Führungskräfte im mittleren Management von \gls{DACH}-Banken den Einsatz generativer KI in Entscheidungsvorbereitungsprozessen erleben und unter welchen Bedingungen diese Technologien die psychologischen Grundbedürfnisse nach Autonomie, Kompetenz und sozialer Eingebundenheit unterstützen oder unterminieren. Mittels leitfadengestützter Interviews und qualitativer Inhaltsanalyse wird ein prozessuales Verständnis der motivationalen Wirkung generativer KI in realen Führungspraktiken entwickelt. Die Studie leistet damit einen Beitrag zur Verbindung von KI-Forschung, motivationspsychologischer Perspektive im Organizational Behavior.


%\thispagestyle{empty}
%\newpage


\newpage
\thispagestyle{empty}
\cleardoublepage

% Inhaltsverzeichnis mit Seitennummerierung starten
\pagenumbering{Roman} % Römische Zahlen starten
\setcounter{page}{1}  % Startet die Nummerierung bei 1
\tableofcontents
\newpage

% Abbildungsverzeichnis
 \listoffigures
 \addcontentsline{toc}{chapter}{\listfigurename} % Zum Inhaltsverzeichnis hinzufügen
 \newpage

% Tabellenverzeichnis
 \listoftables
 \addcontentsline{toc}{chapter}{\listtablename} % Zum Inhaltsverzeichnis hinzufügen
 \newpage

% Abkürzungsverzeichnis
 \chapter*{Abkürzungsverzeichnis}
% \addcontentsline{toc}{chapter}{Abkürzungsverzeichnis}
  \begin{description}
    \item[AI] Artificial Intelligence
    \item[SDT] Self-determination Theory
\end{description}
\newpage

% Arabische Seitennummerierung für den Hauptteil
\pagenumbering{arabic}
\setcounter{page}{1} % Startet die arabische Nummerierung bei 1

%######### KAPITEL EINFÜGEN ##############
\chapter{Einführung}

Generative \gls{ki} verändert, wie in Organisationen gearbeitet, entschieden und geführt wird. Das vorliegende Kapitel umreißt die Problemstellung, die sich daraus für das motivationale Erleben von Führungskräften ergibt, formuliert die Zielsetzung und Forschungsfrage der Arbeit und gibt einen Überblick über den Gang der Argumentation.

\section{Problemstellung und Relevanz}

Generative \gls{ki} hat innerhalb weniger Jahre den Weg aus der Forschungsliteratur in den Arbeitsalltag von Millionen von Wissensarbeiterinnen und Wissensarbeitern gefunden. Was lange als futuristisches Versprechen galt, ist in Unternehmen längst operative Realität: Sprachmodelle wie GPT-4, Claude oder Gemini unterstützen bei der Analyse von Dokumenten, der Vorbereitung von Entscheidungen, der Kommunikation mit Stakeholdern und der Zusammenfassung komplexer Sachverhalte \parencite{brynjolfsson_generative_2023, noyExperimentalEvidenceProductivity2023}. Das Effizienzversprechen, das mit dieser Technologie verbunden wird, ist empirisch nicht ohne Grundlage: Feldstudien dokumentieren substanzielle Produktivitätsgewinne, insbesondere bei wissensintensiven und kommunikativen Tätigkeiten \parencite{dellacqua_navigating_2023}.


Doch hinter der Fassade des Effizienzgewinns tut sich ein blinder Fleck auf. Die Frage, wie Führungskräfte den Einsatz dieser Systeme motivational erleben, bleibt in der bisherigen Forschung weitgehend unbeantwortet. Bestehende Studien konzentrieren sich auf Leistungsoutcomes, Adoptionsraten und organisationale Effizienz \parencite{bankinsMultilevelReviewArtificial2024} -- das subjektive Erleben von Menschen, die täglich mit generativer \gls{ki} arbeiten, gerät dabei in den Hintergrund. Dabei ist aus Sicht des Organizational Behavior längst bekannt, dass Motivation, Wohlbefinden und Leistungsfähigkeit untrennbar miteinander verbunden sind und wesentlich davon abhängen, wie Menschen ihre Arbeitsbedingungen erleben \parencite{deciSelfDeterminationTheoryWork2017}.

Besonders auffällig ist diese Lücke mit Blick auf Führungskräfte des mittleren Managements. Sie nehmen eine strukturell exponierte Stellung ein: Als strategische Übersetzer zwischen Unternehmensführung und operativer Ebene müssen sie \acrshort{ki}-gestützte Arbeitsprozesse nicht nur selbst navigieren, sondern auch in ihren Teams einführen und begleiten \parencite{floydManagingStrategicConsensus1997}. Gleichzeitig ist ihre Arbeit durch hohe Komplexität, Verantwortung und den Anspruch geprägt, Entscheidungen in einem Umfeld zu treffen, das Rechenschaftspflicht und fachliches Urteilsvermögen verlangt. In diesem Kontext verändert generative \gls{ki} die Bedingungen von Führungsarbeit grundlegend -- und damit potenziell auch das Erleben der eigenen Wirksamkeit, Autonomie und Eingebundenheit.

Für den Bankensektor im \acrshort{dach}-Raum gilt das in besonderer Weise. Entscheidungsprozesse sind dort durch regulatorische Anforderungen, Dokumentationspflichten und Verantwortungszurechnung geprägt -- ein Umfeld, in dem kognitive Unterstützungssysteme besonders sichtbare Spuren hinterlassen. Generative \gls{ki} wirkt hier nicht als neutrales Effizienzwerkzeug: Sie verändert, wer welchen Teil eines Entscheidungsprozesses verantwortet, wie Expertise wahrgenommen wird und in welchem Verhältnis menschliches Urteil und algorithmische Ausgabe zueinanderstehen. Ob diese Veränderungen als Entlastung oder als Kontrollverlust erlebt werden, ist nicht technologisch determiniert -- es hängt davon ab, wie Führungskräfte den Einsatz dieser Systeme deuten \parencite{edwardsManagerialControlFeedback2024, tongJanusFaceArtificial2021}.


Genau hier setzt die \gls{sdt} an. Sie beschreibt Motivation und Wohlbefinden als Funktionen dreier psychologischer Grundbedürfnisse: Autonomie -- das Erleben, Entscheidungen selbstbestimmt zu treffen; Kompetenz -- das Erleben, wirksam und fähig zu sein; und soziale Eingebundenheit -- das Erleben von Verbundenheit und Zugehörigkeit \parencite{deciWhatWhyGoal2000, deciSelfDeterminationTheoryWork2017}. Werden diese Bedürfnisse durch Arbeitsbedingungen unterstützt, entstehen Engagement und intrinsische Motivation; werden sie frustriert, folgen Rückzug, Erschöpfung und Amotivation \parencite{vandenbroeckReviewSelfDeterminationTheorys2016}. Generative KI kann, je nach Wahrnehmung und Implementierungskontext, in beide Richtungen wirken \parencite{gagneUnderstandingShapingFuture2022, klonekDoesAIWork2025}. Dieses ambivalente Potenzial macht sie zu einem theoretisch besonders interessanten Gegenstand für die \acrshort{sdt}-Forschung.



Die vorliegende Arbeit nimmt diese Spannung zum Ausgangspunkt. Sie fragt nicht, ob generative KI die Führungsarbeit effizienter macht -- das ist gut dokumentiert. Sie fragt, was dieser Technologieeinsatz mit Menschen macht, die täglich Entscheidungsverantwortung tragen.

\section{Zielsetzung und Forschungsfrage}

Ziel der Arbeit ist es zu verstehen, wie Führungskräfte im mittleren Management von Banken im \acrshort{dach}-Raum den Einsatz generativer \gls{ki} in Entscheidungsvorbereitungsprozessen motivational erleben und unter welchen Bedingungen diese Technologien als unterstützend oder einschränkend für die psychologischen Grundbedürfnisse nach Autonomie, Kompetenz und sozialer Eingebundenheit wahrgenommen werden.

Damit richtet sich die Arbeit gegen eine doppelte Verkürzung im bisherigen Forschungsstand: zum einen gegen die Reduktion von \acrshort{ki}-Wirkungen auf Effizienzmetriken, zum anderen gegen eine technologiedeterministische Betrachtung, die die deutungsabhängige Natur dieser Wirkungen ausblendet. Statt Hypothesen zu testen, verfolgt die Arbeit ein prozessuales Erkenntnisinteresse: Sie will aufzeigen, wie sich motivationale Wirkungen generativer KI in konkreten Führungspraktiken entfalten und welche situativen, organisationalen und individuellen Faktoren dabei eine Rolle spielen.

Die leitende Forschungsfrage lautet:

\begin{quote}
    \textit{Wie erleben Führungskräfte im mittleren Management von Banken den Einsatz generativer \gls{ki} in Entscheidungsvorbereitungsprozessen, und unter welchen Bedingungen wird diese Technologie als unterstützend oder einschränkend für Autonomie, Kompetenz und soziale Eingebundenheit wahrgenommen?}
\end{quote}

Diese Frage trägt zur Schließung mehrerer identifizierbarer Forschungslücken bei. Erstens fehlt es an theoretisch fundierten Untersuchungen, die die \gls{sdt} systematisch auf den Kontext generativer \gls{ki} anwenden -- obwohl das Framework dafür konzeptionell gut geeignet ist \parencite{mcanallySelfDeterminationTheoryWorkplace2024, gagneUnderstandingShapingFuture2022}. Zweitens ist die Gruppe der mittleren Führungskräfte in der \acrshort{ki}-Forschung trotz ihrer strategischen Bedeutung empirisch unterrepräsentiert. Drittens mangelt es an Studien, die das subjektive Erleben in realen organisationalen Kontexten erfassen, statt sich auf experimentelle Settings mit artifiziellen Aufgaben zu beschränken \parencite{bankinsMultilevelReviewArtificial2024}.



Der erwartete Beitrag der Arbeit ist dreifach: Theoretisch erweitert sie die Anwendung der SDT auf einen bislang wenig untersuchten technologischen Kontext. Empirisch gewinnt sie differenzierte Einblicke in das subjektive Erleben einer strategisch zentralen, in der Forschung jedoch unterbeschriebenen Gruppe. Praktisch liefert sie Orientierungspunkte für die motivationsgerechte Gestaltung generativer KI in Führungskontexten -- ein Aspekt, der in den meisten \acrshort{ki}-Implementierungsstrategien bislang zu kurz kommt \parencite{prasadGenerativeAICatalyst2024, quaquebekeNowNewNext2023}.


\section{Aufbau der Arbeit}

Die Arbeit gliedert sich in sechs Kapitel. Das vorliegende Einführungskapitel legt Problemstellung, Relevanz und Forschungsfrage dar.

Kapitel~2 entwickelt den theoretischen Rahmen. Es führt zunächst in generative \gls{ki} als soziotechnisches Arbeitssystem ein, beschreibt ihre Einsatzszenarien in wissensintensiven Organisationen und geht auf spezifische Entwicklungen im Bankensektor der \acrshort{dach}-Region ein. Anschließend wird Führungsarbeit im mittleren Management beleuchtet -- mit Fokus auf Entscheidungsarbeit als Kernaufgabe und den Besonderheiten des Bankenumfelds. Der dritte theoretische Abschnitt entfaltet die \gls{sdt}: Grundannahmen, die drei psychologischen Grundbedürfnisse, ihre Anwendung im Arbeitskontext und die Frage, wie digitale Arbeitssysteme bedürfnisunterstützend oder -frustrierend wirken können. Das Kapitel schließt mit einer theoretischen Synthese, die die motivationale Wirkung generativer KI in der Führungsarbeit konzeptionell rahmt.

Kapitel~3 beschreibt das methodische Vorgehen. Es begründet die Wahl eines qualitativen Forschungsdesigns, stellt den problemzentrierten Interviewansatz vor und erläutert Leitfadenentwicklung, Sampling und Durchführung der Erhebung. Daran schließen sich die Beschreibung des Transkriptionsverfahrens sowie die Darstellung der strukturierenden qualitativen Inhaltsanalyse nach Kuckartz an, die als Auswertungsmethode eingesetzt wird. Das Kapitel endet mit Überlegungen zu Gütekriterien und ethischen Aspekten der Untersuchung.

Kapitel~4 präsentiert die empirischen Ergebnisse. Nach einer Beschreibung der Interviewpartnerinnen und -partner werden die Befunde entlang der drei \acrshort{sdt}-Grundbedürfnisse strukturiert: Autonomieerleben, Kompetenzerleben und soziale Eingebundenheit im Kontext generativer \gls{ki}. Übergreifende Deutungsmuster und situative Bedingungen werden abschließend zusammengeführt.

Kapitel~5 diskutiert die Ergebnisse vor dem Hintergrund der \gls{sdt} und des Forschungsstands, benennt Implikationen für die Gestaltung von \gls{ki} in Führungskontexten und reflektiert die Limitationen der Studie.

Kapitel~6 fasst die zentralen Befunde zusammen, benennt den wissenschaftlichen Beitrag der Arbeit, formuliert praktische Handlungsempfehlungen und skizziert offene Fragen für zukünftige Forschung.
\chapter{Theoretischer Hintergrund}

Die Frage, wie generative KI das motivationale Erleben von Führungskräften beeinflusst, lässt sich nicht aus einer einzelnen Disziplin heraus beantworten. Sie erfordert eine Verbindung technologischer, organisationaler und psychologischer Perspektiven. Das folgende Kapitel legt dieses theoretische Fundament in vier Schritten: Zunächst wird generative KI als soziotechnisches Arbeitssystem charakterisiert, das weit über ein reines Effizienzwerkzeug hinausgeht (Abschnitt~\ref{sec:genai}). Anschließend richtet sich der Blick auf die spezifische Arbeitssituation mittlerer Führungskräfte im Bankensektor, deren Entscheidungsarbeit den unmittelbaren Wirkungskontext der Technologie bildet (Abschnitt~\ref{sec:fuehrung}). Der dritte Abschnitt entfaltet die Self-Determination Theory als motivationspsychologischen Erklärungsrahmen (Abschnitt~\ref{sec:sdt}). Eine abschließende Synthese führt die drei Stränge zusammen und leitet die Forschungslücke ab, die diese Arbeit adressiert (Abschnitt~\ref{sec:synthese}).

\section{Generative KI als soziotechnisches Arbeitssystem}
\label{sec:genai}

Generative KI-Systeme sind in den vergangenen Jahren von einem Forschungsgegenstand zu einem Arbeitsmittel geworden, das kognitive Kernprozesse in Organisationen verändert. Der folgende Abschnitt grenzt den Begriff ab, ordnet die technologische Entwicklung ein und beleuchtet Einsatzszenarien in wissensintensiven Organisationen -- mit besonderem Blick auf den Bankensektor der DACH-Region.

\subsection{Begriffliche Abgrenzung und Entwicklung}

Generative Künstliche Intelligenz (GenAI) bezeichnet eine Klasse von KI-Systemen, die auf Basis umfangreicher Trainingsdaten neuartige Inhalte erzeugen können -- Texte, Bilder, Code, Audio oder multimodale Kombinationen. Der Begriff grenzt sich damit bewusst von früheren KI-Generationen ab, die primär klassifikatorisch oder regelbasiert operierten: Während ein Spam-Filter eingehende E-Mails in Kategorien sortiert, verfasst ein generatives Sprachmodell eigenständig Antworten, Analysen oder Entwürfe \parencite{brynjolfsson_generative_2023}.

Technologisch basieren die derzeit leistungsfähigsten generativen Systeme auf Large Language Models (LLMs). Diese neuronalen Netzwerke mit Milliarden von Parametern werden auf umfangreichen Textkorpora trainiert und nutzen Transformer-Architekturen, um probabilistische Vorhersagen über Textsequenzen zu treffen. Modelle wie GPT-4, Claude oder Gemini generieren auf dieser Grundlage kontextuell kohärente Outputs, die sich sprachlich kaum von menschlich verfassten Texten unterscheiden \parencite{brynjolfsson_generative_2023}. Die Veröffentlichung von ChatGPT im November 2022 markierte einen Wendepunkt: Innerhalb von zwei Monaten erreichte das System 100 Millionen aktive Nutzer und machte generative KI erstmals für breite Anwendergruppen in Organisationen zugänglich.

Was generative KI von traditionellen Informationssystemen -- etwa ERP-Systemen, Business-Intelligence-Dashboards oder Datenbanken -- unterscheidet, ist ihre \textit{generative Kapazität}. Herkömmliche Systeme speichern, verarbeiten und visualisieren vorhandene Informationen. GenAI erzeugt hingegen Inhalte, die so nicht explizit in den Trainingsdaten enthalten sind \parencite{bankinsMultilevelReviewArtificial2024}. Diese Eigenschaft eröffnet qualitativ neue Anwendungsszenarien: automatisierte Texterstellung, Szenarioanalysen, kreative Ideengenerierung und dialogische Interaktionen, bei denen das System auf Rückfragen und Kontextualisierungen reagiert.

Drei Merkmale charakterisieren generative KI-Systeme im organisationalen Kontext besonders. Erstens ihre \textit{probabilistische Kreativität}: Die Outputs sind nicht deterministisch, sondern variieren bei identischen Eingaben -- eine Eigenschaft, die sowohl kreatives Potenzial als auch Unsicherheit erzeugt. Zweitens ihre \textit{Kontextsensitivität}: Moderne LLMs halten Kontext über mehrteilige Dialoge hinweg und passen Antworten an spezifische Nutzerbedürfnisse an. Drittens ihre \textit{Anpassbarkeit}: Durch Fine-Tuning und Retrieval-Augmented Generation (RAG) lassen sich generative Systeme an organisationsspezifische Wissensbasen und Prozesse koppeln, während Nutzer über iteratives Prompting ihre Interaktionskompetenz weiterentwickeln \parencite{brynjolfsson_generative_2023, bankinsMultilevelReviewArtificial2024}.

Diese Merkmale machen GenAI zu einem soziotechnischen Phänomen im engeren Sinne: Das System entfaltet seine Wirkung nicht unabhängig von den Personen, die es nutzen, sondern in einer kontinuierlichen Wechselwirkung zwischen technologischen Möglichkeiten und menschlichen Praktiken. Wie Nutzer Prompts formulieren, welches Vertrauen sie in Outputs setzen, ob sie Ergebnisse kritisch prüfen oder unreflektiert übernehmen -- all das beeinflusst, welche Rolle GenAI in einer Organisation tatsächlich spielt \parencite{bankinsMultilevelReviewArtificial2024}.


\subsection{Einsatzszenarien in wissensintensiven Organisationen}

Generative KI findet primär dort Anwendung, wo Arbeit kognitiv anspruchsvoll, textbasiert und wissensintensiv ist: Dokumentenerstellung und -analyse, Entscheidungsvorbereitung, Strategieentwicklung, Kommunikation und Problemlösung \parencite{brynjolfsson_generative_2023}. Im Unterschied zu früheren Automatisierungswellen, die vorrangig manuelle und repetitive Tätigkeiten adressierten, greift die aktuelle Technologiewelle in den Kern professioneller Wissensarbeit ein.

Empirisch gut dokumentiert ist der Produktivitätseffekt. Eine Feldstudie mit über 5.000 Kundenservice-Mitarbeitenden ergab, dass der Einsatz eines generativen KI-Assistenten die Zahl gelöster Anfragen pro Stunde um 14\,\% steigerte. Aufschlussreich war die Verteilung dieses Effekts: Die Produktivitätsgewinne konzentrierten sich auf weniger erfahrene Mitarbeitende, während hochqualifizierte Experten kaum profitierten \parencite{brynjolfsson_generative_2023}. GenAI scheint demnach als eine Art Kompetenz-Augmentation zu fungieren -- sie hebt das Leistungsniveau weniger Erfahrener an, ohne die Leistung von Experten substanziell zu steigern.

\textcite{bankinsMultilevelReviewArtificial2024} identifizierten in einer Multilevel-Review fünf thematische Pfade, über die KI in Organisationen wirkt: (1) Mensch-KI-Kollaboration und Komplementarität, (2) Wahrnehmung von KI-Fähigkeiten und -Grenzen, (3) KI als Kontrollmechanismus im Sinne algorithmischen Managements, (4) Arbeitsmarktimplikationen wie Job Displacement und Skill Shifts sowie (5) ethische und soziale Implikationen. Diese Pfade interagieren über Analyseebenen hinweg und erzeugen häufig widersprüchliche Effekte -- ein Befund, der vereinfachende Narrative von KI als reinem Effizienzwerkzeug infrage stellt.

Die organisationale Einbettung von GenAI lässt sich entlang dreier Perspektiven konzeptualisieren \parencite{bankinsMultilevelReviewArtificial2024}. Aus \textit{instrumenteller} Sicht wird KI als Produktivitätswerkzeug verstanden: Sie automatisiert repetitive Aufgaben, beschleunigt Informationsverarbeitung und reduziert kognitive Belastung. Organisationen messen den Erfolg hier in Zeitersparnis, Kostenreduktion und Output-Steigerung. Diese Perspektive dominiert in frühen Adoptionsphasen, birgt aber das Risiko, motivationale und identitätsbezogene Effekte zu übersehen \parencite{edwards_managerial_2024}.

Die \textit{strategische} Perspektive geht einen Schritt weiter: KI liefert nicht nur Informationen, sondern generiert Empfehlungen, Prognosen und Handlungsalternativen, die in menschliche Entscheidungsprozesse einfließen. In einer Feldstudie mit Verkaufsmitarbeitenden untersuchten \textcite{tongJanusFaceArtificial2021} diese Integration und fanden einen bemerkenswerten Disclosure-Effekt: KI-basiertes Feedback verbesserte die Leistung, allerdings nur, solange Mitarbeitende nicht wussten, dass es von KI stammte. Die Offenlegung der algorithmischen Quelle reduzierte Akzeptanz und Wirksamkeit -- ein Hinweis darauf, dass strategische KI-Integration mehr erfordert als technische Funktionalität, nämlich Vertrauensaufbau und transparente Kommunikation.

Aus \textit{transformativer} Perspektive schließlich fungiert KI als Katalysator fundamentaler Veränderungen in Arbeitsrollen, Organisationsstrukturen und professionellen Identitäten. Mensch-KI-Kollaboration, hybride Teamstrukturen und algorithmisch vermittelte Koordination sind hier keine Ausnahme mehr, sondern alltägliche Praxis \parencite{bankinsMultilevelReviewArtificial2024}. Die Wahl der Perspektive beeinflusst maßgeblich, wie Mitarbeitende KI wahrnehmen und nutzen: Instrumentelle Framings erleichtern unter Umständen die Akzeptanz, lassen aber transformative Potenziale ungenutzt. Strategische Framings erhöhen Anforderungen an Transparenz und Erklärbarkeit. Transformative Implementierungen erfordern neue Rollen, Kompetenzen und Governance-Strukturen.

Quer zu diesen Perspektiven zeigt sich, dass die Wahrnehmung von KI durch Mitarbeitende nicht technologisch determiniert, sondern sozial konstruiert ist. Eine zentrale Dimension ist die Unterscheidung zwischen KI als Unterstützungs- und als Kontrollsystem \parencite{edwards_managerial_2024}. Dieselben Tools werden in verschiedenen Kontexten unterschiedlich interpretiert: als Arbeitserleichterung oder als Überwachungsinstrument, als Kompetenzerweiterung oder als Bedrohung professioneller Expertise \parencite{monod_worker_2024}. Vertrauen erweist sich dabei als zentraler Mediator. \textcite{prasad_generative_2024} zeigten in einer Studie mit 1.362 Beschäftigten, dass Vertrauen in GenAI die Akzeptanz KI-basierter Praktiken vollständig mediierte: Ohne Vertrauen in Zuverlässigkeit, Fairness und Transparenz blieb auch wahrgenommene Nützlichkeit wirkungslos.


\subsection{Generative KI im Bankensektor (DACH)}

Der Bankensektor zählt international zu den Branchen mit der höchsten GenAI-Adoptionsrate. Branchenerhebungen beziffern den Anteil der Finanzinstitute, die generative KI bereits einsetzen oder dies innerhalb von zwei Jahren planen, auf über 95\,\% \parencite{YourJourneyGenAI}. Typische Anwendungsfelder umfassen Marketing und Kundenkommunikation, Risikomanagement und Compliance, Kreditanalyse sowie interne Wissensmanagement-Prozesse. Für die globale Bankenbranche werden jährliche Produktivitätsgewinne von 200 bis 340 Milliarden US-Dollar prognostiziert \parencite{mckinsey__company_capturing_nodate}.

Im DACH-Raum ergibt sich ein spezifisches Bild, das durch drei Faktoren geprägt wird. Erstens ist der Bankensektor in Deutschland, Österreich und der Schweiz stark reguliert. Die Aufsichtsbehörden -- BaFin, FMA und FINMA -- stellen hohe Anforderungen an Transparenz, Erklärbarkeit und Nachvollziehbarkeit algorithmischer Entscheidungen, insbesondere in der Kreditvergabe und im Risikomanagement. Diese regulatorischen Rahmenbedingungen erzeugen ein Spannungsfeld: Einerseits begrenzen sie die Geschwindigkeit der KI-Adoption, andererseits zwingen sie Organisationen zu einer bewussteren Auseinandersetzung mit Governance-Fragen, die in weniger regulierten Branchen häufig nachgelagert behandelt werden \parencite{hundertmarkIFZGenerativeAI2024}.

Zweitens ist das DACH-Bankensystem durch eine hohe Dichte an Universalbanken, Genossenschaftsbanken und Sparkassen gekennzeichnet. Anders als in angelsächsischen Märkten, wo wenige Großbanken den technologischen Takt vorgeben, existieren im DACH-Raum zahlreiche mittelgroße Institute, deren Digitalisierungsgrad erheblich variiert. Für mittlere Führungskräfte in diesen Organisationen bedeutet dies, dass GenAI-Implementierung selten als Top-down-Projekt mit klarer strategischer Rahmung erfolgt, sondern häufig als bottom-up-getriebene Experimentation einzelner Teams oder Abteilungen.

Drittens unterscheidet sich die Arbeitskultur im DACH-Bankensektor in relevanter Weise. Die Tradition konsensualer Entscheidungsfindung, ausgeprägte Mitbestimmungsstrukturen (insbesondere in Deutschland und Österreich) sowie eine vergleichsweise hohe Bedeutung formaler Qualifikationen und Fachexpertise prägen das Umfeld, in das generative KI eingeführt wird. Wenn ein KI-System Kreditanalysen entwirft, die bislang erfahrene Spezialisten formuliert haben, berührt dies nicht nur Effizienzfragen, sondern auch professionelle Identität und die Legitimation von Expertise \parencite{quaquebeke_now_2023}.

Empirische Studien, die explizit die motivationalen Auswirkungen generativer KI auf Führungskräfte im DACH-Bankensektor untersuchen, liegen bislang nicht vor. Die vorhandene Forschung adressiert entweder den Bankensektor ohne spezifischen Fokus auf Führungsmotivation \parencite{bankinsMultilevelReviewArtificial2024} oder untersucht motivationale Effekte von KI ohne branchenspezifische Differenzierung \parencite{brynjolfsson_generative_2023, edwards_managerial_2024}. Diese Forschungslücke ist insofern bemerkenswert, als der DACH-Bankensektor aufgrund seiner regulatorischen Dichte, organisationalen Heterogenität und kulturellen Spezifika einen Kontext darstellt, in dem die motivationalen Spannungen der KI-Adoption besonders ausgeprägt sein dürften. Die vorliegende Arbeit adressiert dieses Desiderat durch eine qualitative Untersuchung, die das Erleben mittlerer Führungskräfte in diesem spezifischen Branchenkontext in den Mittelpunkt stellt.
\section{Führungsarbeit im mittleren Management}
\label{sec:fuehrung}

Mittlere Führungskräfte stehen organisational an einer Stelle, an der strategische Absicht und operative Wirklichkeit aufeinandertreffen. Ihre Arbeit ist geprägt von Entscheidungen unter Unsicherheit, der Vermittlung zwischen Hierarchieebenen und einem Tätigkeitsprofil, das generative \gls{ki} in besonderer Weise berührt. Der folgende Abschnitt beschreibt diese Arbeitssituation -- zunächst allgemein, dann mit Blick auf die spezifischen Bedingungen des Bankensektors.

\subsection{Entscheidungsarbeit als Kernaufgabe}

Mittlere Führungskräfte operieren an einer organisationalen Nahtstelle: zwischen strategischer Planung des Top-Managements und operativer Ausführung durch Frontline-Mitarbeitende. Diese Position ist weniger komfortabel, als es die Organigramme vermuten lassen. Wer hier arbeitet, muss strategische Vorgaben in operative Handlungen übersetzen, gleichzeitig aber Rückmeldungen und Widerstände der operativen Ebene nach oben kommunizieren -- oft unter Zeitdruck, mit unvollständigen Informationen und widersprüchlichen Erwartungen \parencite{floydManagingStrategicConsensus1997}.

\textcite{floydManagingStrategicConsensus1997} identifizierten vier strategische Rollen, die mittlere Führungskräfte einnehmen: \textit{Championing strategic alternatives} -- das Einbringen innovativer Ideen in strategische Entscheidungsprozesse; \textit{Synthesizing information} -- die Aggregation und Interpretation von Informationen aus verschiedenen organisationalen Quellen; \textit{Facilitating adaptability} -- die Förderung organisationaler Anpassungsfähigkeit; sowie \textit{Implementing deliberate strategy} -- die Umsetzung strategischer Entscheidungen in operative Praxis. Alle vier Rollen haben eines gemeinsam: Sie erfordern Entscheidungen. Nicht die großen, strategischen Richtungsentscheidungen, die dem Top-Management vorbehalten sind, sondern die unzähligen kleineren Urteile darüber, wie Strategie im Alltag konkret wird -- welche Prioritäten gesetzt, welche Informationen weitergegeben, welche Interpretationsspielräume genutzt werden.

Entscheidungsarbeit im mittleren Management ist dabei selten algorithmisch im Sinne klarer Wenn-dann-Regeln. Sie ist vielmehr geprägt von \textit{Sensemaking}: dem Versuch, aus mehrdeutigen Situationen tragfähige Handlungsgrundlagen abzuleiten \parencite{quaquebekeNowNewNext2023}. Mittlere Führungskräfte interpretieren strategische Vorgaben, kontextualisieren sie für ihre Teams, antizipieren Widerstände und justieren ihre Kommunikation entsprechend. Dieses Sensemaking ist nicht bloß ein kognitiver Prozess; es ist auch ein sozialer und emotionaler. Wer Transformation vermitteln soll, muss selbst verstanden haben, was sich verändert und warum -- und muss gleichzeitig mit der eigenen Unsicherheit umgehen können.

Genau hier wird die Einführung generativer KI relevant. Wenn ein Großteil der Entscheidungsarbeit mittlerer Führungskräfte darin besteht, Informationen zu synthetisieren, Optionen abzuwägen und Handlungsempfehlungen zu formulieren, dann adressiert \gls{genai} den Kern ihres Tätigkeitsprofils. \acrshort{ki}-Systeme können Daten schneller aggregieren, Entscheidungsvorlagen erstellen und Szenarien durchspielen. Ob diese Fähigkeit als Unterstützung oder als Bedrohung erlebt wird, hängt davon ab, wie Führungskräfte ihre eigene Rolle definieren -- und ob sie ihre professionelle Identität an den Prozess der Informationsverarbeitung oder an die Qualität des Urteils knüpfen \parencite{quaquebeke_now_2023}.

\textcite{quaquebeke_now_2023} argumentieren, dass \gls{ki} die Natur von Führung fundamental verschieben wird: weg von wissensbasierter Autorität hin zu facilitativer, emotional-intelligenter Führung. Führungskräfte müssen ihre Rolle neu verhandeln -- nicht als allwissende Experten, sondern als Kuratoren und Orchestratoren, die menschliche und algorithmische Ressourcen zusammenführen. Diese Neuverhandlung berührt alle drei psychologischen Grundbedürfnisse der Self-Determination Theory: Autonomie (Wer entscheidet -- Mensch oder Maschine?), Kompetenz (Wessen Expertise zählt noch?) und soziale Eingebundenheit (Wie verändert sich die Beziehung zum Team, wenn \gls{ki} Kommunikation mediiert?).

Hinzu kommt, dass mittlere Führungskräfte in Technologietransformationen eine paradoxe Doppelrolle einnehmen. Sie sollen als Change Agents die Adoption vorantreiben und gleichzeitig sind sie selbst Betroffene, deren Tätigkeitsprofile, Kompetenzen und professionelle Identität durch die neuen Tools transformiert werden \parencite{quaquebeke_now_2023}. Im Kontext generativer \gls{ki} verschärft sich dieses Paradox: Wer die Technologie in seinem Team implementieren soll, muss sie zunächst selbst in die eigene Arbeitsweise integrieren -- mit allen damit verbundenen Unsicherheiten über Verlässlichkeit, Grenzen und langfristige Konsequenzen für die eigene Position.

\textcite{koponen_work_2025} identifizierten durch eine systematische Literaturanalyse zentrale Arbeitscharakteristika, die mittlere Führungskräfte in \acrshort{ki}-integrierten Teams benötigen: Autonomie bei der Gestaltung von Mensch-\acrshort{ki}-Interaktionen, transparente Feedback-Mechanismen für menschliche und algorithmische Leistung, eine Balance zwischen Routine- und strategischen Aufgaben sowie soziale Unterstützung durch Peers und Vorgesetzte bei \acrshort{ki}-bezogenen Unsicherheiten. Fehlen diese Charakteristika, steigt das Risiko für Rollenkonflikte, Ambiguitätsstress und motivationale Erosion.


\subsection{Besonderheiten des Bankensektors}

Der Bankensektor unterscheidet sich von anderen wissensintensiven Branchen in mehreren Dimensionen, die für die motivationale Wirkung generativer \gls{ki} auf mittlere Führungskräfte unmittelbar relevant sind.

Am offensichtlichsten ist die \textit{regulatorische Dichte}. Banken im \acrshort{dach}-Raum unterliegen der Aufsicht durch \gls{bafin}, \gls{fma} und \gls{finma} sowie europäischen Regulierungsrahmen wie der \gls{crr} und der EU-\acrshort{ki}-Verordnung (AI Act). Für Führungskräfte im mittleren Management bedeutet dies, dass Entscheidungen selten in einem Freiraum getroffen werden, sondern innerhalb eng definierter Compliance-Korridore. Kreditentscheidungen folgen standardisierten Ratingprozessen, Beratungsgespräche werden dokumentationspflichtig geführt, Risikoeinschätzungen müssen nachvollziehbar begründet sein. Wenn generative \gls{ki} in diese Prozesse integriert wird, verschärfen sich die Anforderungen an Transparenz und Erklärbarkeit erheblich -- ein algorithmisch generierter Kreditvorschlag, dessen Zustandekommen nicht lückenlos nachvollziehbar ist, widerspricht den aufsichtsrechtlichen Grundprinzipien [Quelle einfügen\footnote{Hier wäre eine regulatorische Quelle sinnvoll, z.\,B. BaFin (2024): \textit{Maschinelles Lernen in der Finanzbranche -- Aufsichtliche Prinzipien}; oder EU AI Act, Art. 6 zu hochriskanten Anwendungen im Finanzsektor.}].

Zweitens prägt eine ausgeprägte \textit{Hierarchie- und Fachexpertisekultur} das mittlere Management im Bankensektor. Entscheidungsbefugnisse sind an formale Kompetenzstufen gebunden; Unterschriftsberechtigungen für Kreditvergaben beispielsweise sind nach Volumen und Risikokategorie gestaffelt. Fachexpertise -- etwa in Bilanzanalyse, Risikomodellierung oder regulatorischer Compliance -- hat traditionell hohen Stellenwert und legitimiert die Autorität mittlerer Führungskräfte gegenüber ihren Teams. Generative \gls{ki}, die in Sekunden Bilanzanalysen erstellt oder Compliance-Prüfungen automatisiert, berührt damit direkt die Grundlage, auf der professionelle Identität und Führungslegitimation aufgebaut sind \parencite{quaquebeke_now_2023}. Anders als in kreativeren Branchen, wo der Wert einer Idee unabhängig von ihrer Quelle beurteilt wird, knüpft der Bankensektor Entscheidungslegitimation eng an formale Expertise und hierarchische Position.

Drittens ist die \textit{Art der Entscheidungsarbeit} im Bankensektor spezifisch. Mittlere Führungskräfte treffen täglich Entscheidungen, die unmittelbare finanzielle Konsequenzen haben -- für die Bank, für Kunden, für regulatorische Kennzahlen. Im Firmenkundengeschäft beurteilen sie Kreditrisiken, im Private Banking beraten sie vermögende Kunden über Anlagestrategien, im Risikomanagement bewerten sie Portfolioexpositionen. Diese Entscheidungen verlangen zweierlei: solide analytische Kompetenz \textit{und} kontextabhängiges Urteilsvermögen, das formale Modelle ergänzt. Die Frage, ob ein langjähriger Firmenkunde mit temporär verschlechterter Bonität weiterhin Kredit erhält, ist eben nicht vollständig formalisierbar -- sie erfordert Kenntnis der Branche, der persönlichen Geschichte und der strategischen Beziehung. Genau in diesem Spannungsfeld zwischen formalisierbarer Analyse und nicht-formalisierbarem Urteil entfaltet generative \gls{ki} ihre ambivalente Wirkung.

Viertens kennzeichnet den \acrshort{dach}-Bankensektor eine \textit{strukturelle Heterogenität}, die sich auf die \acrshort{ki}-Adoption auswirkt. Neben international agierenden Großbanken mit dedizierten Digital-Innovation-Teams existieren zahlreiche Sparkassen, Genossenschaftsbanken und Regionalbanken, deren Digitalisierungsgrad und Ressourcenausstattung erheblich variieren. Für mittlere Führungskräfte in kleineren Instituten bedeutet \acrshort{genai}-Adoption häufig eine individuell getriebene Exploration ohne institutionelle Rahmung -- sie experimentieren eigenständig mit Tools, ohne klare organisationale Leitlinien zu Nutzung, Grenzen und Verantwortlichkeiten. In größeren Häusern wiederum wird \acrshort{ki}-Adoption als Top-down-Projekt implementiert, was Effizienzgewinne verspricht, aber Autonomiespielräume einschränken kann \parencite{edwards_managerial_2024}.

Schließlich ist der Bankensektor durch eine \textit{Vertrauenskultur} geprägt, die über das Kunden-Berater-Verhältnis hinausreicht. Vertrauen ist das Grundkapital des Bankgeschäfts -- Kunden vertrauen ihre finanziellen Ressourcen der Bank an, Mitarbeitende vertrauen auf die Integrität interner Prozesse, Aufsichtsbehörden vertrauen auf die Selbstregulierungsfähigkeit der Institute. Wenn generative \gls{ki} in Entscheidungsprozesse eintritt, stellt sich die Vertrauensfrage auf einer neuen Ebene: Können Führungskräfte \acrshort{ki}-Outputs so weit vertrauen, dass sie Entscheidungen darauf gründen? Können sie dieses Vertrauen gegenüber Kunden und Vorgesetzten begründen? \textcite{prasad_generative_2024} zeigten, dass Vertrauen den zentralen Mediator zwischen wahrgenommener \acrshort{ki}-Nützlichkeit und tatsächlicher Akzeptanz darstellt -- ein Befund, der im vertrauensintensiven Kontext des Bankwesens besondere Relevanz besitzt.

Diese Besonderheiten -- regulatorische Dichte, Fachexpertisekultur, spezifische Entscheidungscharakteristik, strukturelle Heterogenität und Vertrauensabhängigkeit -- erzeugen ein Spannungsfeld, in dem die motivationalen Effekte generativer \gls{ki} auf mittlere Führungskräfte vermutlich anders verlaufen als in weniger regulierten oder stärker technologieaffinen Branchen. Die vorliegende Arbeit nimmt diesen spezifischen Kontext als Untersuchungsfeld, um die Wechselwirkungen zwischen \acrshort{ki}-Nutzung und den psychologischen Grundbedürfnissen von Führungskräften empirisch zu explorieren.

\section{Self-Determination Theory (SDT)}
\label{sec:sdt}

Die bisherigen Abschnitte haben gezeigt, dass generative KI in Kernbereiche professioneller Wissensarbeit eingreift und dass mittlere Führungskräfte im Bankensektor davon in spezifischer Weise betroffen sind. Offen geblieben ist die Frage, über welche psychologischen Mechanismen sich diese Veränderungen auf das motivationale Erleben auswirken. Die Self-Determination Theory bietet dafür einen empirisch breit abgestützten Erklärungsrahmen, der Motivation nicht als Quantität, sondern als Qualität begreift -- und damit einen differenzierteren Zugang eröffnet als reine Akzeptanz- oder Leistungsmodelle.

\subsection{Grundannahmen und Entstehungskontext}

Die Self-Determination Theory (SDT), entwickelt von Edward L. Deci und Richard M. Ryan, ist eine Meta-Theorie menschlicher Motivation. Ihr Ausgangspunkt ist eine Annahme, die im Kontrast zu behavioristischen und rein ökonomischen Motivationsmodellen steht: Menschen sind nicht passive Rezipienten externer Anreize, sondern aktive, wachstumsorientierte Organismen mit einer intrinsischen Tendenz, ihre Umwelt zu explorieren, Kompetenzen zu entwickeln und soziale Beziehungen aufzubauen \parencite{deciWhatWhyGoal2000}. Ob diese Tendenz sich entfaltet oder verkümmert, hängt maßgeblich von den sozialen Kontexten ab, in denen Menschen handeln.

Was SDT von anderen Motivationstheorien unterscheidet, ist ihr Fokus auf die \textit{Qualität} der Motivation, nicht bloß deren Intensität. Zwei Personen können gleich viel Energie auf eine Aufgabe verwenden und dennoch fundamental unterschiedlich motiviert sein: die eine aus genuinem Interesse, die andere aus Angst vor negativen Konsequenzen. SDT argumentiert, dass diese Unterscheidung nicht trivial ist -- sie sagt systematisch vorher, wie nachhaltig, kreativ und gesundheitsförderlich das resultierende Verhalten ausfällt \parencite{deciSelfDeterminationTheoryWork2017}.

Konkret differenziert SDT zwischen \textit{intrinsischer Motivation} -- Verhalten, das um seiner selbst willen ausgeführt wird, aus Interesse und Freude -- und \textit{extrinsischer Motivation} -- Verhalten, das instrumentell auf separate Outcomes gerichtet ist. Der theoretische Beitrag liegt darin, extrinsische Motivation nicht pauschal als defizitär zu behandeln, sondern nach dem Grad der Internalisierung zu differenzieren \parencite{deciWhatWhyGoal2000}. Das resultierende \textit{Selbstbestimmungskontinuum} reicht von Amotivation (keine Handlungsintention) über kontrollierte Formen extrinsischer Motivation -- externale Regulation durch Belohnung und Bestrafung, introjizierte Regulation durch Schuld und Selbstwertdruck -- bis hin zu autonomen Formen: identifizierte Regulation (persönliche Anerkennung des Werts einer Handlung), integrierte Regulation (Übereinstimmung mit dem Selbstkonzept) und schließlich intrinsische Motivation.

Je autonomer die Motivationsform, desto günstiger die Konsequenzen: höheres Wohlbefinden, bessere Leistung, größere Persistenz und kreativeres Problemlösen \parencite{deciSelfDeterminationTheoryWork2017}. Das ist keine bloß theoretische Unterscheidung. In einer Validierungsstudie über sieben Sprachen und neun Länder (N\,>\,3.000) bestätigten \textcite{gagneMultidimensionalWorkMotivation2015} die faktorielle Struktur des Kontinuums und zeigten, dass autonome Motivationsformen konsistent positiv mit Leistung, Wohlbefinden und organisationalem Commitment assoziiert waren, während kontrollierte Motivation schwächere oder inkonsistente Effekte aufwies.

Der entscheidende Mechanismus, der erklärt, \textit{warum} bestimmte Kontexte autonome Motivation fördern und andere sie untergraben, liegt in der Befriedigung psychologischer Grundbedürfnisse -- dem Kernstück der Theorie.


\subsection{Die drei psychologischen Grundbedürfnisse}

SDT postuliert drei fundamentale psychologische Grundbedürfnisse, deren Befriedigung für psychologisches Wachstum, Integrität und Wohlbefinden essenziell ist: Autonomie, Kompetenz und soziale Eingebundenheit \parencite{deciWhatWhyGoal2000, vandenbroeckReviewSelfDeterminationTheorys2016}. Die Theorie versteht diese nicht als individuelle Präferenzen, die von Person zu Person variieren, sondern als universelle Nährstoffe -- vergleichbar mit physiologischen Bedürfnissen, deren Frustration unabhängig von kulturellem Kontext oder persönlicher Disposition zu Beeinträchtigungen führt.

\subsubsection{Autonomie}

Autonomie bezeichnet das Bedürfnis, sich als Ursprung des eigenen Verhaltens zu erleben -- selbstbestimmt und in Übereinstimmung mit den eigenen Werten zu handeln \parencite{deciWhatWhyGoal2000}. Der Begriff wird häufig missverstanden: Autonomie meint nicht Unabhängigkeit oder Isolation, sondern \textit{Volition}. Eine Führungskraft, die eine strategische Vorgabe umsetzt, handelt autonom, solange sie die Vorgabe als sinnvoll anerkennt und den Umsetzungsweg selbst gestalten kann. Dieselbe Vorgabe wird zum Autonomiefresser, wenn sie als willkürliche Kontrolle erlebt wird, der man sich fügen muss.

Autonomie wird gefördert durch Wahlmöglichkeiten, Partizipation an Entscheidungen, Bereitstellung von Rationalen und Minimierung von Kontrolle und Druck. Sie wird frustriert, wenn Verhalten durch externe Kräfte oder internalisierte Druckmechanismen kontrolliert wird \parencite{vandenbroeckReviewSelfDeterminationTheorys2016}. Für Führungskräfte ist Autonomie besonders relevant, da ihre Rolle traditionell mit Entscheidungsfreiheit und strategischem Gestaltungsspielraum assoziiert wird. Technologien, die Entscheidungsspielräume einschränken oder algorithmische Kontrolle ausüben, können daher Reaktanz erzeugen, die weit über das Maß hinausgeht, das bei operativen Mitarbeitenden zu beobachten wäre \parencite{edwardsManagerialControlFeedback2024}.

\subsubsection{Kompetenz}

Kompetenz bezeichnet das Bedürfnis, sich als effektiv und fähig zu erleben -- Herausforderungen erfolgreich zu meistern und kontinuierlich zu lernen \parencite{deciWhatWhyGoal2000}. Kompetenzerleben ist nicht identisch mit objektiver Kompetenz; es bezieht sich auf die subjektive Wahrnehmung von Wirksamkeit und Meisterschaft. Eine Führungskraft kann objektiv kompetent sein und sich dennoch inkompetent fühlen, wenn ein KI-System dieselbe Analyse in Sekunden erstellt, für die sie Stunden benötigt.

Optimales Kompetenzerleben entsteht, wenn Aufgaben weder zu einfach noch zu überfordernd sind, sondern im Bereich der optimalen Herausforderung liegen -- ein Konzept, das Parallelen zum Flow-Erleben aufweist \parencite{csikszentmihalyiFlowPsychologyOptimal2009}, aber breiter gefasst ist. Kompetenz wird gefördert durch klares, konstruktives Feedback, erreichbare aber herausfordernde Ziele und Gelegenheiten zur Kompetenzentwicklung \parencite{vandenbroeckReviewSelfDeterminationTheorys2016}. Sie wird frustriert durch Überforderung, intransparente Bewertungskriterien oder die Entwertung erworbener Expertise.

\subsubsection{Soziale Eingebundenheit}

Soziale Eingebundenheit (Relatedness) bezeichnet das Bedürfnis, sich mit anderen verbunden und zugehörig zu fühlen -- Beziehungen zu pflegen, die durch gegenseitige Fürsorge, Respekt und Vertrauen gekennzeichnet sind \parencite{deciWhatWhyGoal2000}. Im Arbeitskontext umfasst dies Zugehörigkeit zu Teams und Organisationen, unterstützende Beziehungen zu Kollegen und Vorgesetzten sowie das Gefühl, einen Beitrag zu einer größeren Gemeinschaft zu leisten \parencite{vandenbroeckReviewSelfDeterminationTheorys2016}.

Im Vergleich zu Autonomie und Kompetenz wird soziale Eingebundenheit in der SDT-Forschung zum Arbeitskontext manchmal als nachrangig behandelt. \textcite{deciSelfDeterminationTheoryWork2017} argumentieren jedoch, dass Relatedness eine notwendige Voraussetzung für nachhaltige Internalisierung extrinsischer Motivation darstellt: Menschen übernehmen Werte und Praktiken ihrer sozialen Umgebung eher, wenn sie sich dieser Umgebung zugehörig fühlen. Für Führungskräfte, deren Wirksamkeit wesentlich auf Beziehungsqualität beruht -- gegenüber Teams, Peers und Vorgesetzten --, ist dieses Bedürfnis keineswegs sekundär.


\subsection{SDT im Arbeitskontext}

SDT hat sich als produktiver theoretischer Rahmen für die Arbeits- und Organisationspsychologie etabliert, mit umfangreicher empirischer Evidenz zu Arbeitsmotivation, Leistung, Kreativität, Wohlbefinden, Burnout und organisationalem Commitment \parencite{deciSelfDeterminationTheoryWork2017, gagneUnderstandingShapingFuture2022}.

Der zentrale Wirkmechanismus ist gut belegt: Arbeitskontexte, die die drei Grundbedürfnisse befriedigen, fördern autonome Motivation, die wiederum positive Outcomes begünstigt. Meta-Analysen zeigen konsistent, dass Bedürfnisbefriedigung am Arbeitsplatz positiv mit Wohlbefinden und negativ mit Burnout und Ill-Being assoziiert ist -- robust über Kulturen, Berufe und Messmethoden hinweg \parencite{vandenbroeckReviewSelfDeterminationTheorys2016}. Bedürfnisbefriedigung fördert dabei nicht nur Wohlbefinden, sondern auch Leistung und Engagement: \textcite{vandenbroeckReviewSelfDeterminationTheorys2016} fanden signifikante positive Zusammenhänge mit Job Performance, organisationalem Commitment und proaktivem Verhalten.

Konzeptionell und empirisch bedeutsam ist die Unterscheidung zwischen Bedürfnisbefriedigung (need satisfaction) und Bedürfnisfrustration (need frustration). Frustration tritt auf, wenn Bedürfnisse aktiv blockiert oder untergraben werden -- nicht bloß abwesend sind \parencite{vandenbroeckCapturingAutonomyCompetence2010}. Dieser Unterschied ist nicht akademisch: Bedürfnisfrustration erweist sich als stärkerer Prädiktor für negative Outcomes (Burnout, kontraproduktives Verhalten) als Bedürfnisbefriedigung für positive \parencite{vandenbroeckReviewSelfDeterminationTheorys2016}. Die Asymmetrie hat praktische Implikationen: Technologien, die Bedürfnisse aktiv frustrieren -- etwa durch Kontrollerleben oder Expertiseentwertung --, dürften gravierendere motivationale Konsequenzen haben als Technologien, die Bedürfnisse lediglich nicht fördern.

Ein zentrales Konzept der angewandten SDT-Forschung ist \textit{Autonomy Support} -- Führungsverhalten und organisationale Praktiken, die Autonomie, Kompetenz und Relatedness fördern. Autonomieunterstützung umfasst: Perspektiven anerkennen, Wahlmöglichkeiten bieten, Rationale bereitstellen und Kontrolle minimieren \parencite{deciSelfDeterminationTheoryWork2017}. Meta-Analysen bestätigen, dass autonomieunterstützende Führung signifikant positiv mit Bedürfnisbefriedigung, autonomer Motivation und Leistung assoziiert ist, während kontrollierende Führung mit Bedürfnisfrustration und negativen Outcomes einhergeht \parencite{vandenbroeckReviewSelfDeterminationTheorys2016}.

Für die vorliegende Arbeit ist dieser Befund in doppelter Hinsicht relevant. Erstens sind mittlere Führungskräfte selbst Empfänger von Autonomieunterstützung (oder -kontrolle) durch ihre Vorgesetzten und durch organisationale Strukturen -- zu denen zunehmend auch algorithmische Systeme gehören. Zweitens sind sie Gestalter von Autonomieunterstützung für ihre Teams, wobei KI-Tools diese Gestaltungsarbeit sowohl erleichtern als auch erschweren können.


\subsection{Digitale Arbeitssysteme als bedürfnisunterstützende oder -frustrierende Faktoren}

SDT wurde zunehmend auf Technologieakzeptanz und -nutzung angewendet, häufig in Integration mit dem Technology Acceptance Model (TAM) und der Unified Theory of Acceptance and Use of Technology (UTAUT) \parencite{KoenigPascal2024, YangHsiHsun2024}. Die Kernfrage lautet: Unter welchen Bedingungen unterstützen oder frustrieren digitale Arbeitssysteme die drei psychologischen Grundbedürfnisse?

SDT erweitert rein kognitive Akzeptanzmodelle um eine motivationale Erklärungsebene. TAM und UTAUT postulieren, dass wahrgenommene Nützlichkeit und Benutzerfreundlichkeit die Nutzungsintention bestimmen. SDT fragt darüber hinaus: \textit{Wie} wird Technologie erlebt? Fördert sie das Gefühl von Selbstbestimmung, oder erzeugt sie Kontrollerleben? Stärkt sie die eigene Wirksamkeit, oder untergräbt sie professionelle Identität? Verbindet sie Menschen, oder isoliert sie? \parencite{KoenigPascal2024}

Empirisch zeigen sich ambivalente Befunde. Auf der einen Seite können digitale Arbeitssysteme Autonomie erweitern (etwa durch flexible Arbeitsorganisation), Kompetenz fördern (durch Zugang zu Informationen und Feedback-Systemen) und Relatedness stärken (durch Kommunikationstools und kollaborative Plattformen) \parencite{gagneUnderstandingShapingFuture2022}. Auf der anderen Seite dokumentiert die Forschung konsistent auch negative Pfade: Automatisierung kann wahrgenommene Arbeitsbedeutsamkeit und Autonomie reduzieren -- eine Studie fand bei einer 7,5-fachen Steigerung der Robotisierung projizierte Rückgänge von 6,8\,\% bei Bedeutsamkeit und 7,5\,\% bei Autonomie \parencite{nikolovaRobotsMeaningSelfdetermination2024}. Wahrgenommene Technologieunsicherheit fungiert als Demand, die Technostress erhöht und Arbeitszufriedenheit reduziert \parencite{LiuWangLin2023}.

\textcite{KoenigPascal2024} entwickelten ein theoretisches Framework, das drei Akzeptanzperspektiven integriert: Nutzerakzeptanz, Delegationsakzeptanz und gesellschaftliche Adoption. Für jede Perspektive identifizierten sie spezifische SDT-relevante Faktoren: Nutzerakzeptanz hängt primär von Kompetenz- und Autonomieerleben ab; Delegationsakzeptanz von Vertrauen und wahrgenommener Zuverlässigkeit; gesellschaftliche Adoption von kollektiven Werten und sozialen Normen.

Speziell für generative KI zeigt eine Studie mit 565 Designprofessionellen die Erklärungskraft eines integrierten UTAUT-SDT-Modells: Es klärte 52,1\,\% der Varianz in der Verhaltensintention auf. Bemerkenswert war, dass die wahrgenommene Bedrohung durch Jobersatz die Beziehung zwischen Leistungserwartung und Nutzungsintention negativ moderierte -- selbst Personen, die die Leistungsfähigkeit der Tools anerkannten, zeigten geringere Nutzungsbereitschaft, wenn sie KI als Bedrohung ihrer beruflichen Existenz wahrnahmen \parencite{YangHsiHsun2024}.

Besonders aufschlussreich für die vorliegende Arbeit ist ein experimenteller Befund von \textcite{WuLiuRuan2025}. In einer Studie mit 15.105 Teilnehmenden steigerte GenAI-Unterstützung die Aufgabenleistung signifikant, reduzierte aber gleichzeitig die intrinsische Motivation. Der Mechanismus: Nutzer attribuierten den Erfolg der KI statt sich selbst und erlebten dadurch reduzierte Selbstwirksamkeit. Kompetenzerleben wurde also nicht durch objektiven Misserfolg frustriert, sondern durch die Verschiebung der Erfolgsattribution -- ein Paradox, das für Führungskräfte besonders relevant sein dürfte, deren professionelle Identität eng an persönliche Expertise geknüpft ist.

Trotz dieser Fortschritte bleibt die Anwendung von SDT auf generative KI im Arbeitskontext begrenzt. Generative KI unterscheidet sich von den bisher untersuchten Technologien durch drei Eigenschaften, die theoretische Erweiterungen erfordern \parencite{gagneUnderstandingShapingFuture2022}: Erstens ihre \textit{Ko-Kreationsfähigkeit} -- GenAI ist nicht bloß Werkzeug, sondern Dialogpartner, der Inhalte mitgestaltet. Zweitens ihre \textit{Kontextsensitivität} -- die Qualität der Interaktion hängt von der Kompetenz der nutzenden Person ab (Prompt Engineering), was eine rekursive Beziehung zwischen Technologienutzung und Kompetenzerleben erzeugt. Drittens ihr \textit{Potenzial zur Identitätsbedrohung} -- GenAI kann Kernaufgaben professioneller Wissensarbeit substituieren, nicht nur Routinetätigkeiten.

Diese drei Eigenschaften machen generative KI zu einem Untersuchungsgegenstand, der über die bisherige SDT-Technologieforschung hinausgeht und spezifische empirische Zugänge erfordert -- eine Lücke, die die vorliegende Arbeit qualitativ zu adressieren versucht.

\subsection{Grenzen und Randbedingungen der SDT im Kontext organisationaler KI-Forschung}
\label{subsec:sdt-grenzen}

Die bisherige Darstellung hat die Self-Determination Theory als tragfähigen Rahmen für die Analyse motivationaler Wirkungen digitaler Arbeitssysteme eingeführt. Bevor dieser Rahmen auf den spezifischen Untersuchungsgegenstand -- generative KI in der Führungsarbeit -- angewendet wird, verdienen einige theoretische Grenzen Beachtung. Sie schmälern nicht die Eignung der SDT für die vorliegende Fragestellung, verweisen aber auf Bereiche, in denen die Theorie unterspezifiziert bleibt und explorative Zugänge erfordert.

\paragraph{Kulturelle Universalität und der Autonomiebegriff.}
SDT postuliert die drei psychologischen Grundbedürfnisse als universell -- eine Annahme, die wiederholt auf kulturelle Einwände gestoßen ist. Chirkov et al. (2003) zeigten zwar, dass Autonomie auch in kollektivistischen Kontexten motivational wirksam bleibt, wiesen aber zugleich nach, dass die \textit{Art und Weise}, wie Autonomie erlebt und ausgedrückt wird, kulturell variiert.\footnote{%
  Chirkov, V., Ryan, R. M., Kim, Y. \& Kaplan, U. (2003). Differentiating autonomy from individualism and independence: A self-determination theory perspective on internalization of cultural orientations and well-being. \textit{Journal of Personality and Social Psychology}, \textit{84}(1), 97--110. \url{https://doi.org/10.1037/0022-3514.84.1.97} -- [Quelle in Zotero ergänzen]}
Für die vorliegende Arbeit ist das keineswegs trivial: Die DACH-Region zeichnet sich durch eine ausgeprägte Konsens- und Mitbestimmungskultur aus, in der Autonomie weniger als individuelle Unabhängigkeit, sondern als Partizipation an Entscheidungsprozessen verstanden werden dürfte. Ob und wie Führungskräfte in diesem Kontext die KI-induzierte Verschiebung von Entscheidungsstrukturen als autonomiefördernd oder -einschränkend erleben, lässt sich aus der SDT allein nicht ableiten -- hier braucht es empirische Rekonstruktion.

\paragraph{Korrelative Evidenzbasis.}
McAnally und Hagger (2024) haben in einem konzeptionellen Review drei zentrale methodische Schwächen der arbeitspsychologischen SDT-Forschung identifiziert: eine Dominanz querschnittlicher Designs, den Mangel an Interventionsstudien mit explizitem Mediationstest und die unzureichende Berücksichtigung moderierender Variablen \parencite[vgl.][]{mcanallySelfdeterminationTheoryWorkplace2024}. Die allermeisten Befunde zur Bedürfnisbefriedigung am Arbeitsplatz beruhen auf korrelativen Daten, die keine kausalen Schlüsse erlauben. Auch die Debatte um den sogenannten Undermining-Effekt -- die These, dass extrinsische Anreize intrinsische Motivation verdrängen -- wird durch widersprüchliche Befunde aus Labor- und Feldstudien verkompliziert. Saini, Uppal und Howard (2025) zeigten in einer aktuellen Metaanalyse, dass der Effekt unter kontrollierten Laborbedingungen robust auftritt, in Feldstudien mit realen Arbeitsbedingungen jedoch deutlich schwächer und inkonsistenter ausfällt.\footnote{%
  Saini, G. K., Uppal, N. \& Howard, J. L. (2025). The undermining effect of extrinsic rewards: Fact or artefact? \textit{Journal of Occupational and Organizational Psychology}, \textit{98}, e70000. -- [Quelle in Zotero ergänzen]}
Für die vorliegende Studie folgt daraus: Die SDT liefert ein konzeptionell überzeugendes, aber empirisch noch unvollständig gesichertes Erklärungsmodell. Gerade deshalb erscheint ein qualitativer Zugang sinnvoll, der Wirkzusammenhänge nicht voraussetzt, sondern rekonstruiert.

\paragraph{Fehlende Identitäts- und Institutionenperspektive.}
Eine auffällige Leerstelle der SDT betrifft die Rolle professioneller Identität. Die vorangegangenen Abschnitte haben gezeigt, dass generative KI Kompetenzerleben nicht nur über Aufgabenbewältigung, sondern auch über die wahrgenommene Entwertung beruflicher Expertise beeinflusst. Dieser Mechanismus -- die Bedrohung des Selbstverständnisses als fachliche Autorität -- wird von der SDT nicht abgebildet. Sie modelliert Kompetenz als funktionales Erleben von Wirksamkeit, nicht als identitätsgebundene Zuschreibung. Ähnlich verhält es sich mit institutionellen Rahmenbedingungen: Der Bankensektor ist durch eine dichte regulatorische Umgebung geprägt, die Handlungsspielräume nicht nur faktisch einschränkt, sondern auch normativ rahmt. SDT verfügt über keinen Mechanismus, der erklärt, wie regulatorischer Druck die Bedürfnisbefriedigung moderiert -- ob etwa Compliance-Anforderungen als externe Kontrolle (und damit autonomiefrustrierend) oder als professionelle Selbstverständlichkeit (und damit motivational neutral) erlebt werden, hängt von Kontextfaktoren ab, die außerhalb der Reichweite der Theorie liegen.

\paragraph{Teamebene als blinder Fleck.}
Grenier et al. (2024) haben darauf hingewiesen, dass die SDT trotz ihrer breiten Rezeption in der Arbeitspsychologie auf Teamebene kaum systematisch angewendet wurde.\footnote{%
  Grenier, S., Bhatt, M. \& Bhimani, R. (2024). A systematic review of self-determination theory in teams. \textit{Applied Psychology}, \textit{73}(3), 1286--1312. \url{https://doi.org/10.1111/apps.12526} -- [Quelle in Zotero ergänzen]}
Das Bedürfnis nach sozialer Eingebundenheit wird vorwiegend als dyadisches oder individuelles Konstrukt operationalisiert; wie sich Teamdynamiken -- etwa geteilte Verantwortung, kollektive Kompetenzzuschreibung oder die soziale Aushandlung von KI-Nutzung -- auf die Bedürfnisbefriedigung auswirken, bleibt unterbestimmt. Für Führungskräfte im mittleren Management, deren Arbeit wesentlich durch die Koordination von Teams und die Vermittlung zwischen Hierarchieebenen geprägt ist, stellt diese Lücke eine relevante Einschränkung dar.

\paragraph{Konsequenzen für das Forschungsdesign.}
Diese Grenzen entwerten die SDT nicht als Analyserahmen -- sie präzisieren den Anspruch, mit dem die Theorie in dieser Arbeit eingesetzt wird. Die drei psychologischen Grundbedürfnisse strukturieren die Analyse als \textit{heuristische Kategorien}, nicht als kausaltheoretisches Erklärungsmodell. Gerade weil die SDT an den beschriebenen Stellen unterspezifiziert ist, braucht es einen explorativen Zugang, der empirisch rekonstruiert, wie Führungskräfte die motivationalen Wirkungen generativer KI tatsächlich erleben und deuten. Die qualitative Methodik dieser Arbeit -- problemzentrierte Interviews mit induktiver Subkategorienbildung (vgl. Kapitel~3) -- ist darauf angelegt, genau diese Leerstellen zu adressieren, indem sie Phänomene sichtbar macht, die deduktiv aus der SDT allein nicht ableitbar wären.
\section{Synthese: Motivationale Wirkung von KI in der Führungsarbeit}

Die vorangegangenen Abschnitte haben drei Perspektiven entfaltet, die nun zusammengeführt werden: generative KI als soziotechnisches System mit spezifischer Eingriffstiefe in wissensintensive Arbeit (Abschnitt 2.1), Führungsarbeit im mittleren Management des Bankensektors mit ihrer charakteristischen Verknüpfung von Entscheidungskompetenz, Expertiseidentität und regulatorischer Einbettung (Abschnitt 2.2) sowie die Self-Determination Theory als motivationspsychologischer Erklärungsrahmen (Abschnitt 2.3). Der vorliegende Abschnitt integriert diese Stränge zu einem konzeptionellen Analyserahmen, identifiziert die zentrale Forschungslücke und formuliert die Fragestellung der Arbeit.

\subsection{Generative KI und Autonomieerleben.}

Das Autonomiebedürfnis mittlerer Führungskräfte wird durch generative KI auf mindestens zwei Wegen berührt. Erstens kann KI als Ermöglicherin von Autonomie wirken: Sie entlastet von Routineaufgaben, beschleunigt Informationssynthesen und erweitert den Handlungsspielraum für strategische Gestaltung. Zweitens kann sie als subtiles Kontrollsystem erfahren werden -- dann nämlich, wenn algorithmische Empfehlungen faktisch den Entscheidungskorridor verengen, wenn Transparenzanforderungen die Begründungslast erhöhen oder wenn die Organisation KI-gestützte Prozesse als Steuerungsinstrument implementiert.

Die empirische Forschung stützt diese Ambivalenz. \textcite{edwards_managerial_2024} zeigten in einer Drei-Wellen-Studie (N\,=\,401), dass die motivationalen Konsequenzen algorithmischer HR-Systeme davon abhängen, wie Beschäftigte deren Zweck attribuieren: Wer das System als Kontrollinstrument der Führung wahrnahm, erlebte erhöhte extrinsische Motivation und priorisierte selektiv messbare Aufgaben. Wer es als Feedback-Instrument deutete, berichtete höhere intrinsische Motivation und geringere emotionale Erschöpfung. Die Attributionsrichtung -- Kontrolle oder Unterstützung -- erwies sich als entscheidender als das System selbst. \textcite{klonek_does_2025} bestätigten dieses Muster in einer Text-Mining-Studie mit über 2.700 Beiträgen: Hohe KI-Kontrolle war signifikant mit erhöhtem Stress assoziiert, während Mensch-KI-Teamprozesse -- insbesondere aktions- und interpersonenbezogene -- Stress reduzierten und den negativen Effekt der Kontrolle abpufferten.

Für mittlere Führungskräfte im Bankensektor verschärft sich diese Dynamik durch die regulatorische Struktur. Compliance-Anforderungen erzeugen bereits ohne KI enge Entscheidungskorridore; GenAI-Systeme, die regulatorisch gebotene Analysen automatisieren, können die erlebte Autonomie weiter einschränken, selbst wenn sie objektiv Entlastung bieten. Die Frage ist nicht, ob Autonomie tangiert wird, sondern wie Führungskräfte die Verschiebung erleben und deuten.

\subsection{Generative KI und Kompetenzerleben.}

Das Kompetenzerleben mittlerer Führungskräfte ist im Kontext generativer KI besonders verwundbar. Im Unterschied zu klassischen Automatisierungstechnologien, die primär manuelle oder standardisierte Aufgaben betreffen, greift GenAI in Kernbereiche professioneller Wissensarbeit ein: Analysen erstellen, Strategien entwerfen, Kommunikation gestalten \parencite{brynjolfssonGenerativeAIWork2025}. Das sind genau die Tätigkeiten, über die mittlere Führungskräfte im Bankensektor traditionell ihre professionelle Identität definieren.

Die experimentelle Studie von \textcite{WuLiuRuan2025} (N\,=\,15.105) hat den Mechanismus aufgedeckt: GenAI-Unterstützung steigerte die Aufgabenleistung, reduzierte aber gleichzeitig die intrinsische Motivation -- weil Teilnehmende den Erfolg der KI statt sich selbst attribuierten. Kompetenzerleben wurde nicht durch Scheitern frustriert, sondern durch den Verlust der Erfolgsattribution. Für Führungskräfte, deren Wirksamkeitserleben eng an persönliche Expertise geknüpft ist, dürfte dieser Mechanismus besonders ausgeprägt sein.

Gleichzeitig eröffnet GenAI neue Kompetenzdomänen. Die Fähigkeit, effektive Prompts zu formulieren, KI-Outputs kritisch zu evaluieren und Mensch-KI-Zusammenarbeit zu gestalten, wird zunehmend als eigenständige professionelle Kompetenz verstanden \parencite{dellacquaNavigatingJaggedTechnological2023}. \textcite{smith_navigating_2025} zeigten in vier experimentellen Studien (N\,$\approx$\,1.100), dass die freiwillige Einholung von KI-Empfehlungen -- im Vergleich zur erzwungenen Konfrontation -- die Bereitschaft signifikant erhöhte, KI-Ratschläge zu akzeptieren. Die Wahlfreiheit erzeugte ein Erleben von Kompetenz und Kontrolle über die Interaktion, das die SDT-Logik direkt bestätigte: Optionale KI-Nutzung stützt sowohl Autonomie- als auch Kompetenzerleben, obligatorische untergräbt beides.

Das resultierende Spannungsfeld ist für den Bankensektor besonders ausgeprägt. Die Entscheidungsarbeit mittlerer Führungskräfte verbindet dort analytische Kompetenz mit kontextuellem Erfahrungswissen -- etwa bei Kreditentscheidungen, in denen formale Risikomodelle und langjährige Kundenkenntnis zusammenfließen. Wenn GenAI die analytische Komponente substituiert, stellt sich die Frage, wie das verbleibende Erfahrungswissen erlebt wird: als aufgewertete Kernkompetenz oder als Residualkategorie.

\subsection{Generative KI und soziale Eingebundenheit.}

Die Auswirkungen generativer KI auf soziale Eingebundenheit sind bislang am wenigsten empirisch untersucht, theoretisch aber plausibel. Drei Wirkpfade lassen sich antizipieren.

Erstens verändert GenAI die Interaktionsstruktur in Teams. Wenn Aufgaben, die bisher kooperativ gelöst wurden -- Berichterstellung, Datenanalyse, Entscheidungsvorbereitung -- nun einzeln an KI-Systeme delegiert werden, reduziert sich die aufgabenbezogene Interdependenz. \textcite{klonek_does_2025} argumentierten, dass gerade interpersonale Mensch-KI-Teamprozesse als soziale Ressource wirken, die Stress reduziert. Fehlen diese Prozesse, entfällt auch der Puffer.

Zweitens kann sich das Verhältnis zur eigenen Führungsrolle verschieben. \textcite{quaquebeke_now_2023} argumentierten, dass Führung in KI-Kontexten zunehmend weniger über Informationsvorsprung und Expertise definiert wird und stärker über Beziehungsqualität und Sinngebung. Das birgt Chancen -- eine Intensivierung genuin menschlicher Führungsarbeit --, setzt aber voraus, dass Führungskräfte diese Rollenverschiebung nicht als Kompetenzverlust, sondern als Rollenerweiterung erleben.

Drittens berührt KI die wahrgenommene organisationale Wertschätzung. \textcite{lagios_explaining_2022} zeigten in einer Zwei-Wellen-Studie (N\,=\,603), dass organisationale Dehumanisierung -- das Erleben, als bloßes Werkzeug behandelt zu werden -- alle drei psychologischen Grundbedürfnisse frustriert, mit negativen Konsequenzen für Wohlbefinden, Zufriedenheit und Bindung. Wenn Organisationen GenAI primär als Effizienzinstrument implementieren, ohne die Perspektive der Führungskräfte einzubeziehen, besteht das Risiko, dass sich deren Wertschätzungserleben verschlechtert -- nicht weil die Technologie inhärent dehumanisierend wirkt, sondern weil der Implementierungsprozess es ist.

\subsection{Das Paradox der Entlastung.}

Die Synthese der empirischen Befunde offenbart ein zentrales Paradox. Generative KI verspricht Führungskräften Entlastung von Routineaufgaben und informationeller Überflutung -- eine Ressource im Sinne der JD-R-Theorie, die Demands reduziert und Handlungsspielräume erweitert \parencite{bakker_job_2007}. Gleichzeitig greift sie in genau jene Tätigkeiten ein, über die Führungskräfte Kompetenz, Autonomie und Identität definieren. Die Entlastung kann zum Kontrollverlust werden, die Effizienzsteigerung zur Expertiseentwertung, die Prozessoptimierung zur Beziehungsverarmung.

Ob das eine oder das andere eintritt, ist -- das legen die Befunde nahe -- kein Determinismus der Technologie. \textcite{tongJanusFaceArtificial2021} beschrieben KI treffend als Janusgesicht: bedrohlich und vielversprechend zugleich. Die motivationale Wirkung hängt von einer Konfiguration aus individuellen, organisationalen und technologischen Faktoren ab \parencite{bankinsMultilevelReviewArtificial2024}. SDT bietet einen Erklärungsrahmen, der über die binäre Logik von Akzeptanz oder Ablehnung hinausgeht, indem sie die Mechanismen offenlegt: Nicht die Technologienutzung per se, sondern deren Wirkung auf Autonomie, Kompetenz und soziale Eingebundenheit bestimmt, ob GenAI als motivationale Ressource oder als Belastung erlebt wird.

\subsection{Forschungslücke und Fragestellung.}

Trotz der wachsenden Literatur zu KI im Arbeitskontext bestehen substanzielle Lücken, die den Ausgangspunkt dieser Arbeit bilden.

Die bestehende empirische Forschung ist überwiegend quantitativ und erfasst breit angelegte Zusammenhänge zwischen KI-Nutzung und motivationalen oder leistungsbezogenen Outcomes -- typischerweise über standardisierte Fragebögen, die aggregierte Bewertungen abbilden \parencite{WuLiuRuan2025, prasad_generative_2024, YangHsiHsun2024}. Was diese Studien konzeptionell nicht leisten können, ist die Rekonstruktion der subjektiven Erfahrungswelt: Wie erleben Führungskräfte den konkreten Moment, in dem ein KI-System eine Analyse erstellt, die sie bisher selbst verantwortet haben? Welche Deutungsmuster mobilisieren sie, um ihre Rolle im Verhältnis zur Technologie zu verorten? Wie verschiebt sich ihr Erleben über die Zeit?

Zudem fokussiert die Mehrzahl der Studien auf operative Beschäftigte oder breite Populationen -- nicht auf mittleres Management, dessen spezifische Positionierung zwischen strategischer Planung und operativer Umsetzung eine besondere Betroffenheit durch GenAI begründet \parencite{klonekDoesAIWork2025}. Die wenigen Studien, die Führungskräfte explizit adressieren, untersuchen deren Umgang mit KI in der Teamsteuerung \parencite{monod_worker_2024}, nicht deren eigenes motivationales Erleben als KI-Nutzer.

Schließlich fehlt eine sektorbezogene Perspektive für den DACH-Raum. Der Bankensektor mit seiner Kombination aus hoher Regulierungsdichte, wissensintensiver Entscheidungsarbeit und ambivalenter Digitalisierungsgeschichte (vgl. Abschnitt 2.1.3) stellt spezifische Kontextbedingungen bereit, die in der bisherigen Forschung nicht abgebildet sind.

Die vorliegende Arbeit adressiert diese Lücken mit folgender Forschungsfrage:

\begin{quote}
\textit{Wie erleben mittlere Führungskräfte im Bankensektor (DACH-Region) die Auswirkungen generativer KI-Tools auf ihre psychologischen Grundbedürfnisse nach Autonomie, Kompetenz und sozialer Eingebundenheit gemäß der Self-Determination Theory?}
\end{quote}

Der qualitative Zugang über problemzentrierte Leitfadeninterviews ermöglicht es, die subjektive Erfahrungswelt der Betroffenen zu rekonstruieren, Deutungsmuster zu identifizieren und die Bedingungskonfigurationen herauszuarbeiten, unter denen generative KI bedürfnisunterstützend oder -frustrierend wirkt. Die SDT-Grundbedürfnisse dienen dabei als deduktive Analyselinse, die durch induktive Subkategorien ergänzt wird (vgl. Kapitel 3).
\chapter{Methodisches Vorgehen}

Die Forschungsfrage dieser Arbeit zielt auf subjektives Erleben, Deutungsmuster und psychologische Wahrnehmungsprozesse -- Gegenstände, die einen methodischen Zugang erfordern, der Offenheit und theoretische Strukturierung verbindet. Das folgende Kapitel legt das Forschungsdesign dar, beschreibt Erhebung, Aufbereitung und Auswertung der Daten und reflektiert die Gütekriterien sowie die ethischen Rahmenbedingungen der Untersuchung.

\section{Forschungsdesign und -paradigma}
\label{met:forschungsdesign}
Die vorliegende Arbeit folgt einem qualitativen Forschungsdesign. Diese Entscheidung ergibt sich aus der Natur der Forschungsfrage: Im Mittelpunkt steht das subjektive Erleben von Führungskräften im Umgang mit generativer \gls{ki} -- also Sinnzuschreibungen, Deutungsmuster und psychologische Wahrnehmungsprozesse, die sich einer standardisierten 
Messung grundsätzlich entziehen \parencite{flick2017, mayring2016}. Qualitative Methoden bieten sich dort an, wo soziale Phänomene aus der Innenperspektive der Beteiligten erschlossen werden sollen und wo der theoretische Zugang noch vergleichsweise offen ist \parencite{przyborski2021}.
Wissenschaftstheoretisch bewegt sich die Arbeit im interpretativen Paradigma. Ausgangspunkt ist die Überzeugung, dass soziale Wirklichkeit -- hier konkret: das motivationale Erleben von Führungsarbeit unter dem Einsatz generativer KI -- nicht einfach vorgefunden, sondern durch Wahrnehmungen und Deutungen der handelnden Personen erst hervorgebracht wird \parencite{berger1969}. In diesem konstruktivistischen Verständnis geht es nicht darum, ob generative KI die psychologischen Grundbedürfnisse nach Autonomie, Kompetenz und sozialer Eingebundenheit objektiv verändert. Interessanter ist die Frage, wie Führungskräfte solche Veränderungen wahrnehmen und welche Bedeutung sie ihnen in konkreten Handlungssituationen beimessen.
Die Entscheidung für ein qualitatives Design lässt sich auch mit dem Stand der Forschung begründen. Studien zu \gls{ki} in Organisationen setzen bislang überwiegend auf quantitative Designs und arbeiten mit aggregierten Effektmaßen \parencite{bankinsMultilevelReviewArtificial2024}. 
Was dabei weitgehend fehlt, sind Untersuchungen, die die prozessualen und situativen Bedingungen motivationaler Wirkungen in realen Führungskontexten nachzeichnen. Die vorliegende Arbeit versteht sich als explorativer Beitrag mit dem Ziel, theoretisch gehaltvolle Einsichten zu gewinnen -- ohne dabei den Anspruch statistischer Verallgemeinerung zu erheben \parencite{strauss1996}.
\section{Datenerhebung: Problemzentrierte Leitfadeninterviews}

Zur Erhebung der empirischen Daten wurden problemzentrierte Leitfadeninterviews eingesetzt. Diese Methode verbindet die Offenheit qualitativer Gesprächsführung mit einer thematischen Fokussierung auf ein vorab definiertes Problemfeld -- in diesem Fall 
das motivationale Erleben von Führungskräften im Umgang mit generativer \gls{ki} \parencite{witzelProblemcenteredInterview2000}. Im Vergleich zu vollständig narrativen oder unstrukturierten Formaten ermöglicht das problemzentrierte Interview eine gezieltere Erschließung theoretisch relevanter Inhaltsbereiche, ohne die Gesprächspartner in 
vorgegebene Antwortkategorien zu drängen \parencite{flick2017}.

\subsection{Interviewleitfaden und Entwicklungsprozess}

Der Interviewleitfaden wurde in mehreren Schritten entwickelt und dabei konsequent an der Forschungsfrage sowie den zentralen Konstrukten der Selbstbestimmungstheorie ausgerichtet \parencite{deciSelfDeterminationTheoryWork2017}. In einem ersten Schritt wurden auf Basis der theoretischen Vorüberlegungen thematische Blöcke formuliert, die die drei psychologischen Grundbedürfnisse -- Autonomie, Kompetenz und soziale 
Eingebundenheit -- als Strukturierungsprinzip aufgreifen. Ergänzt wurden diese durch einen einleitenden Block zur beruflichen Situation und zum konkreten \acrshort{ki}-Einsatz im Arbeitsalltag sowie durch einen abschließenden Block, der Raum für persönliche Bewertungen und weiterführende Gedanken ließ.

Die Fragen wurden bewusst offen formuliert, um Erzählimpulse zu setzen, anstatt Antwortrichtungen vorzugeben. Auf suggestive oder wertende Formulierungen wurde verzichtet. Nach einer ersten Fassung des Leitfadens erfolgte ein kognitives Pretest-Verfahren, in dem der Leitfaden hinsichtlich Verständlichkeit, Gesprächsfluss und thematischer Vollständigkeit überprüft wurde. Auf dieser 
Grundlage wurden einzelne Fragen umformuliert und die Reihenfolge der Blöcke angepasst. Der finale Leitfaden umfasst [X] Hauptfragen mit jeweils vorbereiteten Nachfragen und dient im Interview als Orientierungsrahmen, nicht als starres Skript \parencite{helfferich_qualitat_2011}.

\subsection{Sampling und Stichprobenbeschreibung}

Die Auswahl der Interviewpersonen folgt einer bewussten, theoriegeleiteten Samplinglogik. Da das Ziel der Arbeit nicht die statistische Repräsentativität, sondern die theoretische Sättigung relevanter Perspektiven ist, wurde ein purposives Sampling-Verfahren gewählt \parencite{flick2017}. Als Einschlusskriterien galten: 
eine Führungsposition im mittleren Management, eine Tätigkeit im Bankensektor der \acrshort{dach}-Region sowie nachweisliche Berührungspunkte mit generativer KI im beruflichen Kontext -- sei es durch aktive Nutzung oder durch die Begleitung entsprechender Implementierungsprozesse im eigenen Verantwortungsbereich.
Die Stichprobe umfasst [X] Führungskräfte aus [X] Instituten unterschiedlicher Größe und Ausrichtung, darunter Universalbanken, Sparkassen und genossenschaftliche Institute. Die Variation entlang dieser Dimensionen wurde bewusst angestrebt, um ein möglichst breites Spektrum organisationaler Kontexte abzubilden. Hinsichtlich 
Geschlecht, Alter und Führungsspanne weist die Stichprobe [kurze Beschreibung der Zusammensetzung] auf. Der Zugang zu den Interviewpersonen erfolgte über [Beschreibung des Zugangswegs, z.\,B. berufliche Netzwerke, direkte Ansprache, Verbandsstrukturen]. Alle Teilnehmenden wurden vorab schriftlich über Ziel, Ablauf und Datenschutz der Studie informiert und gaben ihr Einverständnis zur 
Aufzeichnung und wissenschaftlichen Auswertung der Gespräche.

\subsection{Durchführung und Aufzeichnung}
Die Interviews wurden im Zeitraum [Monat/Jahr -- Monat/Jahr] durchgeführt. Auf Wunsch der Teilnehmenden fanden [X] Gespräche als Videokonferenz statt, [X] wurden persönlich vor Ort geführt. Die Wahl des Settings oblag den Interviewpersonen, um eine möglichst vertraute und störungsarme Gesprächsatmosphäre zu gewährleisten. Die durchschnittliche Gesprächsdauer betrug [X] Minuten, bei einer Spanne von [X] bis [X] Minuten.

Alle Interviews wurden -- nach ausdrücklicher Zustimmung -- vollständig aufgezeichnet. Die Aufzeichnungen wurden anschließend nach einem einheitlichen Transkriptionsschema verschriftlicht \parencite{dresingTranskriptionenQualitativerDaten2017}. Zur Wahrung der Anonymität wurden Namen und identifizierende Angaben bereits im Transkript durch neutrale Kürzel ersetzt. Die fertigen Transkripte wurden den Interviewpersonen auf Wunsch zur Durchsicht und Freigabe zur Verfügung gestellt (Member Checking). Sämtliche Materialien werden gemäß den datenschutzrechtlichen Vorgaben aufbewahrt und nach Abschluss der Arbeit fristgerecht gelöscht.
\section{Datenaufbereitung und Transkription}

Die Audioaufnahmen der Interviews werden vollständig transkribiert, um das Datenmaterial für die qualitative Inhaltsanalyse zugänglich zu machen. Die Transkription folgt den Regeln des einfachen Transkriptionssystems nach \textcite{dresingTranskriptionenQualitativerDaten2017}, das für inhaltsanalytische Auswertungsverfahren empfohlen wird. Dieses System priorisiert semantische Genauigkeit gegenüber phonetischer Detailtreue: Dialektfärbungen werden ins Hochdeutsche übertragen, Wort- und Satzabbrüche dokumentiert, Pausen ab drei Sekunden markiert und nonverbale Äußerungen (Lachen, Zögern) in Klammern notiert. Auf eine Notation von Intonationsverläufen oder exakten Pausenlängen wird verzichtet, da die Analyse auf inhaltliche Deutungsmuster abzielt, nicht auf konversationsanalytische Feinstrukturen \parencite{flick2017}.

Die Transkription erfolgt softwaregestützt. Zunächst wird eine automatische Rohtranskription erstellt, die anschließend manuell anhand der Originalaufnahme korrigiert und den Transkriptionsregeln angepasst wird. Jedes Transkript wird vollständig mit der Aufnahme abgeglichen, um Übertragungsfehler auszuschließen. Zeitmarken werden in regelmäßigen Abständen gesetzt, um die Zuordnung von Textstellen zur Aufnahme zu ermöglichen.

Im Zuge der Transkription werden sämtliche identifizierenden Merkmale pseudonymisiert: Namen der Interviewpartner:innen, Institutionen, Standorte und spezifische Funktionsbezeichnungen werden durch neutrale Kürzel ersetzt (z.\,B. \textit{IP-01}, \textit{Bank-A}). Rückschlüsse auf Einzelpersonen oder Organisationen werden durch Verallgemeinerung oder gezielte Auslassung spezifischer Details verhindert. Die Pseudonymisierungslogik wird in einem separaten, passwortgeschützten Schlüsseldokument dokumentiert, das ausschließlich dem Verfasser zugänglich ist.

Nach Abschluss der Transkription und Validierung werden die fertiggestellten Transkripte den jeweiligen Interviewpartner:innen zur optionalen Durchsicht angeboten (\textit{Member Check}). Dieser Schritt dient nicht der Verifizierung der Analyse, sondern der Absicherung, dass die dokumentierte Gesprächsbasis korrekt und vollständig ist \parencite{flick2017}.


\section{Datenauswertung: Strukturierende qualitative Inhaltsanalyse}

Die Auswertung des Interviewmaterials folgt der inhaltlich-strukturierenden qualitativen Inhaltsanalyse nach Kuckartz, die deduktive Kategorienarbeit mit induktiver Offenheit für unerwartete Muster verbindet. Der folgende Abschnitt stellt die methodische Grundlage vor, operationalisiert die deduktiven Hauptkategorien und beschreibt das Vorgehen bei der induktiven Subkategorienbildung.

\subsection{Methodische Grundlage}

Für die Auswertung des Interviewmaterials wird die inhaltlich-strukturierende qualitative Inhaltsanalyse nach \textcite{kuckartz_qualitative_2018} eingesetzt. Diese Methode eignet sich für die vorliegende Arbeit aus mehreren Gründen: Sie ermöglicht eine theoriegeleitete, systematische Auswertung entlang vorgegebener Kategorien, ist aber gleichzeitig offen für induktive Ergänzungen aus dem Material. Damit verbindet sie die Stärken deduktiver Strukturierung mit der Fähigkeit, unerwartete Muster zu entdecken -- eine Eigenschaft, die angesichts des explorativen Charakters der Forschungsfrage zentral ist.

Das Verfahren unterscheidet sich von der qualitativen Inhaltsanalyse nach Mayring, mit der es häufig verwechselt wird. Während \textcite{mayring2016} stärker regelgeleitet und sequenziell vorgeht, betont Kuckartz den iterativen Charakter des Analyseprozesses und die Verschränkung von Fallarbeit und Kategorienarbeit \parencite{kuckartz_qualitative_2018}. Die Analyseschritte verlaufen nicht streng linear, sondern zirkulär: Kategorien werden im Laufe der Auswertung verfeinert, zusammengelegt oder ausdifferenziert.

Der Analyseprozess folgt sieben Phasen \parencite{kuckartz_qualitative_2018}:

\begin{enumerate}
    \item \textbf{Initiierende Textarbeit:} Sorgfältiges Lesen aller Transkripte, Markierung auffälliger Passagen, Verfassen von Memos zu ersten Eindrücken und Deutungen.
    \item \textbf{Entwicklung thematischer Hauptkategorien:} Ableitung der deduktiven Kategorien aus dem theoretischen Rahmen (hier: \acrshort{sdt}-Grundbedürfnisse).
    \item \textbf{Erster Codierdurchgang:} Codierung des gesamten Materials entlang der Hauptkategorien. Jede Textstelle wird der passenden Kategorie zugeordnet; Mehrfachcodierungen sind zulässig.
    \item \textbf{Zusammenstellung aller codierten Textstellen:} Systematische Zusammenführung aller einer Kategorie zugeordneten Textstellen.
    \item \textbf{Induktive Ausdifferenzierung:} Bildung von Subkategorien am Material, die das deduktive Kategoriensystem ergänzen und verfeinern.
    \item \textbf{Zweiter Codierdurchgang:} Erneute Codierung des gesamten Materials mit dem ausdifferenzierten Kategoriensystem.
    \item \textbf{Analyse und Ergebnisdarstellung:} Kategorienbasierte und fallübergreifende Auswertung, Identifikation von Mustern und Zusammenhängen.
\end{enumerate}

Die Codierung erfolgt softwaregestützt mit MAXQDA, das für qualitative Inhaltsanalysen nach Kuckartz explizit konzipiert ist und Funktionen für Codierung, Memo-Verwaltung, Kategorienvergleich und visuelle Analyse bereitstellt.


\subsection{Deduktive Hauptkategorien (SDT-Grundbedürfnisse)}

Die drei psychologischen Grundbedürfnisse der \gls{sdt} bilden das deduktive Kategoriensystem der Analyse. Dieses Vorgehen begründet sich theoretisch: \gls{sdt} postuliert, dass Autonomie, Kompetenz und soziale Eingebundenheit als universelle Grundbedürfnisse die zentralen Mechanismen motivationalen Erlebens darstellen (vgl. Abschnitt 2.3). Die Fragestellung der Arbeit richtet sich explizit auf diese drei Dimensionen; ein deduktiver Einstieg ist daher methodologisch konsistent und nicht willkürlich.

Die Hauptkategorien werden wie folgt operationalisiert:

\paragraph{HK1 -- Autonomieerleben.} Textstellen, in denen Interviewpartner:innen beschreiben, wie der Einsatz generativer \gls{ki} ihr Erleben von Selbstbestimmung, Entscheidungsfreiheit und Handlungsspielraum beeinflusst. Dazu zählen sowohl autonomieförderliche Erfahrungen (erweiterte Gestaltungsspielräume, Entlastung von fremdbestimmten Aufgaben) als auch autonomiefrustrierende Erfahrungen (algorithmische Kontrolle, Einengung von Entscheidungskorridoren, erlebter Zwang zur \acrshort{ki}-Nutzung).

\paragraph{HK2 -- Kompetenzerleben.} Textstellen, in denen Interviewpartner:innen beschreiben, wie generative \gls{ki} ihr Erleben von Wirksamkeit, Expertise und professioneller Kompetenz beeinflusst. Dies umfasst kompetenzförderliche Erfahrungen (erweiterte Fähigkeiten, neue Kompetenzdomänen, verbesserte Ergebnisqualität) und kompetenzfrustrierende Erfahrungen (Expertiseentwertung, Attributionsverschiebung, Überforderung durch neue Anforderungen).

\paragraph{HK3 -- Soziale Eingebundenheit.} Textstellen, in denen Interviewpartner:innen beschreiben, wie generative \gls{ki} ihr Erleben von Zugehörigkeit, Beziehungsqualität und Wertschätzung im organisationalen Kontext beeinflusst. Dies umfasst Erfahrungen von veränderter Teaminteraktion, Rollenverschiebungen in der Führungsbeziehung, organisationaler Wertschätzung sowie neue Formen der Zusammenarbeit oder Isolation.

Ergänzend wird eine Restkategorie \textbf{HK0 -- Übergreifende Kontextfaktoren} geführt, die Textstellen aufnimmt, die für das Verständnis der motivationalen Dynamik relevant sind, aber keiner der drei Grundbedürfniskategorien eindeutig zugeordnet werden können -- etwa organisationale Rahmenbedingungen, regulatorische Kontexte oder biografische Einbettungen.


\subsection{Induktive Subkategorienbildung}

Die deduktiven Hauptkategorien werden im Verlauf der Analyse durch induktive Subkategorien ausdifferenziert. Das Ziel ist, die theoretische Struktur mit den tatsächlichen Erfahrungsmustern der Interviewpartner:innen zu verschränken und so über die bloße Bestätigung oder Widerlegung theoretischer Annahmen hinauszugehen.

Die Subkategorienbildung erfolgt nach Phase 4 des Analyseprozesses: Nachdem alle Textstellen den Hauptkategorien zugeordnet sind, werden die Textpassagen innerhalb jeder Kategorie systematisch gesichtet und nach wiederkehrenden Themen, Deutungsmustern und Erfahrungsdimensionen differenziert \parencite{kuckartz_qualitative_2018}. Die Subkategorien entstehen am Material, nicht aus theoretischen Vorannahmen -- wenngleich Sensibilisierung durch die SDT-Literatur und die in Kapitel 2 identifizierten Spannungsfelder (Kontrolle vs. Unterstützung, Expertiseentwertung vs. neue Kompetenzdomänen, Teaminteraktion vs. Individualisierung) die Aufmerksamkeit der Analyse lenkt.

Konkret ist folgender Prozess vorgesehen: Für jede Hauptkategorie werden die zugeordneten Textstellen fallübergreifend verglichen. Passagen mit ähnlicher inhaltlicher Aussage werden zu thematischen Clustern zusammengefasst und mit einem beschreibenden Label versehen. Diese vorläufigen Subkategorien werden anschließend geprüft: Überlappende Kategorien werden konsolidiert, zu breite aufgespalten, nicht tragfähige verworfen. Das resultierende Kategoriensystem wird in einem Codebuch mit Definitionen, Ankerbeispielen und Abgrenzungsregeln dokumentiert.

Um die interne Konsistenz der Codierung zu sichern, wird eine Teilmenge von [X]\,\% des Materials von einer zweiten Person unabhängig codiert (\textit{konsensuelle Codierung}). Abweichungen werden diskutiert und das Kategoriensystem gegebenenfalls angepasst \parencite{kuckartz_qualitative_2018}. Dieses Verfahren dient nicht der Berechnung von Intercoder-Reliabilität im quantitativen Sinne, sondern der reflexiven Qualitätssicherung.


\section{Gütekriterien qualitativer Forschung}

Qualitative Forschung folgt eigenen Gütekriterien, die sich von den quantitativen Maßstäben Reliabilität, Validität und Objektivität unterscheiden, ohne den Anspruch auf methodische Strenge aufzugeben \parencite{flick2017, mayring2016}. Die vorliegende Arbeit orientiert sich an den von \textcite{mayring2016} formulierten Qualitätskriterien qualitativer Forschung und ergänzt diese um spezifische Maßnahmen.

\paragraph{Verfahrensdokumentation.} Die Nachvollziehbarkeit des Forschungsprozesses wird durch lückenlose Dokumentation aller methodischen Entscheidungen sichergestellt: Forschungsdesign, Sampling-Strategie, Interviewleitfaden, Transkriptionsregeln, Kategoriensystem und Analyseschritte werden im Methodenkapitel offengelegt. Das Codebuch mit Kategoriendefinitionen, Ankerbeispielen und Codierregeln wird als Anhang bereitgestellt.

\paragraph{Argumentative Interpretationsabsicherung.} Interpretationen werden nicht bloß behauptet, sondern durch Ankerbeispiele aus dem Material belegt und theoretisch eingebettet. Alternative Deutungsmöglichkeiten werden systematisch geprüft und diskutiert. Die Ergebnisdarstellung in Kapitel 4 folgt dem Prinzip, Belege und Interpretation transparent zu verzahnen.

\paragraph{Regelgeleitetheit.} Die Analyse folgt dem systematischen Phasenmodell nach \textcite{kuckartz_qualitative_2018}. Codierregeln werden vorab definiert und im Prozess dokumentiert angepasst. Die Regelgeleitetheit wird durch den Einsatz von MAXQDA unterstützt, das die Codierhistorie nachvollziehbar macht.

\paragraph{Nähe zum Gegenstand.} Die Untersuchung orientiert sich an der Lebens- und Arbeitswelt der Befragten. Die problemzentrierten Interviews ermöglichen es, den Erfahrungshorizont der Interviewpartner:innen als Ausgangspunkt der Analyse zu nehmen, statt vorstrukturierte Antwortmuster abzufragen \parencite{witzelProblemcenteredInterview2000}. Der Interviewleitfaden wurde in Pretests geprüft und enthält erzählgenerierende Stimuli, die an konkreten Arbeitssituationen ansetzen.

\paragraph{Kommunikative Validierung.} Die fertiggestellten Transkripte werden den Interviewpartner:innen zur optionalen Durchsicht angeboten (Member Check). Darüber hinaus werden zentrale Analyseergebnisse im Betreuungsprozess diskutiert, um blinde Flecken der Interpretation zu identifizieren \parencite{flick2017}.

\paragraph{Triangulation.} Da die Arbeit ein monomethodisches Design verfolgt (ausschließlich problemzentrierte Interviews), ist eine Methodentriangulation im engeren Sinne nicht möglich. Stattdessen wird auf theoretische Triangulation gesetzt: Die Ergebnisse werden nicht nur im Licht der \gls{sdt}, sondern auch mit ergänzenden theoretischen Perspektiven (\acrshort{jdr}-Modell, Rollentheorie, Attributionstheorie) diskutiert, um eindimensionale Interpretationen zu vermeiden. Zudem ermöglicht das bewusst heterogene Sampling eine Perspektiventriangulation über unterschiedliche Institutionsgrößen, Tätigkeitsschwerpunkte und Länderkontexte hinweg.

\paragraph{Reflexivität.} Der Verfasser steht möglicherweise bei einzelnen Interviewpartner:innen in einem Beschäftigungsverhältnis beim gleichen Institut. Diese potenzielle Verzerrung durch berufliche Netzwerknähe wird durch reflektierte Interviewführung, transparente Offenlegung und die Dokumentation von Reflexionsmemos adressiert. Memos zu eigenen Vorannahmen, emotionalen Reaktionen und methodischen Entscheidungen werden während des gesamten Forschungsprozesses geführt und bei der Interpretation berücksichtigt \parencite{przyborski2021}.


\section{Ethische Überlegungen und Datenschutz}

Die Studie wurde dem MCI Ethics Assessment unterzogen und positiv begutachtet. Die ethischen Grundsätze orientieren sich an den Prinzipien informierter Einwilligung, Freiwilligkeit, Vertraulichkeit und Schadensvermeidung.

\paragraph{Informierte Einwilligung.} Alle Teilnehmer:innen erhalten vor dem Interview ein Informationsblatt mit Angaben zu Zweck, Ablauf, voraussichtlicher Dauer, Freiwilligkeit, Datenschutz und Anonymisierungsverfahren. Vor Beginn des Interviews wird eine schriftliche Einwilligungserklärung eingeholt, die explizit die Tonaufnahme umfasst. Die Teilnehmer:innen werden darauf hingewiesen, dass sie das Interview jederzeit und ohne Angabe von Gründen abbrechen können, ohne Nachteile zu erfahren. Einzelne Fragen können übersprungen werden.

\paragraph{Risikobewertung.} Das Forschungsthema -- Arbeitserfahrungen mit generativer KI -- birgt keine relevanten physischen Risiken. Psychologische Risiken sind gering, können aber nicht vollständig ausgeschlossen werden: Fragen zur Arbeitsplatzveränderung durch KI können Unsicherheiten berühren. Sollte ein:e Teilnehmer:in Unbehagen signalisieren, wird das Interview sensibel gesteuert und auf Wunsch abgebrochen.

\paragraph{Datenschutz und Datenspeicherung.} Audioaufnahmen und Transkripte werden ausschließlich auf verschlüsselten, passwortgeschützten Geräten und \acrshort{dsgvo}-konformen Cloud-Diensten gespeichert. Zugang haben ausschließlich der Verfasser und die betreuende Lehrperson. Alle personenbezogenen Angaben -- Namen, Institutionen, Standorte, spezifische Funktionsbezeichnungen -- werden im Transkript durch Pseudonyme ersetzt. Die Pseudonymisierungslogik wird in einem separaten Schlüsseldokument geführt, das nach Abschluss der Arbeit gelöscht wird.

\paragraph{Löschfristen.} Originalaufnahmen werden nach Abschluss der Transkription und Validierung durch die Teilnehmer:innen gelöscht, spätestens nach Abschluss der Arbeit (September 2026). Anonymisierte Transkripte werden für die Dauer der Archivierungspflicht aufbewahrt.

\paragraph{Rückmeldung.} Teilnehmer:innen, die eine Rückmeldung wünschen, erhalten nach Abschluss der Arbeit eine Zusammenfassung der zentralen Befunde.

%\chapter{Ergebnisse}

\section{Deskriptive Ergebnisse}

\section{Hypothesentests}

\section{Ergebniszusammenfassung}

%\chapter{Diskussion}
\label{dis:dis}
Die empirischen Befunde aus Kapitel~4 werden im Folgenden in den theoretischen und empirischen Kontext eingeordnet. Das Kapitel interpretiert die Ergebnisse vor dem Hintergrund der \gls{sdt}, diskutiert Anknüpfungspunkte und Abweichungen zum bisherigen Forschungsstand, leitet Implikationen für die Praxis ab und reflektiert die Grenzen der Untersuchung.

\section{Interpretation der Ergebnisse vor dem Hintergrund der SDT}
\section{Einordnung in den Forschungsstand}
\section{Implikationen für die Gestaltung von KI in Führungskontexten}
\section{Limitationen der Studie}


%\chapter{Zusammenfassung und Ausblick}
\label{faz:faz}
Das abschließende Kapitel bündelt die zentralen Erkenntnisse der Arbeit, ordnet ihren wissenschaftlichen Beitrag ein und formuliert Handlungsempfehlungen für die Praxis. Ein Ausblick auf offene Fragen schließt die Arbeit ab.

\section{Zusammenfassung der zentralen Befunde}
\section{Beitrag zur Forschung}
\section{Praktische Handlungsempfehlungen}
\section{Zukünftiger Forschungsbedarf}
\chapter{Ergebnisse}

Die folgenden Abschnitte stellen die empirischen Befunde der Interviewanalyse dar. Nach einer Beschreibung der Interviewpartner:innen werden die Ergebnisse entlang der drei SDT-Grundbedürfnisse -- Autonomie, Kompetenz und soziale Eingebundenheit -- strukturiert präsentiert. Abschließend werden übergreifende Deutungsmuster und situative Bedingungen zusammengeführt, die sich quer zu den einzelnen Bedürfnisdimensionen zeigten.

\section{Darstellung der Interviewpartner:innen}
\section{Autonomieerleben im Kontext generativer KI}
\section{Kompetenzerleben im Kontext generativer KI}
\section{Soziale Eingebundenheit im Kontext generativer KI}
\section{Übergreifende Deutungsmuster und situative Bedingungen}

\chapter{Diskussion}

Die empirischen Befunde aus Kapitel~4 werden im Folgenden in den theoretischen und empirischen Kontext eingeordnet. Das Kapitel interpretiert die Ergebnisse vor dem Hintergrund der SDT, diskutiert Anknüpfungspunkte und Abweichungen zum bisherigen Forschungsstand, leitet Implikationen für die Praxis ab und reflektiert die Grenzen der Untersuchung.

\section{Interpretation der Ergebnisse vor dem Hintergrund der SDT}
\section{Einordnung in den Forschungsstand}
\section{Implikationen für die Gestaltung von KI in Führungskontexten}
\section{Limitationen der Studie}

\chapter{Zusammenfassung und Ausblick}

Das abschließende Kapitel bündelt die zentralen Erkenntnisse der Arbeit, ordnet ihren wissenschaftlichen Beitrag ein und formuliert Handlungsempfehlungen für die Praxis. Ein Ausblick auf offene Fragen schließt die Arbeit ab.

\section{Zusammenfassung der zentralen Befunde}
\section{Beitrag zur Forschung}
\section{Praktische Handlungsempfehlungen}
\section{Zukünftiger Forschungsbedarf}


%###### KAPITEL EINFÜGEN ENDE ############

% Erklärung zu generativer KI
% \chapter*{Erklärung zu generativer KI und KI-gestützten Technologien}
% \addcontentsline{toc}{chapter}{Erklärung zu generativer KI und KI-gestützten Technologien}
% In der vorliegenden Praxisarbeit wurde die Unterstützung von generativen KI-Systemen
als Hilfsmittel für die Recherche, die Strukturierung von Inhalten und die Formulierung
von Texten in Anspruch genommen. Der Autor dieser Arbeit war verantwortlich für die
Konzeption, Planung und Durchführung der Forschungsarbeit, einschließlich der Definition
der Forschungsfragen, der Analyse der Daten und der Interpretation der Ergebnisse.

Generative KI-Systeme wurden insbesondere zur Unterstützung in den folgenden Bereichen
eingesetzt:
\begin{itemize}
    \item \textbf{Recherche und Literatursuche:} Generative KI-Systeme halfen bei der Identifikation relevanter Literaturquellen und der Formulierung von Suchstrategien.
    \item \textbf{Ideenentwicklung und Strukturierung:} Die Modelle wurden genutzt, um verschiedene Perspektiven und Ansätze zur Bearbeitung der Themenstellung zu entwickeln und die Struktur der Arbeit zu optimieren.
    \item \textbf{Textformulierung und -paraphrasierung:} Generative KI-Systeme unterstützten bei der Formulierung und Paraphrasierung von Texten, um präzise und verständliche Darstellungen der Ergebnisse und Diskussionen zu gewährleisten.
\end{itemize}

Die finale Entscheidung über die Einbindung von Inhalten, die Interpretation der Ergebnisse
sowie die Schlussfolgerungen der Arbeit oblagen jedoch ausschließlich dem Autor.
Alle durch Generative KI-Systeme generierten Inhalte wurden sorgfältig geprüft, angepasst
und in den Kontext der spezifischen Fragestellung dieser Arbeit integriert. Die Verantwortung
für den gesamten Inhalt und die wissenschaftliche Integrität dieser Arbeit
liegt beim Autor. % Einbindung der ausgelagerten Datei

% Literaturverzeichnis
%\addcontentsline{toc}{chapter}{\bibname}
%\nocite{*} % alle Einträge aus der .bib-Datei ins Literaturverzeichnis aufnehmen
\printbibliography[heading=bibintoc]

% Anhang
\appendix
%\chapter{Interviewleitfaden}
\chapter{Interviewleitfaden}
\label{anhang:interviewleitfaden}

Der nachfolgende Leitfaden orientiert sich am Ansatz des problemzentrierten Interviews
nach \parencite{witzelProblemcenteredInterview2000} und wurde für den Einsatz in
qualitativen Einzelgesprächen mit Führungskräften im mittleren Management des
\acrshort{dach}-Bankensektors entwickelt. Die Legende der Fragekategorien (E\,=\,Erzählstimulus,
S\,=\,Sondierungsfrage, A\,=\,Ad-hoc-Frage) findet sich auf Seite~1 des Dokuments.

\includepdf[pages=-, pagecommand  = {},
  fitpaper     = true,
  noautoscale  = false]{C_Inhalt/Anhang/Interviewleitfaden_Masterarbeit_Maerker.pdf}
\chapter{Einwilligungserklärung und Informationsblatt}
\chapter{Kategoriensystem}
%\chapter{Anhang}

\clearpage
\begin{landscape}
\section*{OB2SDT}
%\addcontentsline{toc}{section}{OB2SDT}
% Preamble (falls noch nicht vorhanden)
% \usepackage{tikz}
% \usetikzlibrary{positioning, arrows.meta, fit, shapes.geometric}

% ------------------------------------------------------------
% FIGURE 1: Vertikale Theorie-Landkarte (OB -> \gls{SDT} -> Kompetenz)
% ------------------------------------------------------------
\begin{figure}[ht]
\centering
\begin{tikzpicture}[
  font=\small,
  node distance=8mm and 14mm,
  box/.style={draw, rounded corners, align=center, inner sep=6pt, minimum width=60mm},
  sub/.style={draw, rounded corners, align=left, inner sep=6pt, minimum width=70mm},
  note/.style={draw, dashed, rounded corners, align=left, inner sep=6pt, minimum width=70mm},
  arrow/.style={-Latex, line width=0.6pt}
]

% Main vertical spine
\node[box] (ob) {Organizational Behavior (OB)\\\emph{Forschungsfeld: Verhalten in Organisationen}};
\node[box, below=of ob] (mot) {Motivation im OB\\\emph{Erklärungen: Warum handeln Menschen engagiert?}};
\node[box, below=of mot] (bridges) {Brückenebene\\Work/Job Design \& Empowerment\\\emph{Arbeitsmerkmale $\rightarrow$ Erleben}};
\node[box, below=of bridges] (sdt) {Self-Determination Theory (\gls{SDT})\\\emph{Bedürfnisse $\rightarrow$ Motivation \& Wohlbefinden}};
\node[box, below=of sdt] (comp) {Kompetenzerleben (AV)\\\emph{Wirksamkeit \& Fähigkeitsgefühl bei Arbeit/Entscheidung}};

\draw[arrow] (ob) -- (mot);
\draw[arrow] (mot) -- (bridges);
\draw[arrow] (bridges) -- (sdt);
\draw[arrow] (sdt) -- (comp);

% Side theories near Motivation
\node[sub, right=of mot, xshift=10mm] (altmot) {\textbf{Alternative Motivationstheorien (relevant, aber nicht Kern)}\\
-- Expectancy / Valence-Ansätze\\
-- Goal Setting\\
-- Agency / Control-Logiken\\
-- JD-R (Ressourcen vs. Belastung)};

\draw[arrow] (altmot.west) -- ++(-8mm,0) |- (mot.east);

% Bridges details
\node[sub, right=of bridges, xshift=10mm] (bridge_detail) {\textbf{Brücken in Richtung \gls{SDT}}\\
-- Job Characteristics / Work Design\\
-- Feedback, Autonomie, Lerngelegenheiten\\
-- Psychological Empowerment (Dimension Kompetenz)};

\draw[arrow] (bridge_detail.west) -- ++(-8mm,0) |- (bridges.east);

% \gls{SDT} details
\node[sub, right=of sdt, xshift=10mm] (sdt_detail) {\textbf{\gls{SDT}-Mechanik (Mikro)}\\
-- Autonomie\\
-- Kompetenz\\
-- soziale Eingebundenheit\\
\emph{Bedürfnisbefriedigung vs. Need thwarting}};

\draw[arrow] (sdt_detail.west) -- ++(-8mm,0) |- (sdt.east);

% \gls{AI} as overlay context (connects to bridges/sdt)
\node[note, left=of bridges, xshift=-10mm] (ai_ctx) {\textbf{Technologiepfad (OB neu)}\\
Algorithmische Systeme / KI verändern\\
Arbeitsgestaltung und Entscheidungsarbeit.\\
Wirkungen sind \emph{wahrnehmungsabhängig}:\\
\emph{Support vs. Control}.};

\draw[arrow] (ai_ctx.east) -- ++(8mm,0) |- (bridges.west);
\draw[arrow] (ai_ctx.east) -- ++(8mm,0) |- (sdt.west);

\end{tikzpicture}
\caption{Vertikale Theorie-Landkarte: Vom Feld Organizational Behavior zur \gls{SDT} und dem Kompetenzerleben.}
\label{fig:ob_sdt_vertical_map}
\end{figure}


% ------------------------------------------------------------
% FIGURE 2: "Spielwiesen" von KI im OB + Eingrenzung auf generative, chatbasierte Augmentierung
% ------------------------------------------------------------
\begin{figure}[ht]
\centering
\begin{tikzpicture}[
  font=\small,
  node distance=8mm and 10mm,
  domainbox/.style={draw, rounded corners, align=left, inner sep=6pt, minimum width=70mm},
  hi/.style={draw, rounded corners, very thick, align=left, inner sep=6pt, minimum width=70mm},
  micro/.style={draw, rounded corners, align=left, inner sep=6pt, minimum width=70mm},
  arrow/.style={-Latex, line width=0.6pt}
]

% \gls{AI} playground domains (left column)
\node[domainbox] (am) {\textbf{Algorithmic Management}\\
Monitoring, Scheduling, Evaluation\\
\emph{typisch: Steuerung/ Kontrolle}};
\node[domainbox, below=of am] (hr) {\textbf{Algorithmic HR / People Analytics}\\
Scoring, Screening, Feedback-Systeme\\
\emph{Risiko: Kontrollattribution}};
\node[domainbox, below=of hr] (teams) {\textbf{Human--\gls{AI} Teams}\\
Koordination, Vertrauen, Rollen\\
\emph{Signaleffekte, Reliance}};
\node[domainbox, below=of teams] (decision) {\textbf{\gls{AI} in Entscheidungen}\\
Empfehlungen, Prognosen, Analytik\\
\emph{Accountability, Transparency}};

% Focus area (right column): Generative \gls{AI} as augmented chat-based support
\node[hi, right=of teams, xshift=18mm] (genai) {\textbf{Fokus dieser Arbeit: Generative KI}\\
\textbf{chatbasiert \& augmentierend}\\
-- kognitive Unterstützung\\
-- Strukturierung / Synthese\\
-- Entscheidungsvorbereitung\\
\emph{Mechanismus: wahrgenommene Unterstützungsqualität}};
\node[micro, below=of genai] (boundary) {\textbf{Explizite Abgrenzung}\\
Nicht im Fokus: verpflichtende\\
algorithmische Steuerung (AM),\\
automatisierte Leistungs-/Kontrollregime.};

% Connections among \gls{AI} playground
\draw[arrow] (am) -- (hr);
\draw[arrow] (hr) -- (teams);
\draw[arrow] (teams) -- (decision);

% Arrow to focus (why focus)
\draw[arrow] (teams.east) -- ++(8mm,0) |- (genai.west);
\draw[arrow] (decision.east) -- ++(8mm,0) |- (genai.west);

% Boundary link
\draw[arrow] (genai) -- (boundary);

% Add a small mechanism box
\node[micro, right=of decision, xshift=18mm, yshift=-12mm] (mech) {\textbf{OB-Mechanismen im KI-Kontext}\\
-- Support vs. Control Attribution\\
-- Vertrauen / Disclosure-Reaktionen\\
-- Bedürfnisbefriedigung (\gls{SDT})};

\draw[arrow] (mech.west) -- ++(-8mm,0) |- (decision.east);
\draw[arrow] (mech.west) -- ++(-8mm,0) |- (genai.east);

\end{tikzpicture}
\caption{Spielwiesen von KI im OB-Kontext und Eingrenzung auf generative, chatbasierte Augmentierung.}
\label{fig:ai_ob_playground}
\end{figure}
\input{C_Inhalt/Anhang/ai_ob_playground}

\end{landscape}
\clearpage

\end{document}